\chapter[HTTP and Web Services]
{HTTP and Web Services\\
  {\Large by Michael Kifer}}


This chapter describes the API for issuing HTTP requests to Web servers.
This facility could be used for reading and querying Web resources and,
perhaps more importantly, for talking to Web services.

\section{General}

The \ERGO-to-Java API is available in the system module \texttt{\bs{}http}
and calling anything in this module will load that module. If, however, for
some reason it is necessary to load this module without executing any
operations, one can accomplish this by calling 
%% 
\begin{itemize}
\item  \texttt{ensure\_loaded@\bs{}http}. 
\end{itemize}
%% 

\section{The HTTP API}

The most important call in the \ERGO Web API is \texttt{http(...)},
described next. 

%% 
\begin{itemize}
\item  \texttt{?URL[http->(?Result,?Warnings)]@\bs{}http} --- a basic
  request to bring back a Web page or to invoke a RESTfull service via a
  GET HTTP method. The result from the server is bound to \texttt{?Result}
  and the errors/warnings from the server, if any, to \texttt{?Warnings}. A
  result is an atom,
  which typically is in the HTML, XML, or JSON format.  
  The warnings are represented as lists of atoms (one per warning)
  or as an empty list, if no warnings.

  If \texttt{?Result} is a zero-length atom, it means that the request
  failed for various reasons. Such reasons may or may not be explained as a
  warning---depending on the server. 
  %% 
\item  \texttt{?URL[http(?OptionList)->(?Result,?Warnings)]@\bs{}http} ---
  a more complex request to a server, which specifies the requirements by
  passing a list of options. This API call supports GET, POST, PUT, and DELETE
  HTTP requests and can be used both for RESTfull as well as non-RESTfull Web
  services. The option list has this form:
  %% 
  \begin{quote}
     \texttt{[}  \emph{option1, option2, ..., optionN} \texttt{]}  
  \end{quote}
  %% 
  where each option either has the form \emph{optionName} = \emph{value}
  or is a Boolean option of the form \emph{optionName}.   (Currently there
  is only one Boolean option: \texttt{delete}.) 
  No \emph{optionName}  can occur in the list more than once, or an error is
  issued.
  The following options are supported:
  %% 
  \begin{itemize}
  \item    \texttt{redirect}: the value must be \texttt{true} (default) or
    \texttt{false}.   Tells the server whether to follow redirection or
    not.
    %% 
\begin{verbatim}
?- 'https://google.com'[http([redirect=false])->?R]@\http.
\end{verbatim}
    %% 
    This will respond with an HTML document saying ``The document has
    moved.''
  \item   \texttt{secure}: the value must be \texttt{false} (default)
    or a path-name to a local file, which contains certificates.
    The certificates must be in the PEM format
    \url{https://support.ssl.com/index.php?/Knowledgebase/Article/View/19/0/der-vs-crt-vs-cer-vs-pem-certificates-and-how-to-convert-them}. Such
    files typically have the \texttt{.pem} or \texttt{.crt} extension.

    If a file-path is specified, the server is verified with respect to the
    certificate. If unsuccessful, \texttt{?Result} is a zero-length atom.
  \item \texttt{timeout}: the value is a positive integer specifying the
    number of seconds to wait before aborting the request.
  \item \texttt{useragent}: the name of the user agent to use in the HTTP header
    when handshaking with the server. Some servers require this, but most
    do not. Servers have no way of verifying the user agent field.
    Example:
    %% 
\begin{verbatim}
?- 'http://myurl.my'[http([timeout=7, useragent='My Ergo crawler'])
                           -> (?Res,?Warn)]@\http.
\end{verbatim}
    %% 
  \item \texttt{header}: The
    value is either an atom (if just one header needs to be passed) or a
    list of atoms, to specify several headers at once. Examples:
    %% 
\begin{verbatim}
header='User-Agent: just me'
header=['Content-Type: application/json','Authorization: Bearer abcdefg']
\end{verbatim}
    %% 
    \item \texttt{auth}: the value must be user/password. This is used if
      the web site requires authentication. For example,
      %% 
\begin{verbatim}
auth=justme/mypasswd      
\end{verbatim}
      %% 
    \item \texttt{post}, \texttt{put}: the value is an atom, typically in
      the JSON or XML format.
      
      These options, if given,  
      will contact the server using the HTTP methods POST or PUT, respectively.
      If none of these options is given (and no \texttt{delete} option),
      the GET method is used.
      One cannot specify both of these at once, and the HTTP method must
      match what the server expects. In case of a mismatch, the server may
      (or may not) send an error message back, which would then be
      available in the warnings list mentioned above, or as contents in
      \texttt{?Result}. Example:
%% 
\begin{verbatim}
?- 'https://myurl.my'[http([auth='me@my'/'my+pw',
                            post='{"message": "Hello"}'])->?R]@\http.
\end{verbatim}
%% 
    \item \texttt{delete}: this is the only Boolean option, so it is
      specified just as \texttt{delete}.
      It tells the server
      to use the DELETE HTTP method. This option cannot be used together with
      \texttt{post} or \texttt{put} options.  Example:
      %% 
\begin{verbatim}
?- "https://google.com"^^\iri[http([delete])->?R]@\http.
\end{verbatim}
      %% 
  (will respond with an HTML file saying ``The request method
  DELETE is inappropriate for the URL'').
  This example also illustrates the point that the URL can be given as a
  plain symbol or as an \texttt{\bs{}iri} data type. 
  \end{itemize}
  %% 
\end{itemize}
%% 

\section{Miscellaneous}

This package includes a number of other useful calls that are often used
together with the \texttt{http(...)} method.

%% 
\begin{itemize}
\item \texttt{?URL[properties->?Props]@\bs{}http} --- returns a list of
  properties of \texttt{?URL}:
  \texttt{[}\emph{PageSize},\emph{ModTime},\emph{RedirectedURL}\texttt{]}.
  Here
  \emph{PageSize} is the
  size of the page in bytes, and \emph{ModTime} is the last modification
  time expressed as the number of seconds since epoch (1900-01-01).   
  Some servers might not return one or both of the last two parameters, in
  which case -1 is returned.
  The last component is the actual URL of the page. If the page at
  \texttt{?URL} was not redirected then \emph{RedirectedURL} is the same as
  the original URL. If that page has a redirection then this and all the
  intermediate redirections are followed  and \emph{RedirectedURL} is the
  final URL in that chain.
  
  This method contacts the network, but it is lighter than
  \texttt{http(...)} and it retrieves properties only.

  Example:
  %% 
\begin{verbatim}
?- \"http://expedia.com"[properties->?R]@\http.
?R = [124787, 1509446769, 'http://www.expedia.com']
\end{verbatim}
  %% 
\item \texttt{?URL[properties(Options)->?Props]@\bs{}http} --- 
  Same as before, but takes a list of options as a parameter. The options
  are the same as for \texttt{http(...)} but some of them (e.g., put, post,
  delete) are ignored.  Example:
  %% 
\begin{verbatim}
?- 'http://google.com'[properties([redirect=false])->?R]@\http.
?R = [-1, -1, 'http://google.com']
?- 'http://google.com'[properties([redirect=true])->?R]@\http.
?R = [-1, -1, 'http://www.google.com']
\end{verbatim}
  %% 
\item \texttt{?URL[encoding->?Enc]@\bs{}http} --- sometimes it is necessary
  to mangle URLs by replacing special characters like '/', ':', etc., so
  they could be used in various situations, like being sent over the network.
  This process is known as URL-encoding. The above method
  returns a list consisting of three components:
  \texttt{[}\emph{Dir},\emph{File},\emph{Ext}\texttt{]}.
  \emph{Dir} is the directory portion of the URL in the URL-encoded
  form, \emph{File} is the file portion (sans the extension; also
  URL-encoded), \emph{Ext} is the file extension portion.
  
  Network is not contacted in order to produce these results, so this
  method is very fast.
\item \texttt{?Item[base64encode->?Item2]@\bs{}http} --- base 64 encoding.
  When sending information over the network, it is necessary to convert
  some of the special characters into ``benign'' ACSII characters that
  would not be mangled by the network. This is called \emph{base 64}  
  \emph{encoding}.  
  This is similar to URL encoding, but is not specific to URLs. Here
  \emph{Item} (the source) can be a character list, an atom, and in the Web
  situation it is most commonly a file specified as
  \texttt{file(}\emph{Path}\texttt{)}. The output, \emph{Item2}, is always
  an atom or a variable that will be bound to an atom.    
  This is used, for example, to upload files to Web services.
  Note that if the source contains the ASCII character '\bs{}0' then this
  source cannot be represented as an atom, for otherwise it will be encoded
  only partially. So, a list-representation or a file should be used.
\item \texttt{?Item[base64decode->?Item2]@\bs{}http} --- base 64
  decoding. This is the opposite process, used when receiving information from
  the networks and decoding it. Here, \emph{Item}--- the source---is always
  an atom, but \emph{Item2} can be an atom, a variable (that will be bound
  to an atom), \texttt{file(}\emph{Path}\texttt{)}, or
  \texttt{list(}\emph{CharlistOrVariable}\texttt{)}.    
  In case of \texttt{file(}\emph{Path}\texttt{)}, the result of decoding is
  stored directly to the specified file. The last representation is used when
  the result of the decoding cannot be represented as an atom because the
  decoded string contains the ASCII character '\bs{}0' and one does not
  want to store the result in a file. So, when
  \emph{Item2} has the form \texttt{list(...)}, it directs the system to
  decode the source as a list of characters.  
\end{itemize}
%% 





%%% Local Variables: 
%%% mode: latex
%%% TeX-master: "../ergo-packages"
%%% End: 
