
\section{ErgoText}\label{sec-ergotext}

\emph{ExgoText} is a high-level facility intended to facilitate knowledge
entry via natural language (NL) style phrases and is primarily targeting 
engineers who are not highly skilled \FLSYSTEM users.
However, skilled \FLSYSTEM users may also find ErgoText very convenient in
many situations.

ErgoText documents typically have two parts: the actual knowledge base that
contains template NL phrases (with the usual extensions
\texttt{.\ergoext} or \texttt{.\ergoext}) 
and another file, with the extension \texttt{.ergotxt},  which 
contains \emph{template definitions}.
Template definitions are typically created by a highly skilled \FLSYSTEM
user; they provide mappings from NL templates to \FLSYSTEM
statements. The interesting feature of ErgoText is that the NL text from
the templates
can be freely mixed with \FLSYSTEM statements in the same document or even
in the same rule. When an text phrase is found, template
definitions are automatically looked up. If no matching template exists for the
phrase in question, a compile-time error is issued. Otherwise, the
\FLSYSTEM template definition is used instead of the phrase.

\subsection{ErgoText Knowledge Bases}\label{sec-ergotext-kb}

\index{ErgoText}
%%
In this section, we assume that we have a file, \texttt{mykb.\ergoext},
which contains \FLSYSTEM statements intermixed with ErgoText phrases.
Before such phrases can be used, the template definition file (say,
\texttt{mytemplates.ergotxt})\footnote{
  It is perfectly fine to use the same base name for the template file,
  i.e., \texttt{mykb.ergotxt}.
}
%% 
must be loaded either by executing the \emph{runtime}  query
\index{ergotext\{...\} primitive}
%% 
\begin{verbatim}
   ?- ergotext{mytemplates}.
\end{verbatim}
%% 
\emph{before} compiling \texttt{mykb.\ergoext} or it must be declared
at compile time by including the directive
%% 
\begin{verbatim}
   :- ergotext{mytemplates}.
\end{verbatim}
%% 
(note: \texttt{:-}, \textbf{not} \texttt{?-}) at the top of
\texttt{mykb.\ergoext}.  Note that \texttt{mytemplates.ergotxt} must be
\emph{locatable,} i.e., \ERGO must be able to find it. If the above
\texttt{ergotext} query or directive appears inside a file, such as 
\texttt{mykb.\ergoext}, then
\texttt{mytemplates.ergotxt} must be a path \emph{relative to the folder}
where \texttt{mykb.\ergoext} resides. It can also be an absolute path name,
but this is \textbf{strongly discouraged} because it will make your knowledge
base non-portable between machines. 
Thus, if
\texttt{mykb.\ergoext} and \texttt{mytemplates.ergotxt} are in the same
directory then simply \texttt{ergotext\{mytemplates\}} will do. If, say, all
templates are in the \texttt{templates\_dir} subfolder with respect to
the file \texttt{mykb.\ergoext} that contains the directive,
then \texttt{ergotext\{templates\_dir/mytemplates\}}  would
work (the forward slash will work both for Unix \emph{and} Windows). 

If, on the other hand,
\texttt{ergotext\{mytemplates\}} is posed as a query \emph{on command line}
in the \ERGO shell \emph{then} \texttt{mytemplates.ergotxt} would typically
be an absolute path name. It can also be a path name relative to the
\emph{current folder}, but in case of \ERGOAI Studio it is not always obvious
what the current folder is.

Invoking \texttt{ergotext\{?X\}} as a query with an unbound variable
argument will bind the argument to the currently active template file or will
fail if none is active.

\index{active ErgoText template}
\index{ErgoText template!active}
Note that \underline{at most one} ErgoText
  template is \emph{active}  at any given
moment. The active template is determined
by either the most recent query of the form \texttt{?-
  ergotext\{...\}} or (if set within a file) by the nearest previous directive
\texttt{:- ergotext\{...\}} within the same (or \texttt{\#include'd})  file.
This implies that a file can have several declared templates, but all the
statements in-between the consecutive \texttt{:- ergotext} declarations
will be interpreted by exactly one template. For instance, in
%% 
\begin{alltt}
   :- ergotext\{template1\}.
   \emph{statement1}  
   :- ergotext\{template2\}.
   \emph{statement2}  
   \emph{statement3}  
   :- ergotext\{template3\}.
   \emph{statement4}  
   \emph{statement5}  
\end{alltt}
%% 
the statements 2 and 3 will be interpreted using \texttt{template2} and
statements 4 and 5 via \texttt{template3}.
This also means that all the ErgoText
phrases that appear in statements 1 to 3 must be defined in
\texttt{template2.ergotxt}. If one is defined, say, in
\texttt{template3} or in \texttt{template1} but
not in \texttt{template2} then an error will be issued.
It should be said, however, that using more than one template definition file
per \ERGO module
is a bad idea. It is a sign of poor design and \textbf{should be avoided}.

\index{ErgoText phrase}
%%
An \emph{ErgoText phrase} has the form
%% 
\begin{verbatim}
    \(word1 word2 ... wordN\)
\end{verbatim}
%% 
where each word is either an word, a HiLog term, including
variables, or even an ISA or frame formula.
Special symbols like parentheses, commas,
semicolons, +, -, =, $\mid$, etc., are also allowed in ErgoText
phrases. They can be standalone symbols or play the role of punctuation.
Note that NL words are a special case of
HiLog terms. Note that words do not need to be in English or, for that
matter, use Latin characters, but in case of non-Latin characters the words
have to be enclosed in single
quotes, like any atom that contains non-alphanumeric characters.
This also applies to text fragments that contain single or double quotes.
For instance,
%% 
\begin{verbatim}
   \(My parent's house\)
   \(My parent\'s house\)
\end{verbatim}
%% 
will be rejected by the parser, but all of the ones below
%% 
\begin{verbatim}
   \(My 'parent''s' house\)
   \(My 'parent\'s' house\)
   \(My parent''s house\)
\end{verbatim}
%% 
are fine. (Note that in the last case \texttt{parent''s} actually consists
of \emph{three} separate tokens: \texttt{parent}, \texttt{''}, and
\texttt{s} and separating them with spaces will yield the same three tokens,
\texttt{parent} \texttt{''} \texttt{s}, and the same template.)

The following is a sample \ERGO knowledge base that consists mostly of
ErgoText phrases, some of which (S3,S5) are intermixed with the normal
\ERGO syntax.
%% 
\begin{verbatim}
   \(Tom is not only a Student, but also a (salaried) Employee\).      // S1
   \(If ?someone has ?something and paid for it then s/he likes it\).  // S2
   \(?X cares for ?Y\), ?Y[inGoodShape] :-                             // S3
                   \(?X has ?Y\), ?Y:ValuableThing.
   \(insert rule: if ?X has more than 5M then he is rich\).            // S4
   ?- delete{\(Tom is an Employee\)}, ?X:Employee.                     // S5
   ?- deleterule{\(?someone with at least 5M is rich\)}.               // S6
\end{verbatim}
%% 
Each phrase must match one and only one template (matching templates for
the above are given below). All details of the syntax of the corresponding
template must be followed to the letter: commas, parentheses (cf. S1),
letter case, etc. Only variables can be replaced with HiLog terms when
templates are used. For instance, as will be seen shortly, in S1 we replaced the
variables in the template (T1) with the terms \texttt{Tom},
\texttt{Student}, and \texttt{Employee}, respectively.   
The phrases S1, S2, and S4 also illustrate the use of special features, like
parentheses, commas, slashes, etc., which might be desirable for
readability and other purposes.

Needless to say that the template text must be
clear, unambiguous, and easily understood in order for ErgoText to be
useful; achieving these goals is the job of the template designer. Typically, templates
are designed for a particular domain (e.g., legal documents) and thus must
involve both a domain expert to supply the template text and semantics
and a skilled \ERGO user to provide the \ERGO translation.

To summarize, ErgoText phrases are NL phrases that can be mixed
with \FLSYSTEM syntax, such as terms, variables, etc. These phrases must
be bound to \FLSYSTEM statements using template definitions (see
below). These definitions are kept in separate template definition files (with the
extension \texttt{.ergotxt}) and these files must be either declared or
loaded (with the \texttt{ergotext\{...\}} directive) before compiling the
\FLSYSTEM knowledge bases that contain the ErgoText phrases in question.
These phrases can also be used in the \FLSYSTEM shell, provided that the
corresponding template definition file is loaded beforehand---again
using an appropriate \texttt{ergotext\{...\}} directive.

\subsection{ErgoText Template Definitions}

\index{ErgoText template}
%%
ErgoText template definitions are supposed to be stored in files with the
extension \texttt{.ergotxt}. Each template definition has the form 
%% 
\begin{alltt}
    template(\textnormal{\textit{Context}, \textit{ErgoText phrase}, \textit{Ergo translation}})
\end{alltt}
%% 
\emph{ErgoText phrases}
were described in the previous subsection. The \emph{Ergo
translation}   is an Ergo formula into which the phrase in the second
argument is to be
translated. What is allowed as such translations depends on the context
specified as the first argument.
The \texttt{Context} argument is one of the following reserved keywords (they are
reserved
only in the template definitions, not in general):
\index{ErgoText context!insert}
\index{ErgoText context!delete}
\index{ErgoText context!insdel}
\index{ErgoText context!headbody}
\index{ErgoText context!dynrule}
\index{ErgoText context!head}
\index{ErgoText context!body}
\index{head!ErgoText context}
\index{body!ErgoText context}
\index{headbody!ErgoText context}
\index{insert!ErgoText context}
\index{insert!ErgoText context}
\index{delete!ErgoText context}
\index{insdel!ErgoText context}
\index{dynrule!ErgoText context}
%% 
\begin{itemize}
\item  \texttt{head},  \texttt{body},  \texttt{headbody} 
\item  \texttt{insert},  \texttt{delete},  \texttt{insdel} 
\item  \texttt{dynrule}  
\end{itemize}
%% 
Why is context needed? The problem is that the same \ERGO formula may
need to be compiled differently depending on where it occurs: rule head,
rule body, in an insert or delete statement, etc. Some syntax may be allowed, say,
in the rule body and insert/delete statements (e.g., \texttt{@module}) but
not in rule heads; rule Ids and tags can occur only in facts and rules, but
not in rule bodies; delete statements allow syntax that is not allowed in
insert statements, and so on.  Thus, the \texttt{head} context means that
the corresponding template is allowed in rule heads (and facts), but not in
rule bodies (e.g., S3 above). The \texttt{body} context means the template
is suitable only for rule bodies (also S3).  Since often the same template
might be allowable in both contexts, a combined \texttt{headbody} context
is provided for convenience (e.g., for S1 above).  The \texttt{insert} and
\texttt{delete} contexts are provided for the templates that are meant to
be used
inside the \texttt{insert\{...\}} and \texttt{delete\{...\}} statements,
respectively (cf., S5). Some formulas may be suitable for both of these
update commands and also for rule heads and bodies. An
all-encompassing context \texttt{insdel}, which subsumes all four of the
aforesaid contexts, is provided for this kind of
formulas, for convenience.  Finally, \texttt{dynrule} is a context for
formulas that are meant to appear inside \texttt{insertrule\{...\}}
and \texttt{deleterule\{...\}} statements (as in S6 above).

\index{ErgoText context!dynrule}
%%
Apart from the above contexts, there are, of course, rules (outside of
\texttt{insertrule\{...\}} and \texttt{deleterule\{...\}}), like 
S2 above, queries as in S4, and latent queries (not shown). The context for
these is ``anything else'' and any keyword other than the above reserved
ones can be used for them. It is a good practice to use \texttt{rule}, \texttt{fact},
\texttt{query}, and \texttt{latentquery} in these cases, for documentation
purposes (so that you will be able to recognize and easily understand
the purpose of the corresponding templates). Internally, these latter contexts are represented by the keyword
\texttt{toplevel}, and this is what is shown by various debugging
primitives described in Section~\ref{sec-ergotext-debug}. 

Here are examples of template definitions that are suitable for the above
knowledge base (S1-S6) to be placed in \texttt{mytemplates.ergotxt}:
%% 
\begin{verbatim}
   template(headbody,   // headbody is reserved context                  (T1)
            \(?X is not only a ?Y, but also a (salaried) ?Z\),
            (?X:{?Y,?Z})).
   template(rule,       // rule is not a reserved context keyword        (T2)
            \(If ?P has ?Thing and paid for it then s/he likes it\),
            (?P[likes->?Thing] :- ?P[{has,paidfor}->?Thing])).
   template(head,      // reserved context, used for documentation       (T3)
            \(?X cares for ?Y\), ?X[caresfor->?Y]).
   template(body,      // reserved context keyword                       (T4)
            \(?X has ?Y\), ?X[has->?Y]).
   template(query,     // not reserved: query is used for documentation  (T5)
            \(insert rule: if ?X has more than 5M then he is rich\),
            (?- insertrule{?X:rich :- ?X[networth->?W], ?W>50000000})).
   template(insdel,    // reserved for insert{...}/delete{...}           (T6)
            \(?X is an ?Y\), ?X:?Y).
   template(dynrule,   // reserved                                       (T7)
            \(?someone with at least 5M is rich\),
            (?someone:rich :- ?someone[networth->?W], ?W>=50000000))
\end{verbatim}
%% 
The reader may have noticed that
the \FLSYSTEM side of the above definitions (argument 3) is
sometimes enclosed in parentheses, as in (T1), (T2), (T5), and (T7). This is
needed in cases when the definitions contain infix operators, like
\texttt{?-}, \texttt{:-}, \texttt{<==}, and the like, which bind its
arguments weaker than the comma (or when the definitions contain commas,
colons, as in (T1), or other operators, like
semicolons). In these cases, the templates might not parse
without the aforesaid parentheses.

Note that template definitions can be arbitrarily complex as long as they
fit the declared context. For instance,
%% 
\begin{verbatim}
    template(fact,
             \(Every state has a senator\),
             (forall(?State)^exist(?Sen)^senator(?State,?Sen) <== ?State:State))
\end{verbatim}
%% 

In case one wants to create a template for rules or facts with descriptors,
such as rule Ids, tags, and the like, keep in mind that rule descriptor
constructs like \texttt{@\{...\}} and \texttt{@!\{...\}} are not permitted
in ErgoText phrases. Instead, one should just use NL and whatever
symbols are allowed to stick in the variables for rule Ids and tags. For
instance, a template definition that defines a rule with descriptors could
look like this:
%% 
\begin{verbatim}
 template(rule,
          \(Here is a rule with Id=?I and Tag=?T: If it rains, it pours\),
          (@!{?I} @{?T} pours(?X) :- rains(?X))).
\end{verbatim}
%% 
One can then construct a concrete \FLSYSTEM rule via the following
ErgoText phrase:
%% 
\begin{verbatim}
 \(Here is a rule with Id=badweather & Tag = watchout: If it rains, it pours\).
\end{verbatim}
%% 

\subsection{ErgoText Queries and Command Line}

So far we have been talking about ErgoText statements that appear in files.
What if we want to ask an ErgoText query using the \ERGO shell, i.e., 
on command line (in a terminal,
the Studio Listener, or the Studio Query tool)?  Just as with file-based
ErgoText, one must tell the command line processor which templates to
use. There are two ways to do that.

The most common and convenient way to import a template definition file into the \ERGO
shell is to load or add
a file that contain the directive \texttt{:- ergotext\{...\}}.
The template specified in that directive will be then imported into the
module to which the file was loaded or added.
For instance, if  a file, say
\texttt{example.ergo} is loaded into the module \texttt{main} and contains
a directive of the form \texttt{:- ergotext\{'some/place/templ'\}}     
then one can issue a query like
%% 
\begin{verbatim}
    ?- \(my ?H template\).
\end{verbatim}
%% 
in the \ERGO shell (assuming such a template exists in
\texttt{some/place/templ.ergotxt}).
If \texttt{example.ergo} is loaded into the module \texttt{foo} then
%% 
\begin{verbatim}
    ?- \(my ?H template\)@foo.
\end{verbatim}
%% 
can be used as a command line query.

The other method is to load a template definition file explicitly. This can be used if
the first method cannot be used for some reason.
In Section~\ref{sec-ergotext-kb}, we already mentioned that ErgoText
template definition files can be loaded at runtime by executing the query
%%
\index{ergotext\{...\} primitive}
%% 
\begin{verbatim}
   ?- ergotext{mytemplates}.
\end{verbatim}
%% 
where \texttt{mytemplates.ergotxt} is a suitable template file. 
To load a template definition file into a module other than \texttt{main}, a two-argument
directive can be used. For instance,
%% 
\begin{alltt}
    ?- ergotext\{mytemplates,foobar\}.
\end{alltt}
%% 


\subsection{ErgoText and \ERGO Modules}

ErgoText understands \ERGO modules, so the expressions like
%% 
\begin{verbatim}
    p(?H) :- \(my ?H template\)@foo.
\end{verbatim}
%% 
are understood subject to the following:
%% 
\begin{enumerate}
\item  A compiler directive
  %% 
\begin{alltt}
    :- ergotext\{mytemplate,foo\}.   
\end{alltt}
  %% 
  must appear at the top of the file containing a phrase like
  \texttt{\bs(my ?H template\bs)@foo} above and the template definition file
  \texttt{mytemplate.ergotxt} must exist and be locatable, i.e.,
  it must either be an absolute path name (not
  recommended) or a name relative to the directory of the containing file.
\item  The module specification must be a constant;
  a variable or a quasi-constant like \bs$@$ will cause the compiler
  to issue an error. That is, something like the following
%% 
\begin{verbatim}
      ?- ?Mod = foo, \(my ?H template\)@?Mod.
      ?- \(my ?H template\)@\@.
\end{verbatim}
%% 
  are not allowed.
\end{enumerate}
%% 

\subsection{Debugging ErgoText Templates}
\label{sec-ergotext-debug}

\index{ErgoText template!debugging}
\index{ErgoText template!overlapping contexts}
\index{ErgoText template!disjoint phrases}
\index{ErgoText template!ambiguity}
\index{ambiguity!among ErgoText templates}
%%
\FLSYSTEM makes many syntactic checks to verify the templates are
syntactically correct and otherwise make sense. It also checks that
there is no \emph{ambiguity} among templates; that is, that
templates with \emph{overlapping} contexts have {\rm disjoint} template
phrases.

Two contexts overlap if their phrases can be used in
the same syntactic position in \FLSYSTEM statements.
Clearly, identical contexts (like
\texttt{body}/\texttt{body} or \texttt{head}/\texttt{head}) overlap. In
addition, \texttt{headbody} is a shorthand for a set of contexts
\texttt{body}, \texttt{head}, and \texttt{toplevel}. The context
\texttt{head} actually also includes \texttt{toplevel}, while \texttt{insdel} 
includes \texttt{body}, \texttt{head}, \texttt{insert}, \texttt{delete},
and \texttt{toplevel}. So, \texttt{head} overlaps with \texttt{insdel} and
\texttt{toplevel}, while \texttt{insdel} overlaps with \texttt{body},
\texttt{head}, and others.
Two templates are \emph{disjoint}
if they cannot match the same ErgoText
phrase; in other words, if they do not unify.

Besides the checks,
\FLSYSTEM provides several primitives to help the template designer to debug the
templates before releasing them to the end user.
The most typical problem is when an ErgoText phrase \emph{seems} to match
a template, but it does not and the compiler issues an error claiming that
no template definition matches the phrase. To better see the problem,
consider the following template:
%% 
\begin{verbatim}
   template(body,\(Is ?X true?\), ?X).
\end{verbatim}
%% 
and a query
%% 
\begin{verbatim}
   ?- \(Is p:q true?\).
\end{verbatim}
%% 
Intuitively, one might think that this should be translated into the query
\texttt{?- p:q}, but it does not. The reason is that \FLSYSTEM's syntax is
rich with infix operators, like \texttt{:}, \texttt{::}, \texttt{->},
comma, etc.  These operators have precedence values that indicate which
operator binds tighter. The magic of the operator grammars combines the
above text into regular terms of the form
\texttt{functor(argument\_list)}. Since the details of that algorithm are
complicated and one cannot and does not want to consult the table of
precedence values, it is sometimes hard to see how a phrase, like the
above, will be parsed. The bottom line here is that the above query will
\emph{not} match the template---to much surprise and chagrin of the
template developer and the user. To help debug such cases, \FLSYSTEM
provides a parser API call \texttt{show\_ergotext\_phrase\_as\_term/1},
which can be used as follows:
%% 
\index{show\_ergotext\_phrase\_as\_term/1}
\begin{verbatim}
   ?- show_ergotext_phrase_as_term(\(Is p:q true?\))@\prolog(flrparser).
   ErgoText phrase parsed form:
       [Is, p : (q, (true, ?))]
\end{verbatim}
%% 
This means that the above ErgoText query gets parsed as a list of two terms
the second of which
is \texttt{p:(q,(true,?))}. Asking the same question about the
template phrase, we get:
%% 
\begin{verbatim}
   ?- show_ergotext_phrase_as_term(\(Is ?X true?\))@\prolog(flrparser).
   ErgoText phrase parsed form:
       [Is, ?X, true, ?]
\end{verbatim}
%% 
which is quite different. In such a situation, one can force the parser do
the needful by placing commas in strategic places. For instance,
%% 
\begin{verbatim}
   ?- show_ergotext_phrase_as_term(\(Is, p:q, true?\))@\prolog(flrparser).
   ErgoText phrase parsed form:
       [Is, p : q, true, ?]
\end{verbatim}
%% 
yields the same parse as the phrase in the template, and one can verify
that the above ErgoText phrase will now match the template.

\index{ergo\_show\_active\_templates/0}
%%
To list all the active templates (i.e., templates that are
currently affecting the parser), \FLSYSTEM's parser provides
\texttt{ergo\_show\_active\_templates/0}. The following example
shows a use of that call along with some of the results it produces.
%% 
\begin{verbatim}
    ?- ergo_show_active_templates@\prolog(flrparser).
    Active ErgoText templates:

      Template context:    body
      Template phrase:     ?A is not only a ?B but also a (salaried) ?C
      Template definition: ?A : {?B, ?C}

      Template context:    head
      Template phrase:     ?A cares for ?B
      Template definition: ?A[caresfor->{?B}]

      Template context:    body
      Template phrase:     ?A has ?B
      Template definition: ?A[has->{?B}]

      Template context:    delete
      Template phrase:     ?A is an ?B
      Template definition: ?A : ?B
\end{verbatim}
%% 
If one wants to see templates that are \emph{not} currently loaded (active),
the API call \texttt{ergo\_show\_templates/1} will do it:
%% 
\begin{verbatim}
    ?- ergo_show_templates(template3)@\prolog(flrparser).
\end{verbatim}
%% 
The output is similar to \texttt{ergo\_show\_active\_templates/0}.

Given an ErgoText phrase, one might want to
see the templates that match it. This is done via
the calls \texttt{ergo\_show\_matching\_templates/2} and
\texttt{ergo\_show\_active\_matching\_templates/2}.
The difference between these calls is that the latter matches against the
active templates that are in effect in the \emph{module} specified in the second
argument, while the former
takes the template file argument explicitly. For instance,
%% 
\begin{verbatim}
   ?- ergo_show_matching_templates(\(Is,p:q,true?\),template3)@\plg(flrparser).
   Context:    body
   Definition: p : q
\end{verbatim}
%% 
or
%% 
\begin{verbatim}
   ?- ergo_show_active_matching_templates(\(Is,p:q,true?\),main)@\plg(flrparser).
\end{verbatim}
%% 
The latter produces similar output but it contains the matches against
the templates currently active in module \texttt{main}. The context part in the above listing is
either one of the reserved keywords \texttt{body},  \texttt{head},
\texttt{insert}, \texttt{delete}, or  \texttt{dynrule}, or it is
\texttt{toplevel}. The latter corresponds to all other contexts, i.e.,
rule, query, latent query, or fact (facts can match either \texttt{head}
context or \texttt{toplevel}).   

\subsection{ErgoText and Error Reporting}

ErgoText templates present a number of challenges when it comes to error
reporting, and the template designer must be aware of the possible
pitfalls. One such pitfall has to do with templates that obscure so much of the
underlying \FLSYSTEM formulas that it is hard for the end user to
understand the reasons for various errors and warnings.
To illustrate, consider the following buggy template:
%% 
\begin{verbatim}
   template(body,\(test\),(\neg insert{foobar})). 
\end{verbatim}
%% 
Here explicit negation is applied to the insert operator, which is an error.
The problem is that the template compiler uses a shallow parser, which does
not catch all the errors and leaves further error detection to a later
stage. As a result, the above template would be compiled without
any errors or warnings. The error (``illegal use of explicit negation'')
will be reported
only when the template will be actually used at compile or run time:
%% 
\begin{verbatim}
   :- ergotext{mytemplates}. 
   ?- \(test\).
\end{verbatim}
%% 
Although the
error will point to the line number where the above ErgoText phrase was
used, the template designer (who failed
to debug his templates) --- and certainly the end user --- will find it
difficult to
understand what is going on because the template phrase provides no hint
regarding the underlying \FLSYSTEM statement.

For another example, consider the template
%% 
\begin{verbatim}
   template(head,\(test\),p(b)).
\end{verbatim}
%% 
and the knowledge base 
%% 
\begin{verbatim}
   :- ergotext{mytemplates}. 
   f(p(a)).
   \(test\).
\end{verbatim}
%% 
Although this is not a mistake, it is suspicious: the symbol \texttt{p/1} is used in two
different ways here---once explicitly, as a HiLog function symbol, and once
implicitly,
as a predicate symbol (in the template). \FLSYSTEM will issue a warning
about the ambiguous use of \texttt{p/1}  (see Section~\ref{sec:symbol}),
but the end user will likely be at a loss as to why.

The lesson of these examples is that ErgoText templates must be
designed in a way that the template phrases will
have enough information to help the end user resolve the possible issues
with errors and warnings reported by \FLSYSTEM.


\subsection{ErgoText and Text Generation}\label{sec-ergo-textgen}

\emph{Text generation}   refers to the ability to generate textual
description of \ERGO facts, queries, and rules. Currently text generation
is used for generating textual explanations to query answers---see
Section~\ref{sec-expl-answers}.

\index{text generation}
\index{text generation!ErgoText-based}
\index{text generation!TextIt}
\ERGO supports two kinds of text generation: ErgoText-based and a
light-weight \emph{TextIt} facility, described in Section~\ref{sec-textit}.
As far as ErgoText-based text generation is concerned, it is triggered
whenever a subgoal is chosen as an explanation
and that subgoal matches the definitional part in an ErgoText template. In
that case,
the NL phrase in the template will be shown as an explanation
instead of the subgoal.
Only the templates of type \texttt{body} (and, by implication,
\texttt{headbody} and \texttt{insdel}) are used for text generation.
Thus, if you want to \emph{exclude} a certain template from being used for text
generation, use an ErgoText context other than the above three. For instance,
\texttt{rulehead}, \texttt{rulebody}, or some other easily recognizable  
keyword.

Text generation via ErgoText and TextIt can be used together.
If a subgoal matches both an ErgoText template and a TextIt template,
the latter will be shown as an explanation. If only one kind of a template
(ErgoText or TextIt) matches the subgoal, the text generated from that
template will be used.



%%% Local Variables: 
%%% mode: latex
%%% TeX-master: "../ergo-manual"
%%% End: 
