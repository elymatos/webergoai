

\section{Omniformity: General Formulas in Rule Heads}\label{sec-omniform} 

\index{quantifier!forall}
\index{quantifier!exist}
\index{quantifier!exists}
\index{omniformity}
\index{\url{==>}}
\index{\url{<==}}
\ERGO greatly expands the syntactic form of the rules in \FLORA
by allowing
general first-order formulas in the \emph{rule heads}---a feature called
\emph{omniformity}. 
This feature is
particularly important for translating English sentences into logic, but
also in other kinds of reasoning about knowledge, especially 
when contrapositive forms of the rules are required during the reasoning.

\index{omni=on!compiler directive}
\index{compiler directive!omni=on}
However, since this is an advanced feature that requires a very good
understanding of the meaning of such expanded formulas, omniform rules are
not enabled by default: the user must request omniformity explicitly by
placing the compiler directive
%% 
\begin{verbatim}
   :- compiler_options{omni=on}.
\end{verbatim}
%% 
\index{omni\{...\}!primitive}
\index{primitive!omni\{...\}}
More information on compiler options appears in Section~\ref{misc-options}.
In the \ERGO shell, one can also enable this feature by executing the command
%% 
\begin{verbatim}
   ?- omni{on}.    // use omni{off} to turn that off.
\end{verbatim}
%% 

\index{head-formula}
In \FLSYSTEM (with the omniformity feature turned on), the head formulas can be statements composed out of the
following elements in a logically correct syntactic forms. That is, a
\emph{head-formula} is a logical formula recursively defined as follows:
%% 
\begin{itemize}
\item  an atomic formula (frame or HiLog);
\item  $\RULELOGNEG\,head$, where $head$ is a head-formula.
\item  $head_1,head_2$ and $head_1;head_2$, where $head_1$ and $head_2$ are
  head formulas. As before, \texttt{\bs{}and} and \texttt{\bs{}or} can be used in place
  of ``,'' and ``;'', respectively.
\item $(head)$, where $head$ is a head-formula.
\item  $head_1 ~op~ head_2$, where $head_1$ and $head_2$ are
  head formulas and $op$ is \texttt{==>}, \texttt{<==}, or \texttt{<==>}.
\item  $\texttt{forall}(VarList)\widehat{\ }(head)$ and $\texttt{exist}(VarList)\widehat{\ }(head)$, where
  $head$ is a head-formula and \texttt{VarList} is a comma-separated list of
  variables. The keyword \texttt{exists} can be used in place of
  \texttt{exist} and parentheses can be omitted, if \emph{head} is an
  atomic formula.   
\end{itemize}
%% 
Note that $\RULELOGNAF$, the implications \rnafarr, \lnafarr,
\rlnafarr, and the rule connective \texttt{:-}  are \emph{not permitted} in rule heads.
Likewise, builtins (e.g., $=$, $!=$, \texttt{\bs{}is}, etc.), calls to Prolog,
and non-logical
operators (e.g., \texttt{load}, \texttt{compile}, updates, etc.) are
prohibited. This type of formulas must always go into rule bodies.
Also note that formulas like
%% 
\begin{alltt}
   \ensuremath{\phi} <== \ensuremath{\phi}.
   \ensuremath{\phi} ==> \ensuremath{\phi}.
\end{alltt}
%% 
are syntactically classified in \ERGO as \emph{facts}, \emph{not} rules,  even
though it is
tempting to call these constructs rules because they are implications.
Thus, both $\phi$ and $\psi$ appear in the \emph{head} there (since facts
can be viewed as body-less rules). 
In \ERGO, a rule always has the form $\phi$\texttt{:-}$\psi$, and only this
type of constructs have rule bodies ($\psi$ in this case).
We will explain the meaning of $\rnegarr$ and $\lnegarr$ in rule heads
shortly.

For clarity, consider some examples first. Suppose we know that
\texttt{Agent666} is either a terrorist or works for an enemy state and
that either case is dangerous:
%% 
\begin{verbatim}
   enemystate(Agent666) \or terrorist(Agent666).
   dangerous(?X) :- terrorist(?X) ; enemystate(?X).
\end{verbatim}
%% 
In classical logic, it would be possible to derive \texttt{dangerous(Agent666)},
but $\FLSYSTEM$ cannot reason by cases and it cannot reach such a
conclusion just yet.
However, if we also know that \texttt{\RULELOGNEG enemystate(Agent666)}  is
true then
$\ERGO$ would be able to derive \texttt{terrorist(Agent666)} and thus
also \texttt{dangerous(Agent666)}.

The inability of \ERGO to reason by cases is a fundamental limitation
of this class of logic languages, and
working around it is difficult.\footnote{
  But this has very important advantages from the point of view of
  computational complexity the reasoning.
}
%% 
For instance, even if \texttt{foo \bs{}or
  foo} is given, \ERGO still cannot conclude \texttt{foo}. 


For another example, let us represent the statement that U.S. Senate has exactly
two senators from each state. We give two alternative representations:
%% 
\begin{verbatim}
    // version 1
    forall(?St)^exist(?Sn1,?Sn2)^(senator(?St,{?Sn1,?Sn2}) <== state(?St)).
    // version 2
    exist(?Sn1,?Sn2)^senator(?St,{?Sn1,?Sn2}) :- state(?St).
    state(NY).
    state(AL).
    state(TX).
    ...
\end{verbatim}
%% 
The two statements have somewhat different properties with respect
omniformity---a concept discussed next.


\index{omniformity of \texttt{<==}, \texttt{==>}, \texttt{<===>}}
\index{\texttt{<==}!omniformity}
\index{\texttt{==>}!omniformity}
\index{\texttt{<==>}!omniformity}
\subsection{Omniformity of \texttt{<==}, \texttt{==>}, and \texttt{<===>}}
We have already seen that the implication \texttt{<==} and its related forms
are different from $\lnafarr$ and friends in that the latter cannot occur in
the rule heads. They are also different from the rule connective
\texttt{:-} in that the body of a \texttt{:-} can be more general and
\texttt{<==} -- statements are not even called rules.
There are also other differences.
The most interesting one is the \emph{omniformity} property. 
Consider the following statements:
%% 
\begin{verbatim}
    mother1(?X,?Y) :- female(?X), parent(?X,?Y).       // (*)
    mother2(?X,?Y) <== female(?X), parent(?X,?Y).      // (**)
    female(Mary), parent(Mary,Bob).
    \neg mother1(Bob,Bill), \neg mother2(Bob,Bill), parent(Bob,Bill).
    \neg mother1(Sally,Peter), \neg mother2(Sally,Peter), female(Sally).
\end{verbatim}
%% 
Statement (*) lets us derive \texttt{mother1(Mary,Bob)}, while
statement (**) allows us to conclude \texttt{mother2(Mary,Bob)}.
So far the conclusions are similar. However, the facts given in the example
should also let us conclude more, if we use the contrapositive statements
corresponding to the implications
(*) and (**). For instance, since we know that Bob is not the mother of Bill,
we should be able to conclude that Bob is not a female.
The catch is that the semantics of \texttt{:-}  
does \emph{not} sanction any contrapositive inferences, while the semantics
of \texttt{<==} \emph{does}. Thus, the contrapositive forms of (**)
%% 
\begin{verbatim}
    \neg female(?X) <== parent(?X,?Y), \neg mother2(?X,?Y).
    \neg parent(?X,?Y) <== female(?X), \neg mother2(?X,?Y).
\end{verbatim}
%% 
let us conclude \texttt{\RULELOGNEG female(Bob)} and
\texttt{\RULELOGNEG parent(Sally,Peter)}. 
This sanctioning of the contrapositive implications that are not explicitly
written is called \emph{omniformity}, and it is the property of
$\lnegarr$, $\rnegarr$, and $\rlnegarr$ \emph{when they occur in the rule
  heads} (or facts). (Recall from
Section~\ref{pg-strong-vs-weak-implication} that in rule bodies these
connectives act as tests only, no omniformity.)

For a more involved example, let us come back to the statement that
every U.S. state has two senators:
%% 
\begin{verbatim}
    // version 1
    forall(?St)^exist(?Sn1,?Sn2)^(senator(?St,{?Sn1,?Sn2}) <== state(?St)).
    // version 2
    exist(?Sn1,?Sn2)^senator(?St,{?Sn1,?Sn2}) :- state(?St).
    state(NY).
    ...
\end{verbatim}
%%
Since Puerto Rico is not a state, it would not be listed among the states, but
this is still not enough to conclude \texttt{\RULELOGNEG state(PuertoRico)}.\footnote{
  But is enough to evaluate \texttt{\RULELOGNAF state(PuertoRico)} to true.
  However, remember that \texttt{\RULELOGNAF} cannot occur in rule heads, so
  it is not possible to \emph{derive}  \texttt{\RULELOGNAF
    state(PuertoRico)}.
}
%% 
However, if we also assert that Puerto Rico has no senators
%% 
\begin{verbatim}
    \neg senator(PuertoRico,?).              // this
    \neg exist(?X)^senator(PuertoRico,?X).   // or that
\end{verbatim}
%% 
then the version 1 implication above but not (version 2!) will sanction
the derivation of \texttt{\RULELOGNEG state(PuertoRico)}.

Note that one has to be careful in the way the absence of senators is
specified.  The following might seem to be a correct way to say this and,
indeed, it \emph{is} a correct way to express the absence of senators in
\emph{classical} logic:
%% 
\begin{verbatim}
    \neg exist(?X,?Y)^senator(PuertoRico,{?X,?Y})    //  (***)
\end{verbatim}
%% 
The trouble is that this is equivalent to
%% 
\begin{verbatim}
    forall(?X,?Y)^(\neg senator(PuertoRico,?X) \or \neg senator(PuertoRico,?Y))
\end{verbatim}
%% 
and, since \ERGO cannot reason by cases, it cannot
conclude
%% 
\begin{verbatim}
    forall(?X)^(\neg senator(PuertoRico,?X)) 
\end{verbatim}
%% 
from the above.
For that reason, we cannot use a contrapositive instance of
%% 
\begin{verbatim}
    forall(?St)^exist(?Sn1,?Sn2)^(senator(?St,{?Sn1,?Sn2}) <== state(?St))
\end{verbatim}
%% 
namely,
%% 
\begin{verbatim}
   \neg state(PuertoRico):- forall(?Sn1,?Sn2)^(\neg senator(PuertoRico,{?Sn1,?Sn2}).
\end{verbatim}
%% 
since we lack the necessary premise. Therefore, we would not be able to
derive that Puerto Rico is not a state, if we the absence of senators were
given using (***).



\subsection{Omniformity and Builtins}
As follows from the above discussion, the head-formulas that 
use \texttt{<==} and related connectives behave very much like rules (since
they are translated into sets of rules) and can
be used in lieu of them in many contexts. One thing to remember, however,
is that these constructs \emph{cannot} contain builtins. 
Suppose we want to represent the statement ``Customers over 65 years old
are eligible to receive discounts.'' We could try
%% 
\begin{verbatim}
    eligible(?X) <== customer(?X), birthYear(?X,?YoB),
                     \date[now->?[year->?Yr]]@\basetype,
                     ?Yr-?YoB > 64.
\end{verbatim}
%% 
but this implication contains a builtin, which
cannot occur in this kind of statements. The solution is to make a rule out
of the above and put the forbidden predicates in the body:
%% 
\begin{verbatim}
    eligible(?X) <== customer(?X) :-                           //  (****)
                     birthYear(?X,?YoB),
                     \date[now->?[year->?Yr]]@\basetype,
                     ?Yr-?YoB>64.
\end{verbatim}
%% 
Note that the above still enjoys the omniformity feature with respect to
the head-implication. That is, if we know
%% 
\begin{verbatim}
    customer(Bob), birthYear(Bob,1940).
    \neg eligible(Bill), birthYear(Bill,1945).
\end{verbatim}
%% 
then we can use (****) to
conclude \texttt{eligible(Bob)} and use the
contrapositive form of (****) to conclude
\texttt{\RULELOGNEG customer(Bill)}.



%%% Local Variables: 
%%% mode: latex
%%% TeX-master: "../flora2-manual"
%%% End: 
