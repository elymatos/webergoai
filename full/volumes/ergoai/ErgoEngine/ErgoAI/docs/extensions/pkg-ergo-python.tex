\chapter[Python-to-\ERGO Interface]
{pyergo: A Python-to-\ERGO Interface \\ 
  {\Large by Michael Kifer}}

This interface allows Python programs to start \ERGO, load knowledge bases
into it, and then query and modify them.  One can also talk directly to the
underlying Prolog engine, XSB.
This API works both with Python 2.7 and 3.X, but 3.7+ is recommended.

\section{Introduction}

The \emph{pyergo} interface consists of four types of APIs: one for
starting and closing \ERGO sessions, one for querying \ERGO,
one for talking directly to XSB, and one for parsing the results.
A fairly extensive example of a program that does both is found in
%% 
\begin{verbatim}
    .../ErgoAI/python/pyergo_example.py
\end{verbatim}
%% 
in the \ERGO distribution. This program provides several examples of using
the \texttt{pyergo} interface, including various edge cases and exception
handling. 
The easiest way to try these examples is via the provided shell scripts,
%% 
\begin{verbatim}
    .../ErgoAI/python/runpyergo.sh    -- Linux/Mac
    .../ErgoAI/python/runpyergo.bat   -- Windows
\end{verbatim}
%% 
One only has to change the two variables in those scripts \texttt{ERGOROOT}
and \texttt{XSBARCHDIR}.

To make it possible for your Python program
find \ERGO, two parameters are to be provided: the architecture directory,
called \texttt{XSBARCHDIR} in those scripts and in
\texttt{pyergo\_example.py}, and the root directory for the \ERGO reasoner,
which we called \texttt{ERGOROOT}. The names of these variables are, of
course immaterial, but we will use these names here for easy reference.
Pay attention to how this information is passed from the scripts to the
program via the \texttt{sys.argv}  array, as this is one of the most
convenient methods.

How will you know what to substitute for the aforesaid \texttt{ERGOROOT}
and \texttt{XSBARCHDIR}? Easy! Just start \ERGO and type these two
queries:
%% 
\begin{verbatim}
    ?- system{installdir=?Ins}.  // gives ERGOROOT
    ?- system{archdir=?Arch}.    // yields XSBARCHDIR
\end{verbatim}
%% 
The first query will give you \texttt{ERGOROOT} and the second
\texttt{XSBARCHDIR}.

Next your program needs to be told where the interface can be found. In our
example, this is accomplished via
%% 
\begin{verbatim}
    import sys
    sys.path.append(ERGOROOT.replace('\\','/') + '/python')
\end{verbatim}
%% 
Of course, you will have to replace \texttt{ERGOROOT} here appropriately,
as explained.
Note that forward slashes are preferred, although backward slashes are also
recognized in Windows (sometimes they must be escaped with another
backslash to satisfy Python syntax).
Finally, import the API as follows:
%% 
\begin{alltt}
from pyergo import \bs
    pyergo_start_session, pyergo_end_session, \bs  \emph{\color{blue}to start/end Ergo session}
    pyergo_command, pyergo_query,            \bs   \emph{\color{blue}to talk to Ergo}
    HILOGFunctor, PROLOGFunctor,             \bs   \emph{\color{blue}to parse results from Ergo}
    ERGOVariable, ERGOString, ERGOIRI, ERGOSymbol, \bs
    ERGOIRI, ERGOCharlist, ERGODatetime,     \bs
    ERGODuration, ERGOUserDatatype,          \bs
    pyxsb_query, pyxsb_command,              \bs   \emph{\color{blue}to talk to XSB directly}
    XSBFunctor, XSBVariable, XSBAtom,        \bs   \emph{\color{blue}to parse results from XSB}
    XSBString,                               \bs
    PYERGOException, PYXSBException              \emph{\color{blue}to process exceptions}
\end{alltt}
%% 
In most cases you will need only a small subset of these functions, but
importing them all is easy, does not hurt,
and is useful in case you later extend your program.
(Of course, delete the blue comments.)

\section{Connecting to \ERGO}

This part of the
API consists of two commands: \texttt{pyergo\_start\_session()} --- to
start \ERGO and
connect to it, and \texttt{pyergo\_end\_session()} --- to unload \ERGO.  
%% 
\begin{itemize}
\item  \texttt{pyergo\_start\_session()}: This takes two arguments, 
  the aforementioned directories:
  %% 
\begin{verbatim}
     pyergo_start_session(XSBARCHDIR,ERGOROOT)  
\end{verbatim}
  %% 
  Do not forget to replace \texttt{XSBARCHDIR} and \texttt{ERGOROOT}, as
  discussed.  Errors will be thrown if one of these directories is missing,
  unreadable, or does not look like belonging to a valid \ERGO
  installation.
\item \texttt{pyergo\_end\_session()}: Used to unload \ERGO. For example,
  %% 
\begin{verbatim}
    pyergo_end_session()  
\end{verbatim}
  %% 
  This command is not needed if your program exits soon after unloading,
  but it can save resources, if your Python program uses \ERGO in the
  beginning only and then continues to work for a significant period of
  time till the end, without accessing the knowledge base.
\end{itemize}
%% 


\section{Talking to \ERGO}\label{sec-pyergo-query}

This part of the 
API consists of two commands also: \texttt{pyergo\_command()} and
\texttt{pyergo\_query()}. The difference is that the first is not expected
to return any results and exception is thrown if the command returns
\emph{False}. It also throws exceptions if something went wrong during the
compilation or execution of the command. In contrast,
\texttt{pyergo\_query()} throws far fewer exceptions.
%% 
\begin{itemize}
\item  \texttt{pyergo\_command()}: Execute an \ERGO command passed as a
  parameter. Takes a
  query and treats it as a command, i.e., ignores the results and expects
  it to succeed. This is
  usually used to load a file, do something that does not return results
  (e.g., insert facts or rules),
  and when the command returns \emph{False} then it is treated as an error
  so an exception is raised. For example,
  %% 
\begin{verbatim}
    pyergo_command("writeln(Ergo = 'aaaa bbb')@\\io.")
    pyergo_command('insert{qq({11,fff(22)},33)}.')  
    pyergo_command("\\false.")  # will throw an error
\end{verbatim}
  %% 
  Note that whenever \ERGO requires backslashes, they must be doubled and
  the commands must end with a period, as usual.
\item \texttt{pyergo\_query()}: Execute a query passed as a parameter
  and get results.
  Like \texttt{pyergo\_command()}, it takes an \ERGO query, but that
  query may or may not have results and when it does the results are returned.
  The results are returned as an array of 4-tuples, which can be iterated
  over, as explained in the example below.
  Note: \texttt{pyergo\_command()} never throws an exception that has to
  do with the compilation or execution of a query. Instead, it
  returns that exception information as part of the result.
  The only exception it is supposed to throw is when a query returns
  an unsupported variable binding, i.e., something that is not a Prolog or
  HiLog term (like, for example, a reified formula).
\end{itemize}
%% 
Here is how one gets results from a query. Suppose \texttt{qq/1}  was previously
assigned the tuples $\tt(11,33)$ and $\tt(fff(22),33)$. Then:
%% 
\begin{verbatim}
    for row in pyergo_query('qq(?X,?Y).  '):
        print("result: ",row[0],row[1],row[2],row[3].value)
\end{verbatim}
%% 
will print (except the top line, which is added for readability):
%% 
\begin{verbatim}
   # Result                   Compile Status      Truth Value    Exception
   [('?X',11),('?Y',33)]   ('not_eof','success')    True           normal
   [('?X',HILOGFunctor(name=fff,args=[22])),('?Y',33)]
                           ('not_eof','success')    True           normal
\end{verbatim}
%% 
Here the compile status says whether the query was compiled without errors
and whether an end of file (or string)
was reached. In our case, it was not (\texttt{not\_eof}) because the
query string has some white space after the period, but usually it says
\texttt{eof}. The truth value can be \emph{True}, \emph{False}, or
\emph{None}---the latter standing for ``undefined'' in \ERGO terms.    
Finally, exception is \texttt{normal}, if no runtime exception 
happened in the query execution, and the actual exception is shown
otherwise.

The most important component is the first, \texttt{row[0]}.
It is a list of variable name/binding pairs, where the variable names are
taken from the query (\texttt{'?X'} and \texttt{'?Y'} in our case).
These are the actual query answers. If a query has no output variables, but
the query is true, an empty list is returned. Silent variables (the ones
that start with an underscore) are not returned.
Note that the results returned can be complex terms (they are always returned as
Prolog, not HiLog terms), like the Python object
\texttt{HILOGFunctor(name=fff,args=[22])}  in our case.
We will provide the details of that part of the API in 
Section~\ref{sec-py-unpack},
but here is an example of how to unpack such objects:
%% 
\begin{verbatim}
  for row in pyergo_query('qq(?X,?Y).  '):
      [(XVarname,XVarVal),(YVarname,YVarVal)] = row[0]
      if isinstance(XVarVal,HILOGFunctor):
          #Xresult=XVarname+'='+str(XVarVal.name)+' '+str(XVarVal.args)
          Xresult=XVarname+'='+str(XVarVal)
      elif isinstance(XVarVal,PROLOGFunctor):
          #Xresult=XVarname+'='+str(XVarVal.name)+' '+str(XVarVal.args)+'@\\plg'
          Xresult=XVarname+'='+str(XVarVal)
      else:
          Xresult=XVarname+'='+str(XVarVal)
      if isinstance(YVarVal,HILOGFunctor):
          #Yresult=YVarname+'='+str(YVarVal.name)+' '+str(YVarVal.args)
          Yresult=YVarname+'='+str(YVarVal)
      elif isinstance(YVarVal,PROLOGFunctor):
          #Yresult=YVarname+'='+str(YVarVal.name)+' '+str(YVarVal.args)+'@\\plg'
          Yresult=YVarname+'='+str(YVarVal)
      else:
          Yresult=YVarname+'='+str(YVarVal)
      print("result: ",Xresult+" and "+Yresult,row[1],row[2],row[3].value)
\end{verbatim}
%% 
The commented out parts of this example show how to access various parts of
the complex answers returned as HiLog or Prolog terms.

Note: only elementary data types, Prolog, and HiLog terms can be returned
from Ergo to Python. More complex things, like reified predicates and
frames, cannot be returned.

The commands \texttt{pyergo\_query()} and
\texttt{pyergo\_command()} raise Python exceptions, if bad things happen
during the execution. These exceptions are Python objects of the form
%% 
\begin{verbatim}
    PYERGOException(query=..., command=..., message=...)
\end{verbatim}
%% 
Some of the components may be missing in specific cases (e.g.,
\texttt{query} in case of a command, and vice versa).


\section{Talking to XSB}\label{sec-pyxsb-query}

This part of the 
API is \textbf{for expert users only} whose applications require talking to
the underlying XSB engine directly. It
consists of the commands  \texttt{pyxsb\_command()} and
\texttt{pyxsb\_query()}.
The first is like
\texttt{pyergo\_command()} except that Prolog syntax is used for the
query.  The second, \texttt{pyxsb\_query()}, differs more.  Like
\texttt{pyergo\_query()}, it returns an iterable array of tuples, but the
number of components in those tuples is arbitrary and each element
corresponds to a variable binding in the query. The bindings are listed in
the lexical order of appearance of the variables in the query.  Silent variables
are treated as any other variable, duplicate occurrences of the same
variable are omitted, and the names of the variables are not made
available.  Likewise, compilation information is not returned and neither
is the truth value, so it is not easily possible to tell whether a query is
true or undefined.
For example,
%% 
\begin{verbatim}
    for row in pyxsb_query("p(X,Y,Y,_W).") :
            print(row[0],row[1],row[2])
\end{verbatim}
%% 
Here \texttt{row[0]} corresponds to \texttt{X}, \texttt{row[1]}  
to \texttt{Y}, and \texttt{row[2]} to \texttt{\_W} (recall that silent
variables are \emph{not} ignored in the XSB interface).
True and undefined answers to the query will be printed without
distinction. To separate these two types of answers, the XSB predicate
\texttt{call\_tv/2} can be used (read about it in the XSB manual). 

Like with \ERGO queries, the bindings (\texttt{row[0]}, \texttt{row[1]},
etc.)
can be complex terms and variables. Unpacking that information is the
subject of the next section.

One last difference is that \texttt{pyxsb\_query()} and
\texttt{pyxsb\_command()} raise Python exceptions, if bad things happen
during the execution. The exceptions are Python objects of the form
%% 
\begin{verbatim}
   PYXSBException(code=..., query=..., command=..., type=..., message=...)
\end{verbatim}
%% 
Some of the components may be missing in specific cases (e.g.,
\texttt{query} in case of a command, and vice versa).


\section{Unpacking the Results}\label{sec-py-unpack}

Unpacking the results returned by \ERGO and XSB is conceptually similar.
For integers, floats, and lists, both \ERGO and XSB use the same native
Python classes.
However, for more complex data structures, \texttt{pyergo\_query()} and
\texttt{pyxsb\_query()} use different
Python classes. \ERGO has more data types than XSB and thus needs more
classes to represent them, so we discuss these issues separately.


\subsection{Unpacking Results from pyergo\_query()}

%% 
\begin{itemize}
\item  \texttt{ERGOSymbol(value=}\emph{string}\texttt{)}: this represents
  \ERGO abstract symbols (\texttt{"..."$\hat{~}\hat{~}$\bs{}symbol}). To get
  the actual string out of an
  \texttt{XSBAtom}-object, just use \emph{obj}.\texttt{value}.      
\item  \texttt{ERGOString(value=}\emph{string}\texttt{)}: this represents
  the \ERGO datatype \texttt{"..."$\hat{~}\hat{~}$\bs{}string}.
\item  \texttt{ERGOCharlist(value=}\emph{string}\texttt{)}: this represents
  lists or Unicode characters and corresponds to the
  \ERGO datatype \texttt{\bs{}charlist}.
\item \texttt{ERGOIRI(value=}\emph{string}\texttt{)}: this type of an
  object comes from an \ERGO \texttt{\bs{}iri} literal.
\item
  \texttt{ERGODatetime(date=}\emph{date-list},\texttt{time=}\emph{time-list}\texttt{)}. A
  Python object of this form would come from an \ERGO
  \texttt{\bs{}datetime} literal.  For instance,
  the \ERGO \texttt{\bs{}datetime}-literal
  \verb|"2008-6-27T10:30:55.23456-0:20"^^\datetime| will give rise to
  \texttt{ERGODatetime(date=[1,2008,6,27],time=[10,30,55.23456,-1,0,20])}.
  In a date-list like \texttt{[1,2008,6,27]}, 1 means the year is CE and -1
  means BCE. The rest stands for the year, month, and day. In a time-list,
  the elements are hours, minutes, seconds, UTC offset sign (1 or -1), UTC
  offset hour, and UTC offset minutes. Seconds and milliseconds are
  represented together using one positive decimal number.

  A \texttt{\bs{}time} object from \ERGO would give rise to an
  \texttt{ERGODatetime} object in Python in which the \texttt{date}
  component is absent. A \texttt{\bs{}date} object in \ERGO would give rise
  to a \texttt{ERGODatetime} object in which the \texttt{time}-component is
  absent.
\item \texttt{ERGODuration(value=}\emph{duration-list}\texttt{)}.
  This type of a Python object corresponds to an \ERGO
  \texttt{\bs{}duration}-literal. For instance,
  \verb|"-P22Y2M10DT1H2M3.0S"^^\duration| would give rise to a Python
  object of the form
  \texttt{ERGODuration(value=[-1,22,2,10,1,2,3.0])}. The components of the
  \emph{duration-list} are sign (of the duration, 1 or -1), years, months,
  days, hours, minutes, and seconds (which is a decimal number that
  represents both seconds and milliseconds).
\item  \texttt{ERGOVariable(name=}\emph{string}\texttt{)}: this type of
  objects may be returned if query results contain unbound variables.
  Note that the actual names of these variables are immaterial and they are
  almost always different from what was in the query. The only thing that
  matters is the equality among these names. For instance, if a tuple of
  bindings like \texttt{(ERGOVariable(name='\_Var123'), ERGOSymbol(value='abc'),
    ERGOVariable(name='\_Var123'))} is returned, it means that the first and the
  last components in that answer are the same variable.
\item
  \texttt{PROLOGFunctor(name=}\emph{string}, \texttt{args}=\emph{list}, \texttt{module=}\emph{xsb-module}\texttt{)}
  and
  \\
  \texttt{HILOGFunctor(name=}\emph{string}, \texttt{args}=\emph{list}\texttt{)}:
  these classes are used to represent
  Prolog and HiLog terms, respectively.
\end{itemize}
%% 
Typically, unpacking of an answer takes the form or testing what kind of
object it is (e.g., \texttt{isinstance(obj,ERGOVariable)},
\texttt{isinstance(obj,ERGOString)},
\texttt{isinstance(obj,ERGOIRI)},
%%\texttt{isinstance(obj,ERGODatetime)},
\texttt{isinstance(obj,HILOGFunctor)},
\texttt{isinstance(obj,int)})
and then proceeding to extract the relevant attributes of the object. An
example of this was shown in Section~\ref{sec-pyergo-query}.


\subsection{Unpacking Results from pyxsb\_query()}

The function \texttt{pyxsb\_query()} uses these classes for the data types
it returns: 
%% 
\begin{itemize}
\item  \texttt{XSBAtom(name=}\emph{string}\texttt{)}: this represents XSB
  atoms (i.e., \ERGO symbols). To get the actual atom out of an
  \texttt{XSBAtom}-object, just use \emph{obj}.\texttt{name}.      
\item  \texttt{XSBString(value=}\emph{string}\texttt{)}: this represents
  lists or Unicode characters. This class is provided for easier
  readability: this data type does not really exists in XSB in its own
  right.
\item  \texttt{XSBVariable(name=}\emph{string}\texttt{)}: this type of
  objects may be returned if query results contain unbound variables.
  Note that the actual names of these variables are immaterial and they are
  almost always different from what was in the query. The only thing that
  matters is the equality among these names. For instance, if a tuple of
  bindings like \texttt{(XSBVariable(name='\_Var123'), XSBAtom(name='abc'),
  XSBVariable(name='\_Var123'))} is returned, it means that the first and the
last components in that answer are the same variable.
\item
  \texttt{XSBFunctor(name=}\emph{string},\texttt{args}=\emph{list},\texttt{module=}\emph{xsb-module}\texttt{)}:
  this represents a complex term with the functor name \emph{string} and
  arguments \emph{lists}. The elements of the list can be terms, atoms,
  variables, etc.  
  \texttt{XSBFunctor}  objects is used only by \texttt{pyxsb\_query()}.
\end{itemize}
%% 

Typically, unpacking of an answer takes the form or testing what kind of
object it is (e.g., \texttt{isinstance(obj,XSBVariable)},
\texttt{isinstance(obj,XSBString)}, \texttt{isinstance(obj,XSBFunctor)},
\texttt{isinstance(obj,HILOGFunctor)},
\texttt{isinstance(obj,int)})
and then proceeding to extract the relevant attributes of the object. An
example of this was shown in Section~\ref{sec-pyergo-query}.




%%% Local Variables: 
%%% mode: latex
%%% TeX-master: "../ergo-packages"
%%% End: 

