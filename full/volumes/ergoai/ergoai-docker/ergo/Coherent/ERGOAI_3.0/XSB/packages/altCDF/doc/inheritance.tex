\section{Inheritance} \label{sec:inheritance}

Example~\ref{ex:hasAttr} indicates that each object in the class of
{\tt cid(absorbableSutures)} has a {\tt rid(hasPointStyle)} relation
to an object in the domain {\tt cid(pointStyle)}.  Thus, if no
stronger information were provided for, say, part {\tt
oid(sutureU245H)}, then
%--------------------------
{\small 
\begin{tabbing}
foo\=foo\=foo\=foo\=foo\=foo\=foooo\=foooooooooooooooo\=\kill
\> {\tt hasAttr(oid(sutureU245H),rid(hasPointStyle),cid(pointStyle))}$^{\cI}$
\end{tabbing} } 
%--------------------------
\noindent would be true in any model of the CDF instance.  This
consequence can be seen as a primitive form of ``default'', or more
precisely indefinite, reasoning that is provided by CDF.  Indeed,
%--------------------------
{\small 
\begin{tabbing}
foo\=foo\=foo\=foo\=foo\=foo\=foooo\=foooooooooooooooo\=\kill
\> {\tt
hasAttr(cid(suture547466),rid(hasPointStyle),cid(pointStyle))$^{\cI}$} 
\end{tabbing} } 
%--------------------------
\noindent would also be true, but would be of less interest, since
Example~\ref{ex:hasAttr} specifically indicates that {\tt
oid(suture547466)} is related to a subclass of {\tt cid(pointStyle)},
namely {\tt cid(regularCuttingEdge)}.  In contrast to {\tt
rid(endType)} the relation {\tt rid(suturesRusMatch)} was defined via
the {\tt allAttr/3} predicate.  
However the constraint that any {\tt rid(suturesRusMatch)} must be to
a {\tt cid(suturesRusPart)} holds for members of the class {\tt
cid(sutures)}, just as it holds for subclasses of {\tt
cid(sutures)}. 
The following formulas summarize inheritance in the
first argument of Type-0 relations.

%{\sc TLS: coversAttr}

\begin{proposition}[First Argument Inheritance Propagation] \label{prop:inh1}\rm
Let $\cM$ be an ontology model.
\begin{enumerate}
\item If $\cM \models {\tt hasAttr(Id_1,Id_2,Id_3)}^{\cI} \wedge
						{\tt isa(Id_0,Id_1)}^{\cI}$ 
	then $\cM \models  {\tt hasAttr(Id_0,Id_2,Id_3)}^{\cI}$
%
\item If $\cM \models {\tt allAttr(Id_1,Id_2,Id_3)}^{\cI} \wedge
					    {\tt isa(Id_0,Id_1)}^{\cI}$ 
		then $\cM \models {\tt allAttr(Id_0,Id_2,Id_3)}^{\cI}$ 
\item If $\cM \models {\tt minAttr(Id_1,Id_2,Id_3,N)}^{\cI} \wedge
						{\tt isa(Id_0,Id_1)}^{\cI}$ 
	then $\cM \models  {\tt minAttr(Id_0,Id_2,Id_3,N)}^{\cI}$
%
\item If $\cM \models {\tt maxAttr(Id_1,Id_2,Id_3,N)}^{\cI} \wedge
					    {\tt isa(Id_0,Id_1)}^{\cI}$ 
		then $\cM \models {\tt maxAttr(Id_0,Id_2,Id_3,N)}^{\cI}$ 
%
%\item If $\cM \models {\tt coversAttr(Id_1,Id_2,Id_3)}^{\cI} \wedge
%					    {\tt isa(Id_0,Id_1)}^{\cI}$ 
%		then $\cM \models {\tt coversAttr(Id_0,Id_2,Id_3)}^{\cI}$ 
%
\end{enumerate}
\end{proposition}



There is inheritance also in the third argument of relations.
Consider again the {\tt hasAttr/3} facts of Example~\ref{ex:hasAttr}
that states that any element of {\tt cid(absorbableSutures)} is
related via {\tt rid(hasPointStyle)} to an element or subclass of {\tt
cid(pointStyle)}.  By this definition, it also holds that {\tt
rid(hasPointStyle)} constrains any element of {\tt rid(absorbaleSutures)}
to an element or subclass of any {\em superclass} of {\tt
cid(pointStyle)} so that
%--------------------------
{\small 
\begin{tabbing}
foo\=foo\=foo\=foo\=foo\=foo\=foooo\=foooooooooooooooo\=\kill
\> 
{\tt hasAttr(cid(absorbableSutures),rid(hasPointStyle),cid('CDF Classes'))}$^{\cI}$
\end{tabbing} }
%--------------------------
\noindent
should also hold in any model of the CDF instance.  
%-----------------------
Similiarly,
{\small 
\begin{tabbing}
foo\=foo\=foo\=foo\=foo\=foo\=foooo\=foooooooooooooooo\=\kill
\> 
{\tt allAttr(cid(dlaPart),rid(suturesRusMatch),id('CDF Classes'))$^{\cI}$}
\end{tabbing} }
%-----------------------
\noindent
should also hold.  In English, every member or subclass of {\tt
cid(sutures)} that has a relation {\tt rid(suturesRusMatch)} has the
same relation to a member or subclass of {\tt \cid{'CDF Classes'}}.
{\tt minAttr/4} and {\tt classHasAttr/4} behave similarly with regards
to third argument inheritance.

Third argument inheritance is different for {\tt maxAttr/4}, however.
The fact
%-----------------------
{\small 
\begin{tabbing}
foo\=foo\=foo\=foo\=foo\=foo\=foooo\=foooooooooooooooo\=\kill
\> 
{\tt maxAttr(cid(person),rid(hasGeneticRelation),cid(mother))}
\end{tabbing} }
%-----------------------
\noindent
states that each person has at most one genetic mother.  If {\tt
cid(mother)} is a subclass of {\tt cid(parent)}, then it is not true
that each person has at most one genetic parent.  On the other hand,
if {\tt cid(elderlyMother)} is a subclass of {\tt cid(mother)} then it
is true that each person has at most one genetic elderly mother.  The
behavior of {\tt maxAttr/4} in models of CDF instances accords with
this intuition.
%-----------------------

\begin{proposition}[Third-Argument Inheritance Propagation] \rm
\label{prop:inh3} 
\end{proposition} 
\begin{enumerate}
\item If $\cM \models {\tt hasAttr(Id_1,Id_2,Id_3)}^{\cI} \wedge
					{\tt isa(Id_3,Id_4)}^{\cI}$
					then $\cM \models {\tt
					hasAttr(Id_0,Id_2,Id_4)}^{\cI}
					$
%
\item If $\cM \models {\tt allAttr(Id_1,Id_2,Id_3)}^{\cI} \wedge
					{\tt isa(Id_3,Id_4)}^{\cI}$ 
	then $\cM \models {\tt allAttr(Id_0,Id_2,Id_4)}^{\cI} $ 
%
\item If $\cM \models {\tt minAttr(Id_1,Id_2,Id_3,N)}^{\cI} \wedge
					{\tt isa(Id_3,Id_4)}^{\cI}$ 
	then $\cM \models {\tt minAttr(Id_0,Id_2,Id_4,N)}^{\cI} $ 
%
\item If $\cM \models {\tt classHasAttr(Id_1,Id_2,Id_3)}^{\cI} \wedge
			  {\tt isa(Id_3,Id_4)}^{\cI}$ 
	then $\cM \models {\tt classHasAttr(Id_0,Id_2,Id_4)}^{\cI} $ 
%
\item If $\cM \models {\tt maxAttr(Id_1,Id_2,Id_3,N)}^{\cI} \wedge
			  {\tt isa(Id_4,Id_3)}^{\cI}$ 
	then $\cM \models {\tt maxAttr(Id_0,Id_2,Id_4,N)}^{\cI} $ 

\end{enumerate}

%-------------------------------
\mycomment{
\footnote{One can define a relational inverse $Rid^-$ for any relation
identifier $Rid$ as
%
\[ (forall X,Y)[ rel(X,Rid^-,Y) \leftrightarrow rel(Y,Rid,X) ] \]
%
The binary relations $Rid$ and $Rid^-$ can be extended into a Galois
connection between sets of object identifiers using a standard
construction (e.g. \cite{Ore62} Section 11.2).  This connection is
reflected in the two foregoing inheritance propositions.}}
%-------------------------------

In CDF, the inheritance in the second argument of relations
generalizes or specializes relations.  For instance, the relation {\it
parent} can be generalized to {\it ancestor} or specialized to {\it
mother}.  Thus, if {\it Abraham} is the {\it parent} of {\it Isaac},
it is true that he is the {\it ancestor\ } of {\it Isaac\ } but not
necessarily the {\it mother} of {\it Isaac}.  It follows from the
semantics of CDF that {\tt hasAttr/3}, {\tt classHasAttr/3}, and {\tt
minAttr/4} all propigate inheritance in their second argument from
relations to the super-relations that contain them.  {\tt allAttr/3}
works differently, however.  Any {\it mother} of {\it Isaac} is
female, but any {\it parent} is not.  Second argument inhertiance
propagates from relations to subrelations both in {\tt allAttr/3} and
{\tt maxAttr/3}.

\begin{proposition}[Second-Argument Inheritance Propagation]
\label{prop:inh2} \rm 
\end{proposition} 
\begin{enumerate}
\item If $\cM \models {\tt hasAttr(Id_1,Id_2,Id_3)}^{\cI} \wedge
				{\tt isa(Id_2,Id_4)}^{\cI}$ 
	then $\cM \models {\tt hasAttr(Id_0,Id_4,Id_3)}^{\cI} $ 
%
\item If $\cM \models {\tt classHasAttr(Id_1,Id_2,Id_3)}^{\cI} \wedge
				{\tt isa(Id_2,Id_4)}^{\cI}$ 
	then $\cM \models {\tt classHasAttr(Id_0,Id_4,Id_3)}^{\cI} $ 
%
\item If $\cM \models {\tt minAttr(Id_1,Id_2,Id_3,N)}^{\cI} \wedge
				{\tt isa(Id_2,Id_4)}^{\cI}$ 
	then $\cM \models {\tt minAttr(Id_0,Id_4,Id_3,N)}^{\cI} $ 
%
\item If $\cM \models {\tt allAttr(Id_1,Id_2,Id_3)}^{\cI} \wedge
				{\tt isa(Id_4,Id_2)}^{\cI}$ 
	then $\cM \models {\tt allAttr(Id_0,Id_4,Id_3)}^{\cI} $ 
\item If $\cM \models {\tt maxAttr(Id_1,Id_2,Id_3,N)}^{\cI} \wedge
				{\tt isa(Id_4,Id_2)}^{\cI}$ 
	then $\cM \models {\tt maxAttr(Id_0,Id_4,Id_3,N)}^{\cI} $ 
\end{enumerate}

We conclude this chapter with a discussion of irredundant sets and
principle classes, both of which are used in the operational semantics
of the Type-0 interface in \refchap{sec:impl}.

\index{{\sc inh} proof}
The above inheritance propositions can be made into a simple proof
system, {\sc inh}, by considering the facts themselves rather than
their interpretations into an ontology theory.  For instance, give a
CDF instance $\cO$ containing {\tt hasAttr(Id1,Id2,Id3)} and {\tt
isa(Id3,Id4)} then {\tt hasAttr(Id1,Id2,Id4)} can be deduced.  In
other words, {\sc inh} contains an inference rule,
\[
\frac{{\tt hasAttr(Id1,Id2,Id3)},{\tt isa(Id3,Id4)}}{{\tt hasAttr(Id1,Id2,Id4)}}
\]
based on Proposition~\ref{prop:inh3}.1.  The {\sc inh} proof system
contains similar inference rules obtained from Proposition
Proposition~\ref{prop:inh1}.(1-2), Proposition~\ref{prop:inh3}.(2-3),
Proposition~\ref{prop:inh2}.(1-3), and the transitive reflexive
closure of the {\tt isa/2} facts, and forms the main reasoning
mechanism for Type-0 instances.

%-------------------------------------------------------------------------------
\mycomment{
When making use of Propositions~\ref{prop:inh1}-\ref{prop:inh1} on
facts in a CDF instance, we slightly abuse terminology by saying that
a fact $F$ is implied by a set of facts $\cF$ via
Propositions~\ref{prop:inh1}-\ref{prop:inh1}, rather than the more
cumbersome statement that the translation of $F$ is implied by the set
of translations of each fact in $\cF$.  }
%-------------------------------------------------------------------------------

%-------------------------------------------------------------------------------
\index{class!principle} \index{irredundant set} \index{irredundant basis}
\index{irredundant set!more specific} \index{$\succ_{spec}$}
\begin{definition} \label{def:redund} \index{irredundant}
Let $\cO$ be a Type-0 CDF instance, $\cS \subseteq \cO$, and
$\cO_{isa}$ be the set of {\tt isa/2} facts in $\cO$.  Then a fact $f
\in \cS$ is {\em irredundant in $\cS$} if there is no other $f' \in
\cS, f' \not= f$ such that
\[
\cO_{isa},f' \inhdash f
\]
$\cS$ is irredundant if each fact in it is irredundant.  An
irredundant basis for $\cS$ is a irredundant $\cS' \subseteq \cS$ such
that for all $f \in \cS$, 
\[
(\cS' \cup \cO_{isa}) \inhdash f
\]

Given an identifier $I$, a class $C$ is a {\em principle class for
$I$} if $\cO_{isa} \inhdash I \not= C \wedge isa(I,C)$, but $\cO_{isa}
\not \inhdash isa(I,C') \wedge (C' \not = C) \wedge isa(C',C)$.  Similarly
a relation identifier $R$ is a {\em principle relation for object
identifiers $O_1$ and $O_2$} if $\cO_{isa} \inhdash rel(O_1,R,O_1)$
but $\cO_{isa} \not \inhdash rel(O_1,R',O_2) \wedge (R \not = R')
\wedge isa(R',R)$

Finally, if $\cS, \cS' \subseteq \cO$, and $\cS$ and $\cS'$ are both
irredundant sets, then {\em $\cS$ is more specific than $\cS'$ in
$\cO$}, $\cS \succ_{spec} \cS'$ if there is a $f \in \cS$ and $f' \in
\cS'$, $f \not= f'$ such that $\cO_{isa},f \inhdash f'$, but there is
no $g' \in \cS'$ $g \in \cS$, $g \not = g$ such that $\cO_{isa},g
\inhdash g'$.
%
\end{definition}
%-----------------------------------------------------------------------------
\noindent
It is straightforward that any $\cS \subseteq \cO$ contains an
irredundant basis (recall that $\cO$ is finite).  Furthermore, $\cS$
contains a unique irredundant basis if the {\tt isa/2} facts in $\cO$
are ``acyclic'', i.e. if there are no two non-identical identifiers
$I,I' \in \cO$ such that $\inhdash isa(I,I')$ and $\inhdash isa(I',I)$

