\subsection{Models for CDF Instances}~\label{sec:model}
\begin{definition} \label{def:ont-th}
Let $\cO$ be a CDF instance.  We define as $\cL_{\cO}$, the ontology
language whose functions and constant symbols are restricted to the
identifiers in $\cO$.  $TH(\cO)$ is the ontology theory over
$\cL_{\cO}$ whose non-logical axioms consist of the core axioms and
the instance axiom for each fact in $\cO$.
% while $\cO^{\cI}$ is the theory over $\cL_{\cO}$ containing only the
% instance axioms.  
We say that an ontology structure $\cM$ is a model of $\cO$ if it is a
model of $TH(\cO)$.

If $\cO$ is Type-0 it is {\em well-typed} if each instance axiom is
consistent with Core Axioms~\ref{ax:distinct} (Distinct Identifiers)
and Axiom Schema~\ref{ax:downcl} (Downward Closure)
\end{definition}

\begin{theorem} \label{thm:equality}
Let $\cO$ be a well-typed CDF instance containing no {\tt minAttr/4}
or {\tt maxAttr/4} facts.  Then $\cO$ has a model.
\end{theorem}
\input{eqconstproof}
%---------------------------------------------------------
\mycomment{
From a practical point of view, the Core
Axioms~\ref{ax:distinct}-\ref{ax:nonnull}, and \ref{ax:downcl} can be
checked in a simple manner directly from a CDF instance, while Core
Axiom~\ref{ax:contained} is consistent with any CDF instance.
Determining that Core Axiom~\ref{ax:implsc} holds in a CDF instance is
less straightforward.  We now examine when a finite CDF instance
translates into a consistent theory that has a finite model.  The
notion of {\em conformability} is a syntactic property of a CDF
instance $\cO$ that is used to ensure that the translation of $\cO$
will satisfy Axiom~\ref{ax:implsc}.

Let $\cO$ be a CDF instance.  Then an identifier $C$ is a {\em class
identifier} if ${\tt class(Atom,C)} \in \cO$ for some $Atom$; an {\em
object identifier} if ${\tt object(Atom,C)} \in \cO$ for some $Atom$;
and a {\em relation identifier} if ${\tt relation(Atom,C)} \in \cO$
for some $Atom$.
}
%-------------------------------------------------------------

Theorem~\ref{thm:equality} indicates that any well-typed CDF instance
has a model, but in order to satisfy the various axioms, the model may
need to map a number of distinct identifiers into the same individual
in the universe of the model.  Such a model may be an undesirable
semantics for a CDF instance since using it for computation would
require unification over various equality theories.  We therefore
consider {\em freeness axioms}, which contains an axiom schema that
specifies that equality does not hold for any pair of distinct
constants or functions; and an axiom schema stating that no term can
be equal to a subterm it contains.  These axioms are commonly used in
logic programming (e.g. \cite{Lloy84}) and we do not include them
here.

\begin{definition}
Let $\cO$ be a CDF instance.  A {\em free ontology theory} for $\cO$,
$TH_F(\cO)$ is an ontology theory for $\cO$ extended with the freeness
axioms.  An ontology structure for a free ontology theory is called a
free model of $\cO$.
\end{definition}

Example~\ref{ex:equality} shows that CDF instances in which product
identifiers occur may not give rise to free models.  We derive a
condition for the existence of such models.

\begin{definition} \label{def:conformable}
Let $I$ and $I'$ be two identifiers occurring in a CDF instance $\cO$.
Then $I'$ is reachable from $I$ in $\cO$ if $I = I'$ or if $isa(I',I)$
is in $\cO$ or if $isa(I',I_n)$ is in $\cO$ and $I_n$ is reachable
from $I$.  If either $I$ is reachable from $I'$ or $I'$ is reachable
from $I$, $I$ and $I'$ are termed {\em connected}.

An identifier $I$ {\em locally conforms to} an identifier $I'$ if 
\begin{itemize}
\item $I'$ is an atomic identifier; or 
\item The argument of $I'$ is $f(I'_1,...,I'_n)$; the argument of  $I$
is $f(I_1,...,I_n)$ and $I_i$ {\em locally conforms to} $I'_i$, for $1
\leq i \leq n$.
\end{itemize}
$I$ and $I'$ are {\em conformable} if $I$ locally conforms to all
identifiers from which $I'$ is reachable in $\cO$, and $I'$ locally
conforms to identifiers from which $I$ is reachable in $\cO$.  An
identifier $I$ is self-conforming if it is conformable with itself.
\end{definition}
%---------------------------------------------------------

As will be shown, acceptability is necessary for a CDF instance to
have a finite free model.

%---------------------------------------------------------
\begin{definition} \label{def:acceptable}
A CDF instance $\cO$ is {\em acceptable} if
\begin{enumerate}
\item Each class or object identifier $I$ in $\cO$ is self-conforming; 

%\item There is no product idenifier $C_P$ and identifier $C$ such that
%there is a path from $C$ to $C_P$ in $\cO$ and $C$ occurs in $C_P$.

\item Let $allAttr(I_1,R_1,I_3)$ or $maxAttr(I_1,R_1,I_3) \in \cO$,
and $R_2$ reachable from $R_1$.  Then 
\begin{enumerate}
\item if $I_1$ and $I_2$ are connected, and $hasAttr(I_2,R_2,I_4), 
allAttr(I_2,R_2,I_4),$ $minAttr(I_2,R_2,I_4,N)$ or
$maxAttr(I_2,R_2,I_4,N) \in \cO$, then $I_3$ and $I_4$ are
conformable.
%
\item if $classHasAttr(I_1,R_2,I_4) \in \cO$, then $I_3$ and $I_4$ are
conformable. 
\end{enumerate}
\end{enumerate}
\end{definition}

%---------------------------------------------------------
\begin{theorem} \label{thm:acceptable}
Let $\cO$ be a finite well-typed CDF instance.  Then $\cO$ is
acceptable and has a model iff $\cO$ has a finite free model.
\end{theorem}
\begin{proof}
The proof is contained in the appendix.
\end{proof}

\begin{corollary} \label{thm:atomic}
Let $\cO$ be a finite CDF instance in which no product identifiers
occur.  Then $\cO$ has a finite free model iff $\cO$ has a model.
\end{corollary}


\begin{theorem} \label{thm:poly}
Let $\cO$ be a finite CDF instance.  Then it can be determined in
polynomial time whether there exists a finite free model for $\cO$.
\end{theorem}

