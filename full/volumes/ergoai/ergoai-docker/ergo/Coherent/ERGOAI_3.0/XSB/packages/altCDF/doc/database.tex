\section{Database Access for CDF} \label{sec:database} 

\subsection{Storing a CDF in an ODBC database}

The 4 CDF relations, while usually stored in prolog .P files, may also
be stored in 4 relations in a relational database.  Each relation is
stored in its external form.  Each field (except an object ID) is
stored as a Prolog Term.  Object ID's are stored as strings.

These routines allow the loading of a CDF into memory from an ODBC
database and the dumping of an CDF from memory to an ODBC database.

In the future they will allow all updates to such a CDF in memory to
be immediately reflected back out to the stored DB.

\subsection{Lazy Access to an CDF Stored in a RDB}

A facility is also provided to lazily access an CDF stored externally
as 4 relations in external form in a relational database.


\begin{description}

\ourpredmoditem{clear\_db\_cdf/1} {\tt clear\_db\_cdf(+Connection)} deletes all
tuples in the CDF tables in the database accessible through the
database connection named {\tt Connection}, which must have been opened
with odbc\_open/1.



\ourpredmoditem{clear\_db\_cdf\_component/2}
{\tt clear\_db\_cdf(+Connection,+Component)} deletes all tuples for
component {\tt Component} in the CDF tables in the database accessible
through the database connection named {\tt Connection}, which must have
been opened with odbc\_open/1.  



\ourpredmoditem{drop\_db\_cdf/1} {\tt drop\_db\_cdf(+Connection)} removes the
tables used to dump a CDF to a database. Connection must be an opened
ODBC connection (using odbc\_open/4).


\ourpredmoditem{drop\_db\_cdf\_component/2}
{\tt drop\_db\_cdf(+Connection,+Component)} removes the tables used to
dump a CDF {\tt Component} to a database. Connection must be an opened
ODBC connection (using odbc\_open/4).

\ourpredmoditem{create\_db\_cdf/1} {\tt create\_db\_cdf(+Connection)} creates
the 4 tables necessary to dump a CDF to a database.  Connection must
be an opened ODBC connection (using odbc\_open/4).


\ourpredmoditem{create\_db\_cdf\_indices/1} 
{\tt create\_db\_cdf\_indices(+Connection)} creates the appropriate
indices for the tables necessary to dump a CDF to a database. {\tt
Connection} must be an opened ODBC connection (using odbc\_open/4).

\ourpredmoditem{create\_db\_cdf\_component/2}
{\tt create\_db\_cdf\_component(+Connection,+Component)} creates the tables
necessary for dumping a given CDF {\em Component} into a
database. Connection must be an opened ODBC connection (using
odbc\_open/4).


\ourpredmoditem{create\_db\_cdf\_component\_indices/2}
{\tt create\_db\_cdf\_component\_indices(+Connection,+Component)} creates
the appropriate indices for the tables necessary to dump a CDF
{\tt Component} to a database. {\tt Connection} must be an opened ODBC
connection (using odbc\_open/4).


\ourpredmoditem{load\_db\_cdf/1} {\tt load\_db\_cdf(+Connection)} loads an CDF
from tables stored in a database in external form.  Connection must be
an opened ODBC connection created by odbc\_open/4.

\ourpredmoditem{dump\_db\_cdf/1}
{\tt dump\_db\_cdf(+Connection)} dumps a CDF into a database. It
assumes the necessary tables have been created by {\tt
create\_db\_cdf/1}. {\tt Connection} must be an opened ODBC connection
(using odbc\_open/4).

\ourpredmoditem{dump\_db\_component/3}
{\tt dump\_db\_component(+Connection,+Component,+Path)} dumps a CDF
{\tt Component} into appropriate tables in a database. It assumes the
tables have been created by a call to
{\tt create\_db\_cdf\_component/2}. {\tt Connection} must be an opened ODBC
connection (using odbc\_open/4). Component information is created in
the {\tt Path}. This includes necessary initialization files as well as
static dependencies obtained from the current dependencies for the
Component in CDF. The created files should be patched to contain
appropriate path information for dependencies.

\ourpredmoditem{cache\_abstract/3}{\tt cache\_abstract(Connection,
CallTemplate, AbstractCallTemplate)} is a dynamic predicate
which can be used to control call abstraction, thus implementing
a form of cache prefetching.

\end{description}
%-----------------------------------------------------------------------


\subsection{Updatable External Object Data Sources}

{\em Updatable External Object Data Sources} provide a way for CDF object
data (objects and their attributes) to be stored and retrieved from
external tables within a relational database that is accessible
through the ODBC interface.  Objects can be created and deleted and
their attributes can be added, deleted and updated.  The table(s) in
the (external) relational database are updated to reflect such
changes.

An Updatable External Object Data Source is stored as a ""component""
\secref{sec:components}, which is identified by a particular,
unique, {\em Source}.

The external database table(s) of an Updatable External Object Data
Source must be of specific form(s), and represent objects and their
attributes in particular ways.  However, the ways supported are
general enough to allow reasonably natural external data
representations.

The main table in an Updatable External CDF Object Data Source is
called the {\em Object Table}.  An object table is a database table
whose rows represent objects to be seen within a CDF.  Such a set of
objects will share the same ""source"" in the CDF, indicating their
""component"".

An Object Table contains a column which is the CDF object Native ID,
and that column must be a key for the object table.  The table may
have other columns that can be reflected as CDF attributes for the
objects.  Each such attribute must (clearly) be functional.  There may
also be other tables, called Attribute Tables, which have a foreign
key to the object table, and a column that can be reflected as a CDF
attribute for the object.  These attributes need not be functional.

An Object Table must be declared with the following fact.

\begin{verbatim}
ext_object_table(Source,TableName,NOidAttr,NameAttr,
		 MemberofNCid,MemberofSou).
\end{verbatim}

where:

\begin{itemize}
\item
{\tt Source} is the component identifier for this object table.
\item
{\tt TableName} is the name of the object table in the database.
\item
{\tt NOidAttr} is the column name of the key column of the object table.
\item
{\tt NameAttr} is the column name of field that contains the name field
  for the object.  (If there is no special one, it should be the same
  as {\tt NOidAttr}.)
\item
{\tt MemberofNCid} determines the Native ID of the classes of which the
objects are members.  If it is an atom, then it is the name of a
column in the table whose values contain the Native ID for the class
containing the corresponding object.  If it is of the form
{\tt con(Atom)}, then {\tt Atom}, itself, is the Native ID of the single
class that contains {\em all} objects in the table.
\item
{\tt MemberofSou} determines the Sources of the classes containing the
objects in the table.  If it is an atom, then it is the name of a
column in the object table whose values are the sources of the classes
containing the corresponding objects.  If it is of the form
{\tt con(Atom)}, then {\tt Atom}, itself, is the Source for all classes
containing objects in the table.  (Note if {\tt memberofNCid} is of the
form con(\_), then {\tt MemberofSou} should also be of that form.)
\end{itemize}

The caller must have previously opened an ODBC connection, named
Source, by using \pred{cdf\_db\_open/3}, before these routines will work.

For each functional attribute represented in an Object Table, there
must be a fact of the following form:

\begin{verbatim}
ext_functional_attribute_table(Source,RelNatCid,RelSou,
                                      TarAttr,Trans).
\end{verbatim}

where:

\begin{itemize}
\item
{\tt Source} is the component identifier for this object table.
\item
{\tt RelNatCid} is the Native ID of the CDF relationship for this
attribute.
\item
{\tt RelSou} is the Source of the CDF relationship for this attribute.
\item
{\tt TarAttr} is the name of the column(s) in the table containing the
value of this attribute.  If the internal target type is a product
type, then this is a list of the column names of the columns that
contain the product values.  The predicate \pred{coerce\_db\_type/5}
converts from internal Native Ids to (and from) (lists of) data field
values.  Rules for \pred{coerce\_db\_type/5} are provided to do standard
(Trans = std) conversion between primitive CDF and database types.
Product types are unfolded to be a list of primitive CDF and database
types and are converted as such.  If desired, {\tt coerce\_db\_type/5}
could be extended to include special-purpose conversion methods
(probably only of interest for special product types.)
\item
{\tt Trans} is an atom that indicates the type of translation from
internal to external format.  Normally it is '{\tt std}', unless
\pred{coerce\_db\_type/5} has been extended to include special
translation capabilities.  \end{itemize}

There must be a {\tt schrel} in the CDF for each of these CDF
relationships indicating the CDF type of the attribute value.

For each attribute table, there must be a fact of the following form:

\begin{verbatim}
ext_attribute_table(Source,TableName,NOidAttr,
                    RelationNatCid,RelationSou,TarAttr,Trans)
\end{verbatim}

where:
\begin{itemize}
\item
{\tt Source} is the component identifier for this object table.
\item
{\tt TableName} is the name of the attribute table in the database.
\item
{\tt NOidAttr} is the column name of the column of the attribute table which
  is a foreign key to the object table.
\item
{\tt RelationsNatCid} is the Native ID of the CDF relationship for this
  attribute.
\item
{\tt RelationSou} is the Source of the CDF relationship for this attribute.
\item
{\tt TarAttr} is the name(s) of the column(s) in the table containing the
value(s) of this attribute.  It is an atomic name if the value is of a
primitive CDF type; it is a list of names if the value of this
attribute is of a product type.
\item
{\tt Trans} is an atom that indicates the type of translation from
internal to external format.  Normally it is {\tt std}.
\end{itemize}

For each functional {\tt attribute\_object}, there must be a fact of the
following form:

\begin{verbatim}
ext_functional_attribute_object_table(Source,RelationNatCid,RelationSou,
                                            TarAttr,TarSource)
\end{verbatim}

where, as above

\begin{itemize}
\item
{\tt Source} is the component identifier for this object table.
\item
{\tt RelationsNatCid} is the Native ID of the CDF relationship for this
  attribute.
\item
{\tt RelationSou} is the Source of the CDF relationship for this attribute.
\item
{\tt TarAttr} is the name of the column in the table containing the value of
  a native object ID.
\item
{\tt TarSource} is the Source of the native Oids in the TarAttr field.
\end{itemize}

For each attribute\_object table, there must be a fact of the following form:

\begin{verbatim}
ext_attribute_object_table(Source,TableName,NOidAttr,
                    RelationNatCid,RelationSou,TarAttr,TarSource)
\end{verbatim}
where, once again:
\begin{itemize}
\item
{\tt Source} is the component identifier for this object table.
\item
{\tt TableName} is the name of the attribute table in the database.
\item
{\tt NOidAttr} is the column name of the column of the attribute table which
  is a foreign key to the object table.
\item
{\tt RelationsNatCid} is the Native ID of the CDF relationship for this
  attribute.
\item
{\tt RelationSou} is the Source of the CDF relationship for this attribute.
\item
{\tt TarAttr} is the name of the column in the table containing the value of
  a native object ID.
\item
{\tt TarSource} is the Source of the native Oids in the TarAttr field.
\end{itemize}


\begin{description}
\ourpredmoditem{cdf\_db\_open/3}
{\tt cdf\_db\_open(Component,CallToGetPar,Parameter)} opens an odbc
connection to a database for use by cdf\_db\_updatable, or
cdf\_db\_storage, predicates.  {\tt Component} is an atom representing the
component; {\tt CallToGetPar} is a callable term, which will be called
to instantiate variables in {\tt Parameters}.  If {\tt Parameters} is
given as a ground term, then {\tt CallToGetPar} should be {\tt true}.
It can be used, for example, to ask the user for a database and/a
password.  {\tt Parameter} specifies the necessary information for
odbc\_open to open a connection.  It is one of the following forms:
{\tt odbc(Server,Name,Passwd)} or {\tt odbc(ConnectionString)}.  See the
{\tt odbc\_open/1/3} documentation \footnote{See Volume 2 of the XSB
manual.}  for details on what these parameters must be.  

\ourpredmoditem{isa\_int\_udb/2} {\tt isa\_int\_udb(?Arg1,?Arg2)} is used to
provide access to database-resident {\tt isa\_ext} facts. It is
typically used in the definition of {\tt isa\_int/2} in
{\tt cdf\_intensional.P} files for db\_updatable components.

\ourpredmoditem{assert\_cdf\_int\_udb/1} 
{\tt assert\_cdf\_int\_udb(+Term)} is used to assert a {\tt Term} in a
db\_updatable component. It is typically used in the definition of
{\tt assert\_cdf\_int} for db\_updatable components.

\ourpredmoditem{retractall\_cdf\_int\_udb/1}
{\tt retractall\_cdf\_int\_udb(+Term)} is used to retract a {\tt Term}
from a db\_updatable component. It is typically used in the definition
of {\tt retractall\_cdf\_int} for db\_updatable components.

\ourpredmoditem{hasAttr\_int\_udb/3}
{\tt hasAttr\_int\_udb(?Source,?Relation,?Target)} is used to provide
access to database-resident relations in CDF components. It is
typically used in the definition of {\tt hasAttr\_int/3} for
db\_updatable components.

\end{description}