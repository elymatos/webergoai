\chapter{Libraries from Other Prologs}
%================================

XSB is distributed with some libraries that have been provided from
other Prologs.

%Not all XSB libraries are fully documented.  We provide brief
%summaries of some of these other libraries.
%
%\section{Justification}
%%
%\index{Code authors!Guo, Hai-Feng}
%By Hai-Feng Guo
%
%Most Prolog debuggers, including XSB's, are based on a mechanism that
%allows a user to trace the evaluation of a goal by interrupting the
%evaluation at call, success, retry, or failure of various subgoals.
%While this has proved an excellent mechanism for evaluating SLD(NF)
%e%xecutions, it is difficult at best to use such a mechanism during a
%%tabled evaluation.  This is because, unlike with SLD(NF), SLG requires
%answers to be returned to tabled subgoals at various times (depending
%on whether batched or local evaluation is used), negative subgoals to
%be sometimes be delayed and/or simplified, etc.
%
%One approach to understanding tabled evaluation better is to abstract
%away the procedural aspects of debugging and to use the tables
%produced by an evaluation to construct a {\em justification} after the
%%evaluation has finished.  The justification library does just this
%u%sing algorithms described in \cite{GuRR01}.
%
\section{AVL Trees}
\index{Code authors!Carlsson, Mats}

By Mats Carlsson

AVL trees (i.e., triees subject to the Adelson-Velskii-Landis balance
criterion) provide a mechanism to maintain key value pairs so that
loop up, insertion, and deletion all have complexity ${\cal O}(\log
n)$.  The library, {\tt assoc\_xsb} contains predicates to transform a
sorted list to an AVL tree and back, along with predicates to
manipulate the AVL trees~\footnote{This library contains functionality
  not documented here: see the code file for further documentation.}

\begin{description}
\indourmoditem{list\_to\_assoc(+List, ?Assoc)}{list\_to\_assoc/2}{assoc\_xsb}
%
is true when {\tt List} is a proper list of Key-Val pairs (in any
order) and {\tt Assoc} is an association tree specifying the same
finite function from Keys to Values.

\indourmoditem{assoc\_to\_list(+Assoc, ?List)}{assoc\_to\_list/2}{assoc\_xsb}
   assumes that {\tt Assoc} is a proper AVL tree, and is true when
   {\tt List} is a list of Key-Value pairs in ascending order with no
   duplicate keys specifying the same finite function as {\tt Assoc}.
   Use this to convert an {\tt Assoc} to a list.

\indourmoditem{assoc\_vals\_to\_list(+Assoc, ?List)}{assoc\_vals\_to\_list/2}{assoc\_xsb}
   assumes that {\tt Assoc} is a proper AVL tree, and is true when
   {\tt List} is a list of Values in ascending order of Key with no
   duplicate keys specifying the same finite function as {\tt Assoc}.
   Use this to extract the list of Values from {\tt Assoc}.

\indourmoditem{is\_assoc(+Assoc)}{is\_assoc/1}{assoc\_xsb}
%
is true when {\tt Assoc} is a (proper) AVL tree.  It checks both that
the keys are in ascending order and that {\tt Assoc} is properly
balanced.  

\indourmoditem{gen\_assoc(?Key, +Assoc, ?Value)}{gen\_assoc/3}{assoc\_xsb}
   assumes that {\tt Assoc} is a proper AVL tree, and is true when
   Key is associated with Value in {\tt Assoc}.  Can be used to enumerate
   all Values by ascending Keys.

\indourmoditem{get\_assoc(+Key, +OldAssoc,?OldValue,?NewAssoc,?NewValue)}{get\_assoc/5}{assoc\_xsb}
%
is true when {\tt OldAssoc} and {\tt NewAssoc} are AVL trees of the
same shape having the same elements except that the value for {\tt
  Key} in {\tt OldAssoc} is {\tt OldValue} and the value for {\tt Key}
in {\tt NewAssoc} is {\tt NewValue}.

\indourmoditem{put\_assoc(+Key,+OldAssoc,+Val,-NewAssoc)}{put\_assoc/4}{assoc\_xsb}
is true when {\tt OldAssoc} and {\tt NewAssoc} define the same finite
function except that {\tt NewAssoc} associates {\tt Val} with {\tt
  Key}.  {\tt OldAssoc} need not have associated any value at all with
{\tt Key}.

\indourmoditem{del\_assoc(+Key,+OldAssoc,?Val,-NewAssoc)}{del\_assoc/4}{assoc\_xsb}
%
 is true when {\tt OldAssoc} and {\tt NewAssoc} define the same finite
 function except that {\tt OldAssoc} associates {\tt Key} with {\tt
   Val} and {\tt NewAssoc} doesn't associate {\tt Key} with any value.

\end{description}



%\section{Ordered Sets: {\tt ordsets.P}}
%\index{Code authors!O'Keefe, Richard}
%By Richard O'Keefe
%
%%
%(Summary from code documentation) {\tt ordset.P} provides an XSB port
%  of the widely used ordset library, whose summary we paraphrase here.
%  In the ordset library, sets are represented by ordered lists with no
%  duplicates.  Thus {\em \{c,r,a,f,t\}} is represented as {\tt
%    [a,c,f,r,t]}.  The ordering is defined by the \verb|@<| family of
 % term comparison predicates, which is the ordering used by {\tt
%    sort/2} and {\tt setof/3}.  The benefit of the ordered
%  representation is that the elementary set operations can be done in
%  time proportional to the sum of the argument sizes rather than their
%  product.  Some of the unordered set routines, such as {\tt
%    member/2}, {\tt length/2}, or {\tt select/3} can be used
%  unchanged.

\section{Unweighted Graphs: {\tt ugraphs.P}}
\index{Code authors!Carlsson, Mats}

By Mats Carlsson

%
XSB also includes a library for unweighted graphs.  This library
allows for the representation and manipulation of directed and
non-directed unlabelled graphs, including predicates to find the
transitive closure of a graph, maximal paths, minimal paths, and other
features.  This library represents graphs as an ordered set of their
edges and does not use tabling.  As a result, it may be slower for
large graphs than similar predicates based on a datalog representatoin
of edges.

\section{Heaps: {\tt heaps.P}}
\index{Code authors!O'Keefe, Richard}

By Richard O'Keefe

(Summary from code documentation).  A heap is a labelled binary tree
where the key of each node is less than or equal to the keys of its
sons.  The point of a heap is that we can keep on adding new elements
to the heap and we can keep on taking out the minimum element.  If
there are $N$ elements total, the total time is $\cO (Nlg(N))$.  If
you know all the elements in advance, you are better off doing a
merge-sort, but this file is for when you want to do say a best-first
search, and have no idea when you start how many elements there will
be, let alone what they are.

A heap is represented as a triple {\tt t(N, Free, Tree)} where {\tt N}
is the number of elements in the tree, {\tt Free} is a list of
integers which specifies unused positions in the tree, and {\tt Tree}
is a tree made of t terms for empty subtrees and {\tt
  t(Key,Datum,Lson,Rson)} terms for the rest The nodes of the tree are
notionally numbered like this:
%
\begin{verbatim}
                                    1
                    2                               3
             4               6               5               7
         8      12      10     14       9       13      11     15
      ..  ..  ..  ..  ..  ..  ..  ..  ..  ..  ..  ..  ..  ..  ..  ..
\end{verbatim}

The idea is that if the maximum number of elements that have been in
the heap so far is $M$, and the tree currently has $K$ elements, the
tree is some subtreee of the tree of this form having exactly $M$
elements, and the Free list is a list of $K-M$ integers saying which
of the positions in the $M$-element tree are currently unoccupied.
This free list is needed to ensure that the cost of passing N elements
through the heap is $\cO(Nlg(M))$ instead of $\cO (NlgN)$.  For $M$ say
$100$ and $N$ say $10^4$ this means a factor of two.
%


%%% Local Variables:
%%% mode: latex
%%% TeX-master: "manual2"
%%% End:
