\subsection{Non-termination Analysis}\label{sec-terminyzer}
\index{non-termination}
\index{Terminyzer}

It is a fact of life that an \ERGO query might not terminate. In fact, due
to the expressive power of \ERGO, it is \emph{undecidable} whether a query
will terminate or not. However, \ERGO comes with a powerful analysis tool
called \emph{Terminyzer} (for non-\emph{Termin}ation Anal\emph{yzer}),
which can greatly help in determining whether a given query might not
terminate.

Terminyzer examines the trace of the query run and tries to detect symptoms
of non-termination. Since the problem is fundamentally undecidable, it may
not find any problems even if they exist and it may report a
non-terminating loop falsely.  In some cases, the analysis might turn out
to be too time consuming and thus not feasible.
Details of the approach are given in \cite{terminyzer2,liang-padl-13}.
It is guaranteed that Terminyzer will find a problematic loop \emph{given enough
time}. But the time we have is limited and it is pointless to wait beyond
10 or 15 minutes. When Terminyzer reports a loop, however, it means that
the query will indeed not terminate or it might terminate after a very long
period of time and the reported loop may be a computational
bottleneck. It is worth to examine the reported loop and see if you can
spot any inefficiency or redundancy.

Non-termination can happen due to two reasons:
%% 
\begin{itemize}
\item  the execution might be invoking some subgoals repeatedly, with
  larger and larger arguments; or
\item  some subgoal has an infinite number of answers.
\end{itemize}
%% 
Here is a typical example of a non-terminating query due to the first reason:
%% 
\begin{verbatim}
    p(a).
    q(b).
    p(?X) :- q(f1(?X)).
    q(?X) :- p(f2(?X)).
    ?- p(?X).
\end{verbatim}
%% 
Clearly, the two rules on lines 3 and 4 repeatedly call each other with
increasingly growing arguments f1(...), f1(f2(...)), f1(f2(f1(...)))
etc., and Terminyzer will dutifully report the following:
%% 
\begin{verbatim}
*** Report: subgoals that form a possibly infinite call loop ***
    p(f2(f1(f2(f1(?A)))))@main
        in rule on line 4, file test1.ergo, module main
    q(f1(f2(f1(f2(f1(?A))))))@main
        in rule on line 3, file test1.ergo, module main
\end{verbatim}
%% 
The user can then examine the reported subgoals in the rules that occur at
the specified lines and will easily see the problem. In general, several
subgoals in several rules can be involved, and the larger the cycle the
harder it is to understand what is going on. However, this is certainly
much easier than manually finding and
examining a sometimes astronomical number of possible loops in a
knowledge base that has a few dozens to thousands of of rules.

\index{call abstraction}
If the cause of non-termination is due to repeated calls to the same
subgoals with increasingly large arguments, a simple remedy may sometimes
help; it is called \emph{call abstraction}   and can be requested using the
\texttt{setruntime} directive described in Section~\ref{sec-size-control}. 
Call abstraction is a technique that the underlying XSB inference engine
can use to prevent calls with increasingly deeply nested arguments by
replacing deeply nested subterms with variables. In our example, subgoal
abstraction at term size 5 will take the call
such as
%% 
\begin{verbatim}
    p(f1(f2(f1(f2(f1(?X))))))
\end{verbatim}
%% 
and transform it into an equivalent call
%% 
\begin{verbatim}
    p(f1(f2(f1(?Y)))), ?Y = f2(f1(?X))
\end{verbatim}
%% 
The effect is that the new call has no more than 5 symbols and the first
cause of non-termination is impossible. In our case, issuing the command
%% 
\begin{verbatim}
    ?- setruntime{goalsize(5,abstract)}.
\end{verbatim}
%% 
prior to running the query \texttt{p(?X)} would terminate the query with
the answer 
%% 
\begin{verbatim}
    ?X =  a
\end{verbatim}
%% 
Call abstraction is no a panacea, however. In some cases, it may transform a
non-terminating program with an infinite loop of subgoal calls into a
non-terminating program in which some subgoal has an infinite number of
answers. That is, the first cause of non-termination will morph into the
second. To see if call abstraction might help your query, check the box
``Abstract large subgoals?'' when you start Terminyzer (which is in the
sequel).

If the cause of non-termination is an infinite number of answers
to some subgoal or a set of subgoals, Terminyzer will try to identify such
subgoals. To illustrate, consider the following knowledge base and a query:
%% 
\begin{verbatim}
    p(a).
    q(b).
    p(f1(?X)) :- q(?X).
    q(f2(?X)) :- p(?X).
    ?- p(?X).
\end{verbatim}
%% 
This is similar to the previous query, but the function symbols are in the
rule heads. It is easy to see that the query has the answers
\texttt{p(f1(a))}, \texttt{p(f1(f2(f1(a))))},
\texttt{p(f1(f2(f1(f2(f1(a))))))}, etc., ad infinity.
In this case, Terminyzer will report
%% 
\begin{verbatim}
*** Report: subgoals that form a possibly infinite answer-producing pattern ***
    q(?A)@main
        in rule on line 3, file test2.ergo, module main
    p(?A)@main
        in rule on line 4, file test2.ergo, module main
\end{verbatim}
%% 
What this means is that answers to the subgoal \texttt{q(?A)} that 
feeds into the rule on line 3 might cause production of new answers for the
head \texttt{p(...)}. But the next line in the report says that production
of new answers for \texttt{p(...)} might cause production of new answers
for \texttt{q(...)} because \texttt{p(...)} feeds its answers into the rule
on line 4, which generates answers for its head \texttt{q(...)}.

\index{answer abstraction}
In the rare case when
you find that the knowledge base is correct and the infinite set of
answers is inevitable, the computation will not terminate and one will not
get any answers. However, not everything is lost. The engine also supports
\emph{answer abstraction} --- a technique similar to call abstraction.
Since we cannot cheat the nature, something will have to give, and what we
have to give is precision.
Executing both of these commands (where the sizes are chosen for the
sake of an example):
%% 
\begin{verbatim}
    ?- setruntime{goalsize(7,abstract)},
       setruntime{answersize(3,abstract)}.
\end{verbatim}
%% 
will give the following answers:
%% 
\begin{verbatim}
    ?X = a
    ?X = f1(b)
    ?X = f1(f2(a))
    ?X = f1(f2(f1(b)))
    ?X = f1(f2(f1(?(?))))  - undefined
\end{verbatim}
%% 
what this means is that we got four certain answers and the instances of
the fifth answer are \emph{undefined}. This is because some of them are
true answers and some are not. 

Answer abstraction together with call abstraction is guaranteed to
terminate any query that has the following properties:
%% 
\begin{itemize}
\item  uses only tabled (non-transactional predicates and methods)
\item  does not use arithmetic operators
\item  does not perform updates of any kind.
\end{itemize}
%% 

\noindent
To see how Terminyzer works in practice, start \ERGO and load the knowledge
base. Then type
%% 
\begin{verbatim}
    terminyzer{}.
\end{verbatim}
%% 
\index{tripwire}
at the command prompt or select ``\emph{Use Terminyzer}'' 
from the \emph{Debug} menu, if using the Studio. 

If the monitor has graphical capabilities,
a window will pop up with suggested values of various ``tripwires.''
A tripwire is a condition that will cause XSB to pause if certain events happen.
Terminyzer uses the following tripwires:
%% 
\begin{itemize}
\item  \emph{timer}: the query computation will pause if a timeout occurs
\item  \emph{goal size limit}: if the computation generates a subgoal of
  the size exceeding the specified  number, the computation will pause;
  the Terminyzer window also lets the user check a box to request that call
  abstraction should be used.
\item  \emph{answer size limit}: the computation will pause if a subgoal
  answer gets generated that exceeds the given limit.
\item  \emph{limit on the number of currently unfinished subgoals}:
  the computation reaches the state where the number of unfinished subgoals
  (that it is still trying to evaluate) exceeds the given limit.
\end{itemize}
%% 
When the computation pauses, various messages will be printed for your
perusal and Terminyzer will start its analysis. When it is done, a window
will pop up telling you the results. It may find the problem on the first try
and you should then try to find the problem in the knowledge base based on
Terminyzer's report.  It is also possible that no problems were found.
In that case, you might want to ask Terminyzer to try harder.

When Terminyzer finishes its analysis, it asks the user whether to continue
the analysis further or to stop. If you would like to continue, respond
appropriately and the analysis will continue after you resume the query.
(The query is not resumed by itself, giving the user an opportunity to change
mind or to examine the reported statistics and to use other tools.)
When the query is resumed, the computation will pause again if any of the
tripwires is tripped. Note that if the pause happened due to the times then
the timer is reset to the same value as before. If the pause was due to any
other reason, say, reaching the goal size limit, that limit is
automatically increased by 20\%. (Any other tripwire limit is not
affected unless the computation pauses due to that tripwire.)

Note that with each new pause the time it takes to analyze the computation
increases, since Terminyzer now tries to look at a bigger chunk of the
computation trace. It does try to give the user an idea about how long the
wait might be, but it is only a very rough figure.

Finally, if you decide to stop Terminyzer, this does not stop the query.
You can still resume it, use other analysis tools, or you can abort it.




%%% Local Variables: 
%%% mode: latex
%%% TeX-master: "../ergo-manual"
%%% End: 
