
\section{Regular Expression Matching}
\label{sec-regexp}

\index{regular expression}
Regular expression matching is important in many applications. \ERGO
provides convenient facilities for such operations based on the well-known
PCRE package,\footnote{
  \url{http://pcre.org/}
  }
which supports Perl-style extended regular expressions, which are
documented at this RL:
\url{http://perldoc.perl.org/perlre.html#Regular-Expressions}. The Web
contains many sites that include various tutorials for this type of
expressions.

On Linux and Mac,
in order to use the pattern matching facility, PCRE must be installed
before \ERGO is installed. On Windows, PCRE gets installed together with \ERGO,
so the below instructions can be skipped.

To install PCRE on Linux, use the system's package manager and search for
\texttt{pcre}. The names of the packages differ from distribution to
distribution. On Debian, Ubuntu, and Mint, two packages need to be installed:
\texttt{libpcre3} and \texttt{libpcre++-dev}; on Fedora and CentOS, the package is
\texttt{pcre-devel}; on the Mac, use Homebrew to install the \texttt{pcre}
package. (The exact name of the package may differ slightly depending on the
version of the OS, e.g., \texttt{libpcre3-dev}.)

If, on Linux or Mac, \ERGO was installed \emph{first} and PCRE is
installed afterwards, \ERGOAI must be reinstalled.
Alternatively, it can be reconfigured as follows: change to a
directory that has \texttt{ErgoAI} and
\texttt{XSB} as subdirectories (create it if it does not exist) and then execute
%% 
\begin{verbatim}
    sh ./ErgoAI/ergoAI_config.sh
\end{verbatim}
%% 
Follow the prompts.

\index{matchOne method}
\index{matchAll method}
\paragraph{Pattern matching commands.}
Like other string and symbol operations, \ERGO provides the pattern matching
facility through its \texttt{\bs{}basetype}  (or \texttt{\bs{}btp}) module.
Three API calls are supported:
%% 
\begin{itemize}
\item  \texttt{?\emph{Symbol}[matchOne(?\emph{Pattern}) -> ?\emph{Result}]}:
  find the first match.\\
  Here
  \texttt{?\emph{Symbol}} must be a Prolog atom (which includes the data
  types \texttt{\bs{}url}, \texttt{\bs{}string}, or \texttt{\bs{}symbol});
  it is a string to be matched against the \texttt{\emph{?Pattern}} regular
  expression---also a Prolog atom that represents a Perl regular
  expression. \textbf{Important}: Perl regular expressions include various
  keywords that start with a backslash, e.g., \texttt{\bs{}d}. These
  backslashes must be \emph{doubled} in \texttt{\emph{?Pattern}}, i.e.,
  \texttt{\bs{}\bs{}d}.  

  \texttt{?\emph{Result}} is a HiLog term of the form
  \texttt{match(?\emph{Match},?\emph{Prematch},?\emph{Postmatch},?\emph{Submatches})},
  that represents the \emph{first} match.
  Here the first argument, \texttt{?\emph{Match}}, is the substring of
  \texttt{?\emph{Symbol}} that matches the pattern;
  \texttt{?\emph{Prematch}} is the substring of \texttt{?\emph{Match}}
  that precedes
  \texttt{?\emph{Match}};   \texttt{?\emph{Postmatch}} is the substring
  that follows \texttt{?\emph{Match}}; and \texttt{?\emph{Submatches}} is a
  list of submatches. These submatches are substrings of the pattern that match
  the \emph{subpatterns} of \texttt{?\emph{Pattern}} that are enclosed in
  parentheses (see the examples below).
\item \texttt{?\emph{Symbol}[matchAll(?\emph{Pattern}) -> ?\emph{Result}]}: 
  find all matches.\\
  Here
  \texttt{?\emph{Symbol}} and \texttt{\emph{?Pattern}} are as before and
  \texttt{?\emph{Result}} is a \emph{list} of the terms
  of the form
  \texttt{match(?\emph{Match},?\emph{Prematch},?\emph{Postmatch},?\emph{Submatches})}
  described above.
\item \texttt{?\emph{Symbol}[substitute(?\emph{Pattern},?\emph{Substitution}) -> ?Result]}:
  substitute a string for all matching substrings.
  \\
  Here \texttt{?\emph{Symbol}}
  and \texttt{\emph{?Pattern}} are as before and
  \texttt{?\emph{Substitution}} is a Prolog atom. \texttt{?\emph{Result}}
  is \texttt{?\emph{Symbol}} with each matching substring replaced
  with \texttt{?\emph{Substitution}}.
\end{itemize}
%% 
Examples:
%% 
{\small
\begin{verbatim}
    ergo> 'Hello12345-6789 NYwalk'[matchOne('(\\d{5}-\\d{4})\\ [A-Z]{2}')->?R]@\btp.
    ?R = match('12345-6789 NY',Hello,walk,['12345-6789'])

    ergo> 'a@b.com@c.net@d.edu'[matchAll('[a-z]+@[a-z]+\.(com|net|edu)')->?Result]@\btp.
    ?Result = [match('a@b.com',,'@c.net@d.edu',[com]),
               match('com@c.net','a@b.','@d.edu',[net]),
               match('om@c.net','a@b.c','@d.edu',[net]),
               match('m@c.net','a@b.co','@d.edu',[net]),
               match('net@d.edu','a@b.com@c.',,[edu]),
               match('et@d.edu','a@b.com@c.n',,[edu]),
               match('t@d.edu','a@b.com@c.ne',,[edu])]

    ergo> 'This is a Mississippi issue'[substitute(is,was)->?Result]@\btp.
    ?Result = 'Thwas was a Mwasswassippi wassue'
\end{verbatim}
  }
%% 



%%% Local Variables: 
%%% mode: latex
%%% TeX-master: "../ergo-manual"
%%% End: 
