\section{Concurrency Control in CDF} \label{sec:concurrency}

Logging has multiple purposes in CDF use.  It is used to allow
incremental checkpointing, which is faster and takes less space than
continual saving of an entire CDF.  Logging is also used to support
concurrent use of a CDF.

Logging can be turned on and/or off.  When it is on, every update to
the CDF is ""logged"" in an in-memory predicate.  This log can then be
saved to disk in a ""checkpoint"" file, and later used to recreate the
CDF state as it was at the time of the file-saving.  The checkpoint
file contains the locations of the versions of the components needed
to reconstruct the state.

When used to support multiple concurrent use of a CDF, first logging
is turned on, and CDF components are loaded from a stored (shared)
CDF, and their versions are noted.  Subsequent updates to the
in-memory CDF are logged as they are done.  Then when the in-memory
CDF is to be written back to disk to create new versions of the
updated components, using update\_all\_components(in\_shared\_place), the
following is done for each component.  If the current
most-recent-version on disk is the same as the one orginally loaded to
memory, then update\_all\_components works normally (in\_place),
incrementing the version number and writing out the current in-memory
component as that new version.  Otherwise, there is a more recent
version of the CDF on disk (written by a ""concurrent user"".)  The most
recent version is loaded into memory, and the log is used to apply all
the updates to that new version.  (If conflicts are detected, they
must be resolved.  At the moment, no conflict detection is done.)
Then update\_all\_components(in\_place) is used to store that updated
CDF.  After update\_all\_components is run, the log is emptied, and the
process can start again.  

\begin{description}
\ourpredmoditem{cdf\_log\_component\_dirty/1}
This is a dynamic predicate.  After restoring a checkpoint file and
applying the updates, {\tt cdf\_log\_component\_dirty/1} is true of all
components that differ from their stored versions.  It is a ""local
version"" of cdf\_flags(dirty,\_), and should be ""OR-ed"" with it to
find the components that have been updated from last stored version.

\ourpredmoditem{cdf\_set\_log\_on/2} 
{\tt cdf\_set\_log\_on(+LogFile,+Freq)} This predicate creates a new log
and ensures that logging will be performed for further updates until
logging is turned off.

\ourpredmoditem{cdf\_set\_log\_suspend/0} {\tt cdf\_set\_log\_suspend/0}
temporarily turns logging off, if it is on.  It is restarted by
{\tt cdf\_set\_log\_unsuspend/0}.

\ourpredmoditem{cdf\_reset\_log/0} If logging is on, this predicate deletes
the current log, and creates a new empty one.  If logging is off, no
action is taken

\ourpredmoditem{cdf\_log/1} 
{\tt cdf\_log(ExtTermUpdate)} takes a term of the form assert(ExtTerm)
or retractall(ExtTerm) and adds it to the log, if logging is on.
ExtTerm must be a legal extensional fact in the CDF.

\ourpredmoditem{cdf\_apply\_log/0}
{\tt cdf\_apply\_log} applies the log to the current in-memory CDF.  For
example, if the in-memory CDF has been loaded from a saved CDF
version, and the log represents the updates made to that CDF saved in
a checkpoint file, then this will restore the CDF to state at the time
the checkpoint file was written.

When applying the updates, it does NOT update the CDF dirty flags.
However, it does add the name of any modified component to the
predicate {\tt cdf\_log\_component\_dirty/1}.  This allows a user to
determine both when a change has been made since the last checkpoint
has been saved and since the last saved component version.  (See also
{\tt cdf\_log\_OR\_dirty\_flags/0}.) 

\ourpredmoditem{cdf\_apply\_merge\_log/0}
{\tt cdf\_apply\_merge\_log} applies the current log to the current CDF.
The CDF may not be the one that formed the basis for the current log.
I.e., it may have been updated by some other process.  This function
depends on the user-defined predicate,
{\tt check\_log\_merge\_assert(+Term,-Action)} to provide information on
whether the assert actions should be taken or not, and which provide a
conflict.  (Retract actions are always assumed to be acceptable.)

\ourpredmoditem{cdf\_log\_OR\_dirty\_flags/0}
{\tt cdf\_log\_OR\_dirty\_flags} makes every component in {\tt
cdf\_log\_component\_dirty/1} to be dirty, i.e., {\tt
cdf\_flags(dirty,CompName)} to be made true.

\ourpredmoditem{cdf\_save\_log/1}
{\tt cdf\_save\_log(LogFile)} writes a checkpoint file into the file
named {\tt LogFile}.  The file contains the current in-memory log and
the components and their versions from which these updates created the
current state.

\ourpredmoditem{cdf\_remove\_log\_file/1}
{\tt cdf\_remove\_log\_file(LogFile)} renames the indicated file to the
name obtained by appending a \verb|~| to the file (deleting any
previous file with this name.)  This effectively removes the indicated
file, but allows for external recovery, if necessary.

\ourpredmoditem{cdf\_restore\_from\_log/1}
{\tt cdf\_restore\_from\_log(LogFile)} recreates the CDF state represented
by the chekpoint file named {\tt LogFile}.  The current CDF is assumed
initialized.  It loads the versions of the components indicated in the
checkpoint file, and then applies the logged updates to that state.

\end{description}