\chapter{Introduction to XSB Packages} \label{packages}
%====================================================

An XSB package is a piece of software that extends XSB functionality
but is not critical to programming in XSB.  Around a dozen packages
are distributed with XSB, ranging from simple meta-interpreters to
complex software systems.  Some packages provide interfaces from XSB
to other software systems, such as Perl, SModels or Web interfaces (as
in the {\tt libwww} package).  Others, such as the CHR and Flora packages,
extend XSB to different programming paradigms.

Each package is distributed in the {\tt \$XSB\_DIR/packages}
subdirectory, and has two parts: an initialization file, and a
subdirectory in which package source code files and executables are
kept.  For example, the {\tt xsbdoc} package has files {\tt xsbdoc.P},
{\tt xsbdoc.xwam}, and a subdirectory, {\tt xsbdoc}.  If a user
doesn't want to retain {\tt xsbdoc} (or any other package) he or she
may simply remove the initialization files and the associated
subdirectory without affecting the core parts of the XSB system.

%Several of the packages are documented in this manual in the various
%chapters that follow.  However, many of the packages contain their own
%manuals.  For these packages, we provide only a summary of their
%functionality in Chapter \ref{sec:otherpackages}.

