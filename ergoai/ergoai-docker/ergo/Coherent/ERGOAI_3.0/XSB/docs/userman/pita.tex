%%% Local Variables: 
%%% mode: latex
%%% TeX-master: "manual2"
%%% End: 
\chapter{ PITA: Probabilistic  Inference}
\label{package:pita} 

  \begin{center}
    {\Large {\bf By Fabrizio Riguzzi}}
  \end{center}

\index{PITA}
\index{Logic Programs with Annotated Disjunction}
\index{CP-logic}
\index{LPADs}
\index{PRISM}
\index{Probabilistic Logic Programming}
\index{Possibilistic Logic Programming}

%\chapter{ PITA: Probabilistic  Inference}
\label{package:pita} 

  \begin{center}
    {\Large {\bf By Fabrizio Riguzzi}}
  \end{center}

\index{PITA}
\index{Logic Programs with Annotated Disjunction}
\index{CP-logic}
\index{LPADs}
\index{PRISM}
\index{Probabilistic Logic Programming}
\index{Possibilistic Logic Programming}

%\chapter{ PITA: Probabilistic  Inference}
\label{package:pita} 

  \begin{center}
    {\Large {\bf By Fabrizio Riguzzi}}
  \end{center}

\index{PITA}
\index{Logic Programs with Annotated Disjunction}
\index{CP-logic}
\index{LPADs}
\index{PRISM}
\index{Probabilistic Logic Programming}
\index{Possibilistic Logic Programming}

%\chapter{ PITA: Probabilistic  Inference}
\label{package:pita} 

  \begin{center}
    {\Large {\bf By Fabrizio Riguzzi}}
  \end{center}

\index{PITA}
\index{Logic Programs with Annotated Disjunction}
\index{CP-logic}
\index{LPADs}
\index{PRISM}
\index{Probabilistic Logic Programming}
\index{Possibilistic Logic Programming}

%\input{pita}

%\ifnum\pdfoutput>0 % pdflatex compilation


{\em Probabilistic Inference with Tabling and Answer subsumption}
(PITA) \cite{RigS13,RigS11a} is a package for reasoning under
uncertainty. In particular, PITA supports various forms of
Probabilistic Logic Programming (PLP) and Possibilistic Logic
Programming (PossLP). It accepts the language of Logic Programs with
Annotated Disjunctions (LPADs)\cite{VenVer03-TR,VenVer04-ICLP04-IC}
and CP-logic programs
\cite{VenDenBru-JELIA06,DBLP:journals/tplp/VennekensDB09}.

An example of LPAD/CP-logic program is as follows (the syntax in the
PITA implementation is slightly different, as explained in
Section~\ref{sec:pita-syntax})
\begin{eqnarray*}
(heads(Coin):0.5)\vee (tails(Coin):0.5)&\leftarrow&
toss(Coin),\neg biased(Coin).\\
(heads(Coin):0.6)\vee (tails(Coin):0.4)&\leftarrow&
toss(Coin), biased(Coin).\\
(fair(Coin):0.9) \vee (biased(Coin):0.1).&&\\
toss(Coin).&&
\end{eqnarray*}

The first clause states that if we toss a coin that is not biased it
has equal probability of landing heads and tails. The second states
that if the coin is biased it has a slightly higher probability of
landing heads. The third states that the coin is fair with probability
0.9 and biased with probability 0.1 and the last clause states that we
toss a coin with certainty.

PITA computes the probability of queries by tranforming the input
program into a normal logic program and then calling a modified
version of the query on the transformed program.  In order to combine
probabilities or possibilities from different derivations of a goal,
PITA makes use of tabled answer subsumption.  For PLPs, PITA's answer
subsumption makes use of the BDD package CUDD to combine the possibly
non-independent probabilities of different derivations.  CUDD is
included in the XSB distribution.

\section{Installation}

To install PITA with XSB, run XSB \texttt{configure} in the
\texttt{build} directory with option \texttt{--with-pita} and then run
\texttt{makexsb} as usual.  On most Linux systems, this is all that is
needed.

\begin{itemize}
\item {\em Windows} When compiling in cygwin, also build the cygwin
  dll with \texttt{makexsb cygdll}.

\item {\em MacOS} When compiling on MacOS, it should be noted that
  recent versions of {\tt xcode} do not include {\tt autoconf} and
  {\tt automake}, both of which are needed for the PITA installation.
  If these tools are not installed on your system, they can be
  easily installed via these commands:

{\tt 
\noindent
sudo brew install autoconf \\
sudo brew install automake
}

Note that your account must have the permission to execute \texttt{sudo}. 

\noindent
or 

{\tt 
\noindent
sudo port autoconf \\
sudo port automake
}
\end{itemize}

\section{Syntax} \label{sec:pita-syntax}

Disjunction in the head is represented with a semicolon and atoms in the head are separated from probabilities by a colon. For the rest, the usual syntax of Prolog is used.
For example, the  CP-logic clause
$$h_1:p_1\vee \ldots \vee h_n:p_n\leftarrow b_1,\dots,b_m ,\neg c_1,\ldots,\neg c_l$$
is represented by
\begin{verbatim}
    h1:p1 ; ... ; hn:pn :- b1,...,bm,\+ c1,....,\+ cl
\end{verbatim}
No parentheses are necessary. The \texttt{pi} are numeric
expressions. It is up to the user to ensure that the numeric
expressions are legal, i.e. that they sum up to less than one.

Note that only \texttt{\\+} can be used as the negation operator in PITA,
i.e., neither \texttt{tnot}  nor \texttt{not} are allowed.
Other points about {\tt pita} syntax are: 

\begin{itemize}
\item If the clause has an empty body, it can be represented as:
\begin{verbatim}
    h1:p1 ; ... ;hn:pn.
\end{verbatim}
\item If the clause has a single head with probability 1, the annotation can be omitted and the clause takes the form of a normal prolog clause, i.e. 
\begin{verbatim}
    h1:- b1,...,bm,\+ c1,...,\+ cl.
\end{verbatim}
stands for 
\begin{verbatim}
    h1:1 :- b1,...,bm,\+ c1,...,\+ cl.
\end{verbatim}

\item The probabilities in the heads may sum to a number less than 1.
  For instance, the LPAD clause 

\[h_1:p_1\vee null:(1-p_1)\leftarrow b_1,\dots,b_m ,\neg c_1,\ldots,\neg c_l\]

%\noindent
is represented in {\tt pita} by dropping the {\tt null} conjunct, i.e., 

\begin{verbatim}
     h_1:p_1 :-  b_1,\dots,b_m ,\+ c_1,\ldots,\+ c_l.
\end{verbatim}

\item Finally, the body of clauses can contain a number of built-in predicates including:
\begin{verbatim}
    is/2 >/2 </2 >=/2 =</2 =:=/2 =\=/2 true/0 false/0
    =/2 ==/2 \=/2 \==/2 length/2 member/2
\end{verbatim}
\end{itemize}

The directory {\tt \$XSB\_DIR/packages/pita/examples} contains several
examples of LPADs, including the program \texttt{coin.cpl} above,
which is written in PITA's syntax as:

\begin{verbatim}
    heads(Coin):1/2 ; tails(Coin):1/2:- 
         toss(Coin),\+biased(Coin).
    heads(Coin):0.6 ; tails(Coin):0.4:- 
         toss(Coin),biased(Coin).
    fair(Coin):0.9 ; biased(Coin):0.1.
    toss(coin).
\end{verbatim}

\section{Using PITA}
\subsection{Probabilistic Logic Programming}
PITA accepts input programs in two formats: \texttt{.cpl} and \texttt{.pl}.
In both cases they are translated into an internal form that has extension \texttt{.P}.
In the \texttt{.cpl} format, files consist of a sequence of LPAD clauses.
In the \texttt{.pl} format, files use the syntax of \texttt{cplint} for SWI-Prolog, see
\url{http://friguzzi.github.io/cplint/_build/html/index.html}. In the  \texttt{.pl}
format, the same file can be used for PITA in XSB and PITA in \texttt{cplint} for SWI-Prolog.

If you want to use inference on LPADs load PITA in XSB with
\begin{verbatim}
?- [pita].
\end{verbatim}

Then you have different commands for loading the input file.

If the input file is in the \texttt{.cpl} format, you can translate it into the internal
representation and load it with
\begin{verbatim}
?- load_cpl(coin).
\end{verbatim}
Note that {\tt coin.cpl}, which is not in Prolog syntax {\bf cannot}
be loaded via the normal command to compile and load a Prolog file
({\tt ?- [coin]}).

This commands reads  {\tt coin.cpl}, translates it into  {\tt coin.cpl.P}
and loads   {\tt coin.cpl.P}.

For files in the  \texttt{.pl} format, the command is
\begin{verbatim}
?- load_pl(coin).
\end{verbatim}
that  reads  {\tt coin.pl}, translates it into  {\tt coin.pl.P}
and loads   {\tt coin.pl.P}.

You can also use command
\begin{verbatim}
?- load('coin.pl').
\end{verbatim}
that requires the full file name, including the extension, compiles it into a file with the same name with the added extension
\texttt{.P} and loads it.

You can also load directly the translated (compiled) version of a file with the command
\begin{verbatim}
?- load_comp('coin.cpl.P').
\end{verbatim}
of
\begin{verbatim}
?- load_comp('coin.pl.P').
\end{verbatim}
that loads directly the compiled file.

Next, the probability of query atom \texttt{heads(coin)} can be
computed by
\begin{verbatim}
?- prob(heads(coin),P).
\end{verbatim} 
%
PITA, which is based on the distribution semantics
(cf. \cite{Sato95astatistical}) will give the answer {\tt P = 0.51} to
this query.  

The package also includes a test file that can be run to check that the installation was successful. 
The file is \texttt{testpita.pl} and it can be loaded and run with
\begin{verbatim}
?- [testpita].
?- test_pita.
\end{verbatim} 
The package also includes MCINTYRE, which performs approximate inference with Monte Carlo algorithms.
MCINTYRE accepts the same input formats as PITA and the same commands for loading input files.
See \url{http://friguzzi.github.io/cplint/_build/html/index.html} for a description of the available commands.

For loading MCINTYRE use
\begin{verbatim}
?- [mcintyre].
\end{verbatim} 

File \texttt{testmc.pl} can be used for testing MCINTYRE. The command to run the tests is
\begin{verbatim}
?- [testmc].
?- test_mc.
\end{verbatim} 
The \texttt{examples} folder contains various examples of use of MCINTYRE. You can 
also look at the file \texttt{test\_mc.pl} for a list of example and queries over them.

The package also includes SLIPCOVER, an algorithm for learning LPADs. Input files should 
follow the syntax specified in \url{http://friguzzi.github.io/cplint/_build/html/index.html} and should have the
\texttt{.pl} extension. They can loaded for example with
\begin{verbatim}
?- load_pl(bongard).
\end{verbatim}
for \texttt{bongard.pl}.

File \texttt{testsc.pl} can be used for testing SLIPCOVER. The command to run the tests is
\begin{verbatim}
?- [testsc].
?- test_sc.
\end{verbatim} 
The \texttt{examples/learning} folder contains various examples of use of SLIPCOVER. You can 
also look at the file \texttt{test\_sc.pl} for a list of example and goals.



\subsection{Modeling Assumptions}
The probability of  \texttt{heads(coin)} above is calculated by adding the probability of the
{\em composite choices}

\[ head(coin),fair(coin) = 0.45 \]
and 
\[ head(coin),biased(coin) = 0.06\]

These two composite choices are mutually exclusive since they differ
in their atomic choices (in this case, the atoms {\tt fair(coin)} and
{\tt biased(coin)}).  Accordingly, their probabilities can be added
leading the total 0.51.  More about the theory that underlies the
distribution semantics can be found in the survey article
\cite{RigS15}.

In the above discussion of the coin example, we combined probabilities
according to the full distribution semantics.  However, some programs
may satisfy a set of modeling assumptions that allows programs to be
evaluated much more efficiently.
%
\begin{itemize}
\item {\em The independence assumption}: The assumption that different
  calls to a probabilistic atom can we be evaluated independently.
  This leads to the ability to compute the probability of a
  conjunction $(A,B)$ as the product of the probabilities of $A$ and
  $B$;
\item {\em The exclusiveness assumption} The assumption that different
  derivations of an atom $A$ depend on exclusive composite choices.
  This leads to the ability to compute the probability of an atom as
  the sum of the probabilities of its derivations.
\end{itemize}
%
While these assumptions are in fact satisfied by the {\tt coin}
program, they may be fairly strong for larger programs.

These assumptions are fairly strong -- note that the {\tt coin}
program discussed above does not satisfy the exclusiveness assumption,
since the two derivations of {\tt head(coins)} share the probabilistic
atom , as used for instance in the PRISM system
\cite{DBLP:conf/ijcai/SatoK97}, i.e.:

\begin{example} \rm 
An example of a program that does not satisfy the exclusiveness
assumption is {\tt \$XSB\_DIR/packages/pita/examples/flu.cpl} 

\begin{verbatim} 
sneezing(X):0.7 :- flu(X).  
sneezing(X):0.8 :- hay_fever(X).  
flu(bob).
hay_fever(bob).
\end{verbatim}

Given the query {\tt sneezing(bob)}, four possible total composite choices or {\em worlds} must
be considered.

\begin{tabbing}
fooooooooooo\=foooooooooooooooo\=foooooooooooooooo\=foooooooooooooooo\=ooooooooooooo\=\kill
{\em Clause 1}    \> {\tt sneezing(bob)} \> {\tt sneezing(bob)} \> {\em null}          \> {\em null} \\
{\em Clause 2}    \> {\tt sneezing(bob)} \> {\em null}          \> {\tt sneezing(bob)} \> {\em null} \\
{\em Probability} \> 0.56               \> 0.14          \> 0.24          \> 0.06
\end{tabbing}
\noindent
Note that unlike in the {\tt coin} program, the derivations of {\tt
  sneezing(bob)} in the first two clauses are not mutually exclusive;
rather they need to be expanded into mutually exclusive worlds, and
the probabilities of those worlds in which {\tt sneezing(bob)} is true
can then be summed.  In this case, probability of {\tt sneezing(bob)}
is the probability of all worlds in which {\tt sneezing(bob)} is true,
which is $0.56+0.14+0.24=0.94$.
\end{example}

If you know that your program satisfies the independence and exclusion
axioms, you can perform faster inference with the PITA package
\texttt{pitaindexc.P}, which accepts the same commands of
\texttt{pita.P}.  Due to its assumptions, it does not need to maintain
information about composite choices in the CUDD BDD
system~\footnote{Computing the full distribution semantics for a
  ground program $P$ is $\#P$-complete, while computing the restricted
  distribution semantics has the same low polynomial complexity as
  computing the well-founded semantics: ${\cal O}(size(P) \times
  atoms(P))$.}

If you want to compute the Viterbi path and probability of a query
(the Viterbi path is the explanation with the highest probability) as
with the predicate \texttt{viterbif/3} of PRISM, you can use package
\texttt{pitavitind.P}.

The package \texttt{pitacount.P} can be used to count the explanations
for a query, provided that the independence assumption holds. To count
the number of explanations for a query use
\begin{verbatim}
    :- count(heads(coin),C).
\end{verbatim}
\texttt{pitacount.P} does not need to maintain composite choices as
BDDs in Cudd, and so can be much faster than computing the full
distribution semantics, or the Viturbi path.

\subsection{Possibilistic Logic Programming}
PITA can be used also for answering queries to possibilistic logic
program \cite{DBLP:conf/iclp/DuboisLP91}, a form of logic progamming
based on possibilistic logic \cite{DubLanPra-poss-94}. The package
\texttt{pitaposs.P} provides possibilistic inference.  You have to
write the possibilistic program as an LPAD in which the rules have a
single head whose annotation is the lower bound on the necessity of
the clauses. To compute the highest lower bound on the necessity of a
query use
\begin{verbatim}
    :- poss(heads(coin),P).
\end{verbatim}
Like {\tt pitaindexc} and {\tt pitacount}, {\tt pitaposs} does not
require maintenance of composite choices through BDDs in CUDD.

%%% Local Variables: 
%%% mode: latex
%%% TeX-master: "manual2"
%%% End: 


%\ifnum\pdfoutput>0 % pdflatex compilation


{\em Probabilistic Inference with Tabling and Answer subsumption}
(PITA) \cite{RigS13,RigS11a} is a package for reasoning under
uncertainty. In particular, PITA supports various forms of
Probabilistic Logic Programming (PLP) and Possibilistic Logic
Programming (PossLP). It accepts the language of Logic Programs with
Annotated Disjunctions (LPADs)\cite{VenVer03-TR,VenVer04-ICLP04-IC}
and CP-logic programs
\cite{VenDenBru-JELIA06,DBLP:journals/tplp/VennekensDB09}.

An example of LPAD/CP-logic program is as follows (the syntax in the
PITA implementation is slightly different, as explained in
Section~\ref{sec:pita-syntax})
\begin{eqnarray*}
(heads(Coin):0.5)\vee (tails(Coin):0.5)&\leftarrow&
toss(Coin),\neg biased(Coin).\\
(heads(Coin):0.6)\vee (tails(Coin):0.4)&\leftarrow&
toss(Coin), biased(Coin).\\
(fair(Coin):0.9) \vee (biased(Coin):0.1).&&\\
toss(Coin).&&
\end{eqnarray*}

The first clause states that if we toss a coin that is not biased it
has equal probability of landing heads and tails. The second states
that if the coin is biased it has a slightly higher probability of
landing heads. The third states that the coin is fair with probability
0.9 and biased with probability 0.1 and the last clause states that we
toss a coin with certainty.

PITA computes the probability of queries by tranforming the input
program into a normal logic program and then calling a modified
version of the query on the transformed program.  In order to combine
probabilities or possibilities from different derivations of a goal,
PITA makes use of tabled answer subsumption.  For PLPs, PITA's answer
subsumption makes use of the BDD package CUDD to combine the possibly
non-independent probabilities of different derivations.  CUDD is
included in the XSB distribution.

\section{Installation}

To install PITA with XSB, run XSB \texttt{configure} in the
\texttt{build} directory with option \texttt{--with-pita} and then run
\texttt{makexsb} as usual.  On most Linux systems, this is all that is
needed.

\begin{itemize}
\item {\em Windows} When compiling in cygwin, also build the cygwin
  dll with \texttt{makexsb cygdll}.

\item {\em MacOS} When compiling on MacOS, it should be noted that
  recent versions of {\tt xcode} do not include {\tt autoconf} and
  {\tt automake}, both of which are needed for the PITA installation.
  If these tools are not installed on your system, they can be
  easily installed via these commands:

{\tt 
\noindent
sudo brew install autoconf \\
sudo brew install automake
}

Note that your account must have the permission to execute \texttt{sudo}. 

\noindent
or 

{\tt 
\noindent
sudo port autoconf \\
sudo port automake
}
\end{itemize}

\section{Syntax} \label{sec:pita-syntax}

Disjunction in the head is represented with a semicolon and atoms in the head are separated from probabilities by a colon. For the rest, the usual syntax of Prolog is used.
For example, the  CP-logic clause
$$h_1:p_1\vee \ldots \vee h_n:p_n\leftarrow b_1,\dots,b_m ,\neg c_1,\ldots,\neg c_l$$
is represented by
\begin{verbatim}
    h1:p1 ; ... ; hn:pn :- b1,...,bm,\+ c1,....,\+ cl
\end{verbatim}
No parentheses are necessary. The \texttt{pi} are numeric
expressions. It is up to the user to ensure that the numeric
expressions are legal, i.e. that they sum up to less than one.

Note that only \texttt{\\+} can be used as the negation operator in PITA,
i.e., neither \texttt{tnot}  nor \texttt{not} are allowed.
Other points about {\tt pita} syntax are: 

\begin{itemize}
\item If the clause has an empty body, it can be represented as:
\begin{verbatim}
    h1:p1 ; ... ;hn:pn.
\end{verbatim}
\item If the clause has a single head with probability 1, the annotation can be omitted and the clause takes the form of a normal prolog clause, i.e. 
\begin{verbatim}
    h1:- b1,...,bm,\+ c1,...,\+ cl.
\end{verbatim}
stands for 
\begin{verbatim}
    h1:1 :- b1,...,bm,\+ c1,...,\+ cl.
\end{verbatim}

\item The probabilities in the heads may sum to a number less than 1.
  For instance, the LPAD clause 

\[h_1:p_1\vee null:(1-p_1)\leftarrow b_1,\dots,b_m ,\neg c_1,\ldots,\neg c_l\]

%\noindent
is represented in {\tt pita} by dropping the {\tt null} conjunct, i.e., 

\begin{verbatim}
     h_1:p_1 :-  b_1,\dots,b_m ,\+ c_1,\ldots,\+ c_l.
\end{verbatim}

\item Finally, the body of clauses can contain a number of built-in predicates including:
\begin{verbatim}
    is/2 >/2 </2 >=/2 =</2 =:=/2 =\=/2 true/0 false/0
    =/2 ==/2 \=/2 \==/2 length/2 member/2
\end{verbatim}
\end{itemize}

The directory {\tt \$XSB\_DIR/packages/pita/examples} contains several
examples of LPADs, including the program \texttt{coin.cpl} above,
which is written in PITA's syntax as:

\begin{verbatim}
    heads(Coin):1/2 ; tails(Coin):1/2:- 
         toss(Coin),\+biased(Coin).
    heads(Coin):0.6 ; tails(Coin):0.4:- 
         toss(Coin),biased(Coin).
    fair(Coin):0.9 ; biased(Coin):0.1.
    toss(coin).
\end{verbatim}

\section{Using PITA}
\subsection{Probabilistic Logic Programming}
PITA accepts input programs in two formats: \texttt{.cpl} and \texttt{.pl}.
In both cases they are translated into an internal form that has extension \texttt{.P}.
In the \texttt{.cpl} format, files consist of a sequence of LPAD clauses.
In the \texttt{.pl} format, files use the syntax of \texttt{cplint} for SWI-Prolog, see
\url{http://friguzzi.github.io/cplint/_build/html/index.html}. In the  \texttt{.pl}
format, the same file can be used for PITA in XSB and PITA in \texttt{cplint} for SWI-Prolog.

If you want to use inference on LPADs load PITA in XSB with
\begin{verbatim}
?- [pita].
\end{verbatim}

Then you have different commands for loading the input file.

If the input file is in the \texttt{.cpl} format, you can translate it into the internal
representation and load it with
\begin{verbatim}
?- load_cpl(coin).
\end{verbatim}
Note that {\tt coin.cpl}, which is not in Prolog syntax {\bf cannot}
be loaded via the normal command to compile and load a Prolog file
({\tt ?- [coin]}).

This commands reads  {\tt coin.cpl}, translates it into  {\tt coin.cpl.P}
and loads   {\tt coin.cpl.P}.

For files in the  \texttt{.pl} format, the command is
\begin{verbatim}
?- load_pl(coin).
\end{verbatim}
that  reads  {\tt coin.pl}, translates it into  {\tt coin.pl.P}
and loads   {\tt coin.pl.P}.

You can also use command
\begin{verbatim}
?- load('coin.pl').
\end{verbatim}
that requires the full file name, including the extension, compiles it into a file with the same name with the added extension
\texttt{.P} and loads it.

You can also load directly the translated (compiled) version of a file with the command
\begin{verbatim}
?- load_comp('coin.cpl.P').
\end{verbatim}
of
\begin{verbatim}
?- load_comp('coin.pl.P').
\end{verbatim}
that loads directly the compiled file.

Next, the probability of query atom \texttt{heads(coin)} can be
computed by
\begin{verbatim}
?- prob(heads(coin),P).
\end{verbatim} 
%
PITA, which is based on the distribution semantics
(cf. \cite{Sato95astatistical}) will give the answer {\tt P = 0.51} to
this query.  

The package also includes a test file that can be run to check that the installation was successful. 
The file is \texttt{testpita.pl} and it can be loaded and run with
\begin{verbatim}
?- [testpita].
?- test_pita.
\end{verbatim} 
The package also includes MCINTYRE, which performs approximate inference with Monte Carlo algorithms.
MCINTYRE accepts the same input formats as PITA and the same commands for loading input files.
See \url{http://friguzzi.github.io/cplint/_build/html/index.html} for a description of the available commands.

For loading MCINTYRE use
\begin{verbatim}
?- [mcintyre].
\end{verbatim} 

File \texttt{testmc.pl} can be used for testing MCINTYRE. The command to run the tests is
\begin{verbatim}
?- [testmc].
?- test_mc.
\end{verbatim} 
The \texttt{examples} folder contains various examples of use of MCINTYRE. You can 
also look at the file \texttt{test\_mc.pl} for a list of example and queries over them.

The package also includes SLIPCOVER, an algorithm for learning LPADs. Input files should 
follow the syntax specified in \url{http://friguzzi.github.io/cplint/_build/html/index.html} and should have the
\texttt{.pl} extension. They can loaded for example with
\begin{verbatim}
?- load_pl(bongard).
\end{verbatim}
for \texttt{bongard.pl}.

File \texttt{testsc.pl} can be used for testing SLIPCOVER. The command to run the tests is
\begin{verbatim}
?- [testsc].
?- test_sc.
\end{verbatim} 
The \texttt{examples/learning} folder contains various examples of use of SLIPCOVER. You can 
also look at the file \texttt{test\_sc.pl} for a list of example and goals.



\subsection{Modeling Assumptions}
The probability of  \texttt{heads(coin)} above is calculated by adding the probability of the
{\em composite choices}

\[ head(coin),fair(coin) = 0.45 \]
and 
\[ head(coin),biased(coin) = 0.06\]

These two composite choices are mutually exclusive since they differ
in their atomic choices (in this case, the atoms {\tt fair(coin)} and
{\tt biased(coin)}).  Accordingly, their probabilities can be added
leading the total 0.51.  More about the theory that underlies the
distribution semantics can be found in the survey article
\cite{RigS15}.

In the above discussion of the coin example, we combined probabilities
according to the full distribution semantics.  However, some programs
may satisfy a set of modeling assumptions that allows programs to be
evaluated much more efficiently.
%
\begin{itemize}
\item {\em The independence assumption}: The assumption that different
  calls to a probabilistic atom can we be evaluated independently.
  This leads to the ability to compute the probability of a
  conjunction $(A,B)$ as the product of the probabilities of $A$ and
  $B$;
\item {\em The exclusiveness assumption} The assumption that different
  derivations of an atom $A$ depend on exclusive composite choices.
  This leads to the ability to compute the probability of an atom as
  the sum of the probabilities of its derivations.
\end{itemize}
%
While these assumptions are in fact satisfied by the {\tt coin}
program, they may be fairly strong for larger programs.

These assumptions are fairly strong -- note that the {\tt coin}
program discussed above does not satisfy the exclusiveness assumption,
since the two derivations of {\tt head(coins)} share the probabilistic
atom , as used for instance in the PRISM system
\cite{DBLP:conf/ijcai/SatoK97}, i.e.:

\begin{example} \rm 
An example of a program that does not satisfy the exclusiveness
assumption is {\tt \$XSB\_DIR/packages/pita/examples/flu.cpl} 

\begin{verbatim} 
sneezing(X):0.7 :- flu(X).  
sneezing(X):0.8 :- hay_fever(X).  
flu(bob).
hay_fever(bob).
\end{verbatim}

Given the query {\tt sneezing(bob)}, four possible total composite choices or {\em worlds} must
be considered.

\begin{tabbing}
fooooooooooo\=foooooooooooooooo\=foooooooooooooooo\=foooooooooooooooo\=ooooooooooooo\=\kill
{\em Clause 1}    \> {\tt sneezing(bob)} \> {\tt sneezing(bob)} \> {\em null}          \> {\em null} \\
{\em Clause 2}    \> {\tt sneezing(bob)} \> {\em null}          \> {\tt sneezing(bob)} \> {\em null} \\
{\em Probability} \> 0.56               \> 0.14          \> 0.24          \> 0.06
\end{tabbing}
\noindent
Note that unlike in the {\tt coin} program, the derivations of {\tt
  sneezing(bob)} in the first two clauses are not mutually exclusive;
rather they need to be expanded into mutually exclusive worlds, and
the probabilities of those worlds in which {\tt sneezing(bob)} is true
can then be summed.  In this case, probability of {\tt sneezing(bob)}
is the probability of all worlds in which {\tt sneezing(bob)} is true,
which is $0.56+0.14+0.24=0.94$.
\end{example}

If you know that your program satisfies the independence and exclusion
axioms, you can perform faster inference with the PITA package
\texttt{pitaindexc.P}, which accepts the same commands of
\texttt{pita.P}.  Due to its assumptions, it does not need to maintain
information about composite choices in the CUDD BDD
system~\footnote{Computing the full distribution semantics for a
  ground program $P$ is $\#P$-complete, while computing the restricted
  distribution semantics has the same low polynomial complexity as
  computing the well-founded semantics: ${\cal O}(size(P) \times
  atoms(P))$.}

If you want to compute the Viterbi path and probability of a query
(the Viterbi path is the explanation with the highest probability) as
with the predicate \texttt{viterbif/3} of PRISM, you can use package
\texttt{pitavitind.P}.

The package \texttt{pitacount.P} can be used to count the explanations
for a query, provided that the independence assumption holds. To count
the number of explanations for a query use
\begin{verbatim}
    :- count(heads(coin),C).
\end{verbatim}
\texttt{pitacount.P} does not need to maintain composite choices as
BDDs in Cudd, and so can be much faster than computing the full
distribution semantics, or the Viturbi path.

\subsection{Possibilistic Logic Programming}
PITA can be used also for answering queries to possibilistic logic
program \cite{DBLP:conf/iclp/DuboisLP91}, a form of logic progamming
based on possibilistic logic \cite{DubLanPra-poss-94}. The package
\texttt{pitaposs.P} provides possibilistic inference.  You have to
write the possibilistic program as an LPAD in which the rules have a
single head whose annotation is the lower bound on the necessity of
the clauses. To compute the highest lower bound on the necessity of a
query use
\begin{verbatim}
    :- poss(heads(coin),P).
\end{verbatim}
Like {\tt pitaindexc} and {\tt pitacount}, {\tt pitaposs} does not
require maintenance of composite choices through BDDs in CUDD.

%%% Local Variables: 
%%% mode: latex
%%% TeX-master: "manual2"
%%% End: 


%\ifnum\pdfoutput>0 % pdflatex compilation


{\em Probabilistic Inference with Tabling and Answer subsumption}
(PITA) \cite{RigS13,RigS11a} is a package for reasoning under
uncertainty. In particular, PITA supports various forms of
Probabilistic Logic Programming (PLP) and Possibilistic Logic
Programming (PossLP). It accepts the language of Logic Programs with
Annotated Disjunctions (LPADs)\cite{VenVer03-TR,VenVer04-ICLP04-IC}
and CP-logic programs
\cite{VenDenBru-JELIA06,DBLP:journals/tplp/VennekensDB09}.

An example of LPAD/CP-logic program is as follows (the syntax in the
PITA implementation is slightly different, as explained in
Section~\ref{sec:pita-syntax})
\begin{eqnarray*}
(heads(Coin):0.5)\vee (tails(Coin):0.5)&\leftarrow&
toss(Coin),\neg biased(Coin).\\
(heads(Coin):0.6)\vee (tails(Coin):0.4)&\leftarrow&
toss(Coin), biased(Coin).\\
(fair(Coin):0.9) \vee (biased(Coin):0.1).&&\\
toss(Coin).&&
\end{eqnarray*}

The first clause states that if we toss a coin that is not biased it
has equal probability of landing heads and tails. The second states
that if the coin is biased it has a slightly higher probability of
landing heads. The third states that the coin is fair with probability
0.9 and biased with probability 0.1 and the last clause states that we
toss a coin with certainty.

PITA computes the probability of queries by tranforming the input
program into a normal logic program and then calling a modified
version of the query on the transformed program.  In order to combine
probabilities or possibilities from different derivations of a goal,
PITA makes use of tabled answer subsumption.  For PLPs, PITA's answer
subsumption makes use of the BDD package CUDD to combine the possibly
non-independent probabilities of different derivations.  CUDD is
included in the XSB distribution.

\section{Installation}

To install PITA with XSB, run XSB \texttt{configure} in the
\texttt{build} directory with option \texttt{--with-pita} and then run
\texttt{makexsb} as usual.  On most Linux systems, this is all that is
needed.

\begin{itemize}
\item {\em Windows} When compiling in cygwin, also build the cygwin
  dll with \texttt{makexsb cygdll}.

\item {\em MacOS} When compiling on MacOS, it should be noted that
  recent versions of {\tt xcode} do not include {\tt autoconf} and
  {\tt automake}, both of which are needed for the PITA installation.
  If these tools are not installed on your system, they can be
  easily installed via these commands:

{\tt 
\noindent
sudo brew install autoconf \\
sudo brew install automake
}

Note that your account must have the permission to execute \texttt{sudo}. 

\noindent
or 

{\tt 
\noindent
sudo port autoconf \\
sudo port automake
}
\end{itemize}

\section{Syntax} \label{sec:pita-syntax}

Disjunction in the head is represented with a semicolon and atoms in the head are separated from probabilities by a colon. For the rest, the usual syntax of Prolog is used.
For example, the  CP-logic clause
$$h_1:p_1\vee \ldots \vee h_n:p_n\leftarrow b_1,\dots,b_m ,\neg c_1,\ldots,\neg c_l$$
is represented by
\begin{verbatim}
    h1:p1 ; ... ; hn:pn :- b1,...,bm,\+ c1,....,\+ cl
\end{verbatim}
No parentheses are necessary. The \texttt{pi} are numeric
expressions. It is up to the user to ensure that the numeric
expressions are legal, i.e. that they sum up to less than one.

Note that only \texttt{\\+} can be used as the negation operator in PITA,
i.e., neither \texttt{tnot}  nor \texttt{not} are allowed.
Other points about {\tt pita} syntax are: 

\begin{itemize}
\item If the clause has an empty body, it can be represented as:
\begin{verbatim}
    h1:p1 ; ... ;hn:pn.
\end{verbatim}
\item If the clause has a single head with probability 1, the annotation can be omitted and the clause takes the form of a normal prolog clause, i.e. 
\begin{verbatim}
    h1:- b1,...,bm,\+ c1,...,\+ cl.
\end{verbatim}
stands for 
\begin{verbatim}
    h1:1 :- b1,...,bm,\+ c1,...,\+ cl.
\end{verbatim}

\item The probabilities in the heads may sum to a number less than 1.
  For instance, the LPAD clause 

\[h_1:p_1\vee null:(1-p_1)\leftarrow b_1,\dots,b_m ,\neg c_1,\ldots,\neg c_l\]

%\noindent
is represented in {\tt pita} by dropping the {\tt null} conjunct, i.e., 

\begin{verbatim}
     h_1:p_1 :-  b_1,\dots,b_m ,\+ c_1,\ldots,\+ c_l.
\end{verbatim}

\item Finally, the body of clauses can contain a number of built-in predicates including:
\begin{verbatim}
    is/2 >/2 </2 >=/2 =</2 =:=/2 =\=/2 true/0 false/0
    =/2 ==/2 \=/2 \==/2 length/2 member/2
\end{verbatim}
\end{itemize}

The directory {\tt \$XSB\_DIR/packages/pita/examples} contains several
examples of LPADs, including the program \texttt{coin.cpl} above,
which is written in PITA's syntax as:

\begin{verbatim}
    heads(Coin):1/2 ; tails(Coin):1/2:- 
         toss(Coin),\+biased(Coin).
    heads(Coin):0.6 ; tails(Coin):0.4:- 
         toss(Coin),biased(Coin).
    fair(Coin):0.9 ; biased(Coin):0.1.
    toss(coin).
\end{verbatim}

\section{Using PITA}
\subsection{Probabilistic Logic Programming}
PITA accepts input programs in two formats: \texttt{.cpl} and \texttt{.pl}.
In both cases they are translated into an internal form that has extension \texttt{.P}.
In the \texttt{.cpl} format, files consist of a sequence of LPAD clauses.
In the \texttt{.pl} format, files use the syntax of \texttt{cplint} for SWI-Prolog, see
\url{http://friguzzi.github.io/cplint/_build/html/index.html}. In the  \texttt{.pl}
format, the same file can be used for PITA in XSB and PITA in \texttt{cplint} for SWI-Prolog.

If you want to use inference on LPADs load PITA in XSB with
\begin{verbatim}
?- [pita].
\end{verbatim}

Then you have different commands for loading the input file.

If the input file is in the \texttt{.cpl} format, you can translate it into the internal
representation and load it with
\begin{verbatim}
?- load_cpl(coin).
\end{verbatim}
Note that {\tt coin.cpl}, which is not in Prolog syntax {\bf cannot}
be loaded via the normal command to compile and load a Prolog file
({\tt ?- [coin]}).

This commands reads  {\tt coin.cpl}, translates it into  {\tt coin.cpl.P}
and loads   {\tt coin.cpl.P}.

For files in the  \texttt{.pl} format, the command is
\begin{verbatim}
?- load_pl(coin).
\end{verbatim}
that  reads  {\tt coin.pl}, translates it into  {\tt coin.pl.P}
and loads   {\tt coin.pl.P}.

You can also use command
\begin{verbatim}
?- load('coin.pl').
\end{verbatim}
that requires the full file name, including the extension, compiles it into a file with the same name with the added extension
\texttt{.P} and loads it.

You can also load directly the translated (compiled) version of a file with the command
\begin{verbatim}
?- load_comp('coin.cpl.P').
\end{verbatim}
of
\begin{verbatim}
?- load_comp('coin.pl.P').
\end{verbatim}
that loads directly the compiled file.

Next, the probability of query atom \texttt{heads(coin)} can be
computed by
\begin{verbatim}
?- prob(heads(coin),P).
\end{verbatim} 
%
PITA, which is based on the distribution semantics
(cf. \cite{Sato95astatistical}) will give the answer {\tt P = 0.51} to
this query.  

The package also includes a test file that can be run to check that the installation was successful. 
The file is \texttt{testpita.pl} and it can be loaded and run with
\begin{verbatim}
?- [testpita].
?- test_pita.
\end{verbatim} 
The package also includes MCINTYRE, which performs approximate inference with Monte Carlo algorithms.
MCINTYRE accepts the same input formats as PITA and the same commands for loading input files.
See \url{http://friguzzi.github.io/cplint/_build/html/index.html} for a description of the available commands.

For loading MCINTYRE use
\begin{verbatim}
?- [mcintyre].
\end{verbatim} 

File \texttt{testmc.pl} can be used for testing MCINTYRE. The command to run the tests is
\begin{verbatim}
?- [testmc].
?- test_mc.
\end{verbatim} 
The \texttt{examples} folder contains various examples of use of MCINTYRE. You can 
also look at the file \texttt{test\_mc.pl} for a list of example and queries over them.

The package also includes SLIPCOVER, an algorithm for learning LPADs. Input files should 
follow the syntax specified in \url{http://friguzzi.github.io/cplint/_build/html/index.html} and should have the
\texttt{.pl} extension. They can loaded for example with
\begin{verbatim}
?- load_pl(bongard).
\end{verbatim}
for \texttt{bongard.pl}.

File \texttt{testsc.pl} can be used for testing SLIPCOVER. The command to run the tests is
\begin{verbatim}
?- [testsc].
?- test_sc.
\end{verbatim} 
The \texttt{examples/learning} folder contains various examples of use of SLIPCOVER. You can 
also look at the file \texttt{test\_sc.pl} for a list of example and goals.



\subsection{Modeling Assumptions}
The probability of  \texttt{heads(coin)} above is calculated by adding the probability of the
{\em composite choices}

\[ head(coin),fair(coin) = 0.45 \]
and 
\[ head(coin),biased(coin) = 0.06\]

These two composite choices are mutually exclusive since they differ
in their atomic choices (in this case, the atoms {\tt fair(coin)} and
{\tt biased(coin)}).  Accordingly, their probabilities can be added
leading the total 0.51.  More about the theory that underlies the
distribution semantics can be found in the survey article
\cite{RigS15}.

In the above discussion of the coin example, we combined probabilities
according to the full distribution semantics.  However, some programs
may satisfy a set of modeling assumptions that allows programs to be
evaluated much more efficiently.
%
\begin{itemize}
\item {\em The independence assumption}: The assumption that different
  calls to a probabilistic atom can we be evaluated independently.
  This leads to the ability to compute the probability of a
  conjunction $(A,B)$ as the product of the probabilities of $A$ and
  $B$;
\item {\em The exclusiveness assumption} The assumption that different
  derivations of an atom $A$ depend on exclusive composite choices.
  This leads to the ability to compute the probability of an atom as
  the sum of the probabilities of its derivations.
\end{itemize}
%
While these assumptions are in fact satisfied by the {\tt coin}
program, they may be fairly strong for larger programs.

These assumptions are fairly strong -- note that the {\tt coin}
program discussed above does not satisfy the exclusiveness assumption,
since the two derivations of {\tt head(coins)} share the probabilistic
atom , as used for instance in the PRISM system
\cite{DBLP:conf/ijcai/SatoK97}, i.e.:

\begin{example} \rm 
An example of a program that does not satisfy the exclusiveness
assumption is {\tt \$XSB\_DIR/packages/pita/examples/flu.cpl} 

\begin{verbatim} 
sneezing(X):0.7 :- flu(X).  
sneezing(X):0.8 :- hay_fever(X).  
flu(bob).
hay_fever(bob).
\end{verbatim}

Given the query {\tt sneezing(bob)}, four possible total composite choices or {\em worlds} must
be considered.

\begin{tabbing}
fooooooooooo\=foooooooooooooooo\=foooooooooooooooo\=foooooooooooooooo\=ooooooooooooo\=\kill
{\em Clause 1}    \> {\tt sneezing(bob)} \> {\tt sneezing(bob)} \> {\em null}          \> {\em null} \\
{\em Clause 2}    \> {\tt sneezing(bob)} \> {\em null}          \> {\tt sneezing(bob)} \> {\em null} \\
{\em Probability} \> 0.56               \> 0.14          \> 0.24          \> 0.06
\end{tabbing}
\noindent
Note that unlike in the {\tt coin} program, the derivations of {\tt
  sneezing(bob)} in the first two clauses are not mutually exclusive;
rather they need to be expanded into mutually exclusive worlds, and
the probabilities of those worlds in which {\tt sneezing(bob)} is true
can then be summed.  In this case, probability of {\tt sneezing(bob)}
is the probability of all worlds in which {\tt sneezing(bob)} is true,
which is $0.56+0.14+0.24=0.94$.
\end{example}

If you know that your program satisfies the independence and exclusion
axioms, you can perform faster inference with the PITA package
\texttt{pitaindexc.P}, which accepts the same commands of
\texttt{pita.P}.  Due to its assumptions, it does not need to maintain
information about composite choices in the CUDD BDD
system~\footnote{Computing the full distribution semantics for a
  ground program $P$ is $\#P$-complete, while computing the restricted
  distribution semantics has the same low polynomial complexity as
  computing the well-founded semantics: ${\cal O}(size(P) \times
  atoms(P))$.}

If you want to compute the Viterbi path and probability of a query
(the Viterbi path is the explanation with the highest probability) as
with the predicate \texttt{viterbif/3} of PRISM, you can use package
\texttt{pitavitind.P}.

The package \texttt{pitacount.P} can be used to count the explanations
for a query, provided that the independence assumption holds. To count
the number of explanations for a query use
\begin{verbatim}
    :- count(heads(coin),C).
\end{verbatim}
\texttt{pitacount.P} does not need to maintain composite choices as
BDDs in Cudd, and so can be much faster than computing the full
distribution semantics, or the Viturbi path.

\subsection{Possibilistic Logic Programming}
PITA can be used also for answering queries to possibilistic logic
program \cite{DBLP:conf/iclp/DuboisLP91}, a form of logic progamming
based on possibilistic logic \cite{DubLanPra-poss-94}. The package
\texttt{pitaposs.P} provides possibilistic inference.  You have to
write the possibilistic program as an LPAD in which the rules have a
single head whose annotation is the lower bound on the necessity of
the clauses. To compute the highest lower bound on the necessity of a
query use
\begin{verbatim}
    :- poss(heads(coin),P).
\end{verbatim}
Like {\tt pitaindexc} and {\tt pitacount}, {\tt pitaposs} does not
require maintenance of composite choices through BDDs in CUDD.

%%% Local Variables: 
%%% mode: latex
%%% TeX-master: "manual2"
%%% End: 


%\ifnum\pdfoutput>0 % pdflatex compilation


{\em Probabilistic Inference with Tabling and Answer subsumption}
(PITA) \cite{RigS13,RigS11a} is a package for reasoning under
uncertainty. In particular, PITA supports various forms zof
Probabilistic Logic Programming (PLP) and Possibilistic Logic
Programming (PossLP). It accepts the language of Logic Programs with
Annotated Disjunctions (LPADs)\cite{VenVer03-TR,VenVer04-ICLP04-IC}
and CP-logic programs
\cite{VenDenBru-JELIA06,DBLP:journals/tplp/VennekensDB09}.

An example of LPAD/CP-logic program is as follows (the syntax in the
PITA implementation is slightly different, as explained in
Section~\ref{sec:pita-syntax})
\begin{eqnarray*}
(heads(Coin):0.5)\vee (tails(Coin):0.5)&\leftarrow&
toss(Coin),\neg biased(Coin).\\
(heads(Coin):0.6)\vee (tails(Coin):0.4)&\leftarrow&
toss(Coin), biased(Coin).\\
(fair(Coin):0.9) \vee (biased(Coin):0.1).&&\\
toss(Coin).&&
\end{eqnarray*}

The first clause states that if we toss a coin that is not biased it
has equal probability of landing heads and tails. The second states
that if the coin is biased it has a slightly higher probability of
landing heads. The third states that the coin is fair with probability
0.9 and biased with probability 0.1 and the last clause states that we
toss a coin with certainty.

PITA computes the probability of queries by tranforming the input
program into a normal logic program and then calling a modified
version of the query on the transformed program.  In order to combine
probabilities or possibilities from different derivations of a goal,
PITA makes use of tabled answer subsumption.  For PLPs, PITA's answer
subsumption makes use of the BDD package CUDD to combine the possibly
non-independent probabilities of different derivations.  CUDD is
included in the XSB distribution.

\section{Installation}

To install PITA with XSB, run XSB \texttt{configure} in the
\texttt{build} directory with option \texttt{--with-pita} and then run
\texttt{makexsb} as usual.  On most Linux systems, this is all that is
needed.

\begin{itemize}
\item {\em Windows} When compiling in cygwin, also build the cygwin
  dll with \texttt{makexsb cygdll}.

\item {\em MacOS} When compiling on MacOS, it should be noted that
  recent versions of {\tt xcode} do not include {\tt autoconf} and
  {\tt automake}, both of which are needed for the PITA installation.
  If your these tools are not installed on your system, they can be
  easily installed by

{\tt 
\noindent
(sudo) brew install autoconf \\
(sudo) brew install automake
}

\noindent
or 

{\tt 
\noindent
sudo port autoconf \\
sudo port automake
}
\end{itemize}

\section{Syntax} \label{sec:pita-syntax}

Disjunction in the head is represented with a semicolon and atoms in the head are separated from probabilities by a colon. For the rest, the usual syntax of Prolog is used.
For example, the  CP-logic clause
$$h_1:p_1\vee \ldots \vee h_n:p_n\leftarrow b_1,\dots,b_m ,\neg c_1,\ldots,\neg c_l$$
is represented by
\begin{verbatim}
    h1:p1 ; ... ; hn:pn :- b1,...,bm,\+ c1,....,\+ cl
\end{verbatim}
No parentheses are necessary. The \texttt{pi} are numeric
expressions. It is up to the user to ensure that the numeric
expressions are legal, i.e. that they sum up to less than one.
Other points about {\tt pita} syntax are: 

\begin{itemize}
\item If the clause has an empty body, it can be represented as:
\begin{verbatim}
    h1:p1 ; ... ;hn:pn.
\end{verbatim}
\item If the clause has a single head with probability 1, the annotation can be omitted and the clause takes the form of a normal prolog clause, i.e. 
\begin{verbatim}
    h1:- b1,...,bm,\+ c1,...,\+ cl.
\end{verbatim}
stands for 
\begin{verbatim}
    h1:1 :- b1,...,bm,\+ c1,...,\+ cl.
\end{verbatim}

\item The probabilities in the heads may sum to a number less than 1.
  For instance, the LPAD clause 

\[h_1:p_1\vee null:(1-p_1)\leftarrow b_1,\dots,b_m ,\neg c_1,\ldots,\neg c_l\]

%\noindent
is represented in {\tt pita} by dropping the {\tt null} conjunct, i.e., 

\begin{verbatim}
     h_1:p_1 :-  b_1,\dots,b_m ,\+ c_1,\ldots,\+ c_l.
\end{verbatim}

\item Finally, the body of clauses can contain a number of built-in predicates including:
\begin{verbatim}
    is/2 >/2 </2 >=/2 =</2 =:=/2 =\=/2 true/0 false/0
    =/2 ==/2 \=/2 \==/2 length/2 member/2
\end{verbatim}
\end{itemize}

The directory {\tt \$XSB\_DIR/packages/pita/examples} contains several
examples of LPADs, including the program \texttt{coin.cpl} above,
which is written in PITA's syntax as:

\begin{verbatim}
    heads(Coin):1/2 ; tails(Coin):1/2:- 
         toss(Coin),\+biased(Coin).
    heads(Coin):0.6 ; tails(Coin):0.4:- 
         toss(Coin),biased(Coin).
    fair(Coin):0.9 ; biased(Coin):0.1.
    toss(coin).
\end{verbatim}

\section{Using PITA}
\subsection{Probabilistic Logic Programming}
PITA accepts input programs in two formats: \texttt{.cpl} and \texttt{.pl}.
In both cases they are translated into an internal form that has extension \texttt{.P}.
In the \texttt{.cpl} format, files consist of a sequence of LPAD clauses.
In the \texttt{.pl} format, files use the syntax of \texttt{cplint} for SWI-Prolog, see
\url{http://friguzzi.github.io/cplint/_build/html/index.html}. In the  \texttt{.pl}
format, the same file can be used for PITA in XSB and PITA in \texttt{cplint} for SWI-Prolog.

If you want to use inference on LPADs load PITA in XSB with
\begin{verbatim}
?- [pita].
\end{verbatim}

Then you have different commands for loading the input file.

If the input file is in the \texttt{.cpl} format, you can translate it into the internal
representation and load it with
\begin{verbatim}
?- load_cpl(coin).
\end{verbatim}
Note that {\tt coin.cpl}, which is not in Prolog syntax {\bf cannot}
be loaded via the normal command to compile and load a Prolog file
({\tt ?- [coin]}).

This commands reads  {\tt coin.cpl}, translates it into  {\tt coin.cpl.P}
and loads   {\tt coin.cpl.P}.

For files in the  \texttt{.pl} format, the command is
\begin{verbatim}
?- load_pl(coin).
\end{verbatim}
that  reads  {\tt coin.pl}, translates it into  {\tt coin.pl.P}
and loads   {\tt coin.pl.P}.

You can also use command
\begin{verbatim}
?- load('coin.pl').
\end{verbatim}
that requires the full file name, including the extension, compiles it into a file with the same name with the added extension
\texttt{.P} and loads it.

You can also load directly the translated (compiled) version of a file with the command
\begin{verbatim}
?- load_comp('coin.cpl.P').
\end{verbatim}
of
\begin{verbatim}
?- load_comp('coin.pl.P').
\end{verbatim}
that loads directly the compiled file.

Next, the probability of query atom \texttt{heads(coin)} can be
computed by
\begin{verbatim}
?- prob(heads(coin),P).
\end{verbatim} 
%
PITA, which is based on the distribution semantics
(cf. \cite{Sato95astatistical}) will give the answer {\tt P = 0.51} to
this query.  

The package also includes a test file that can be run to check that the installation was successful. 
The file is \texttt{testpita.pl} and it can be loaded and run with
\begin{verbatim}
?- [testpita].
?- test_pita.
\end{verbatim} 
The package also includes MCINTYRE, which performs approximate inference with Monte Carlo algorithms.
MCINTYRE accepts the same input formats as PITA and the same commands for loading input files.
See \url{http://friguzzi.github.io/cplint/_build/html/index.html} for a description of the available commands.

For loading MCINTYRE use
\begin{verbatim}
?- [mcintyre].
\end{verbatim} 

File \texttt{testmc.pl} can be used for testing MCINTYRE. The command to run the tests is
\begin{verbatim}
?- [testmc].
?- test_mc.
\end{verbatim} 
The \texttt{examples} folder contains various examples of use of MCINTYRE. You can 
also look at the file \texttt{test\_mc.pl} for a list of example and queries over them.

The package also includes SLIPCOVER, an algorithm for learning LPADs. Input files should 
follow the syntax specified in \url{http://friguzzi.github.io/cplint/_build/html/index.html} and should have the
\texttt{.pl} extension. They can loaded for example with
\begin{verbatim}
?- load_pl(bongard).
\end{verbatim}
for \texttt{bongard.pl}.

File \texttt{testsc.pl} can be used for testing SLIPCOVER. The command to run the tests is
\begin{verbatim}
?- [testsc].
?- test_sc.
\end{verbatim} 
The \texttt{examples/learning} folder contains various examples of use of SLIPCOVER. You can 
also look at the file \texttt{test\_sc.pl} for a list of example and goals.



\subsection{Modeling Assumptions}
The probability of  \texttt{heads(coin)} above is calculated by adding the probability of the
{\em composite choices}

\[ head(coin),fair(coin) = 0.45 \]
and 
\[ head(coin),biased(coin) = 0.06\]

These two composite choices are mutually exclusive since they differ
in their atomic choices (in this case, the atoms {\tt fair(coin)} and
{\tt biased(coin)}).  Accordingly, their probabilities can be added
leading the total 0.51.  More about the theory that underlies the
distribution semantics can be found in the survey article
\cite{RigS15}.

In the above discussion of the coin example, we combined probabilities
according to the full distribution semantics.  However, some programs
may satisfy a set of modeling assumptions that allows programs to be
evaluated much more efficiently.
%
\begin{itemize}
\item {\em The independence assumption}: The assumption that different
  calls to a probabilistic atom can we be evaluated independently.
  This leads to the ability to compute the probability of a
  conjunction $(A,B)$ as the product of the probabilities of $A$ and
  $B$;
\item {\em The exclusiveness assumption} The assumption that different
  derivations of an atom $A$ depend on exclusive composite choices.
  This leads to the ability to compute the probability of an atom as
  the sum of the probabilities of its derivations.
\end{itemize}
%
While these assumptions are in fact satisfied by the {\tt coin}
program, they may be fairly strong for larger programs.

These assumptions are fairly strong -- note that the {\tt coin}
program discussed above does not satisfy the exclusiveness assumption,
since the two derivations of {\tt head(coins)} share the probabilistic
atom , as used for instance in the PRISM system
\cite{DBLP:conf/ijcai/SatoK97}, i.e.:

\begin{example} \rm 
An example of a program that does not satisfy the exclusiveness
assumption is {\tt \$XSB\_DIR/packages/pita/examples/flu.cpl} 

\begin{verbatim} 
sneezing(X):0.7 :- flu(X).  
sneezing(X):0.8 :- hay_fever(X).  
flu(bob).
hay_fever(bob).
\end{verbatim}

Given the query {\tt sneezing(bob)}, four possible total composite choices or {\em worlds} must
be considered.

\begin{tabbing}
fooooooooooo\=foooooooooooooooo\=foooooooooooooooo\=foooooooooooooooo\=ooooooooooooo\=\kill
{\em Clause 1}    \> {\tt sneezing(bob)} \> {\tt sneezing(bob)} \> {\em null}          \> {\em null} \\
{\em Clause 2}    \> {\tt sneezing(bob)} \> {\em null}          \> {\tt sneezing(bob)} \> {\em null} \\
{\em Probability} \> 0.56               \> 0.14          \> 0.24          \> 0.06
\end{tabbing}
\noindent
Note that unlike in the {\tt coin} program, the derivations of {\tt
  sneezing(bob)} in the first two clauses are not mutually exclusive;
rather they need to be expanded into mutually exclusive worlds, and
the probabilities of those worlds in which {\tt sneezing(bob)} is true
can then be summed.  In this case, probability of {\tt sneezing(bob)}
is the probability of all worlds in which {\tt sneezing(bob)} is true,
which is $0.56+0.14+0.24=0.94$.
\end{example}

If you know that your program satisfies the independence and exclusion
axioms, you can perform faster inference with the PITA package
\texttt{pitaindexc.P}, which accepts the same commands of
\texttt{pita.P}.  Due to its assumptions, it does not need to maintain
information about composite choices in the CUDD BDD
system~\footnote{Computing the full distribution semantics for a
  ground program $P$ is $\#P$-complete, while computing the restricted
  distribution semantics has the same low polynomial complexity as
  computing the well-founded semantics: ${\cal O}(size(P) \times
  atoms(P))$.}

If you want to compute the Viterbi path and probability of a query
(the Viterbi path is the explanation with the highest probability) as
with the predicate \texttt{viterbif/3} of PRISM, you can use package
\texttt{pitavitind.P}.

The package \texttt{pitacount.P} can be used to count the explanations
for a query, provided that the independence assumption holds. To count
the number of explanations for a query use
\begin{verbatim}
    :- count(heads(coin),C).
\end{verbatim}
\texttt{pitacount.P} does not need to maintain composite choices as
BDDs in Cudd, and so can be much faster than computing the full
distribution semantics, or the Viturbi path.

\subsection{Possibilistic Logic Programming}
PITA can be used also for answering queries to possibilistic logic
program \cite{DBLP:conf/iclp/DuboisLP91}, a form of logic progamming
based on possibilistic logic \cite{DubLanPra-poss-94}. The package
\texttt{pitaposs.P} provides possibilistic inference.  You have to
write the possibilistic program as an LPAD in which the rules have a
single head whose annotation is the lower bound on the necessity of
the clauses. To compute the highest lower bound on the necessity of a
query use
\begin{verbatim}
    :- poss(heads(coin),P).
\end{verbatim}
Like {\tt pitaindexc} and {\tt pitacount}, {\tt pitaposs} does not
require maintenance of composite choices through BDDs in CUDD.
