
\chapter{{\tt rdf}: The XSB RDF Parser} \label{chapter:RDF}

\begin{center}
  {\Large {\bf By Aneesh Ali}}
\end{center}

\section{Introduction}

RDF is a W3C standard for representing meta-data about documents
on the Web as well as exchanging frame-based data (e.g. ontologies). RDF
has a formal
data model defined in terms of {\it triples}. In addition, a
{\it graph} model is defined for visualization and an XML serialization
for exchange.
This chapter describes the API provided by the XSB RDF parsing package.
The package and its documentation are adaptations
from SWI Prolog.

{\em Note that this package only handles RDF-XML.  For RDF in Turtle,
  ntriples, etc. use xsbpy together with the Python rdflib package as
  indicated in Chapter~\ref{chap:xsbpy}.}
  
\section{High-level API}

The RDF translator is built in Prolog on top of the {\bf sgml2pl}
package, which provides XML parsing.
The transformation is realized in two passes.
It is designed to operate in various environments and therefore
provides interfaces at various levels. First we describe the top level,
which parses RDF-XML file into a list of
triples. These triples are {\em not} asserted into the Prolog database
because it is not necessarily the final format the user wishes to use
and it is not clear how the user might want to deal with multiple RDF
documents.  Some options are using global URI's in one pool, in Prolog
modules, or using an additional argument.

\begin{description}
    \item[load\_rdf({\it +File, -Triples})]\mbox{}\\
    Same as {\tt load\_rdf(}{}{\it +File, -Triples, []}{)}.

    \item[load\_rdf({\it +File, -Triples, +Options})]\mbox{}\\
    Read the RDF-XML file {\it File} and return a list of {\it Triples}.
{\it Options} is a list of additional processing options.  Currently defined
options are:

\begin{description}
  \item[{\bf base\_uri}{\bf (}{\it BaseURI}{\bf)}]\mbox{}\\
If provided, local identifiers and identifier-references are globalized
using this URI.  If omitted, local identifiers are not tagged.

  \item[{\bf blank\_nodes}{\bf (}{\it Mode}{\bf)}]\mbox{}\\
If {\it Mode} is {\tt share} (default), blank-node properties (i.e.\
complex properties without identifier) are reused if they result in
exactly the same triple-set. Two descriptions are shared if their
intermediate description is the same. This means they should produce the
same set of triples in the same order. The value {\tt noshare} creates
a new resource for each blank node.

  \item[{\bf expand\_foreach}{\bf (}{\it Boolean}{\bf)}]\mbox{}\\
If {\it Boolean} is {\tt true}, expand {\tt rdf:aboutEach} into
a set of triples. By default the parser generates
\texttt{rdf}(\texttt{each}(\emph{Container}), \emph{Predicate}, \emph{Subject}).

  \item[{\bf lang}{\bf (}{\it Lang}{\bf)}]\mbox{}\\
Define the initial language (i.e.\ pretend there is an {\tt xml:lang}
declaration in an enclosing element).

  \item[{\bf ignore\_lang}{\bf (}{\it Bool}{\bf)}]\mbox{}\\
If {\tt true}, {\tt xml:lang} declarations in the document are
ignored.  This is mostly for compatibility with older versions of
this library that did not support language identifiers.

  \item[{\bf convert\_typed\_literal}{\bf (}{\it :ConvertPred}{\bf)}]\mbox{}\\
If the parser finds a literal with the {\tt rdf:datatype}={\it Type}
attribute, call \emph{ConvertPred}({\it +Type, +Content, -Literal}).
{\it Content} is the XML element contents returned by the XML
parser (a list). The predicate must unify {\it Literal}
with a Prolog representation of {\it Content} according to
{\it Type} or throw an exception if the conversion cannot be made.

This option serves two purposes.  First of all it can be used
to ignore type declarations for backward compatibility of this
library.  Second it can be used to convert typed literals to 
a meaningful Prolog representation (e.g., convert '42' to the
Prolog integer 42 if the type is {\tt xsd:int} or a related
type).

  \item[{\bf namespaces}{\bf (}{\it -List}{\bf)}]\mbox{}\\
Unify {\it List} with a list of {\it NS}={\it URL} for each
encountered {\tt xmlns}:{\it NS}={\it URL} declaration found
in the source.

  \item[{\bf entity}{\bf (}{\it +Name, +Value}{\bf)}]\mbox{}\\
Overrule entity declaration in file.  As it is common practice
to declare namespaces using entities in RDF/XML, this option
allows changing the namespace without changing the file.
Multiple such options are allowed.
   \end{description}

The {\it Triples} list is a list of the form \texttt{rdf}({\it Subject, Predicate,
Object}) triples.  {\it Subject} is either a plain resource (an atom),
or one of the terms \texttt{each}({\it URI}) or \texttt{prefix}({\it URI}) with the
usual meaning.  {\it Predicate} is either a plain atom for
explicitly non-qualified names or a term 
\mbox{{\it NameSpace}{\bf :}{\it Name}}.  If {\it NameSpace} is the
defined RDF name space it is returned as the atom {\tt rdf}.
{\it Object} is a URI, a {\it Predicate} or a term of the
form \texttt{literal}({\it Value}) for literal values.  {\it Value} is
either a plain atom or a parsed XML term (list of atoms and elements).

\end{description}


\subsection{RDF Object representation}		\label{sec:rdfobject}

The \emph{Object} (3rd) part of a triple can have several different
types.  If the object is a resource it is returned as either a plain
atom or a term \mbox{{\it NameSpace}{\bf :}{\it Name}}.  If it is a
literal it is returned as \texttt{literal}({\it Value}), where {\it Value}
can have one of the form below.

\begin{itemize}
    \item {An atom}\mbox{}\\
If the literal {\it Value} is a plain atom is a literal value not
subject to a datatype or {\tt xml:lang} qualifier.

    \item {\bf lang}{\bf (}{\it LanguageID, Atom}{\bf )}\mbox{}\\
If the literal is subject to an {\tt xml:lang} qualifier
{\it LanguageID} specifies the language and {\it Atom} the
actual text.

    \item {A list}\mbox{}\\
If the literal is an XML literal as created by
\mbox{\tt parseType="Literal"}, the raw output of the XML parser for the
content of the element is returned. This content is a list of
\texttt{element}({\it Name, Attributes, Content}) and atoms for CDATA parts as
described with the {\tt sgml} package.

    \item {\bf type}{\bf (}{\it Type, StringValue}{\bf )}\mbox{}\\
If the literal has an {\tt rdf:datatype}={\it Type} a term of this
format is returned.
\end{itemize}

\subsection{Name spaces}

RDF name spaces are identified using URIs. Unfortunately various URI's
are in common use to refer to RDF. The RDF parser
therefore defines the \texttt{rdf\_name\_space/1} predicate as
\texttt{multifile}, which can be
extended by the user. For example, to parse Netscape
OpenDirectory (\url{http://www.mozilla.org/rdf/doc/inference.html})
given in the {\tt structure.rdf} file
(\url{http://rdf.dmoz.org/rdf/structure.rdf.u8.gz}), the following declarations are used:

\begin{verbatim}
:- multifile
        rdf_parser:rdf_name_space/1.

rdf_parser:rdf_name_space('http://www.w3.org/TR/RDF/').
rdf_parser:rdf_name_space('http://directory.mozilla.org/rdf').
rdf_parser:rdf_name_space('http://dmoz.org/rdf').
\end{verbatim}
%%
The above statements will then extend
the initial definition of this predicate provided by the parser:
%%
\begin{verbatim}
rdf_name_space('http://www.w3.org/1999/02/22-rdf-syntax-ns#').
rdf_name_space('http://www.w3.org/TR/REC-rdf-syntax').
\end{verbatim}
%%


\subsection{Low-level access}

The predicates {\tt load\_rdf/2} and \texttt{load\_rdf/3} described earlier are
not always sufficient. For example, they
cannot deal with documents where the RDF statement is embedded in an XML
document. It also cannot deal with really large documents (e.g.\ the
Netscape OpenDirectory project, currently about 90 MBytes), without requiring
huge amounts of memory.

For really large documents, the {\bf sgml2pl} parser can be instructed
to handle the content of a specific element (i.e. \verb$<rdf:RDF>$)
element-by-element.  The parsing primitives defined in this section
can be used to process these one-by-one.

\begin{description}
    \item[xml\_to\_rdf({\it +XML, +BaseURI, -Triples})]\mbox{}\\
Process an XML term produced by \texttt{sgml}'s \texttt{load\_structure/4}  using the
\texttt{dialect(xmlns)}  output option.  {\it XML} is either
a complete \verb$<rdf:RDF>$ element, a list of RDF-objects
(container or description), or a single description of container.
\end{description}

\section{Testing the RDF translator}

A test-suite and a driver program are provided by {\tt rdf\_test.P} in
the {\tt XSB/examples/rdf} directory. To run these tests, load this file
into Prolog and execute \texttt{test\_all}. 
The test files found in the directory
{\tt examples/rdf/suite} are then converted into triples.
The expected output is in {\tt examples/rdf/expectedoutput}. One can
also run the tests selectively, using the following predicates:
%%
\begin{description}
    \item[suite({\it +N})]\mbox{}\\
Run test {\it N} using the file {\tt suite/tN.rdf} and display its
RDF representation and the triples.
    \item[test\_file({\it +File})]\mbox{}\\
Process {\tt File} and display its RDF representation and the triples.
\end{description}



%%% Local Variables: 
%%% mode: latex
%%% TeX-master: "manual2"
%%% End: 
