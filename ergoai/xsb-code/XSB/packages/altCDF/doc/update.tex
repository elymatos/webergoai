\section{Updating CDF Instances} \label{sec:update}

Both extensional facts and intensional rules are dynamic, so that they
can be asserted or retracted.  Any attempt to directly assert or
retract to a CDF instance, however, will almost certainly lead to
disaster.  When the state of a CDF instance changes, tables that
support the Type-0 interface may need to be abolished; consistency
checks may need to be rerun; various components may need to be marked
as dirty (so that they will be written out the next time the component
is updated); the update logged to support future merges; and the XJ
cache may need to be updated, and so on.  The Update API supports all
the tasks that need to be done when a CDF instance changes.

Abolishing tables for the Type-0 interface, can be seen as an instance
of the {\em table update problem}.  This problem can occur when a
table depends on dynamic information.  When the dynamic information
changes the table may become out-of-date.  In principle, this problem
might be addressed by changing the tables themselves -- adding or
deleting answers as needed.  However, relating specific answers to
dynamic information is not always easy to do, (as when recursive
predicates are tabled).  As an alternative various tables may be
abolished when there is a danger that they are based on out-of-date
information.  Using this approach, one can abolish entire tables for
certain queries, or all tables for certain predicates.

CDF takes the simple but sound solution of abolishing all calls to a
tabled system predicate whenever that predicate might depend on
updated dynamic information.  Such dependencies are most common for
the Type-0 API.  Various tables are used to compute the {\tt xxxAttr}
relations and {\tt isa/2} (when {\tt isa/2} is tabled).  If, say, a
{\tt hasAttr\_ext/3} fact is added, then the {\tt hasAttr/3} tables
should be abolished: otherwise the tables will represent outdated
information.  Similarly, if a change is made to an {\tt isa/2}
extensional fact or intensional rule, all of the tables used to
compute the {\tt xxxAttr} predicates must be abolished, as well as any
tables used to compute {\tt isa/2} itself.

A more difficult problem can arise when intensional rules are added or
deleted.  If the intensional rule depends on tabled predicates, and
the tabled predicates themselves depend on Type-0 predicates, the
tabled predicates supporting the rule must be abolished.  To address
this when a CDF intensional rule $R$ is updated, CDF must be informed
of those Type-0 predicates upon which tables in $R$ depend using the
predicate {\tt addTableDependency/2} defined below.

% TLS concurrency

\subsection{The Update API}

The following predicates form the update API for CDF.  No exceptions
are listed for these predicates: rather a user can chose the semantic
checks she wants by adding checks to various contexts, or removing
them, as described in Section~\ref{sec:consist}.

\begin{description}

\ourpredmoditem{newExtTerm/1}{cdf\_init\_cdf}
{\tt newExtTerm(+Term)} is used to add a new extensional fact to the
CDF instance.  This predicate applies those consistency checks that
have been specified for addition of a single extensional facts (see
\refsec{sec:config} for the default checks, and \refsec{sec:consist}
for a description of the checks themselves).  It then logs the fact
that {\tt Term} has been asserted, marks the component to which {\tt
Term} belongs as $dirty$, invalidates the XJ cache, and finally
abolishes the appropriate tables.

\ctxtexc{newExtTermSingle}

\ourpredmoditem{retractallExtTerm/1}{cdf\_init\_cdf}
{\tt retractallExtTerm(?Term)} retracts all extensional CDF facts that
unify with {\tt Term}.  This predicate applies those consistency
checks that have been specified for retraction of extensional facts
unifying with a term (see \refsec{sec:config} for the default checks,
and \refsec{sec:consist} for a description of the checks themselves).
It then logs the fact that the facts unifying with {\tt Term} have
been retracted, marks the components to which those facts belong as
$dirty$, invalidates the XJ cache, and finally abolishes the
appropriate tables.  Note that this operation does not affect
information derived via intensional rules, and may not affect
information derived via inheritance.

\ctxtexc{retractallExtTermSingle}

\mycomment{
\ourpredmoditem{updateExtTerm/3}{cdf\_init\_cdf}
{\tt updateExtTerm(ExtTerm,Vars,NValList)} updates a set of values,
only invalidating those that changed.  {\tt ExtTerm} is a term of the
form of an \_ext cdf predicate; Vars is a term containing the
variables appearing in ExtTerm, and NValList is a list of ground
instances of Vars.  The goal is to minimize invalidation.  The
semantics is: (retractallExtTerm(ExtTerm), member(Vars,NValList),
newExtTerm(ExtTerm)), fail ; true).
}

\ourpredmoditem{newIntRule/2}{cdf\_init\_cdf}
{\tt newIntRule(+Head,+Body)} is used to add a new intensional rule to
the CDF instance.  This predicate applies those consistency checks
that have been specified for addition of a single intensional rule
(see \refsec{sec:config} for the default checks, and
\refsec{sec:consist} for a description of the checks themselves.  It
then logs that the rule has been asserted, marks the component to
which the rule belongs as $dirty$, invalidates the XJ cache, and
finally abolishes the appropriate tables.

\ourpredmoditem{retractallIntRule/2}{cdf\_init\_cdf}
{\tt retractallExtTerm(?Head,?Body)} retracts all intensional rules for
which {\tt clause(Head,Body)} is true.  This predicate applies those
consistency checks that have been specified for retraction of rules
(see \refsec{sec:config} for the default checks, and
\refsec{sec:consist} for a description of the checks themselves.  Note
that this operation does not affect information derived via
extensional facts, and may not affect information derived via
inheritance.

\index{predicate indicator}
\ourpredmoditem{addTableDependency/2}{cdf\_init\_cdf}
{\tt addTableDependencies(+TableList,+DependencyList} informs CDF that
each table in {\tt TableList} depends on all dynamic predicates in
{\tt DependencyList}.  Both lists use predicate indicators (i.e. terms
of the form {\tt Functor/Arity}, see the XSB manual) and the predicate
specifiers in {\tt DependencyList} must be of type
\domain{isDynSupportedPred/1}.  These dependencies ensure that
tables for predicates in {\tt TableList} are abolished whenever
extensional facts or intensional rules for tables in {\tt
DependencyList} are updated.

\ourpreddomitem{isDynSupportedPred/1}{cdf\_init\_cdf}
The goal {\tt isDynSupportedPred(?Pred)} succeeds if {\tt Pred}
unifies with a specifier for a predicate defined to be the Type-0 API,
currently {\tt isa/2}, {\tt allAttr/3}, {\tt hasAttr/3}, {\tt
maxAttr/4}, {\tt minAttr/4}, {\tt classHasAttr/3} or {\tt
necessCond/2}.

\ourpredmoditem{abolish\_cdf\_tables/0}{cdf\_init\_cdf}
This predicate abolishes all tables used by CDF.  It does not
reinitialize any other aspects of the CDF system state as does
\pred{initialize\_state/0} (see \refsec{sec:configapi}).

%\ourpredmodrptitem{make\_cdf\_clean/0}{cdf\_init\_cdf}
%\ourpredmoditem{make\_cdf\_clean/1}{cdf\_init\_cdf}

\end{description}
