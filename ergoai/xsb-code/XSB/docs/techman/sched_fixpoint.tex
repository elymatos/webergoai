\section{Scheduling Fixpoint in Completion (D.S.Warren 9 July 2019)} \label{sec:fixpoint}

During completion of a nontrivial component, XSB calls find\_fixpoint,
defined in emu/schedrev\_xsb\_i.h, to schedule all pending answers to
all goals in the component, i.e., those from completion stack frames
between the leader and the top of the stack.  The previous algorithm
simply scanned all the completion frames in the component to find
pending answers to schedule.  However, if the component is large (and
the number of pending answers is small) this is costly.  For some
propositional programs, it can make evaluation quadratic in the size
of the program.

I have modified this code to make the cost proportional to the number
of answers to be scheduled.  The idea is simply to keep a list of all
completion frames which have answers.  This list is anchored in a
global variable, fp\_sched\_list.  Every link is tagged (with
in\_fp\_sched\_tag) so that, when an answer is returned at a frame, it
is known whether that frame has already been added to the
fp\_sched\_list.  Completion frames are added to this list when a new
answer is found in new\_answer\_dealloc (defined in
slginsts\_xsb\_i.h).  This list is scanned in find\_fixpoint to find
answers to schedule.  Because of local scheduling, answers for the top
component are at the beginning of the list, so when a frame is
encountered that is not in the top component, scheduling can stop
since it is known that all frames in the top component have already
been encountered and processed.

This algorithm works for local scheduling since it depends on the order
of answers generated in that answers from the component must be
generated before answers outside of the component.  Without this
assumption the list of completion frames that must be scanned includes
those outside the component and this may re-introduce redundant
scanning and (I believe) the possibility of nonlinear behavior on
propositional programs.  To handle that situation, one might want to
order the items on the scheduling list using a priority queue or
something similar.

