\chapter{Wildcard Matching}
\label{chap-wildcard}

\begin{center}
{\Large {\bf By Michael Kifer}}
\end{center}

XSB has an efficient interface to POSIX 
wildcard matching functions.  To take advantage of this feature, you
must build XSB using a C compiler that supports
POSIX 2.0 (for wildcard
matching).  This includes GCC and
probably most other compilers. This also works under Windows,
provided you install CygWin and use GCC to
compile~\footnote{This package has not yet been ported to the
  multi-threaded engine.}.


The \texttt{wildmatch}  package provides the following functionality: 
%%
\begin{enumerate}
\item Telling whether a wildcard, like the ones used in Unix shells, match
  against a given string. Wildcards supported are of the kind available in
  tcsh or bash. Alternating characters ({\it e.g.}, ``\verb|[abc]|'' or
  ``\verb|[^abc]|'') are supported.
\item Finding the list of all file names in a given directory that match a
  given wildcard. This facility generalizes {\tt directory/2} (in module {\tt
    directory}), and it is much more efficient.
\item String conversion to lower and upper case.
\end{enumerate}
%%

To use this package, you need to type:
%%
\begin{verbatim}
| ?- [wildmatch].  
\end{verbatim}
%%
If you are planning to use it in an XSB program, you need
this directive:
%%
\begin{verbatim}
:- import glob_directory/4, wildmatch/3, convert_string/3 from wildmatch.
\end{verbatim}
%%

The calling sequence for \verb|glob_directory/4| is:
%%
\begin{verbatim}
   glob_directory(+Wildcard, +Directory, ?MarkDirs, -FileList)  
\end{verbatim}
%%
The parameter {\tt Wildcard} can be either a Prolog atom or a Prolog
string. {\tt Directory} is also an atom or a string; it specifies the
directory to be globbed. {\tt MarkDirs} indicates whether directory names
should be decorated with a trailing slash: if {\tt MarkDirs} is bound, then
directories will be so decorated. If MarkDirs is an unbound variable, then
trailing slashes will not be added.

{\tt FileList} gets the list of files in {\tt Directory} that match {\tt
  Wildcard}.  If {\tt Directory} is bound to an atom, then {\tt FileList}
gets bound to a list of atoms; if {\tt Directory} is a Prolog string, then
{\tt FileList} will be bound to a list of strings as well.

This predicate succeeds is at least one match is found. If no matches are
found or if {\tt Directory} does not exist or cannot be read, then the
predicate fails.

The calling sequence for {\tt wildmatch/3}  is as follows:
%%
\begin{verbatim}
    wildmatch(+Wildcard, +String, ?IgnoreCase)  
\end{verbatim}
%%
{\tt Wildcard} is the same as before. {\tt String} represents the string to
be matched against {\tt Wildcard}. Like {\tt Wildcard}, {\tt String} can be
an atom or a string. {\tt IgnoreCase} indicates whether case of letters
should be ignored during matching. Namely, if this argument is bound to a
non-variable, then the case of letters is ignored. Otherwise, if {\tt
  IgnoreCase} is a variable, then the case of letters is preserved.

This predicate succeeds when {\tt Wildcard} matches {\tt String} and fails
otherwise.

The calling sequence for {\tt convert\_string/3}  is as follows:
%%
\begin{verbatim}
    convert_string(+InputString, +OutputString, +ConversionFlag)  
\end{verbatim}
%%
The input string must be an atom or a character list. The output string
must be unbound. Its type will be ``atom'' if so was the input and it will be
a character list if so was the input string. The conversion flag must be
the atom {\tt tolower} or {\tt toupper}. 

This predicate always succeeds, unless there was an error, such as wrong
type argument passed as a parameter.




%%% Local Variables: 
%%% mode: latex
%%% TeX-master: "manual2"
%%% End: 
