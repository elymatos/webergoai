\section{Configuration, Loading and Start-up} \label{px:config}

%{\sc Working draft -- not complete.
%  \bi
%  \item Not mentioning LD\_LIBRARY or XSB\_ROOT issues.
%  \item  only describing Linux.
%  \item Once the above kinks are out, I need to integrate this
%    documentation it with the xsbpy configuration.
%    \ei
%    }

%Also, because \janus{} is installed as a site
%package for a given version of Python, we strongly recommend using a
%Python virtual envonment, just as a virtual Python environment is
%usually necessary for installation via {\tt pip}.  Fortunately, once
%the appropriate Python packages are present and a user has set up a
%virtual Python environment, \janus{} is easy to set up.

\subsection{Installation on Linux and macOS}

\subsubsection{System Prerequisites}

On Linux and MacOS, both \janus-py{} and \janus-plg{} are
automatically configured as part of the source code configuration and
making process for XSB, while on Windows \janus-py{} is not yet
working properly.  This configuration and make process itself requires
most of the tools to build \janus-py (e.g., {\tt build-essential} or
{\tt xcode}).  In XSB \janus{} has been tested at various times on
versions 3.6-3.11 of Python.

\begin{itemize}
\item {\bf Linux}

%To properly execute these steps the proper Python system packages must
%have been installed.

%Beyond Linux packages like {\tt build-essential} that are needed to
  %<build and make XSB itself,

\janus{} requires Python development packages to have been installed
for the Python version of interest.  (See
Section~\ref{sec:xsbpy-linux} for an overview of installing such
dependencies.) On Ubuntu, and other Debian-derived Linuxes
installation also requires installing {\tt \$\{PYTHON\}-distutils},
though this is not required for other Linuxes. For a given version of
Python {\tt \$PYTHON} on Ubuntu, the command to install these Python
packages would be:

{\tt sudo apt-get install \$\{PYTHON\}-dev \$\{PYTHON\}-distutils}

On Fedora-based Linuxes, the {\tt dnf} command must be
used.\footnote{\januspy{} has been tested on Ubuntu v. 18 and v. 20,
and on Fedora 35; and Python versions 3.7-3.11 have been tested.  We
believe that \januspy{} will work on other recent Unix distributions
and newer Python versions as well.}

%Since \janus{} requires {\tt distutils}, the {\tt python-distutils}
%package also needs to have been installed.

%On Linux, setting up a Python virtual environment requires the {\tt
%  python-dev} and {\tt python-venv} packages to be installed,

\item {\bf macOS} For masOS, you'll need a development package of
  Python that includes {\tt libpython.dylib} and {\tt python.h}, both
  of which can be installed via homebrew, macports or other means.
%    libpython.dylib} and {\tt python.h}.  You'll also need a C

%  Future versions of PX will likely support automatic installation,
%  but for now you'll have to make sure the following prerequisites are
%  installed, using Xcode, brew or some other installer.

  %A version of
%  Python 3.6-3.10 is recommended, but for whatever Python version you
%  use you'll need the development package that includes {\tt
%    libpython.dylib} and {\tt python.h}.  You'll also need a C
%  compiler either clang or gcc, wither of whcih can easily be
%  installed via xcode.  
  
\end{itemize}

%\subsubsection{Configuring \janus{} on Linux and macOS}

%\paragraph{Quick Start}
%\begin{enumerate}
%\item cd to {\tt \$XSB\_ROOT/XSB/packages/xsbpy} and execute {\tt
%  bash px\_configure.sh} or {\tt
%  zsh px\_configure.sh}
%\item {\tt source \$XSB\_HOME/packages/xsbpy/px\_activate} from anywhere
%\item start using px!
%\end{enumerate}
%

\subsubsection{Discussion}
%If the correct Python libraries have been installed, configuration of
%{\tt px} should be simple, and nearly the same on Linux and
%macOS.  When XSB is first configured, via
%\begin{verbatim}
%$XSB_ROOT/XSB/build/configure
%\end{verbatim}
%the {\tt xsbpy} package, which allows XSB to call Python, is also
%configured and compiled.  Using files created during the {\tt xsbpy}
%build, \janus{} is configured and build by cd-ing to
%\begin{verbatim}
%$XSB_ROOT/XSB/packages/xsbpy
%\end{verbatim}
%\noindent
%and executing
%\begin{verbatim}
%  bash px_configure.sh
%\end{verbatim}
%This second configuration step uses XSB to generate a {\tt setup.py}
%file for the Python {\tt distutils} package, and then builds and
%installs {\tt px}.  \footnote{At some point, this confuration will be
%  changed to use Python's {\tt setuptools}.}  The compiled C
%extensions for Python are build and put in the user's {\tt
%  site-packages} directory, i.e.,
%\begin{verbatim}
%~/.local/lib/python3.10/site-packages/
%\end{verbatim}
XSB's configuration/make process creates a file, {\tt
  janus\_activate}, somewhat analogous to activation code in {\tt
  venv} or other virtual environments.\footnote{This activation file
updates the {\tt LD\_LIBRARY\_PATH} used by the {\tt ld} command, and
adds the {\tt janus} directory to {\tt PYTHONPATH}.  On macOS {\tt
  LD\_LIBRARY\_PATH} is also updated.} In {\tt bash} or {\tt zsh},
simply type
\begin{verbatim}
source $XSB_HOME/packages/xsbpy/px_activate
\end{verbatim}
and you're ready to go.

%\begin{itemize}
%\item As a default, if the command
%
%  {\tt \$XSB\_ROOT/XSB/packages/xsbpy/configure}
%
%  is issued, \janus{} is installed in the directory corresponding to {\tt
%    sys.USER\_SITE} in the version of Python {\tt \$Python} with which
%  {\tt xsbpy} was configured. In Linux this is under the users home
%  directory (\verb|~|):
%\begin{verbatim}
%~/.local/lib/$PYTHON/site-packages
%\end{verbatim}
%If this default directory is used, {\tt px} will only be available to
%the user who installed it, and {\tt px} will be available for all
%Python projects for that user.
%
%\item If \janus{} is to be installed in a Python virtual environment
%  that uses {\tt venv}, the virtual environment will first need to be
%  activated in a given shell.  In {\tt bash} this is done as:
%
%\begin{verbatim}
%source $VENV_DIR/bin/activate
%\end{verbatim}
%
%Where {\tt \$VENV\_DIR} is the root directory for the virtual
%environment.  Once this is done, the command
%
%\begin{verbatim}  
%$XSB_ROOT/XSB/packages/xsbpy/configure ???
%\end{verbatim}
%
%is issued, \janus{} will be installed in 
%
%\begin{verbatim}
%$VENV_DIR/lib/$PYTHON/site-packages
%\end{verbatim}
%
%\item For system administrators, if the command
%
%\begin{verbatim}  
%sudo $XSB_ROOT/XSB/packages/xsbpy/configure ???
%\end{verbatim}
%
%  is issued, \janus{} will be installed in
%
%\begin{verbatim}  
%/usr/local/lib/$PYTHON/dist-packages
%\end{verbatim}  
%
%  If this option is used, \janus{} will be avaiable to anyone who uses
%  the version {\tt \$PYTHON}.
%
%\item As a final alternative, the command 
%
%\begin{verbatim}  
%  $XSB_ROOT/XSB/packages/xsbpy/configure ???
%\end{verbatim}
%
%will install \janus{} directly in the specified directory.
%\end{itemize}
%
%{\sc TES: If possible, I need to redo setup.py to include px.py in the
%  egg so that px.py does not need to be in the library path.}

Once \janus{} has been built and activated, \janus{} can be used just
as any other package installed as a personal or site package for {\tt
  \$\{PYTHON\}}.

\begin{verbatim}
$PYTHON
>>> import janus as jns
[xsb_configuration loaded]
[sysinitrc loaded]
[xsbbrat loaded]
[janus loaded, cpu time used: 0.001 seconds]
janus_initted_with_python(auto(python3.11))
[janus_py loaded]
\end{verbatim}

\noindent
Note that XSB is initialized in the Python process when the \janus{}
module is loaded.

As a final point, you can test \januspy{} by changing directory to

{\tt \$XSB\_ROOT/xsbtests/janus\_tests}

and executing the command {\tt bash test.sh } $< xsb\_executable\_path>$
   
The test script executes a number of \janus{} examples, which may be
useful for trying out various features.

%------------------------------------------------------------

%\noindent
%Once the appropriate packages are present, a virtual environment can
%be set up, usually in a user's home directory via a command like:

%{\tt \$\{PYTHON\} -m venv {\tt \$VENV\_DIR}}

%Next, the virtual environment needs to be activated in a command
%shell,  In {\tt bash} this is done as:

%{\tt source \$VENV\_DIR/bin/activate  }

%In the same shell the \janus{} configuration will be run.  If XSB has
%not already been configured and made, you'll need to do so as
%discussed in Volume 1 of this manual, which will configure \janus{}
%(and {\tt xsbpy}).  Otherwise, {\tt cd} to the {\tt xsbpy}
%subdirectory and run the command
%
%{\tt ./configure}
%%

%\subsection{Configuring on Linux Systems}

%Currently {\tt setup.py} requires include and
%library directories as shown below.  The paths to these directories
%must be changed to include the proper {\tt \$XSB\_ROOT} and, in the
%case of the library directory the path of the XSB binary appropriate
%to your system.  An example is:

%\begin{footnotesize}
%  \begin{verbatim}
%module1 = Extension('xsbext',
%                    include_dirs = ['/usr/include/python3.8',
%                                    '/home/tswift/xsb-code/XSB/emu',
%                                    '/home/tswift/xsb-code/XSB/packages/xsbpy',
%                                    '/home/tswift/xsb-code/XSB/config/x86_64-unknown-linux-gnu'],
%                    libraries = ['xsb'],
%                    library_dirs = ['/home/tswift/xsb-code/XSB/config/x86_64-unknown-linux-gnu/bin'],
%                    sources = ['xsbext.c','xsbpy_common.c'])
%  \end{verbatim}
%\end{footnotesize}
%
%Once {\tt setup.py} is properly configured, the user need only enter a
%python comaand to build and install the module ({\tt sudo} privilege
%may be needed for this).  For the above file the command is:
%
%\begin{verbatim}
%%python3.8 setup.py install
%end{verbatim}
%\noindent
%This command compiles the px sources, builds a Python egg, and
%installs the egg in the Python libary for distributed packages
%(e.g. {\tt dist-packages})

%\subsection{Configuring PX on Windows}

%The {\tt px} configuration is not yet working for Windows.  If you
%have experience in Windows configuration and would like to help,
%please let us know.

