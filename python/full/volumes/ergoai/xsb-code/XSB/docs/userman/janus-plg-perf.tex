
The core \janus{} routines -- {\tt py\_func/[3,4]} and {\tt py\_dot/4}
-- are written almost entirely in C, have shown good performance so
far, and continue to be optimized.  Calling a simple Python function
to increment a number from XSB and then returning the incremented
value to XSB should take about a microsecond or less on a reasonably
fast machine.  Of course, the overhead for passing large terms from
and to Python will be somewhat higher.  For instance, the time to pass
a list of integers from Python to XSB has been timed at about 20
nanoseconds per list element.  Nonetheless, for nearly any practical
application the time to perform useful functionality within Python
will far outweigh any \janus{} overhead.

Apart from system resource limitations, there is virtually no upper
limit on the size of Python structures passed back to Prolog: stress
tests have passed lists of integers of length 100 million from Python
to Prolog without problems. However, it should be noted that \janus{}t
must ensure that XSB's heap is properly expanded before copying a
large structure from Python to XSB, a topic to which we now turn.

\subsection{Memory Management} \label{sec:janus-memory}

\subsubsection{Space Management for XSB's Heap}
Because Python data structures are directly copied onto the XSB heap
stack, the heap serves as a buffer for the return of information from
Python.  XSB currently relies on the Python C API size routine {\tt
  Py\_SIZE} to estimate the size of a structure, since accessing this
routine has a constant-time overhead.\footnote{XSB estimates the size
  multiplying {\tt Py\_SIZE} by a constant factor.}  However, {\tt
  Py\_SIZE} only returns the length of a structure, and not its exact
size.  Accordingly long lists of large structures, heavily nested
dictionaries and other such structures may present a problem.  It
should be noted that when \janus{} is initialized, XSB's default
heap margin is reset to 1 megabyte so that any data structure whose
size is less than 1 megabyte will be copied safely, even if its size
is under-estimated.\footnote{XSB's default heap margin is 64 kbytes.}

Fortunately, this default works for most users.  In using \janus{}
the only times large data structures have proven a problem is when
returning large bulk queries from Elasticsearch, or returning large
sets of ontology instances from Wikidata.  In such a case there are
two options.  First, one may reset the heap margin to an even larger
value (see Volume 1 for details).  Alternately, one may use the option
{\tt sizecheck(true)} for those {\tt py\_func} or {\tt py\_dot} calls that
are expected to return large data structures.  If this option is
specified, the call performs two traversals of the data structure to
be returned: one to determine its size and ensure heap space, and another
to copy the data structure.

\subsection{Python Space Management}

When using Python's C interface, every Python object that is declared
in C must be explicitly released, a requirement that supports Python's
garbage collection.  \janus{} memory usage has been checked using
the {\tt guppy} package, which indicates that there are no memory
leaks.

%\subsection{Allowing Python to Reclaim Space}
%
%TBD: Discuss space issues for Python how they are now addressed and
%how they will be addressed in the future.
