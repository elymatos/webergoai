\section{CDF Components and I/O} \label{sec:components}

\subsection{CDF Components}

Typically, a CDF instance can be partitioned into several separate
cells representing information that arises from different sources.
For instance, in the example from the previous chapter, the taxonomy
shown in Example~\ref{ex:suture1} is largly drawn from the UNSPSC
taxonomy, while the relations and domains of Example~\ref{ex:suture2}
and later are adapted from DLA's FIIG meta-data.  Practical systems
often need to update data from different sources separately, and may
need to incorporate information from one source independantly from
that of another.  The CDF components system attempts to address this
need by allowing ontologies to be built from discrete {\em components}
that can be maintained separately.

First of all, the CDF component system provides an implicit partition
of all facts and rules in a CDF instance.  Recall from
Defintion~\ref{def:cdfids} all identifiers have an outer 2-ary functor
whose component tag or {\em source} is in the second argument.
%-----------------------------
\mycomment{
Thus, {\tt cid(absorbableSuture)} would actually be maintained as {\tt
cid(absorbableSuture,unspsc)}, {\tt rid(needleDesign)} as {\tt
rid(needleDesign,dlafiig)} etc.  }
%
Extensional facts in a CDF instance can be assigned a component by
choosing a {\em component argument} for each predicate symbol, and
assigning as the component of each fact the component tag of the
identifier in the component argument.  If the identifier is a product
identifier, the source is the source of the outer function symbol).
Intensional rules can be assigned a component in a similar way, by
assigning a component argument to each intensional rule predicate
symbol and requiring that the component tag of the component argument
in each intensional rule be ground.

By default, CDF uses a {\em relation-based component system}, which
chooses as argument identifiers

\begin{itemize}
\item the first argument of all \pred{isa\_ext/2} and
\pred{necessCond\_ext/2} facts, along with \pred{isa\_int/2} and
\pred{necessCond\_int/2} rules.  

\item the second argument of all \pred{hasAttr\_ext/3},
\pred{allAttr\_ext/3}, \pred{minAttr\_ext/4}, \pred{maxAttr\_ext/4}
and \pred{classHasAttr\_ext/3} facts and their associated {\tt \_int}
rules.
\end{itemize}

\index{component dependency}
Given such a partition, a dependency relation between components
arises in a natural way.  Let $Id: components.tex,v 1.3 2010-08-19 15:03:38 spyrosh Exp $ range over CDF identifier functors.
A component $C_1$ {\em directly depends} on component $C_2$ if {\em
Id($Nid_2$,$C_2$)} is an argument in a fact or rule head in component
$C_1$.  In addition, $C_1$ durectly depends on $C_2$ if {\em
Id($Nid_p$,$C_1$)} is a product identifier in a component argument,
and {\em Id($Nid_2$,$C_2$)} occurs as an argument in $Nid_p$.  It is
easy to see that component dependency need not be hierarchical so that
two components may directly depend on each other; furthermore each
component must directly depend on itself.  Component dependency is
defined as the transitive closure of direct dependency.

Dependency information is used to determine how to load a component
and when to update it and is usually computed by the CDF system.
Computing dependency information is easy for extensional facts, but
computing dependency information for intensional rules is harder, as
the component system would need to compute all answer substitutions to
determine all dependencies, and this in impractical for some sets of
intensional rules.  Rather, dependencies are computed by checking the
top-level arguments of intensional rules, which leads to an
under-approximation of the dependencies.

\subsubsection{Component Names, Locations, and Versions}

\index{component!name}
\index{component!version}
\index{component!location}
A component is identified by a structured {\em component name}, which
consists of a {\em location} and a {\em source}.  For example,
information in the directory {\tt /home/tswift/unspsc} would have a
location of {\tt /home/tswift} and source {\tt unspsc}.  For
efficiency, only the source is used as a source argument for
identifiers within a CDF instance; the location is maintained
separately.  The structuring of component names has implications for
the behavior of the component system.  If two components with the same
sources and different locations are loaded, facts and rules from the
two different components cannot be distinguished, as only the source
is maintained in identifiers.  The attempt to load two such components
with the same source and different locations can be treated as an
error; or the load can be allowed to succeed unioning the information
from both components, implicitly asserting an axiom of equality for
the two (structured) component names.

The same component name can have multiple {\em versions}.  An CDF
state can contain only one version for each component name, and an
attempt to load two different versions for the same name always gives
rise to an error.  On the other hand, if two component names $C_1$ and
$C_2$ have the same source and different locations, they do not have
the same name.  If they are loaded so that their information is
unioned together an error is not raised if $C_1$ and $C_2$ have
different versions.

\subsection{I/O for CDF}

The component system allows users to create or update components from
a CDF state, as well as load a component along with its dependencies.
In the present version of CDF, it provides a basis for concurrent
editing of ontologies by different users sharing the same file system
(see Section~\ref{sec:concurrency}).  Users can either load the latest
version of a component, or work with previous versions, or move their
work to a separate directory where they can work until they feel it is
time to merge with another branch.  While the current version of CDF
only supports loading and saving components to a file system, work is
underway to load and save components as RDF resources, using
translators between XSB and RuleML \cite{RuleML} \footnote{These
translators were written by Carlos Dam\'asio.}.

At a somewhat lower level, the CDF I/O system allows a user to load or
dump a file of extensional facts or intensional rules  The I/O system
also forms the building blocks of the component system.


%-----------------------------------------------------------------
\subsection{Component and I/O API}

\subsubsection{Component API}

\begin{description}
\ourpredmoditem{load\_component/1}{cdf\_comps\_share}
{\tt load\_component(Name,Loc,Parameter\_list)} loads the component
{\tt Name}, from location {\tt Loc} and recursively, all other
components upon which the component depends.  If a version conflict is
detected between a component tag to be loaded and one already in the
CDF state or about to be loaded, {\tt load\_component/3} aborts
without changing CDF extensional facts or intensional rules.  In
\version{} of CDF, {\tt Loc} must be a file system path, and the rules
for filename expansion of relative and other paths is the same as in
XSB.

The order of loading is as follows.  First, all extensional facts are
loaded for the component {\tt Name} at location {\tt Loc}, and then
recursively for all components on which {\tt Name} at {\tt Loc}
depends.  Next, the dependency graph is re-traversed so that
intensional rules are loaded and initialization files are consulted in
a bottom-up manner (i.e. in a post-order traversal of the dependency
graph).

{\tt Parameter\_list} may contain the following elements:
\begin{itemize}

\item {\tt action(Action)} where {\tt Action} is {\tt check} or
{\tt union}.  If the action is {\tt check}, two components with the
same component but different locations or versions cannot be loaded:
an attempt to do so will cause an error.  If the action is {\tt
union}, two components with the same name and different locations may
be loaded, and the effect will look as if the two components had been
unioned together.  However an error will occur if two components with
the same name and location but different versions are loaded.

\item {\tt force(Bool)} where {\tt Bool} is {\tt yes} or {\tt no} (default
{\tt no}).  If {\tt Force} is {\tt yes}, any components that have
previously been loaded into the CDF are reloaded, and their
initialization files reconsulted.  If {\tt Force} is {\tt no}, no
actions will be taken to load or initialize components already loaded
into the CDF.

\item {\tt version(V)} where {\tt V} is a version number.  If the
parameter list contains such a term, the loader attempts to load
version {\tt V} of component.  The default action is to load the latest
version of a component.
\end{itemize}

\ourpredmoditem{update\_all\_components/2}{cdf\_comps\_noshare}
{\tt update\_all\_components(Loc,Option\_list)} analyzes components of
a CDF state and their dependencies, determining whether they need to
be updated or not, and creating components when necessary.  When a
component is created with location {\tt Loc}, the files {\tt
cdf\_extensional.P} and {\tt cdf\_intensional.P} are created.
Initialization files must be added manually for new components.
Currently, {\tt Dir} must specify a file system directory. {\tt
Option\_list} contains a list of parameters which currently specifies
the effect on previously existing components:

\begin{itemize} 
\item {\tt action(Action)}.  
\begin{itemize} 

\item If {\tt Action} is {\tt create}, then a new set of components is
created in {\tt Loc}.  Information not previously componentized is
added to new components whose location is {\tt Loc} and whose version
number is 0.  Facts that are parts of previously created components
are also written to {\tt Loc}; if their previous location was {\tt
Loc}, their versions are updated if needed (i.e. if any facts or rules
in the component have changed).  Otherwise, if the location of a
previously created component {\tt C} was not {\tt Loc}, {\tt C} is
written to {\tt Loc} its location information updated, and its version
is set to 0.  
%In addition, the initialization file for {\tt C} is
%copied to {\tt Dir}.

\item If {\tt Action} is {\tt in\_place}, then components are created in
{\tt Loc} only for information that was not previously componentized.
Newly created components are written to {\tt Loc} which serves as
their locations, and their version number is 0.  Previously created
components whose locations were not {\tt Loc} are updated using their
present location, if needed.

\end{itemize}
In either case all dependency information reflects new component and
location and version information.
\end{itemize}

\end{description}

\subsubsection{I/O API}

\begin{description}	
\ourpredmoditem{load\_extensional\_facts/1}{cdf\_io}
{\tt load\_extensional\_facts(DirectoryList)}: loads the file
{\tt cdf\_extensional.P} from directories in {\tt DirectoryList}.  The
files loaded must contain extensional data.
\pred{load\_extensional\_facts/1} does not abolish any extsnsional
information already in memory; rather, it merges the information from
the various files with that already loaded.  Intensional rules will
not be affected by this predicate.  

\ourpredmoditem{load\_intensional\_rules/1}{cdf\_io}
{\tt load\_intensional\_rules(Dir)} ynamically loads intensional rules
from {\tt cdf\_intensional.P} in {\tt Directory}.  This predicate is
designed for the component system, but can be used outside of it.  The
leaf directory name in {\tt Dir} is assumed to be the component name
of the rules.  As the intensional rules are loaded, their functors are
rewritten from {\tt XXX\_int} to {\tt XXX\_int\_Name}, to avoid any
conflicts with intensional rules loaded from other components or
directories.

\ourpredmoditem{merge\_intensional\_rules/0}{cdf\_io}
{\tt merge\_intensional\_rules/0}: This utility predicate takes the
current intensional rules for all sources and transforms them to
extensional form by backtracking through them, and asserting them to
the Prolog store.  All intensional information is then retracted.

\ourpredmoditem{dump\_extensional\_facts/1}{cdf\_io}
{\tt dump\_extensional\_facts(Dir)} writes extensional facts to the
file {\tt cdf\_extensional.P} in {\tt Directory}.  No intensional
rules are dumped by this predicate.
\pred{dump\_extensional\_facts/0} writes the {\tt cdf\_extensional.P} file
to the current directory. 

\ourpredmoditem{dump\_intensional\_rules/1}{cdf\_io}

\ourpredmoditem{cdf\_exists/1}{cdf\_io}
{\tt cdf\_exists(Dir)} checks whether {\tt cdf\_extensional.P} file is
present in directory {\tt Dir}.

\end{description}
