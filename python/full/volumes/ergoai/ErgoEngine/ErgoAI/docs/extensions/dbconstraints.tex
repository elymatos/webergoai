
\section{Integrity Constraints}
\label{sec-kb-constraints}

\index{constraint!automatically maintained}
\index{constraint!integrity}
\index{integrity constraint}
Database systems have a very useful notion, called \emph{integrity constraint}. 
An integrity constraint is a logical statement that specifies a condition
that must be maintained at all times regardless of the changes made to the
knowledge base. We have seen two examples of integrity constraints in
\FLSYSTEM: type constraints and cardinality constraints. These particular
constraints are dealt with in greater detail in
Sections~\ref{sec-typechecking} and \ref{sec-cardinality}, but those
sections deal with
\emph{type-checking}, i.e., with verifying that the knowledge base
satisfies  typing and cardinality constraints and with finding out what violates
them. The primitives described in those sections must be invoked by
the user \emph{explicitly} each time a type-check is desired and then the
knowledge base must be repaired manually.
In this section, however, we are concerned with \emph{automatic} maintenance of
constraints: timely warning the user when a violation occurs in the cause of an
update transaction and, if possible, rolling back the violating
transaction. The primitives introduced in this section can be used in
tandem with Sections~\ref{sec-typechecking} and \ref{sec-cardinality}
to enable automatic maintenance of typing and cardinality constraints.

Importantly, integrity constraints are \emph{not} to be confused with
\emph{constraint solving} described in Section~\ref{sec-clp}.

Conceptually, an integrity constraint is a query that is supposed
to be \emph{false} at all times and this fact is ascertained automatically
after every transaction in \ERGO. If the query is true for some bindings of its
arguments, each of those bindings should be viewed as  a witness of an
integrity violation.
(Yes, normally people think of constraints as something that \emph{must}
always be true, not as something that must always be false, but
constraints-as-false-statements---i.e., requiring that something must \emph{not}
happen---are more useful, since this produces the culprits
responsible for constraint violation.)

To define an integrity constraint, one must first define a suitable 
predicate or a frame to be used as a constraint.
For instance, suppose that it is not allowed to both receive a salary from a
company and also be a consultant for it:
%% 
\begin{verbatim}
   // A Conflict Of Interest (COI) constraint:
   //    it should NOT be the case that, for some binding p for ?Person,
   //    some binding c for ?Company, and some binding s for ? (the salary
   //    argument) both salary(p,c,s) and consults(p,c) are true.
   COI(?Pers,?Company) :- salary(?Pers,?Company,?), consults(?Pers,?Company).
\end{verbatim}
%% 
If it so happens that someone called John both works and consults for
Acme, Inc., i.e., that something like \texttt{salary(John,Acme,100000)}  
and \texttt{consults(John,Acme)} both become true, we would like to know
this immediately and, if possible, the transaction that introduced this
violation should be rolled back.

\index{constraint\{...\} primitive}
\index{primitive!constraint\{...\}}
\index{constraint\{...\}!adding}
If automatic maintenance of this constraint is what we want, the second step
is to tell \ERGO about this want by executing this command:
%% 
\begin{verbatim}
   ?-  +constraint{COI(?,?)}.  // activate constraint
\end{verbatim}
%% 
The plus sign says that we want to \emph{activate} the aforesaid query as an
integrity constraint; the minus sign is used when we want to
\emph{deactivate}  a constraint:
\index{constraint\{...\}!deleting}
%% 
\begin{verbatim}
   ?-  -constraint{COI(?,?)}.  // deactivate constraint
\end{verbatim}
%% 
The first argument of \texttt{constraint\{...\}} must be a HiLog predicate
or a frame. It can also be \texttt{\bs{}neg} of a predicate or a frame, but
it cannot be a \texttt{\bs{}naf}, conjunction, disjunction, or any kind of
more complex subgoals.

Once a constraint is activated, it is first checked against the current
knowledge base. If it is not violated (i.e., the constraint
query is false), everything is good. Otherwise, a
warning is issued. For instance, suppose John and Mary \emph{do}
work and also consult for
Acme. Then the following warning will be triggered after the above
\texttt{+constraint\{...\}} statement is executed: 
%% 
\begin{verbatim}
   *** A violation of the constraint activated on line 5 in file mykb.ergo 
       existed prior to adding this constraint
       The offending instances of the constraint are:
           COI(John,Acme)
           COI(Mary,Acme)
\end{verbatim}
%% 
In this case, the violation existed prior to the moment the constraint was
activated, so there is little one can do automatically. Although a number of
theoretical
approaches to resolving such situations exist, their practical value is
questionable.  The practical approach offered by \ERGO is to use Studio and
ask for an explanation for each of the instances of the query that
indicates a violation. Then one can decide what to do (perhaps the
fact that John consults for Acme was supposed to be, but was not,
deleted once he was hired by Acme as an employee).
If, on the other hand, it is \emph{known} how to resolve
a particular constraint automatically then constraints with callbacks
(described later) might be the way to go.
Another way to resolve constraints automatically is to \emph{always} use
transactional updates (Section~\ref{sec-trans-updates}), as shown next. 

Suppose the knowledge base has only the fact
\texttt{consults(Bob,GroupLTD)}
but not any fact of the form \texttt{salary(Bob,GroupLTD,...)}, i.e.,
\texttt{mykb.ergo} is like this:
%% 
\begin{verbatim}
   COI(?Pers,?Company) :- salary(?Pers,?Company,?), consults(?Pers,?Company).
   consults(Bob,GroupLTD).
   ?-  +constraint{COI(?,?)}.
\end{verbatim}
%% 
Suppose that next we execute the transaction
%% 
\begin{verbatim}
   ?- tinsert{salary(Bob,GroupLTD,200000)}.
\end{verbatim}
%% 
If this insertion is to be performed, the knowledge base would get a
constraint violation
and so the following warning is going to be issued:
%% 
\begin{verbatim}
*** A violation of the constraint activated on line 3 in file mykb.ergo
    is detected after the transaction t_insert{salary(Bob,GroupLTD,100000)}
    The offending instances of the constraint are:
        COI(Bob,GroupLTD)
\end{verbatim}
%% 
\index{tinsert}
\index{t\_insert}
%%
If we now check the query \texttt{salary(Bob,GroupLTD,?Salary)} then we will
see that it has \emph{no} answers,
i.e., the offending insertion was rolled back!
This magic happened because we used \emph{transactional} insertion,
\texttt{tinsert} (or, synonymously, \texttt{t\_insert}) --- see
Section~\ref{sec-trans-updates} for the details.
If we used  a \emph{non}-transactional insertion, such as 
\texttt{insert\{salary(Bob,GroupLTD,100000)\}}, then the above warning would
also appear, but the insertion would stick and the user would
have to resolve the issue manually afterwards.

If a constraint is no longer needed, it can be deactivated:
%% 
\begin{verbatim}
   ?- -constraint{COI(?,?)}.
\end{verbatim}
%% 
After this, John, Bob, and Mary can both work and consult for Acme and
other companies at the same time---with no conflict of interest questions
asked.

\paragraph{Unresolved constraint violations.}
Note that if constraint violations are not resolved, \ERGO will keep
reminding about them when new update transactions are executed whether
these transactions add new violations or not.
In contrast, queries that produce no change to the underlying state of the
knowledge
base will \emph{not} trigger such warnings, even if violations exist
prior to the query execution---to reduce
the annoyance factor.

For instance, suppose that Mary and John violate the conflict of interests
constraint and a new transaction added a COI for Bob with respect to
GroupLTD. Then the warning will be
%% 
\begin{verbatim}
       ... ... ...
       The offending instances of the constraint are:
           COI(John,Acme)
           COI(Mary,Acme)
           COI(Bob,GroupLTD)
\end{verbatim}
%% 
If the user does not want to be reminded about the old violations then,
perhaps, an integrity constraint is not the right tool for the situation at
hand. After all, what is the purpose of an integrity \emph{constraint} if
the violations of the constraint are not being resolved?
To address this type situations, \ERGO provides a different mechanism,
called \emph{truth}-type \emph{alerts}, described in Section~\ref{sec-alerts}.
This is a more computationally expensive mechanism than constraints and so
it should be used only if resolving constraint violations is not always
desirable or when the conditions being checked are not conceptualized as
integrity constraints that must not be violated.


\index{callback!in integrity constraint}
\index{integrity constraint!with callback}
\paragraph{Constraints with callbacks.}
Sometimes, the knowledge engineer might need to take a programmatic action
in response to a constraint violation. In most of the
previous examples, constraint
violations were reported in print and manual intervention was needed
to take a corrective action. To let the user specify programmatic
corrective actions, \ERGO provides a 2-argument version of
the \texttt{+constraint\{...\}} primitive:
%% 
\begin{alltt}
     +constraint\{\textnormal{\emph{ConstraintQuery},\emph{CallbackPredicate}}\}.
\end{alltt}
%% 
The callback predicate must be a HiLog predicate with at least one
argument; it can be also a \texttt{\bs{}neg} of such a predicate. 


If a callback is provided, then constraint-violating alerts won't be
printed and \textit{CallbackPredicate} is called instead. 
If a violation is detected,
that first argument will be bound to the list of
violating instances of the constraint, so the callback could take a
corrective action. For instance, for our conflict-of-interests constraint, we could
define the following corrective action:
%% 
\begin{verbatim}
   resolve_COI(?Violators,?File,?Line) :-
       deleteall{?Y| COI(?Empl,?Comp) \in ?Violators, ?Y=consults(?Empl,?Comp)},
       (
           write('Violation of constraint in file '), write(File),
           write(' on line '), write(Line), write(' was detected: '),
           writeln(?Violators).
           write('Corrective action was deletion of '),
           write(consults(?Person,?Company)),
           writeln(' for all violated instances of the constraint')
       )@\io.
\end{verbatim}
%% 
which deletes the facts that cause the violation and then
prints a message about what was done to correct the violation. (In
databases, this is called a ``compensating'' transaction.)
To tell \ERGO that this corrective action is to be taken, the constraint
must be activated using the 2-argument version of
\texttt{+constraint\{...\}}  
instead of the 1-argument version used before:
%% 
\begin{verbatim}
   ?- +constraint{COI(?,?), resolve_COI(?,\@F,\@L}.
\end{verbatim}
%% 
We see that the first argument (the variable ?) is left unbound because it will be bound
by the system to the list of violating instances of the constraint; arguments 2
and 3 are instantiated with the quazi-constants \texttt{\bs{}@F} and
\texttt{\bs{}@L}. To remind (from Section~\ref{sec-quasi-constants}),
the first quazi-constant is replaced by the
compiler with the name of the file in which the above
\texttt{+constraint\{...\}} statement appears and \texttt{\bs{}@L} is
replaced with the appropriate line number. This is how the callback
in our example gets its file and line arguments.

Finally, in the \texttt{-constraint\{...\}} statement, the second argument
is checked for syntax, but is ignored. Therefore,
it makes little sense to provide a second argument here---regardless of whether the constraint being deleted does or does not use a
callback.

\paragraph{Performance considerations.}
One should keep in mind that an active constraint injects a query (to find
all violating instances) after each top-level query, which cannot but
affect the performance. Therefore, one should strive to keep the number of
active constraints to a minimum and to design the constraints to be
lightweight. It would be an understatement to say that,
if a constraint check goes into an
infinite loop, the resulting problem would be very hard to figure out.



%%% Local Variables: 
%%% mode: latex
%%% TeX-master: "../ergo-manual"
%%% End: 
