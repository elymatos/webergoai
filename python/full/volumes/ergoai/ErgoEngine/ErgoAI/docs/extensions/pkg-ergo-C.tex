\chapter[C and C++ Interface to \FLSYSTEM]{Querying
  Knowledge Bases from C and C++
  Programs\\{\Large by Michael Kifer}}

This API enables one to start a \FLSYSTEM session from within a C or C++
program, load knowledge bases, and query them via that API.
The details appear in the document
``\emph{About Querying Ergo from C and C++}'' in the \ERGOAI \emph{Example Bank}
\url{https://drive.google.com/open?id=1bfoj-yegVdpBnJ6BfgKqKW6p-qmlhfkwU2p0oBYf2oU}.
A well-documented running example is also provided in that example bank:
\url{https://drive.google.com/open?id=1Mux-wkYRXwCV1B9ZDOO01xwFxEGzGnYq}.
\ifdef{\isfloraman}{
Although the document and the program refer to \ERGO, they equally apply to
\FLORA, if \texttt{ergo\_query()} there is replaced with
\texttt{flora\_query()}.   
}

The main issues one should to pay attention to are: 
%% 
\begin{itemize}
\item  Compilation of C/C++ programs that invoke \FLSYSTEM. The process is
  different for Windows, Linux, and Mac.
\item Writing C/C++ programs that interface with \FLSYSTEM.
\end{itemize}
%% 
The last aspect has four major steps:
%% 
\begin{itemize}
\item  Initialization of XSB.
\item  Initialization of \FLSYSTEM.
\item  Issuing queries to \FLSYSTEM and getting the results.
\item  Error handling.
\end{itemize}
%% 

All of these issues are detailed in the aforesaid document and the
accompanying sample program.


%%% Local Variables: 
%%% mode: latex
%%% TeX-master: "ergo-packages"
%%% End: 
