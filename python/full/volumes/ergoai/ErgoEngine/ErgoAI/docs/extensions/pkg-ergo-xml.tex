\chapter[SGML and XML Parser for \FLSYSTEM]
{SGML and XML Import for \FLSYSTEM\\
  {\Large by Rohan Shirwaikar and Michael Kifer}}

This chapter documents the \FLSYSTEM package that provides XML
and XPath
parsing capabilities. The main predicates support parsing
SGML, XML, and HTML documents, and create \FLSYSTEM objects in the user
specified module.
Other predicates evaluate XPath queries
on XML documents and create \FLSYSTEM objects in user specified
modules. The predicates make use of the {\bf sgml} and {\bf xpath}
packages of XSB.



\section{Introduction}\label{sec-xml-intro}

This package supports parsing SGML, XML, and HTML documents,
converting them to sets of \FLSYSTEM objects stored in user-specified
\FLSYSTEM modules. The SGML interface
provides facilities to parse input in the form of files,
URLs and strings (Prolog atoms).  

For example, the following XML snippet

\begin{verbatim}
<greeting id='1'>
    <first ssn='111'>
        John
    </first>
</greeting>
\end{verbatim}

will be converted into the following \FLSYSTEM objects:

\begin{verbatim}
obj1[ greeting -> obj2]
obj2[ attribute(id) -> '1']
obj2[ first -> obj3]
obj3[ attribute(ssn) -> '111']
obj3[ \text -> 'John']
\end{verbatim}

To load the {\tt XML} package, just call any of the API calls
at the \FLSYSTEM prompt.

The following calls are provided by the package.  They
take SGML, XML, HTML, or XHTML documents and create the corresponding
\FLSYSTEM objects as specified in Section~\ref{xml-to-flora}.

\begin{description}
\item[{\tt ?InDoc[load\_sgml(?Module) -> ?Warn]}@\bs{}xml]~~\\
  Import XML data as \FLSYSTEM objects.
\item[{\tt ?InDoc[load\_xml(?Module) -> ?Warn]}@\bs{}xml]~~\\
  Import SGML data as \FLSYSTEM objects.
\item[{\tt ?InDoc[load\_html(?Module) -> ?Warn]}@\bs{}xml]~~\\
  Import HTML data as \FLSYSTEM objects.
\item[{\tt ?InDoc[load\_xhtml(?Module) -> ?Warn]}@\bs{}xml]~~\\
  Import XHTML as \FLSYSTEM objects.
\end{description}
%%
The arguments to these predicates have the following meaning:

{\tt ?InDoc} is an input SGML, XML, HTML, or XHTML document.
It must have one of these forms: {\tt url({\textnormal{'\emph{url}'}})},
{\tt file(\textnormal{'\emph{file name}'})} or {\tt
  string(\textnormal{'\emph{document as a Prolog atom}}')}.  If \texttt{?InDoc}
is just a plain Prolog atom (\FLSYSTEM symbol) then \texttt{file(?Source)}
is assumed.  
{\tt ?Module} is the name of the \FLSYSTEM module where the objects created
by the above calls  should be placed; it must be bound.
{\tt ?Warn} gets bound to a list of warnings, if any are generated, or to
an empty list; it is an output variable.

  
\section{Import Modes for XML in Ergo}

XML can be imported into \FLSYSTEM in several different ways, which can be
specified via the \texttt{set\_mode(...)@\bs{}xml}  primitive. These modes
control two aspects of the import:
%% 
\begin{itemize}
\item  white space handling, and
\item  navigation links that may be added to the imported data.
\end{itemize}
%% 

\subsection{White Space Handling}

The XML standard requires that white space (blanks, tabs, newlines, etc.)
must be preserved by XML parsers. However, in the applications
where \FLSYSTEM is used, XML typically is viewed as a format for data in
which white space is immaterial. For that reason, by default, the
\FLSYSTEM's XML parser operates in the \emph{data mode} in which every
string is trimmed on both sides to remove the white space. In addition, the
empty strings \texttt{''} are ignored.    
This implies that, for example, there will be no \texttt{\bs{}text}
attribute to represent a situation like this:
%% 
\begin{verbatim}
<doc>
    <spaceonly>   </spaceonly>
</doc>
\end{verbatim}
%% 
and the only data created to represent the above document will be 
%% 
\begin{verbatim}
obj1[doc->obj2]@bar
obj2[spaceonly->obj3]@bar
\end{verbatim}
%% 
(plus some additional navigational data about order, siblings, parents,
etc.).
This means that, if capturing certain white space is needed, it should be
encoded explicitly in some way, e.g., 
%% 
\begin{verbatim}
<spaceonly>___</spaceonly>
\end{verbatim}
%% 
instead of three spaces.

Alternatively, one can request to change the XML parsing mode
to \emph{raw}:
%% 
\begin{verbatim}
   ?- set_mode(raw)@\xml.
\end{verbatim}
%% 
In this case, the parser will switch to the pedantic way XML parsers are
supposed to interpret XML and all white space will be
preserved. However, beware what you wish because even for the above tiny
example the representation will end up not pretty because every little bit
of white space will be there (even the one that comes from line breaks):
%% 
\begin{verbatim}
obj1[doc->obj2]@bar
obj2[\text->obj3]@bar
obj2[\text->obj6]@bar
obj2[spaceonly->obj4]@bar
obj3[\string->'
  ']@bar
obj4[\text->obj5]@bar
obj5[\string->'   ']@bar
obj6[\string->'
']@bar
\end{verbatim}
%% 
It is more than likely an \FLSYSTEM user will not want objects like
\texttt{obj3} and \texttt{obj6}.  

Finally, if the \emph{raw} mode is not what is desired, one can always
switch back to the data mode: 
%% 
\begin{verbatim}
   ?- set_mode(data)@\xml.
\end{verbatim}
%% 

\subsection{Requesting Navigation Links}

This aspect can be changed via the calls
%% 
\begin{verbatim}
   ?- set_mode(nonavlinks)@\xml.  // the default
   ?- set_mode(navlinks)@\xml.
\end{verbatim}
%% 
where \texttt{nonavlinks} is the default.

The \texttt{nonavlinks} method uses a slightly simpler translation from XML
to \FLSYSTEM objects and no extra navigation links are provided.
This mode is used when the imported XML document has known tructure
and is viewed simply as
set of data to be ingested (e.g., payroll data).

In the \texttt{navlinks} mode, the representation is slightly more complex
but, most importantly, that imported data includes additional information 
that provides parent/child/sibling links among XML objects as well as the
ordering information, which allows one to reconstruct the original XML
document.  This mode is used when the structure
of the input XML has high variability or may even be arbitrary. This
arises, for instance, when one needs to transform arbitrary XML import or
to extract certain information from unknown structures.
The exact representation of this navigational information is described in 
subsequent sections.


%%\section{Summary of the Predicates}
%%
%%\begin{longtable}[l]{lp{6cm}}
%%{\tt ?Doc[load\_xml(?Module)->?Warn]}@\bs{}xml&import XML data as \FLSYSTEM objects\\
%%{\tt ?Doc[load\_sgml(?Module)->?Warn]}@\bs{}xml&import SGML data as \FLSYSTEM objects\\
%%{\tt ?Doc[load\_html(?Module)->?Warn]}@\bs{}xml&import HTML data as \FLSYSTEM objects\\
%%{\tt ?Doc[load\_xhtml(?Module)->?Warn]}@\bs{}xml&import XHTML as \FLSYSTEM objects\\
%%\end{longtable}


\section{Mapping XML to \FLSYSTEM Objects}\label{xml-to-flora}

This mapping is based on an XML-to-\FLORA object correspondence developed
by Guizhen Yang.
It specifies how an XML parser can construct the
corresponding F-logic objects after parsing an input XML document.
The basic ideas are as follows:
\begin{itemize}
\item XML elements, attribute values,
  and text strings are modeled as objects in F-logic.

\item XML elements are reachable from parent objects
  via F-logic frame attributes of the same
  name as the XML element name.

\item XML element attributes are also modeled as F-logic  frame attributes
    but their name is
    \texttt{attribute(\textnormal{\emph{XML attribute name}})}.
\end{itemize}

This mapping does not address 
comments or processing instructions---they are simply ignored.
However, this mapping does address the issue of mixed
text/element content in which plain text and subelements are interspersed.
This mapping also assumes that XML entities are
resolved by the XML parser.



\subsection{Invention of Object Ids for XML Elements}

According to the XML specification 1.0, an XML element can be identified
by an oid that is unique across the document. The import mechanism invents
such an oid automatically.
Sitting on top of the XML root element, there is an additional root
object which just functions as the access point to the entire object
hierarchy corresponding to the XML document.
The oids of leaf nodes, which have no outgoing arcs and carry
plain text only, are just the string values themselves.

For example, the following XML document

%%
\begin{verbatim}
<?xml version="1.0"?>
<person ssn="111-22-3333">
  <name first="John" 
        last="Smith"/>
</person>
\end{verbatim}
%%
is represented via the following F-logic objects:
%%
\begin{verbatim}
    obj1[person -> obj2].
    obj2[attribute(ssn) -> '111-22-3333', name -> obj3].
    obj3[attribute(first) -> John, attribute(last) -> Smith].
\end{verbatim}
%%
Here \texttt{obj1}  is the root object, \texttt{obj2} is the object
corresponding to the \texttt{person} element, and \texttt{obj3} is the
object that represents the \texttt{name} element. The strings
\texttt{'111-22-3333'}, \texttt{John}, and \texttt{Smith} are oids that
stand for themselves.       


\subsection{Text and Mixed Element Content}

The content of an XML element may consist of plain text, or
subelements interspersed with plain text as in
%%
\begin{verbatim}
  <greeting>Hi! My name is <first>John</first><last>Smith</last>.</greeting> 
\end{verbatim}
%%
How text is actually handled in the translation to F-logic depends on the
mode of import: \texttt{nonavlinks} or \texttt{navlinks}. The former is
simpler because it discards all the information about the order of the text
nodes with respect to subelements and other text nodes.
%% 
\begin{itemize}
\item  \textbf{In the} \texttt{nonavlinks} \textbf{mode}:\\
  Each text segment is modeled as a value of the attribute
  \texttt{\bs{}text} of the parent element-object of that text
  segment.\footnote{
    Of course, XML does not allow such names for tags and attributes, and this
    is the whole point: adding such an invented name to the F-logic
    translation  will not clash with other
    tag names that might be used in the XML documents.
    }
  Thus, for the above XML fragment, the translation would be
  %% 
\begin{verbatim}
   obj1[greeting -> obj2].
   obj2[\text -> {'Hi! My name is ', '.'},
        first -> obj3,
        last  -> obj4
   ].
   obj3[\text -> John].
   obj4[\text -> Smith].
\end{verbatim}
  %% 
\item \textbf{In the} \texttt{navlinks} \textbf{mode}:\\
  Here the order of the text and subelement nodes must be preserved and so
  each text node is modeled as if it were a value of a special attribute
  \texttt{\bs{}string} in an empty XML element named \bs{}text, e.g., 
  %% 
\begin{verbatim}
   <\text \string="John"/>  
\end{verbatim}
  %% 
  As a consequence, a separate F-logic object is created to represent each
  text segment. (Compare this to the translation in the \texttt{nonavlinks}
  mode, which does not create separate objects for text nodes.) 
  Thus, for the aforesaid \texttt{greetings} element the translation will be 
\begin{verbatim}
   obj1[greeting -> obj2].
   obj2[\text -> {obj3, obj8},
        first -> obj4,
        last  -> obj6
   ].
   obj3[\string -> 'Hi! My name is '].
   obj4[\text -> obj5].
   obj5[\string -> John].
   obj6[\text -> obj7].
   obj7[\string -> Smith].
   obj8[\string -> '.'].
\end{verbatim}
  How exactly the aforesaid order is preserved in the \texttt{navlinks} 
  mode is explained later.
\end{itemize}
%% 

\subsection{Translation of XML Attributes}

An XML attribute, \emph{attr},  in an element is translated as an attribute
by the name \texttt{attribute}(\emph{attr}) attached to the object that
``owns'' that element.  

XML attributes of type IDREFS are multivalued, in the sense
that their value is a string consisting of one or more oids separated
by whitespaces. Therefore, the value of such an attribute is a set.
The value of an XML IDREFS
attribute is represented as a list.

For example, the following XML segment:
%%
\begin{verbatim}
   <paper id="yk00" references="klw95 ckw91">
     <title>paper title</title>
   </paper>
\end{verbatim}
%%
will generate the following F-logic atoms, assuming that the
{\tt reference}  attribute is of type IDREFS:
%%
\begin{verbatim}
   obj1[paper -> obj2]
   obj2[title -> obj4]
   obj2[attribute(id) -> yk00]
   obj2[attribute(references) -> 'klw95 ckw91'
   obj4[\text -> obj5]      // here we assume that the navlinks mode was used
   obj5[\string -> 'paper title']
\end{verbatim}
%%

However: if the document has an associated DTD \emph{and} the attribute
\texttt{references} were specified there as \texttt{IDREFS} as in
%% 
\begin{verbatim}
    <!ATTLIST paper references IDREFS #IMPLIED> 
\end{verbatim}
%% 
then that attribute is translated as
%% 
\begin{verbatim}
   obj2[attribute(references)->[klw95,ckw91]] 
\end{verbatim}
%% 
i.e., the value becomes a list.

With this, we are done describing the \texttt{nonavlinks} mode. The
remaining subsections in the current section apply to the
\texttt{navlinks} mode only.  


\subsection{Ordering}

This section applies to the \texttt{navlinks} mode only. 

XML is order-sensitive and the order in which elements and text appear is
significant, in general. The order of the attributes within the same
element tag is \emph{not} significant, however. 

While the \texttt{nonavlinks} mode is sufficient for most data-intensive
uses of XML in \FLSYSTEM, more complex tasks may require the knowledge of
how items are ordered within XML documents. 
Specifying a total order among the elements and text
in an XML document suffices for that purpose, if this order agrees with the
local order within each element's content.

Consider the following XML document
%%
\begin{verbatim}
<?xml version="1.0"?>
<person ssn="111-22-3333">
  <name>
    <first>John</first> 
    <last>Smith</last>
  </name>
  <email>jsmith@abc.com</email>
</person>
\end{verbatim}
%%
It can be represented by the
tree in Figure \ref{fig-total-order}
in which the parenthesized integers 
show the total order assigned to the F-logic objects.

\begin{figure}[bht]
\begin{verbatim}
                    obj0 (0)
                    |
                    | person
                    |
        id /----  obj1 (1)
          /         ^
         /         / \
111-22-3333  name /   \ email
                 /     \
          (2) obj2     obj7 (7)
               / \        \ 
        first /   \last    \ \text
             /     \        \
      (3) obj3     obj5 (5)  obj8 (8)
            |      |          \
      \text |      | \text     \  \string
            |      |            \
      (4) obj4     obj6 (6)    'jsmith@abc.com'
            |      |
    \string |      | \string
            |      |
           John  Smith
\end{verbatim}
  \caption{Total ordering of the F-logic objects arising from XML ordering}
  \label{fig-total-order}
  \end{figure}

The ordering information that exists in XML documents is captured in F-logic
via a special attribute called
{\tt \bs{}order}, which tells position within the total ordering for each
element and text node. It is for that purpose that text segments are
modeled in the \texttt{navlinks} mode as element-style objects (each
segment having its own oid) and not simply as attributes, as is the case
with the simpler \texttt{nonavlinks} mode. 

%%\begin{verbatim}
%%<bibliography>
%%<paper id="sb97">
%%  <author>
%%    <first>John</first>
%%    <last>Smith</last>
%%  </author>
%%  <author>
%%    <first>David</first>
%%    <last>Brown</last>
%%  </author>                                  (Example 3)
%%</paper>
%%
%%<paper id="s91">
%%  <author>
%%    <first>John</first>
%%    <last>Smith</last>
%%  </author>
%%</paper>
%%</bibliography>
%%\end{verbatim}


\subsection{Additional Attributes and Methods in the
  \texttt{navlinks} Mode }

Since the \texttt{navlinks} mode is intended for applications that need to
navigate from children to parents, to siblings, and more, the importer
adds the following additional attributes and methods to the F-logic objects 
into which XML elements and text are mapped.

\begin{enumerate}
\item {\tt \bs{}in\_arc} \\
For each node, {\tt \bs{}in\_arc}  returns the unordered set of labels of the
arcs pointing to this node, i.e., this node's in-arcs. Roughly,
{\tt \bs{}in\_arc}  is defined as follows:
  %%
  \begin{verbatim}
     ?O[\in_arc -> ?InArc] :- ?[?InArc -> ?O].
  \end{verbatim}
  %%
Note that for a node representing a text segment, the value of
its {\tt \bs{}in\_arc}  attribute is {\tt \bs{}text}. 

\item {\tt \bs{}parent} \\
For each node, {\tt \bs{}parent}  returns the oid of the parent node.

\item {\tt \bs{}leftsibling} \\
For each node, {\tt \bs{}leftsibling}  returns the oid of the node appearing
immediately before the current node. This attribute is not defined
for the nodes without a left sibling.

\item {\tt \bs{}rightsibling} \\
For each node, {\tt \bs{}rightsibling}  returns the oid of the node appearing
immediately after the current node. This attribute is not defined
for the nodes without a right sibling.

\item {\tt \bs{}childcount} \\
For each element node, {\tt \bs{}childcount}  returns the number of the immediate
children of that element,
which includes subelements and text segments.

\item {\tt \bs{}childlist} \\
For each element node, {\tt \bs{}childlist}  returns a list of the
oids of the immediate children (subelements and text segments)
of that element.

\item {\tt \bs{}child(N)} \\
For each node, {\tt \bs{}child(N)} returns the {\tt N}-th child, where {\tt 0
  $\leq$ N < \bs{}childcount}.
Note: the first child is the 0-th child.


\item {\tt \bs{}in\_child\_arc(N)} \\
For each node, {\tt \bs{}in\_child\_arc(N)}  returns the in-arcs of the {\tt N}-th child, where {\tt 0
  $\leq$ N < \bs{}childcount}.
This attribute is defined as follows:
%%
\begin{verbatim}
 ?O[\in_child_arc(?N)->?InArc] :- ?O[\child(?N)->?[\in_arc->?InArc]].
\end{verbatim}
%%
\end{enumerate}

\section{Inspection Predicates}

This section applies both to the \texttt{nonavlinks} mode and the
\texttt{navlinks} mode.  

It is sometimes hard to see which objects have actually been created to
represent an XML document or an element. This is especially true in case of
the \texttt{navlinks} mode, which includes a host of special navigational
attributes. 
The purpose of inspection predicates is to provide a simple way to view the
objects, and they also filter the navigational attributes out.
Consider the document \texttt{foo.xml} below: 
%% 
\begin{verbatim}
   <mydoc id='1'><first ssn='111'>John</first></mydoc>
\end{verbatim}
%% 
Even for such a simple document, the query
%% 
\begin{verbatim}
   ?- 'foo.xml'[load_xml(bar) -> ?W]@\xml.    // load foo.xml into module bar
   ?- ?_X[?_Y->?_Z]@bar, ?Z = ${?_X[?_Y->?_Z]}.  // get all facts
\end{verbatim}
%% 
that asks for all the facts---stored and derived---will yield 56 results in
the \texttt{navlinks} mode,
which is overwhelming to inspect visually. However, the core facts that
describe these objects are only 8, and they can be obtained by asking the
query
%% 
\begin{verbatim}
   ?- bar[show->?P]@\xml.
\end{verbatim}
%% 
One furthermore might want to see the representation of individual 
elements (e.g., element named \texttt{first}):
%% 
\begin{verbatim}
   ?- bar[show(first)->?P]@\xml.
\end{verbatim}
%% 
and this is much more manageable:
%% 
\begin{verbatim}
   ?P = ${obj4[\text->obj5]@bar}
   ?P = ${obj4[attribute(ssn)->'111']@bar}
\end{verbatim}
%% 
or of elements that have particular attributes (\texttt{ssn} in this
example): 
%% 
\begin{verbatim}
   ?- bar[show(attribute(ssn))->?P]@\xml.
\end{verbatim}
%% 
which yields the same result as above (because the element \texttt{first}
has the attribute \texttt{ssn}).  


\section{XPath Support}

The XPath support is based on the XSB {\tt xpath} package, which must be
configured as explained in the XSB manual. This package, in turn, relies on
the XML parser called {\tt libxml2}. It comes with most Linux
distributions and is also available for Windows, MacOS, and other
Unix-based systems from {\tt http://xmlsoft.org}. 
Note that both the library itself and the {\tt .h} files of that
library must be installed.

\textbf{Note}: XPath support does not currently work under Windows 64 bit
due to the fact that we could not produce a working
\texttt{libxml2.lib} file (\texttt{xmlsoft.org} provides
\texttt{linxml2.dll} for Windows 64, but not \texttt{libxml2.lib}).     

The following predicates are provided. They select parts of the input
document using the provided XPath expression and create \FLSYSTEM objects as
specified in Section~\ref{xml-to-flora}. These predicates handle XML, SGML,
HTML, and XHTML, respectively.

%%
\begin{longtable}[l]{lp{6cm}}
{\tt ?InDoc[xpath\_xml(?XPathExp,?NS,?Mod)->?Warn]}&apply XPath expression to an XML
  document and import the result\\
%%{\tt ?InDoc[xpath\_sgml(?XPathExp,?NS,?Mod)->?Warn]}&apply XPath expression
%% to SGML and import the result\\
%%{\tt ?InDoc[xpath\_html(?XPathExp,?NS,?Mod)->?Warn]}&apply XPath expression
%% to HTML and import the result\\
{\tt ?InDoc[xpath\_xhtml(?XPathExp,?NS,?Mod)->?Warn]}&apply XPath expression
   to XHTML and import the result
  \end{longtable}
  %%
The arguments have the following meaning:

{\tt InDoc} specifies the input document; this parameter has the same
format as in Section~\ref{sec-xml-intro}. {\tt ?XPath} is an XPath expression
specified as a Prolog atom. \texttt{?Module} is the module where the resulting
\FLSYSTEM objects should be placed. {\tt ?Module} must be bound. \texttt{?Warn}
gets bound to a list of warnings, if any are generated during the
processing---or to an empty list, if none.

{\tt ?NamespacePrefList} is a string that has the form of a
space separated list of items of the form {\tt
  \emph{prefix} = \emph{namespaceURL}}. This allows one to use namespace
prefixes in the \texttt{?XPath} parameter.
For example if the XPath expression is {\tt '/x:html/x:head/x:meta'}
where {\tt x} 
stands for {\tt 'http://www.w3.org/1999/xhtml'}, then this
prefix would have to be
defined in {\tt ?NamespacePrefList}:

%%
\begin{alltt}	
    url('http://w3.org')[xpath_xhtml('/x:html/x:head/x:meta',
                                     'x=http://www.w3.org/1999/xhtml',
                                     foomodule)
                         -> ?Warnings]@\bs{}xml.
\end{alltt}
%%

\section{Low-level Predicates}

This section describes low-level predicates in the XML package. These
predicates parse the input documents into Prolog terms that then must be
further traversed recursively in order to get the desired information.
%% 
\begin{itemize}
\item
  \texttt{parse\_structure(?InDoc,?InType,?Warnings,?ParsedDoc)@\bs{}xml}
  --- take the document \texttt{?InDoc} or type \texttt{?InType}
  (\texttt{xml}, \texttt{xhtml},  \texttt{html},  \texttt{sgml}) and parse
  it as a Prolog term (will not be imported 
  into any module as an object).
\item
  \texttt{apply\_xpath(?InDoc,?InType,?XPathExp,?NamespacePrefList,?Warnings,?ParsedDoc)@\bs{}xml}
  --- like the above, but first applies the XPath expression
  \texttt{?XPathExp} to \texttt{?InDoc}. The \texttt{?InType} parameter
  must be bound to \texttt{xml} or \texttt{xhtml}.      
\end{itemize}
%% 
The output, \texttt{?ParsedDoc},  is a Prolog term that represents the
parse of the input XML document in case of \texttt{parse\_structure} and the
result of application of \texttt{?XPathExp} to the input document in case
of \texttt{apply\_xpath}. The format of that parse is described in the 
\emph{XSB Manual, Volume 2: Interfaces and Packages}, in the chapter on
\emph{SGML/XML/HTML Parsers and XPath}.  


%%% Local Variables: 
%%% mode: latex
%%% TeX-master: "../ergo-packages"
%%% End: 
