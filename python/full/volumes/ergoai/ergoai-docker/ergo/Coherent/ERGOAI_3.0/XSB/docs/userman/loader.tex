The dynamic (or automatic) loader comprises one of \ourprolog's 
differences from other Prolog systems.
In \ourprolog, the loading of user modules Prolog libraries (including
the \ourprolog\ compiler itself) is delayed until predicates in them
are actually needed, saving program space for large Prolog
applications.  The delay in the loading is done automatically, unlike
other systems where it must be explicitly specified for non-system
libraries.

When a predicate imported from another module (see section~\ref{Modules})
is called during execution, the dynamic loader is invoked automatically
if the module is not yet loaded into the system, 
The default action of the dynamic loader is to search for the
byte code file of the module 
first in the system library directories (in the order {\tt lib, syslib}, 
and then {\tt cmplib}), and finally in the current working directory.
If the module is found in one of these directories, then it will 
be loaded ({\em on a first-found basis}). Otherwise, an error 
message will be displayed on the current output stream
reporting that the module was not found.

In fact, \ourprolog\ loads the compiler and most system modules this way.
Because of dynamic loading, the time it takes to compile a file 
is slightly longer than usual the first time the compiler is 
invoked in a session.


\subsection{Changing the default Search Path}
%============================================
Users are allowed to supply their own library directories and also to
override the default search path of the dynamic loader. 
User-supplied library directories are searched by the dynamic loader 
{\em before} searching the default library directories.

The default search path of the dynamic loader can easily be changed
by having a file named {\verb|./xsb/xsbrc.P|} in the user's home directory.  
The {\verb|./xsb/xsbrc.P|} file, which is automatically consulted by the
 \ourprolog\ interpreter, must be a module (see section~\ref{Modules}),
and might look like the following:
\begin{verbatim}
             :- export library_directory/1.

             library_directory('./').
             library_directory('~/').
             library_directory('~my_friend/').
             library_directory('/usr/lib/sbprolog/').
\end{verbatim}
(The precise form of the directory names is operating system dependent.)
After loading the module of the above 
example, the current working directory is searched first (as opposed to the
default action of searching it last).  Also, \ourprolog's system
library directories ({\tt lib, syslib}, and {\tt cmplib}), will now
be searched {\em after} searching the user's, {\tt my\_friend}'s 
and the {\tt "/usr/lib/sbprolog/"} directory.

In \version\ the {\verb|./xsb/xsbrc.P|} file {\em must} define and 
export the predicate {\tt library\_directory/1} if it is used, although
in the future {\verb|./xsb/xsbrc.P|} may be used for other purposes as well.

We emphasize that in the presence of a {\verb|./xsb/xsbrc.P|} file
{\em it is the user's responsibility to avoid module name clashes 
with modules in \ourprolog's system library directories}.
Such name clashes can cause the system to behave strangely since these
modules will probably have with semantics different from that expected
by the \ourprolog\ system code.  The list of module names in
\ourprolog's system library directories can be found in
appendix~\ref{module_names}.


\subsection{Dynamically loading predicates in the interpreter}
%=============================================================
Modules are usually loaded into an environment when they are consulted
(see section~\ref{Consulting}).  Specific predicates from a module can
also be imported into the run-time environment through the standard 
predicate {\tt import PredList from Module}\index{{\tt import/1}}.
Here, {\tt PredList} can either be a Prolog list or a comma list.  (The
{\tt import/1} can also be used as a directive in a source module 
(see section~\ref{Modules}).

We provide a sample session for compiling, dynamically loading, and 
querying a user-defined module named {\tt quick\_sort}.
For this example we assume that {\tt quick\_sort} is a file in the 
current working directory, and contains the definitions of the
predicates {\tt concat/3} and {\tt qsort/2}, both of which are exported.

{\footnotesize
\begin{verbatim}
             | ?- compile(quick_sort).
             [Compiling ./quick_sort]
             [quick_sort compiled, cpu time used: 1.439 seconds]

             yes
             | ?- import concat/3, qsort/2 from quick_sort. 

             yes
             | ?- concat([1,3], [2], L), qsort(L, S).

             L = [1,3,2]
             S = [1,2,3]

             yes.
\end{verbatim}
}

The standard predicate {\tt import/1} does not load the module 
containing the imported predicates, but simply informs the system 
where it can find the definition of the predicate when (and if) the
predicate is called.
