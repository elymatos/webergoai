\chapter{Library Utilities} \label{library_utilities}
%====================================================

In this chapter we introduce libraries of some useful predicates that
are supplied with XSB.  Interfaces and more elaborate packages are
documented in later chapters.  These predicates are available only
when imported them from (or explicitly consult) the corresponding
modules.


\section{List (and Tree) Processing}
%========================
The XSB libraries contain various list utilities, some of which are
listed below.  These predicates should be explicitly imported from the
module specified after the skeletal specification of each predicate.
%There are more useful list processing predicates in various modules of
%the XSB system, and the interested user can find them by looking at
%the sources.

\begin{description}
\ourmoditem{append(?List1, ?List2, ?List3)}{append/3}{basics}
    Succeeds if list {\tt List3} is the concatenation of lists 
    {\tt List1} and {\tt List2}.

\ourmoditem{member(?Element, ?List)}{member/2}{basics}
    Checks whether {\tt Element} unifies with any element of list 
    {\tt List}, succeeding more than once if there are multiple 
    such elements.

\ourmoditem{memberchk(?Element, ?List)}{memberchk/2}{basics}
    Similar to {\tt member}/2, except that {\tt memberchk}/2 is
    deterministic, i.e.\ does not succeed more than once for any call.

\ourmoditem{absmember(+Element, +List)}{absmember/2}{listutil}
    Similar to {\tt member}/2, except that it checks for identity
    (through the use of predicate {\tt '=='/2}) rather than unifiability 
    (through {\tt '='/2}) of {\tt Element} with elements of {\tt List}.

% called member2 (ugh) in listutil.    
\ourmoditem{opentail\_member(?Element,?List)}{opentail\_member/2}{basics}
%
Checks whether {\tt Element} unifies with any of the actual elements
of {\tt List}.  The only difference between this predicate and {\tt
  member/2} is on lists having a variable tail,
e.g.\ \verb'[a, b, c | _ ]': while {\tt member/2} would insert {\tt
  Element} at the end of such a list if it did not find it, Predicate
{\tt opentail\_member/2} only checks for membership but does not insert the
{\tt Element} into the list if it is not there.

\ourmoditem{closetail(?List)}{closetail/1}{listutil}
    Predicate {\tt closetail/1} closes the tail of an open-ended list.
    It succeeds only once.

\ourmoditem{last(List,Elt)}{last/2}{lists}
%
Succeeds if {\tt Elt} unifies with the last element of {\tt List}.

\ourmoditem{length(?List, ?Length)}{length/2}{basics}
    Succeeds if the length of the list {\tt List} is {\tt Length}.
    This predicate is deterministic if {\tt List} is instantiated 
    to a list of definite length, but is nondeterministic if 
    {\tt List} is a variable or has a variable tail.  If {\tt List}
    is uninstantiated, it is unified with a list of length {\tt Length}
    that contains variables.

\ourmoditem{same\_length(?List1, ?List2)}{same\_length/2}{lists}
    Succeeds if list {\tt List1} and {\tt List2} are both lists of
    the same number of elements.  No relation between the types or
    values of their elements is implied.  This predicate may be used
    to generate either list (containing variables as elements) given
    the other, or to generate two lists of the same length, in which
    case the arguments will be bound to lists of length $0,1,2,\ldots$.

\ourmoditem{delete(?Element, ?L1, ?L2)}{select/3}{swi} {\tt List2}
derives from {\tt List1} by deleting an {\tt Element}
non-deterministically.

\ourmoditem{reverse(+List, ?ReversedList)}{reverse/2}{basics}
    Succeeds if {\tt ReversedList} is the reverse of list {\tt List}.
    If {\tt List} is not a proper list, {\tt reverse/2} can succeed
    arbitrarily many times.  It works only one way.

\ourmoditem{perm(+List, ?Perm)}{perm/2}{basics}
    Succeeds when {\tt List} and {\tt Perm} are permutations of each
    other.  The main use of {\tt perm/2} is to generate permutations
    of a given list.  {\tt List} must be a proper list.
    {\tt Perm} may be partly instantiated.

\ourmoditem{ith(?Index, ?List, ?Element)}{ith/3}{basics}
    Succeeds if the ${\tt Index^{th}}$ element of the list {\tt List} 
    unifies with {\tt Element}.  Fails if {\tt Index} is not a positive
    integer or greater than the length of {\tt List}.
    Either {\tt Index} and {\tt List}, or {\tt List} and {\tt Element}, 
    should be instantiated (but not necessarily ground) at the time of 
    the call.

\ourmoditem{delete\_ith(+Index, +List, ?Element, ?RestList)}{delete\_ith/4}{listutils}
    Succeeds if the ${\tt Index^{th}}$ element of the list {\tt List}
    unifies with {\tt Element}, and {\tt RestList} is {\tt List} with
    {\tt Element} removed.  Fails if {\tt Index} is not a positive
    integer or greater than the length of {\tt List}.

\ourmoditem{subseq(?Sequence, ?SubSequence, ?Complement)}{subseq/3}{basics}
    Succeeds when {\tt SubSequence} and {\tt Complement} are both
    subsequences of the list {\tt Sequence} (the order of corresponding
    elements being preserved) and every element of {\tt Sequence} which
    is not in {\tt SubSequence} is in the {\tt Complement} and vice
    versa.  That is,
    \[ length({\tt Sequence}) =
                length({\tt SubSequence})+length({\tt Complement}) \]
    for example, {\tt subseq([1,2,3,4], [1,3], [2,4]).}
    The main use of {\tt subseq/3} is to generate subsets and their
    complements together, but can also be used to interleave two lists
    in all possible ways.

\ourmoditem{list\_valcount(+List,?Counts)}{list\_valcount/2}{basics}
%
Counts the occurrences of each unique term in {\tt List} (i.e., unique
via {\tt ==/2}) and returns a sorted list containinng a pair {\tt
  (Term,Count)} for each such term.

As an exmaple, note that while {\tt p(X)} unifies with all ground terms
of the list, it is counted separately.
\begin{verbatim}
?- list_valcount([p(b),p(c),p(X),p(a),p(X),p(c)],Out).

Out = [(p(_h407),2),(p(a),1),(p(b),1),(p(c),2)]
\end{verbatim}

Since comparisons are based on {\tt ==/2} terms with distinct
variables are also counted separately: 
\begin{verbatim}
?- list_valcount([p(b),p(c),p(X),p(a),p(Y),p(c)],Out).

Out = [(p(_h407),1),p(_h483),1),(p(a),1),(p(b),1),(p(c),2)]
\end{verbatim}

\ourmoditem{list\_hist(+List,+Start,+Incr,?Counts)}{list\_hist/4}{basics}
%
Given a list of numbers, produces a partial histogram.  The histogram
begins with {\tt Start} and is divided into intervals of length {\tt
  Incr}, but any intervals with count 0 are omitted.  Its behavior is
perhaps easily seen through an example.
\begin{verbatim}
?- list_hist( [3.4,4,8,3.7,4.5],1,1,Out).

Out = [(3.0,2),(4.0,2),(8,1)]
\end{verbatim}
Because {\tt Start} and {\tt Incr} are both 1, intervals in the
histogram are of length 1 and are aligned on integer values.  There
are 2 values in the interval beginning with 3.0, 2 in the interval
beginning with 4.0 and 1 in the interval beginning with 8.  All other
intervals have count 0 and are omitted.
    
\end{description}

\subsection{Library Predicates for Trees}

The following predicates have a functionality similar to the list
processing predicates described above; however they are based on
simple n-ary trees can be much faster than lists if information about
a large number of elements must be maintained.

\begin{description}

\ourmoditem{log\_ith(?Index, ?Tree, ?Element)}{log\_ith/3}{basics}
    Succeeds if the ${\tt Index^{th}}$ element of the Tree {\tt Tree}
    unifies with {\tt Element}.  Fails if {\tt Index} is not a
    positive integer or greater than the number of elements that can
    be in {\tt Tree}.  Either {\tt Index} and {\tt Tree}, or {\tt
    Tree} and {\tt Element}, should be instantiated (but not
    necessarily ground) at the time of the call.  Tree is a list of
    full binary trees, the first being of depth 0, and each one being
    of depth one greater than its predecessor.  So {\tt log\_ith/3} is
    very similar to {\tt ith/3} except it uses a tree instead of a
    list to obtain log-time access to its elements.

\ourmoditem{log\_ith\_bound(?Index, ?Tree, ?Element)}{log\_ith\_bound/3}{basics}
    is like {\tt log\_ith/3}, but only if the ${\tt Index^{th}}$ element
    of {\tt Tree} is non-variable and equal to Element.  This predicate
    can be used in both directions, and is most useful with Index
    unbound, since it will then bind {\tt Index} and {\tt Element} for
    each non-variable element in {\tt Tree} (in time proportional to
    $N*logN$, for $N$ the number of non-variable entries in {\tt
    Tree}.)

\ourmoditem{log\_ith\_new(?Index, ?Tree, ?Element)}{log\_ith\_new/3}{basics}
    binds {\tt Element} to the ``end'' of the log\_list {\tt Tree},
    and unifies {\tt Index} with its corresponding index.  The ``end''
    of a log\_list is the first element after the one with the largest
    index that has been added.  This can be used to simulate adding an
    element to the end of an open-tailed list, but with better
    complexity.

\ourmoditem{log\_ith\_to\_list(?Tree, ?List)}{log\_ith\_to\_list/2}{basics}
    constructs the {\tt List} that contains all bound values in the
    log\_list {\tt Tree}, in the order they appear in {\tt Tree}.
    
v% TES took these two out -- use ordsets for this.  Removing duplicates between lists and
% keeping them within each list is too funky for a library pred.
%    \ourmoditem{merge(+List1, +List2, ?List3)}{merge/3}{listutil}
%
%Succeeds if {\tt List3} is the list resulting from ``merging'' lists
%{\tt List1} and {\tt List2}, i.e.\ the elements of {\tt List1}
%together with any element of {\tt List2} not occurring in {\tt List1}.
%In contrast to {\tt append/3}, elments in {\tt List2} that occur in
%{\tt List1} are ignored.  However, if~{\tt List1} or~{\tt List2}
%contain duplicates, {\tt List3} may also contain duplicates.

%\ourmoditem{absmerge(+List1, +List2, ?List3)}{absmerge/3}{listutil}
%    Predicate {\tt absmerge/3} is similar to {\tt merge/3}, except that 
%    it uses predicate {\tt absmember/2} described above rather than 
%    {\tt member/2}.

   
\end{description}

\subsection{Processing Comma Lists}
%===========================

It is often useful to process comma lists when meta-interpreting or
preprocessing.  XSB libraries include the following simple utilities.

\begin{description}

\ourmoditem{comma\_to\_list(+CommaList,-List)}{comma\_to\_list/2}{basics}

%
Transforms {\tt CommaList} to {\tt List}.

\ourrepeatmoditem{comma\_append(?CL1,?CL2,?CL3)}{comma\_append/3}{basics}
%
\ourrepeatmoditem{comma\_length(?CommaList,?Length)}{comma\_length/2}{basics}

%
\ourrepeatmoditem{comma\_member(?Element,?CommaList)}{comma\_member/2}{basics}

%
\ourmoditem{comma\_memberchk(?Element,?CommaList)}{comma\_memberchk/2}{basics}
%
Analogues for comma lists of {\tt append/3}, {\tt length/3}, {\tt member/2} and {\tt
memberchk/2}, respectively.

\end{description}
	
\section{Attributed Variables} \label{sec:attributed-variables}
%=============================

\index{Prologs!hProlog}
\index{Prologs!SWI}
%
Attributed variables are a special data type that associates variables
with arbitrary attributes as well as supports extensible unification.
Attributed variables have proven to be a flexible and powerful
mechanism to extend a classic logic programming system with the
ability of constraint solving.  Our low-level API for constraints
closely resembles that of hProlog~\cite{hprolog} and
SWI~\cite{SWI-manual}.

\subsection{Low-level Interface}
%
Attributes of variables are pairs of attribute module names and
values.  An attribute module name can be any atom.  A value can be any
XSB value (term, variable, atom, \ldots). Any variable has at most one
attribute for a particular attribute module.  Attribute modules are
distinct from XSB modules: although it is most efficient to keep each
handlers for each attribute module in their own XSB module.

Attributes can be manipulated with the following three predicates
(\texttt{get\_attr/3}, \texttt{put\_attr/3} and \texttt{del\_attr/2})
defined in the module \texttt{machine}.

\begin{description}
\ourmoditem{get\_attr(-Var,+Mod, ?Val)}{get\_attr/3}{machine}
%
Gets the value of the attribute of \texttt{Var} in attribute module
\texttt{Mod}. Non-variable terms in \texttt{Var} cause a type
error. \texttt{Val} will be unified with the value of the attribute,
if it exists. Otherwise the predicate fails.

\ourmoditem{put\_attr(-Var,+Mod, ?Val)}{put\_attr/3}{machine}
%
Sets the value of the attribute of \texttt{Var} in attribute module
\texttt{Mod}. Non-variable terms in \texttt{Var} cause a type
error. The previous value of the attribute is overwritten, if it
exists.

\ourmoditem{del\_attr(-Var,  +Mod)}{del\_attr/2}{machine} 
%
Removes the attribute of \texttt{Var} in attribute module
\texttt{Mod}. Non-variable terms in \texttt{Var} cause a type
error. The previous value of the attribute is removed, if it exists.
\end{description}

One has to extend the default unification algorithm for used
attributes by installing a handler in the following way:

\noindent
{\tt
:- install\_verify\_attribute\_handler($+Mod, -AttrValue, -Target, +Handler, +WarningFlag$)}
\\
{\tt
  :- install\_verify\_attribute\_handler($+Mod, -AttrValue, -Target, +Handler$)}

\noindent The
predicates
\texttt{install\_verify\_attribute\_handler/5}\index{\texttt  {install\_verify\_attribute\_handler/5}}
and
\texttt{install\_verify\_attribute\_handler/4}\index{\texttt  {install\_verify\_attribute\_handler/4}}
are defined in module \texttt{machine}. $Mod$ is the
attribute {\tt Module} and $Handler$ is a term with arguments
$AttrValue$ and $Target$. The $Handler$ term has to correspond to a
handler predicate that takes the value of the attribute ($AttrValue$)
and the term that the attributed value is bound to ($Target$) as
arguments. The argument $WarningFlag$ in the 5-argument version of the
predicate can be used to suppress the warning issued when replacing the
{\tt verify\_attribute\_handler} for a module. If the argument is
{\tt warning\_on} then the warning is issued if a handler for the module
already exists. Otherwise, the warning is suppressed. 
The 4-argument version of the predicate does \emph{not} suppress the
warning. 

To get good efficiency, it is usually best to keep the
handlers for each attribute module in separate XSB modules.
The handler is called after the unification of an attributed variable
with a term or other attributed variable, if the attributed variable
has an attribute in the corresponding module. The two arguments of the
unification are already bound at the time the handler is called,
i.e. the handler is a post-unify handler.

Here, by giving the implementation of a simple finite domain
constraint solver (see the file \texttt{fd.P} below), we show how
these low-level predicates for attributed variables can be used.  In
this example, an attribute in the module \texttt{fd} is used and the
value of this attribute is a list of terms.

\begin{small}
\begin{verbatim}
%% File: fd.P
%%
%% A simple finite domain constrait solver implemented using the low-level 
%% attributes variables interface.  

:- import put_attr/3, get_attr/3, del_attr/2, 
   install_verify_attribute_handler/4 from machine.
:- import member/2 from basics.

:- install_verify_attribute_handler(fd,AttrValue,Target,fd_handler(AttrValue,Target)).

fd_handler(Da, Target) :-
        (var(Target),                       % Target is an attributed variable
         get_attr(Target, fd, Db) ->        % has a domain
           intersection(Da, Db, [E|Es]),    % intersection not empty
           (Es = [] ->                      % exactly one element
              Target = E                    % bind Var (and Value) to E
           ;  put_attr(Target, fd, [E|Es])  % update Var's (and Value's)
           )
        ;  member(Target, Da)               % is Target a member of Da?
        ).

intersection([], _, []).
intersection([H|T], L2, [H|L3]) :-
        member(H, L2), !,
        intersection(T, L2, L3).
intersection([_|T], L2, L3) :-
        intersection(T, L2, L3).

domain(X, Dom) :- 
        var(Dom), !, 
        get_attr(X, fd, Dom). 
domain(X, List) :- 
        List = [El|Els],                     % at least one element 
        (Els = []                            % exactly one element
         -> X = El                           % implied binding 
        ;  put_attr(Fresh, fd, List),        % create a new attributed variable
           X = Fresh                         % may call verify_attributes/2
        ).

show_domain(X) :-                            % print out the domain of X
        var(X),                              % X must be a variable
        get_attr(X, fd, D),
        write('Domain of '), write(X),
        write(' is '), writeln(D).

\end{verbatim}
\end{small}

When writing or porting a constraint package, it is usually useful to
adjust the way that correct answer substitutions are shown in the
command line.  This can be controlled using the following two predicates:

\begin{description}

\ourmoditem{install\_attribute\_portray\_hook(Module,Attribute,Handler)}{install\_attribute\_portray\_hook/3}{machine}
%
This hook is called by the command-line interpreter when printing out
the value of each variable in a top-level query.  When a printing out
an attributed variable, any appropriate handlers are called to portray
the constraints represented by the attribute.  As an example, the {\tt
  bounds} package (cf. Volume II: Constraints Packages) uses a hook to
print out the bounds of variables:
%
\begin{small}
\begin{verbatim}
| ?- X in 1..10,Y in 1..10,X + 4 #< Y -3.

X = _h629 { bounds : 1 .. 2 }
Y = _h673 { bounds : 9 .. 10 }
\end{verbatim}
\end{small}
%
Writing a handler can be as simple as possible or as elaborate as
desired.  In the case of {\tt bounds} the handler is simple:

{\small
{\tt bounds\_attr\_portray\_hook(bounds(L,U,\_)) :- write(L..U).}
}

The hook is installed when the constraint package is loaded by placing
in the package loader directive such as:

{\small
{\tt  :- install\_attribute\_portray\_hook(bounds,Attr,bounds\_attr\_portray\_hook(Attr)).}
}

Note that the hook will be indexed on the module associated with the
attribute (in this case {\tt bounds}).  XSB's command-line interpreter
will unify the second argument of the portray hook with the attribute,
and then call {\tt Handler}.

\ourmoditem{install\_attribute\_constraint\_hook(Module,Vars,Names,Handler)}{install\_attribute\_constraint\_hook/4}{machine}
%
For some constraint packages, it may not be particularly useful to
associate constraints with variables: instead, the projection of
global constraints onto the variables of the top-level query may be
more useful.  This is the case in the CLP(R) package (cf. Volume II
Constraints Packages), where the command-line interaction may look as
follows:
%
{\small
\begin{verbatim}
| ?- {X = 2*Y,Y >= 7},inf(X,F).
 { X >= 14.0000 }
 { Y = 0.5000 * X }

X = _h8841
Y = _h9506
F = 14.0000
\end{verbatim}
}
%
In XSB, the (projection of the) global constraints in CLP(R) are
displayed by the following routines:
%
{\small
\begin{verbatim}
clpr_portray_varlist(Vars,Names):- 
	filter_varlist(Vars,Names,V1,N1),
	dump(V1,N1,Constraints),
	member(C,Constraints),
	console_write(' { '), console_write(C),	console_writeln(' } '),
	fail.
clpr_portray_varlist(_V,_N).

filter_varlist([],[],[],[]).
filter_varlist([V1|R1],[N1|R2],[V1|R3],[N1|R4]):- 
	var(V1),!,
	filter_varlist(R1,R2,R3,R4).
filter_varlist([_V1|R1],[_N1|R2],R3,R4):- 
	filter_varlist(R1,R2,R3,R4).
\end{verbatim}
}
%
This predicate sets up a call to the CLP(R) library predicate {\tt
  dump/3}, whose constraints it then writes out to the console.
Analogous to the portray hook, the console hook is installed using the
directive:
%
{\small
\begin{verbatim}
:- install_constraint_portray_hook(clpr,Vars,Names,clpr_portray_varlist(Vars,Names)).
\end{verbatim}
}
%
If the {\tt clpr} module is loaded, the command line interpreter
checks any constraint portray hooks upon the first success of a
top-level goal.  It then unifies the second argument {\tt Vars} with
the variables of the goal, and {\tt Names} with the names of the
variables of the goal which are then passed on to {\tt Handler}

\end{description}

\section{{\tt constraintLib}: a library for CLP} \label{constraintLib}
%====================================================================

XSB supports constraint logic programming through its engine-level
support of attributed variables
(Section~\ref{sec:attributed-variables}), and its support for
constraint handling rules (CHR) (cf. Volume II: Constraint Handling
Rules).  The {\tt constraintLib} library includes routines for
delaying and examining bindings that are commonly used to implement
CHR and other constraint libraries.

When processing constraints, it is often useful to delay a goal based
on the instantiation level of a term or set of terms.  For instance a
{\tt 3 > X + Y} should be delayed until both {\tt X} and {\tt Y} are
instantiated.  However the goal should be reinvoked as soon as
possible after both are instantiated in order to prune search paths
that may not be useful to pursue.  The predicate {\tt when/2} provides
a useful mechanism to delay goals based on instantiation
patterns~\footnote{Despite the similar name, this method of delaying
is conceptually different from SLG {\sc delaying} discussed in Volume
1 of this manual, which is used for resolving cycles of dependencies
in computing the well-founded semantics, and is not based on the state
of instantiation of a term.}.

\begin{description}
\ourmoditem{when(+Condition,Goal)}{when/2}{constraintLib}
%
Delays the execution of {\tt Goal} until {\tt Condition} is satisfied,
whereupon {\tt Goal} will be executed.  Condition can have the form
\begin{itemize}
\item {\tt ?=(Term1,Term2)}
\item {\tt nonvar(Term)}
\item {\tt ground(Term)}~\footnote{To use {\tt ground/1} in the
condition, it must be imported into the file where it is used.}
\item {\tt (Condition,Condition)}
\item {\tt (Condition ; Condition)}
\end{itemize}

{\bf Example:} The following session illustrates the use of when/2 to
delay a goal.
%
\begin{small}
\begin{verbatim}
|?- when(nonvar(X),writeln(test(1-2,nonvar))),writeln(test(1,nonvar)),X = f(_Y).

test(1,nonvar)
test(1 - 2,nonvar)

X = f(_h245)
\end{verbatim}
\end{small}

\ourmoditem{unifiable(X, Y, -Unifier)}{unifiable/3}{constraintLib}
%
If {\tt X} and {\tt Y} can unify, succeeds unifying {\tt Unifier} with
a list of terms of the form {\tt Var = Value} representing a most
general unifier of {\tt X} and {\tt Y}.\  {\tt unifiable/3} can handle
cyclic terms. Attributed variables are handled as normal
variables. Associated hooks are not executed~\footnote{In \version ,
  {\tt unifiable/3} is implemented as a Prolog predicate and so is slower
  than many of the predicates in this section.}.

\ourmoditem{setarg(+Index,+Term,+Value)}{setarg/3}{constraintLib}
\ourrepeatmoditem{set\_arg(+Index,+Term,+Value)}{set\_arg/3}{machine}
%
The predicate {\tt setarg/3} provides an efficient but non-logical way
to update argument {\tt Index} of a Prolog term {\tt Term} to {\tt
  Value} via destructive assignment and without the necessity of
copying {\tt Term}.  {\tt setarg/3} should be used sparingly, to
ensure both clarity and portability of code.

{\bf Example}
{\small
\begin{verbatim}
|?- X = p(f(1),g(2),r([a])),
    writeln(zero(X)),
    (  setarg(2,X,g([b])),
       writeln(one(X)),
       fail
     ; writeln(two(X))).
zero(p(f(1),g(2),r([a])))
one(p(f(1),g([b]),r([a])))
two(p(f(1),g(2),r([a])))

X = p(f(1),g(2),r([a]))
\end{verbatim}
}

{\bf Error Cases} 
\bi
\item 	{\tt Index} is a variable
\bi
\item    {\tt instantiation\_error}
\ei
\item 	{\tt Index} neither a variable nor an integer
\bi
\item    {\tt type\_error(integer,Index)}
\ei
\item 	{\tt Index} is less than 0
\bi
\item    {\tt domain\_error(not\_less\_than\_zero,Index)}
\ei
\item 	{\tt Term} is a variable
\bi
\item    {\tt instantiation\_error}
\ei
\item 	{\tt Term} is neither a variable nor a compound term
\bi
\item    {\tt type\_error(compound,Term)}
\ei
\ei

\ourmoditem{term\_variables(+Term,-Variables)}{term\_variables/2}{constraintLib}

% 
Given any Prolog term {\tt Term} as input, returns a sorted list of
variables in the term.  
\end{description}

\section{Formatted Output}

\begin{description}
\ourrepeatmoditem{format(+String,+Control)}{format/2}{format}
\ourmoditem{format(+Stream,+String,+Control)}{format/3}{format}

{\tt format/2} and {\tt format/3} act as a Prolog analog to the C {\tt
stdio} function {\tt printf()}, allowing formatted
output~\footnote{The {\tt format} family of predicates is due to
Quintus Prolog, by way of Ciao.}.

Output is formatted according to {\tt String} which can contain either
a format control sequence, or any other character which will appear
verbatim in the output. Control sequences act as place-holders for the
actual terms that will be output.  Thus
\begin{verbatim}
        ?- format("Hello ~q!",world).
\end{verbatim}

\noindent
will print {\tt Hello world!}.

If there is only one control sequence, the corresponding element may
be supplied alone in {\tt Control}.  If there are more, {\tt Control}
must be a list of these elements. If there are none then {\tt Control}
must be an empty list. There have to be as many elements in {\tt
Control} as control sequences in {\tt String}.

The character \verb|~| introduces a control sequence. To print 
        a \verb|~| just repeat it:  
        \begin{verbatim}
        ?- format("Hello ~~world!", []).
        \end{verbatim}
      \noindent
        will output  \verb|Hello ~world!|.

%--------------------------------------------------------------------------------------------------
\comment{
% TLS: the \\c does not work properly.

A format may be spread over several lines. The control
        sequence {\tt \\c} followed by a @key{LFD} will translate to the 
        empty string:  
        \begin{verbatim}
        ?- format("Hello \\c
        world!", []).
        \end{verbatim}
        \noindent
        will result in {\tt Hello world!}.
}
%--------------------------------------------------------------------------------------------------

The general format of a control sequence is \verb|~NC|.  The character
{\tt C} determines the type of the control sequence.  {\tt N} is an
optional numeric argument.  An alternative form of {\tt N} is {\tt
*}. {\tt *} implies that the next argument in {\tt Arguments} should
be used as a numeric argument in the control sequence.  For example:

\begin{verbatim}
?- format("Hello~4cworld!", [0'x]).
\end{verbatim}

\noindent
and

\begin{verbatim}
?- format("Hello~*cworld!", [4,0'x]).
\end{verbatim}

\noindent
both produce

\begin{verbatim}
Helloxxxxworld!
\end{verbatim}

The following control sequences are available in XSB.

%TLS: printf-style integers and floats are not currently supported in XSB!

\begin{itemize}

\item \verb|~a|
The argument is an atom.  The atom is printed without quoting.  

\item \verb|~Nc|
(Print character.)  The argument is a number that will be interpreted as an
UTF-8 code. {\tt N} defaults to one and is interpreted as the number of
times to print the character.  

\item \verb|~f|
(Print float).  The argument is a float.  The float will be printed out by XSB.

\item \verb|~d|
(Print integer).  The argument is an integer, and will be printed out by XSB.

%-------------------------------------------------------------------------
\comment{
Various shadings don't work -- TLS
\item \verb|~Ne|
\item \verb|~NE|
\item \verb|~Nf|
\item \verb|~Ng|
\item \verb|~NG|
\begin{verbatim}
printf("%.@var{N}e", @var{Arg})
printf("%.@var{N}E", @var{Arg})
printf("%.@var{N}f", @var{Arg})
printf("%.@var{N}g", @var{Arg})
printf("%.@var{N}G", @var{Arg})
\end{verbatim}

If @var{N} is not supplied the action defaults to

\begin{verbatim}
printf("%e", @var{Arg})
printf("%E", @var{Arg})
printf("%f", @var{Arg})
printf("%g", @var{Arg})
printf("%G", @var{Arg})
\end{verbatim}

\item ~@var{N}d
(Print decimal.) The argument is an integer. @var{N} is interpreted as the
number of digits after the decimal point.  If @var{N} is 0 or missing, no
decimal point will be printed.  Example:  

\begin{verbatim}
?- format("Hello ~1d world!", [42]).
?- format("Hello ~d world!", [42]).
\end{verbatim}

\noindent
will print as

\begin{verbatim}
Hello 4.2 world!
Hello 42 world!
\end{verbatim}

\noindent
respectively.

\item ~@var{N}D
(Print decimal.) The argument is an integer.  Identical to {\tt ~@var{N}d}
except that {\tt ,} will separate groups of three digits to the left of the
decimal point.  Example:  

\begin{verbatim}
?- format("Hello ~1D world!", [12345]).
\end{verbatim}

\noindent
will print as

\begin{verbatim}
Hello 1,234.5 world!
\end{verbatim}

\item ~@var{N}r
(Print radix.) The argument is an integer. @var{N} is interpreted as a radix.
@var{N} should be >= 2 and <= 36.  If @var{N} is missing the radix defaults to
8.  The letters {\tt a-z} will denote digits larger than 9.  Example:  

\begin{verbatim}
?- format("Hello ~2r world!", [15]).
?- format("Hello ~16r world!", [15]).
\end{verbatim}

\noindent
will print as

\begin{verbatim}
Hello 1111 world!
Hello f world!V
\end{verbatim}

\noindent
respectively.

\item ~@var{N}R
(Print radix.) The argument is an integer.  Identical to {\tt ~@var{N}r} except
that the letters {\tt A-Z} will denote digits larger than 9.  Example:  

\begin{verbatim}
?- format("Hello ~16R world!", [15]).
\end{verbatim}

\noindent
will print as

\begin{verbatim}
Hello F world!
\end{verbatim}

}
%-------------------------------------------------------------------------

\item \verb|~Ns|
(Print string.) The argument is a list of UTF-8 codes.  Exactly {\tt
N} characters will be printed. {\tt N} defaults to the length of the
string.  Example:

\begin{verbatim}
?- format("Hello ~4s ~4s!", ["new","world"]).
?- format("Hello ~s world!", ["new"]).
\end{verbatim}

\noindent
will print as

\begin{verbatim}
Hello new worl!
Hello new world!
\end{verbatim}

\noindent
respectively.

\item \verb|~i|
(Ignore argument.) The argument may be of any type.  The argument will be
ignored.  Example:  

\begin{verbatim}
?- format("Hello ~i~s world!", ["old","new"]).
\end{verbatim}

\noindent
will print as

\begin{verbatim}
Hello new world!
\end{verbatim}

\item \verb|~k|
(Print canonical.) The argument may be of any type.  The argument will be
passed to {\tt write\_canonical/2} ).  Example:  

\begin{verbatim}
?- format("Hello ~k world!", a+b+c).
\end{verbatim}

\noindent
will print as

\begin{verbatim}
Hello +(+(a,b),c) world!
\end{verbatim}

%------------------------------------------------------------------------------------------
\comment{
\item ~p
(print.) The argument may be of any type.  The argument will be passed to
{\tt print/2}.  Example:

\noindent
supposing the user has defined the predicate

\begin{verbatim}
portray([X|Y]) :- print(cons(X,Y)).
\end{verbatim}

\noindent
then

\begin{verbatim}
?- format("Hello ~p world!", [[a,b,c]]).
\end{verbatim}

\noindent
will print as

\begin{verbatim}
Hello cons(a,cons(b,cons(c,[]))) world!
\end{verbatim}
}
%------------------------------------------------------------------------------------------

\item \verb|~q|
(Print quoted.) The argument may be of any type.  The argument will be
passed to {\tt writeq/2}.  Example:

\begin{verbatim}
?- format("Hello ~q world!", [['A','B']]).
\end{verbatim}

\noindent
will print as

\begin{verbatim}
Hello ['A','B'] world!
\end{verbatim}

\item \verb|~w|
(write.) The argument may be of any type.  The argument will be passed
to {\tt write/2}.  Example:

\begin{verbatim}
?- format("Hello ~w world!", [['A','B']]).
\end{verbatim}

\noindent
will print as

\begin{verbatim}
Hello [A,B] world!
\end{verbatim}

\item \verb|~Nn|
(Print newline.) Print {\tt N} newlines. {\tt N} defaults to 1.
Example:

\begin{verbatim}
?- format("Hello ~n world!", []).
\end{verbatim}

\noindent
will print as

\begin{verbatim}
Hello
 world!
\end{verbatim}

\end{itemize}

%-------------------------------------------------------------------------------------------------------
\comment{
\item ~N
(Fresh line.) Print a newline, if not already at the beginning of a line.

% TLS: dont keep track of line_position.
The following control sequences are also available for compatibility,
but do not perform any useful functions.

\begin{itemize}
\item ~@var{N}|
(Set tab.) Set a tab stop at position @var{N}, where @var{N} defaults to
the current position, and advance the current position there.

\item ~@var{N}+
(Advance tab.) Set a tab stop at @var{N} positions past the current
position, where @var{N} defaults to 8, and advance the current position
there.

\item ~@var{N}t
(Set fill character.) Set the fill character to be used in the next
position movement to @var{N}, where @var{N} defaults to @key{SPC}.
\end{itemize}
}
\end{description}
%-------------------------------------------------------------------------------------------------------

%-------------------------------------------------------------------------------------
\comment{

\section{Lower-Level I/O}
\label{sec-low-level-io}
%======================

\begin{description}


\ourmoditem{file\_write(+IOport,+Term)}{file\_write/2}{xsb\_writ}
    Writes the term {\tt Term} to the file (or string) with descriptor {\tt
    IOport}.

\ourmoditem{file\_read(+IOport,-Term)}{file\_read/2}{xsb\_read}
    Reads a term from the file (or string) with descriptor {\tt
    IOport} into {\tt Term}.  Note that the term must be terminated
    with a period (.) (whether it appears in a file or in a string.)

\ourmoditem{file\_read(+IOport,-Term,-Vars)}{file\_read/3}{xsb\_read}
    Reads a term from the file (or string) with descriptor {\tt
    IOport} into {\tt Term}, and returns in {\tt Vars} an open-tailed list of
    pairs of names of variables and the variables themselves that
    appear in Term.  For example, reading a term {\tt f(a,X,Y,X)}
    would result in {\tt term} being bound to {\tt
    f(a,\_25,\_26,\_25)} (for some internal variables) and {\tt Vars}
    being bound to {[vv('X',\_25) vv('Y',\_26) $\mid$ \_83]}.  Note that the
    pairing functor symbol is {\tt vv/2} and it must be imported from
    {\tt xsb\_read} along with this read predicate.  Also note that 
    {\tt Vars} is not a proper list, but has a free variable instead 
    of [] at its end.

\ourmoditem{file\_read\_canonical(+IOport,-Term,-Psc)}{file\_read\_canonical/3}{machine}
    Reads a term that is in canonical format from the the I/O port
    indicated by {\tt IOport} (as returned by {\tt file\_open/3} or
    by {\tt stat\_flag(10,IOport))}, and returns it in {\tt Term}.
    It also returns (in {\tt Psc}) the psc address of the main functor
    symbol of the term, if it is the same as that of the previously
    read term, and the current term is a ground (non 0-ary) fact.
    (This is used for efficiency in the implementation of {\tt
    load\_dync/1} and {\tt load\_dynca/1}).  Otherwise {\tt Psc} is set to 0.  To initialize
    its previous psc value to zero, this predicate can be called with
    {\tt IOport} of -1000.

\end{description}

}
%-------------------------------------------------------------------------------------

%%$ -- so that font-lock-mode works in Emacs

\section{Low-level Atom Manipulation Predicates}
\label{sec-strings}

XSB has a number of low-level predicates that, despite their names,
examine properties of atoms.  The functionality of these predicates is
often a subset of the ISO predicate {\tt sub\_atom/5}, but these
predicates are faster, as they are more specialized, and have been
written in C.

These predicates are especially powerful when they are combined with
pattern-matching facilities provided by the {\tt pcre} package
described in Volume 2 of this manual).

It is important to note, that not all string manipulation predicates
have been made thread-safe in \version{}.  In addition, as noted, the
predicates may or may not properly handle (non-ASCII) UTF-8
characters.

\begin{description}
\ourrepeatmoditem{str\_sub(+Sub, +Str, ?Pos)}{str\_sub/3}{string}
\ourmoditem{str\_sub(+Sub, +Str)}{str\_sub/3}{string}

Succeeds if {\tt Sub} is a substring of {\tt Str}. In that case, {\tt
  Pos} unifies with the position where the match occurred. Positions
start from 0. {\tt str\_sub/2} is also available, which is equivalent
to having {\tt \_} in the third argument of {\tt
  str\_sub/3}~\footnote{Currently, {\tt str\_sub/2} works properly for
  UTF-8 characters, but {\tt str\_sub/3} does not.}.

\ourmoditem{str\_match(+Sub, +Str, +Direction, ?Beg, ?End)}{str\_match/5}{string}
%\index{\texttt{str\_match/5}}

This is an enhanced version of the previous predicate.
{\tt Direction} can be {\tt forward} or {\tt reverse} (or any abbreviation
of these). If {\tt forward}, the predicate finds the first match of {\tt
  Sub} from the beginning of {\tt Str}. If {\tt reverse}, it finds the
first match from the end of the string ({\it i.e.}, the last match of {\tt
  Sub} from the beginning of {\tt Str}). {\tt Beg} and {\tt End} must be
integers or unbound variables. (It is possible that one is bound and
another is not.)
{\tt Beg} unifies with the offset of the first character where {\tt Sub}
matched, and {\tt End} unifies with the offset of the next character to the
right of {\tt Sub} (such a character might not exist, but the offset is
still defined). Offsets start from 0.

Both {\tt Beg} and {\tt End} can be bound to negative integers.
In this case, the value represents the offset from the \emph{second}
character past the end of {\tt Str}. Thus {\tt -1} represents the character
next to the end of {\tt Str} and can be used to check where the end of {\tt
  Sub}  matches in {\tt Str}. In the following examples
%%
\begin{verbatim}
    ?- string_match(Sub,Str,forw,X,-1).  
    ?- string_match(Sub,Str,rev,X,-1).  
    ?- string_match(Sub,Str,forw,0,X).  
\end{verbatim}
%%
the first checks if the \emph{first} match of {\tt Sub} from the
beginning of {\tt Str} is a suffix of {\tt Str} (because {\tt End}
represents the character next to the last character in {\tt Sub}, so
{\tt End=-1} means that the last characters of {\tt Sub} and of {\tt
  Str} occupy the same position). If so, {\tt X} is bound to the
offset (from the end of {\tt Str}) of the first character of {\tt
  Sub}. The second example checks if the \emph{last} match of {\tt
  Sub} in {\tt Str} is a suffix of {\tt Str} and binds {\tt X} to the
offset of the beginning of that match (counted from the beginning of
{\tt Str}).  The last example checks if the first match of {\tt Sub}
is a prefix of {\tt Str}. If so, {\tt X} is bound to the offset (from
the beginning of {\tt Str}) of the last character of {\tt
  Sub}~\footnote{Currently, {\tt string\_match/5} does not work
  properly for UTF-8 characters.}.

%-----------------------------------------------------------------------

%%
\ourmoditem{substring(+String, +BeginOffset, +EndOffset, -Result)}{substring/4}{string}
%%
{\tt String} can be an atom or a list of characters, and the offsets must
be integers.  If {\tt EndOffset} is negative, endof({\tt String})+{\tt
  EndOffset}+1 is assumed. Thus, {\tt -1} means end of string.  If {\tt
  BeginOffset} is less than 0, then 0 is assumed; if it is greater than the
length of the string, then string end is assumed. If {\tt EndOffset} is
non-negative, but is less than {\tt BeginOffset}, then empty string is
returned.

Offsets start from 0.

The result returned in the fourth argument is a string, if {\tt String} is
an atom, or a list of characters, if so is {\tt String}.

The \verb|substring/4| predicate always succeeds (unless there is an error,
such as wrong argument type).

Here are some examples: 
%%
\begin{verbatim}
| ?- substring('abcdefg', 3, 5, L).

L = de

| ?- substring("abcdefg", 4, -1, L).

L = [101,102]
\end{verbatim}
%%
({\it i.e.}, L = {\tt ef} represented using ASCII
codes)~\footnote{Currently, {\tt substring/4} works properly for UTF-8
  characters.}.

%-----------------------------------------------------------------------

%%
\ourmoditem{str\_repl(+String, +SourceList, +TargetList, ?Result)}{substring/4}{string}
%%
This predicate, {\tt str\_repl/4}, replaces substrings in an atom with
new substrings.  {\tt String} is the atom whose substrings are to be
replaced.  {\tt SourceList} is a list of atoms that are to be replaced
in {\tt String}.  {\tt TargetList} is a corresponding list of atoms
that are to replace occurrences of {\tt SourceList} atoms in {\tt
  String}.  {\tt Result} is the atom resulting from doing the
replacements.

{\tt SourceList} and {\tt TargetList} may be single atoms if they
would otherwise be lists of length one.

For example:
\begin{verbatim}
| ?- string:str_repl(aaaabcdeaaaabcdefg,[aaa,d,g],['AA','DD','GG'],RES).

RES = AAabcDDeAAabcDDefGG;

no

| ?- string:str_repl(aaba,a,aa,RES).

RES = aaaabaa;

no
| ?- 
\end{verbatim}

\end{description}
%%
%------------------------------------------------------------------------------

\section{Script Writing Utilities}
%========================

Prolog, (in particular XSB!) can be useful for writing scripts.
Prolog's simple syntax and declarative semantics make it especially
suitable for scripts that involve text processing.  There are several
ways to access script-writing commands from XSB.  The first is to
execute the command via the predicates {\tt shell/1} or {\tt shell/2}.
These predicates can execute any command but they do not provide
streamability across UNIX and Windows commands, and they do not return
any output of commands to Prolog.  Special predicates are provided to
handle cross-platform compatibility and to bring output into XSB.

Effort has been made to make the these thread-safe; however in
\version, calls to the XSB script writing utilities go through a
single mutex, and may cause contention if many threads seek to
concurrently use sockets.

\begin{description}
\ourmoditem{expand\_filename(+FileName,-ExpandedName)}{expand\_filename/2}{machine}
%
Expands the file name passed as the first argument and binds the
variable in the second argument to the expanded name. This includes
(1) expanding Unix tildes, (2) prepending {\tt FileName} to the
current directory, and (3) ``rectifying'' the expanded file name.  In
rectification, the expanded file name is ``rectified'' so that
multiple repeated slashes are replaced with a single slash, the
intervening ``./'' are removed, and ``../'' are applied so that the
preceding item in the path name is deleted. For instance, if the
current directory is {\tt /home}, then {\tt abc//cde/..///ff/./b} will
be converted into {\tt /home/abc/ff/b}.

Under Windows, this predicates does rectification as described above,
(using backslashes when appropriate), but it does not expand the
tildes.

\ourmoditem{expand\_filename\_no\_prepend(+FileName,-ExpandedName)}{expand\_filename\_no\_prepend/2}{shell}
This predicate behaves as {\tt expand\_filename/2}, but only expands
tildes and does rectification. It does not prepend the current working
directory to relative file names.

\ourmoditem{parse\_filename(+FileName,-Dir,-Base,-Extension)}{parse\_filename/4}{machine}
This predicate parses file names by separating the directory part, the base
name part, and file extension. If file extension is found, it is removed
from the base name. Also, directory names are rectified and if a directory
name starts with a tilde (in Unix), then it is expanded. Directory names
always end with a slash or a backslash, as appropriate for the OS at hand.

For instance, {\tt $\sim$john///doe/dir1//../foo.bar} will be parsed into:
{\tt /home/john/doe/}, {\tt foo}, and {\tt bar} (where we assume that    
{\tt /home/john} is what {\tt $\sim$john} expands into).  

\ourmoditem{sys\_pid(-Pid)}{pid/1}{shell}
%
Get Id of the current process.

\ourmoditem{sys\_main\_memory(-RamInBytes)}{sys\_main\_memory/1}{shell}
%
Provides a platform-independent way to return the amount of RAM for
the current machine, in bytes.~\footnote{Based on code by David Robert
  Nadeau under the Creative Commons license.}

\end{description}

%---------------------------------------------------------------------------------------
\subsection{Communication with Subprocesses} \label{sec:subprocesses}

In the previous section, we have seen several predicates that allow XSB to
create other processes. However, these predicates offer only a very limited
way to communicate with these processes. The predicate
\verb|spawn_process/5| and friends come to the rescue. It allows a user to spawn
any process (including multiple copies of XSB) and redirect its
standard input and output to XSB streams. XSB can then write to the
process and read from it. The section of socket I/O describes yet
another mode of interprocess communication.

In addition, the predicate {\tt pipe\_open/2} described in this
section lets one create any number of pipes (that do not need to be
connected to the standard I/O stream) and talk to child processes
through these pipes.
%
All predicates in this section, except {\tt pipe\_open/2} and {\tt
fd2stream/2}, must be imported from module {\tt shell}.  The
predicates {\tt pipe\_open/2} and {\tt fd2stream/2} must be imported
from {\tt file\_io}.

\begin{description}
\ourmoditem{spawn\_process(+CmdSpec,-StreamToProc,-StreamFromProc,-ProcStderrStream,-ProcId)}{spawn\_process/5}{shell}
Spawn a new process specified by {\tt CmdSpec}. {\tt CmdSpec} must be
either a single atom or a \emph{list} of atoms.
If it is an atom, then it must represent a shell command.
If it is a list, the first member of the list must be the name of the
program to run and the 
other elements must be arguments to the program. Program name must be specified
in such a way as to make sure the OS can find it using the contents of the
environment variable {\tt PATH}.
Also note that pipes, I/O redirection and such are not allowed in command
specification. That is, {\tt CmdSpec} must represent a single command.
(But read about process plumbing below and about the related predicate
{\tt shell/5}.)

The next three parameters of \verb|spawn_process| are XSB I/O stream
identifiers for the process (leading to the subprocess standard
input), from the process (from its standard output), and a stream
capturing the subprocess standard error output. The last parameter is
the system process id.
\end{description}
%%

\noindent
Here is a simple example of how it works.

%%
\begin{verbatim}
| ?- import file_flush/2, file_read_line_atom/2 from file_io.
| ?- import file_nl/1 , file_write/2 from xsb_writ.  

| ?- spawn_process([cat, '-'], To, From, Stderr, Pid),
     writeln(To,'Hello cat!'), flush_output(To,_), file_read_line_atom(From,Y).

To = 3
From = 4
Stderr = 5
Pid = 14328
Y = Hello cat!

yes
\end{verbatim}
%%

Here we created a new process, which runs the ``{\tt cat}'' program
with argument ``--''. This forces {\tt cat} to read from standard
input and write to standard output. The next line writes an atom and
newline to the XSB stream {\tt To}, which is bound to the standard
input of the {\tt cat} process (proc id {\tt 14328}). The {\tt cat}
process then copies the input to its standard output.  Since standard
output of the cat process is redirected to the XSB stream {\tt From}
in the parent process, the last line in our program is able to read it
and return in the variable {\tt Y}.
%
Note that in the second line we used {\tt flush\_output/2}. Flushing
the output is extremely important here, because XSB I/O pipe (file)
streams are buffered.  Thus, {\tt cat} might not see its input until
the buffer is filled up, so the above clause might hang. {\tt
flush\_output/2} makes sure that the input is immediately available to
the subprocess.

In addition to the above general schema, the user can tell
\verb|spawn_process/5| not to open one of the communication streams or to
use one of the existing communication streams.  This is useful when
you do not expect to write or read to/from the subprocess or when one
process wants to write to another (see the process plumbing example
below).
%
To tell that a certain stream is not needed, it suffices to bind that
stream to an atom.  For instance,
%%
\begin{verbatim}
| ?- spawn_process([cat, '-'], To, none, none, _),
     nl(To), writeln(To,'Hello cat!'), flush_output(To).


To = 3,
Hello cat!
\end{verbatim}
%%
reads from XSB and copies the result to standard output. Likewise,
%%
\begin{verbatim}
| ?- spawn_process('cat library.tex', none, From, none, _),
     file_read_line_atom(From, S).

From = 4
S = \chapter{Library Utilities} \label{library_utilities}
\end{verbatim}
%%
In each case, only one of the streams is open. (Note that the shell
command is specified as an atom rather than a list.) Finally, if both
streams are suppressed, then \verb|spawn_process| reduces to the usual
{\tt shell/1} call (in fact, this is how {\tt shell/1} is
implemented):
%%
\begin{verbatim}
| ?- spawn_process([pwd], none, none).

/usr/local/foo/bar
\end{verbatim}
%%
On the other hand, if any one of the three stream variables in
\verb|spawn_process| is bound to an already existing file stream, then the
subprocess will use that stream (see the process plumbing example
below).

One of the uses of XSB subprocesses is to create XSB servers that spawn
subprocesses and control them. A spawned subprocess can be another XSB
process. The following example shows one XSB process spawning another,
sending it a goal to evaluate and obtaining the result:
%%
\begin{verbatim}
| ?- spawn_process([xsb], To, From,Err,_),
     write(To,'assert(p(1)).'), flush_output(To,_),
     write(To,'p(X), writeln(X).'), flush_output(To,_), 
     file_read_line_atom(From,XX).  

XX = 1

yes
| ?-
\end{verbatim}
%%
Here the parent XSB process sends ``\verb|assert(p(1)).|'' and then
``\verb|p(X), writeln(X).|'' to the spawned XSB subprocess. The latter
evaluates the goal and prints (via ``\verb|writeln(X)|'') to its
standard output. The main process reads it through the {\tt From}
stream and binds the variable {\tt XX} to that output.

Finally, we should note that the stream variables in the
\verb|spawn_process| predicate can be used to do process plumbing, {\it
  i.e.}, redirect output of one subprocess into the input of another. Here
is an example:
%%
\begin{verbatim}
| ?- open(test,write,Stream),
     spawn_process([cat, 'data'], none, FromCat1, none, _),
     spawn_process([sort], FromCat1,Stream, none, _).  
\end{verbatim}
%%
Here, we first open file {\tt test}. Then \verb|cat data| is spawned.
This process has the input and standard error stream blocked (as
indicated by the atom {\tt none}), and its output goes into stream
{\tt FromCat1}.  Then we spawn another process, {\tt sort}, which
picks the output from the first process (since it uses the stream {\tt
FromCat1} as its input) and sends its own output (the sorted version
of {\tt data}) to its output stream {\tt Stream}.  However, {\tt
Stream} has already been open for output into the file {\tt
test}. Thus, the overall result of the above clause is tantamount to
the following shell command:
%%
\begin{verbatim}
        cat data | sort > test  
\end{verbatim}
%%

\paragraph{{\em Important notes about spawned processes\/}:}
\begin{enumerate}
\item Asynchronous processes spawned by XSB do not disappear (at least on
  Unix) when they terminate, \emph{unless} the XSB program executes a
  \emph{wait} on them (see {\tt process\_control} below). Instead, such
  processes become defunct \emph{zombies} (in Unix terminology); they do
  not do anything, but consume resources (such as file descriptors). So,
  when a subprocess is known to terminate, it must be waited on.
  
\item The XSB parent process must know how to terminate the asynchronous
  subprocesses it spawns. The drastic way is to kill it (see {\tt
    process\_control} below). Sometimes a subprocess might terminate by
  itself ({\it e.g.}, having finished reading a file). In other cases, the
  parent and the child programs must agree on a protocol by which the
  parent can tell the child to exit. The programs in the XSB subdirectory
  {\tt examples/subprocess} illustrate this idea. If the child subprocess
  is another XSB process, then it can be terminated by sending the atom
  {\tt end\_of\_file} or {\tt halt} to the standard input of the child.
  (For this to work, the child XSB must waiting at the prompt).
\item It is very important to not forget to close the streams that the
  parent uses to communicate with the child. These are the streams
  that are provided in arguments 2,3,4 of {\tt spawn\_process}. The
  reason is that the child might terminate, but these streams to the
  standard input of the child will remain open, since they belong to
  the parent process. As a result, the parent will own defunct I/O
  streams and might eventually run out of file descriptors or streams.
\end{enumerate}

\begin{description}
  \ourmoditem{process\_status(+Pid,-Status)}{process\_status/2}{shell}
  This predicate always succeeds. Given a process id, it binds the second
  argument (which must be an unbound variable) to one of the following
  atoms: {\tt running}, {\tt stopped}, {\tt exited\_normally}, {\tt
    exited\_abnormally}, {\tt aborted}, {\tt invalid}, and {\tt unknown}.
  The {\tt invalid} status is given to processes that never existed or that
  are not children of the parent XSB process. The {\tt unknown} status is
  assigned when none of the other statuses can be assigned.
  
  Note: process status (other than {\tt running}) is system dependent.
  Windows does not seem to support {\tt stopped} and {\tt aborted}.  Also,
  processes killed using the \verb|process_control| predicate (described
  next) are often marked as {\tt invalid} rather than {\tt exited}, because
  Windows seems to lose all information about such processes. Process
  status might be inaccurate in some Unix systems as well, if the process
  has terminated and {\tt wait()} has been executed on that process.

\ourmoditem{process\_control(+Pid,+Operation)}{process\_control/2}{shell}
    Perform a process control {\tt operation} on the process with the given
    {\tt Pid}. 
    Currently, the only supported operations are {\tt kill} (an atom) and {\tt
    wait(Code)} (a term).
    The former causes the process to exit unconditionally, and the latter
    waits for process completion. When the process exits, {\tt Code} is
    bound to the process exit code. The code for normal termination is 0.

    This predicate succeeds, if the operation was performed successfully.
    Otherwise, it fails. The {\tt wait} operation fails if the process
    specified in {\tt Pid} does not exist or is not a child of the parent
    XSB process. 
    
    The {\tt kill} operation might fail, if the process to be killed does
    not exist or if the parent XSB process does not have the permission to
    terminate that process. Unix and Windows have different ideas as to
    what these permissions are. See \emph{kill(2)} for Unix and
    \emph{TerminateProcess} for Windows.
    
    \emph{Note}: under Windows, the programmer's manual warns of dire
    consequences if one kills a process that has DLLs attached to it.
%
\ourmoditem{get\_process\_table(-ProcessList)}{get\_process\_table/1}{shell}
    This predicate is imported from module {\tt shell}.
    It binds {\tt ProcessList} to the list of terms, each describing one of
    the active XSB subprocesses (created via \verb|spawn_process/5|).
    Each term has the form:
    %%
    \begin{center}
      \verb|process(Pid,ToStream,FromStream,StderrStream,CommandLine)|. 
    \end{center}
    %%
    The first argument in the term is the process id of the corresponding
    process, the next three arguments describe the three standard streams
    of the process, and the last is an atom that shows the command line used
    to invoke the process.
    This predicate always succeeds.

\ourmoditem{shell(+CmdSpec,-StreamToProc, -StreamFromProc, -ProcStderr,
  -ErrorCode)}{shell/5}{shell}
%
The arguments of this predicate are similar to those of
\verb|spawn_process|, except for the following: (1) The first argument
is an atom or a list of atoms, like in \verb|spawn_process|. However,
if it is a list of atoms, then the resulting shell command is obtained
by string concatenation. This is different from \verb|spawn_process|
where each member of the list must represent an argument to the
program being invoked (and which must be the first member of that
list).  (2) The last argument is the error code returned by the shell
command and not a process id. The code -1 and 127 mean that the shell
command failed.
    
    The {\tt shell/5} predicate is similar to \verb|spawn_process| in
    that it spawns another process and can capture that process' input
    and output streams.
%    
    The important difference, however, is that XSB will wait until the
    process spawned by {\tt shell/5} terminates. In contrast, the process
    spawned by \verb|spawn_process| will run concurrently with XSB.  In
    this latter case, XSB must explicitly synchronize with the spawned
    subprocess using the predicate \verb|process_control/2| (using the {\tt
      wait} operation), as described earlier.
    
    The fact that XSB must wait until {\tt shell/5} finishes has a very
    important implication: the amount of data the can be sent to and from
    the shell command is limited (1K is probably safe). This is because the
    shell command communicates with XSB via pipes, which have limited
    capacity.  So, if the pipe is filled, XSB will hang waiting for {\tt
      shell/5} to finish and {\tt shell/5} will wait for XSB to consume data
    from the pipe.  Thus, use \verb|spawn_process/5| for any kind of
    significant data exchange between external processes and XSB.
  
  Another difference between these two forms of spawning subprocesses is
  that {\tt CmdSpec} in {\tt shell/5} can represent \emph{any} shell
  statement, including those that have pipes and I/O redirection. In
  contrast, \verb|spawn_process| only allows command of the form ``program
  args''. For instance,
%%
\begin{verbatim}
| ?- open(test,write,Stream),
     shell('cat | sort > data', Stream, none, none, ErrCode)
\end{verbatim}
%%
As seen from this example, the same rules for blocking I/O streams
apply to {\tt shell/5}. Finally, we should note that the already
familiar standard predicates {\tt shell/1} and {\tt shell/2}
(documented in Volume 1) are implemented using {\tt shell/5}, and {\tt
  shell/5} shares their error cases.

\paragraph{\em Notes:}
%%
\begin{enumerate}
  \item  With {\tt shell/5}, you do not have to worry about terminating
    child processes: XSB waits until the child exits automatically.
    However, since communication pipes have limited capacity, this method
    can be used only for exchanging small amounts of information between
    parent and child.
  \item The earlier remark about the need to close I/O streams to the child
    \emph{does} apply.
\end{enumerate}
%%

\ourmoditem{pipe\_open(-ReadPipe, -WritePipe)}{pipe\_open/2}{shell}
Open a new pipe and return the read end and the write end of that
pipe.  If the operation fails, both {\tt ReadPipe} and {\tt WritePipe}
are bound to negative numbers.
%  
The pipes returned by the {\tt pipe\_open/2} predicate are small
integers that represent file descriptors used by the underlying
OS. They are {\bf not XSB I/O streams}, and they cannot be used for
I/O directly. To use them, one must convert them to streams using {\tt
open/3} or {\tt open/4}.~\footnote{
%% 
XSB does not convert pipe file descriptors into I/O streams
automatically.  Because of the way XSB I/O streams are represented,
they are not inherited by the child process and they do not make sense
to the child process (especially if the child is not another XSB
process). Therefore, we must pass the child processes an OS file
descriptor instead. The child then converts these descriptor into XSB
I/O streams.  
%% 
}

    The best way to illustrate how one can create a new pipe to a child
    (even if the child has been created earlier) is to show an example.
    Consider two programs, {\tt parent.P} and {\tt child.P}. The parent
    copy of XSB consults {\tt parent.P}, which does the following: First, it
    creates a pipe and spawns a copy of XSB. Then it tells the
    child copy of XSB to assert the fact {\tt pipe(RP)}, where {\tt RP} is
    a number representing the read part of the pipe. Next, the parent XSB tells
    the child XSB to consult the program {\tt child.P}. Finally, it sends
    the message {\tt Hello!}.

    The {\tt child.P} program gets the pipe from predicate {\tt pipe/1}
    (note that the parent tells the child XSB to first assert {\tt
    pipe(RP)} and only then to consult the {\tt child.P} file).
  After that, the child reads a message from the pipe and prints it to its
    standard output. Both programs are shown below:
    %%
    \begin{verbatim}
%% parent.P      
:- import pipe_open/2 from file_io.
%% Create the pipe and pass it to the child process
?- pipe_open(RP,WP),
   %% WF is now the XSB I/O stream bound to the write part of the pipe
   open(pipe(WP),write,WF),
   %% ProcInput becomes the XSB stream leading directly to the child's stdin
   spawn_process(nxsb1, ProcInput, block, block, Process),
   %% Tell the child where the reading part of the pipe is
   fmt_write(ProcInput, "assert(pipe(%d)).\n", arg(RP)),
   fmt_write(ProcInput, "[child].\n", _),
   flush_output(ProcInput, _),
   %% Pass a message through the pipe
   fmt_write(WF, "Hello!\n", _),
   flush_output(WF, _),
   fmt_write(ProcInput, "end_of_file.\n",_), % send end_of_file atom to child
   flush_output(ProcInput, _),
   %% wait for child (so as to not leave zombies around; 
   %% zombies quit when the parent finishes, but they consume resources)
   process_control(Process, wait),
   %% Close the ports used to commuicate with the process
   %% Otherwise, the parent might run out of file descriptors 
   %% (if many processes were spawned)
   close(ProcInput), close(WF).
    \end{verbatim}
    %%
    %%
    \begin{verbatim}
%% child.P
:- import file_read_line_atom/2 from file_io.
:- dynamic pipe/1.
?- pipe(P), open(pipe(P),read,F),
   %% Acknowledge receipt of the pipe
   fmt_write("\nPipe %d received\n", arg(P)),
   %% Get a message from the parent and print it to stdout
   file_read_line_atom(F, Line), write('Message was: '), writeln(Line).
    \end{verbatim}
    %%
    This produces the following output:
    %%
    \begin{verbatim}
| ?- [parent].                    <- parent XSB consults parent.P
[parent loaded]
yes
| ?- [xsb_configuration loaded]   <- parent.P spawns a child copy of XSB
[sysinitrc loaded]                   Here we see the startup messages of
[packaging loaded]                   the child copy
XSB Version 2.0 (Gouden Carolus) of June 27, 1999
[i686-pc-linux-gnu; mode: optimal; engine: slg-wam; scheduling: batched]
| ?- 
yes
| ?- [Compiling ./child]          <- The child copy of received the pipe from
[child compiled, cpu time used: 0.1300 seconds]     the parent and then the
[child loaded]                                      request to consult child.P
Pipe 15 received                  <- child.P acknowledges receipt of the pipe
Message was: Hello!               <- child.P gets the message and prints it
yes       
    \end{verbatim}
    %%
    
    Observe that the parent process is very careful about making sure
    that the child terminates and also about closing the I/O streams
    after they are no longer needed.
    
    Finally, we should note that this mechanism can be used to
    communicate through pipes with non-XSB processes as well. Indeed,
    an XSB process can create a pipe using {\tt pipe\_open}
    (\emph{before} spawning a child process), pass one end of the pipe
    to a child process (which can be a C program), and use {\tt
    open/3} to convert the other end of the pipe to an XSB
    stream. The C program, of course, does not need {\tt open/3},
    since it can use the pipe file handle directly. Likewise, a C
    program can spawn off an XSB process and pass it one end of a
    pipe.  The XSB child-process can then convert this pipe fd to a
    file using {\tt fd2iostream} and then talk to the parent C program.

\ourmoditem{fd2iostream(+Pipe, -IOstream)}{fd2iostream/2}{shell}
Take a file descriptor and convert it to an XSB I/O stream. This
predicate should be used only for user-defined I/O.  Otherwise, use
{\tt open/\{3,4\}} when possible.

%------------------------------------------------------------------------------------------
\end{description}



\section{Socket I/O}

The XSB socket library defines a number of predicates for
communication over BSD-style sockets. Most are modeled after and are
interfaces to the socket functions with the same name. For detailed
information on sockets, the reader is referred to the Unix man pages
(another good source is \emph{Unix Network Programming}, by W.
Richard Stevens).  Several examples of the use of the XSB sockets
interface can be found in the {\tt XSB/examples/} directory in the XSB
distribution.

XSB supports two modes of communication via sockets:
\emph{stream-oriented} and \emph{message-oriented}. In turn,
stream-oriented communication can be \emph{buffered} or
\emph{character-at-a-time}.

To use \emph{buffered} stream-oriented communication, system socket
handles must be converted to XSB I/O streams using {\tt
  fd2iostream/2}.  In these stream-oriented communication, messages
have no boundaries, and communication appears to the processes as
reading and writing to a file.  At present, buffered stream-oriented
communication works under Unix only.

\emph{Character-at-a-time} stream communication is accomplished using
the primitives {\tt socket\_put/3} and {\tt socket\_get0/3}. These
correspond to the usual Prolog {\tt put/1} and {\tt get0/1} I/O primitives.

In message-oriented communication, processes exchange messages that have
well-defined boundaries. The communicating processes use {\tt
  socket\_send/3} and {\tt socket\_recv/3} to talk to each other.
XSB messages are represented as strings where the first four bytes
({\tt sizeof(int)}) is an integer (represented in the binary network format
--- see the functions {\tt htonl} and {\tt ntohl} in socket documentation)
and the rest is the body of the message. The integer in the header
represents the length of the message body.

Effort has been made to make the socket interface thread-safe; however
in \version, calls to the XSB socket interface go through a single
mutex, and may cause contention if many threads seek to concurrently
use sockets.

We now describe the XSB socket interface.  All predicates below must be
imported from the module {\tt socket}. Note that almost all predicates have
the last argument that unifies with the error code returned from the
corresponding socket operation. This argument is explained separately.

\paragraph{General socket calls.}
These are used to open/close sockets, to establish connections, and set
special socket options.
\begin{description}
\ourmoditem{socket(-Sockfd, ?ErrorCode)}{socket/2}{socket}
    A socket {\tt Sockfd}  in the AF\_INET domain is created.
    (The AF\_UNIX domain is not yet implemented). 
    {\tt Sockfd} is bound to a small integer, called socket descriptor or
    socket handle.

\ourmoditem{socket\_set\_option(+Sockfd,+OptionName,+Value)}{socket\_set\_option/3}{socket}
    Set socket option. At present, only the {\tt linger} option is
    supported. ``Lingering'' is a situation when a socket continues to live
    after it was shut down by the owner. This is used in order to let the
    client program that uses the socket to finish reading or writing
    from/to the socket. {\tt Value} represents the number of seconds to linger.
    The value -1 means do not linger at all.

\ourmoditem{socket\_close(+Sockfd, ?ErrorCode)}{socket\_close/2}{socket}
    {\tt Sockfd} is closed. Sockets used in {\tt socket\_connect/2}  should
    not be closed by {\tt socket\_close/1}  as they will be closed when the
    corresponding stream is closed.

\ourmoditem{socket\_bind(+Sockfd,+Port, ?ErrorCode)}{socket\_bind/3}{socket}
   The socket {\tt Sockfd}  is bound to the specified local port number.

\ourmoditem{socket\_connect(+Sockfd,+Port,+Hostname,?ErrorCode)}{socket\_connect/4}{socket}
    The socket Sockfd is connected to the address ({\tt Hostname}  and
{\tt Port}). If {\tt socket\_connect/4} terminates abnormally for any reason
(connection refused, timeout, etc.), then XSb closes the socket {\tt
    Sockfd} automatically, because such a socket cannot be used according
    to the BSD semantics. Therefore, it is always a good idea to check to
    the return code and reopen the socket, if the error code is not
    {\tt SOCK\_OK}.

\ourmoditem{socket\_listen(+Socket, +Length, ?ErrorCode)}{socket\_listen/3}{socket}
    The socket {\tt Sockfd}  is defined to have a maximum backlog queue of
{\tt Length}  pending connections.

\ourmoditem{socket\_accept(+Sockfd,-SockOut, ?ErrorCode)}{socket\_accept/3}{socket}
    Block the caller until a connection attempt arrives. If the incoming 
    queue is not empty, the first connection request is accepted, the call
    succeeds and returns a new socket, {\tt SockOut}, which can be used for
    this new connection.
\end{description}

\paragraph{Buffered, message-based communication.}
These calls are similar to the {\tt recv} and {\tt send} calls in C, except
that XSB wraps a higher-level message protocol around these low-level
functions. More precisely, {\tt socket\_send/3} prepends a 4-byte field
to each message, which indicates the length of the message
body. When {\tt socket\_recv/3} reads a message, it first reads the 4-byte
field to determine the length of the message and then reads the remainder
of the message. 

All this is transparent to the XSB user, but you should know these details
if you want to use these details to communicate with external processes
written in C and such. All this means that these external programs must
implement the same protocol. The subtle point here is that different
machines represent integers differently, so an integer must first be
converted into the machine-independent network format using the functions
{\tt htonl} and {\tt ntohl} provided by the socket library. For instance,
to send a message to XSB, one must do something like this:
%%
\begin{verbatim}
char *message, *msg_body;
unsigned int msg_body_len, network_encoded_len;

  msg_body_len = strlen(msg_body);
  network_encoded_len = (unsigned int) htonl((unsigned long int) msg_body_len);
  memcpy((void *) message, (void *) &network_encoded_len, 4);
  strcpy(message+4, msg_body);
\end{verbatim}
%%
To read a message sent by XSB, one can do as follows:
%%
\begin{verbatim}
int actual_len;
char lenbuf[4], msg_buff;
unsigned int msglen, net_encoded_len;  

  actual_len = (long)recvfrom(sock_handle, lenbuf, 4, 0, NULL, 0);
  memcpy((void *) &net_encoded_len, (void *) lenbuf, 4);
  msglen = ntohl(net_encoded_len);

  msg_buff = calloc(msglen+1, sizeof(char))); // check if this succeeded!!!
  recvfrom(sock_handle, msg_buff, msglen, 0, NULL, 0);
\end{verbatim}
%%
If making the external processes follow the XSB protocol is not practical
(because you did not write these programs), then you should use the
character-at-a-time interface or, better, the buffered
stream-based interface both of which are described in this section.
At present, however, the buffered stream-based interface does not work on
Windows.
%%
\begin{description}
\ourmoditem{socket\_recv(+Sockfd,-Message, ?ErrorCode)}{socket\_recv/3}{socket}
    Receives a message from the connection identified by the socket descriptor
    {\tt Sockfd}. Binds {\tt Message} to the message. {\tt socket\_recv/3}
    provides a message-oriented interface. It understands message
    boundaries set by {\tt socket\_send/3}.

\ourmoditem{socket\_send(+Sockfd,+Message, ?ErrorCode)}{socket\_send/3}{socket}
    Takes a message (which must be an atom) and sends it through the
    connection specified by {\tt Sockfd}. {\tt socket\_send/3} provides
    message-oriented communication. It prepends a 4-byte header to the
    message, which tells {\tt socket\_recv/3} the length of the message body.

\end{description}

\paragraph{Stream-oriented, character-at-a-time interface.}
Internally, this interface uses the same {\tt sendto} and {\tt recvfrom}
socket calls, but they are executed for each character separately.
This interface is appropriate when the message format is not known or when
message boundaries are determined using special delimiters.

{\tt socket\_get0/3} creates the end-of-file condition when it receives the
end-of-file character {\tt CH\_EOF\_P} (a.k.a. 255) defined in {\tt
  char\_defs.h} (which must be included in the XSB program). C programs
that need to send an end-of-file character should send {\tt (char)-1}.
%%
\begin{description}
\ourmoditem{socket\_get0(+Sockfd, -Char, ?ErrorCode)}{socket\_get0/3}{socket}
The equivalent of {\tt get0} for sockets.

\ourmoditem{socket\_put(+Sockfd, +Char, ?ErrorCode)}{socket\_put/3}{socket}
Similar to put/1, but works on sockets.
\end{description}
%%

\paragraph{Socket-probing.}
With the help of the predicate {\tt socket\_select/6} one can establish a
group of asynchronous or synchronous socket connections. In the synchronous
mode, this call is blocked until one of the sockets in the group becomes
available for reading or writing, as described below.  In the asynchronous
mode, this call is used to probe the sockets periodically, to find out
which sockets have data available for reading or which sockets have room in
the buffer to write to.

The directory {\tt XSB/examples/socket/select/} has a number of examples of
the use of the socket-probing calls.
%%
\begin{description}
  \ourmoditem{socket\_select(+SymConName,+Timeout,-ReadSockL,-WriteSockL,-ErrSockL,?ErrorCode)}{socket\_select/6}{socket}
  {\tt SymConName} must be an atom that
  denotes an existing connection group, which must be previously created with
  {\tt socket\_set\_select/4} (described below). {\tt ReadSockL}, {\tt
    WriteSockL}, {\tt ErrSockL} are lists of socket handles (as returned by
  {\tt socket/2}) that specify the available sockets that are available for
  reading, writing, or on which exception conditions occurred.  {\tt
    Timeout} must be an integer that specifies the timeout in seconds (0
  means probe and exit immediately). If {\tt Timeout} is a variable, then
  wait indefinitely until one of the sockets becomes available.

\ourmoditem{socket\_set\_select(+SymConName,+ReadSockFdLst,+WriteSockFdLst,+ErrorSockFdLst)}{socket\_set\_select/4}{socket}
Creates a connection group with the symbolic name {\tt SymConName}
(an atom) for subsequent use by {\tt socket\_select/6}.
{\tt ReadSockFdLst}, {\tt WriteSockFdLst}, and {\tt ErrorSockFdLst} are
lists of sockets for which {\tt socket\_select/6} will be used to monitor read,
write, or exception conditions.

\ourmoditem{socket\_select\_destroy(+SymConName)}{socket\_select\_destroy/1}{socket}
Destroys the specified connection group.

\end{description}


\paragraph{Error codes.}
The error code argument unifies with the error code returned by the
corresponding socket commands. The error code -2 signifies
\emph{timeout} for timeout-enabled primitives (see below). The error code
of zero signifies normal termination. Positive error codes denote specific
failures, as defined in BSD sockets. When such a failure occurs, an error
message is printed, but the predicate succeeds anyway. The specific error
codes are part of the socket documentation. Unfortunately, the symbolic
names and error numbers of these failures are different between Unix
compilers and Visual C++. Thus, there is no portable, reliable way to refer
to these error codes. The only reliably portable error codes that can be 
used in XSB programs defined through these symbolic constants:
%%
\begin{verbatim}
#include "socket_defs_xsb.h"  

#define SOCK_OK       0      /* indicates successful return from socket      */
#define SOCK_EOF     -1      /* end of file in socket_recv, socket_get0     */

#include "timer_defs_xsb.h"

#define TIMEOUT_ERR -2                 /* Timeout error code */
\end{verbatim}
%%

\paragraph{Timeouts.}
XSB socket interface allows the programmer to specify timeouts for
certain operations. If the operations does not finish within the
specified period of time, the operation is aborted and the
corresponding predicate succeeds with the {\tt TIMEOUT\_ERR} error
code. The following primitives are timeout-enabled: {\tt
  socket\_connect/4}, {\tt socket\_accept/3}, {\tt socket\_recv/3},
{\tt socket\_send/3}, {\tt socket\_get0/3}, and {\tt socket\_put/3}.
To set a timeout value for any of the above primitives, the user
should execute {\tt set\_timer/1} right before the subgoal to be
timed. Note that timeouts are disabled after the corresponding
timeout-enabled call completes or times out. Therefore, one must use
{\tt set\_timer/1} before each call that needs to be controlled by a
timeout mechanism.

The most common use of timeouts is to either abort or retry the operation
that times out. For the latter, XSB provides the {\tt sleep/1} primitive,
which allows the program to wait for a few seconds before retrying.

The {\tt set\_timer/1} and {\tt sleep/1} primitives are described below.
They are standard predicates and do not need to be explicitly imported.
%%
\begin{description}
\ourstandarditem{set\_timer(+Seconds)}{set\_timer/1}
Set timeout value. If a timer-enabled goal executes after this value is
set, the clock begins ticking. If the goal does not finish in time, it
succeeds with the error code set to {\tt TIMEOUT\_ERR}. The timer is turned
off after the goal executes (whether timed out or not and whether it
succeeds or fails). This goal always succeeds.

Note that if the timer is not set, the timer-enabled goals execute
``normally,'' without timeouts. In particular, they might block (say, on
{\tt socket\_recv}, if data is not available).

\ourstandarditem{sleep(+Seconds)}{sleep/1}
Put XSB to sleep for the specified number of seconds. Execution resumes
after the {\tt Seconds} number of seconds. This goal always succeeds.

\ourstandarditem{sleep\_ms(+MilliSeconds)}{sleep\_ms/1}
Like \texttt{sleep/1} but puts 
XSB to sleep for the specified number of milliseconds instead of seconds.
\end{description}
%%
Here is an example of the use of the timer:
%%
\begin{samepage}
\begin{verbatim}
:- compiler_options([xpp_on]).
#include "timer_defs_xsb.h"

?- set_timer(3),  % wait for 3 secs
   socket_recv(Sockfd, Msg, ErrorCode),
   (ErrorCode == TIMEOUT_ERR
   -> writeln('Socket read timed out, retrying'),
      try_again(Sockfd)
   ;  write('Data received: '), writeln(Msg)
   ).
\end{verbatim}
\end{samepage}
%%

\noindent
Apart from the above timer-enabled primitives, a timeout value can be given
to {\tt socket\_select/6} directly, as an argument.


\paragraph{Buffered, stream-oriented communication.}
In Unix, socket descriptors can be ``promoted'' to file streams and the
regular read/write commands can be used with such streams. In XSB, such
promotion can be done using the following predicate:
%%
\begin{description}
\ourmoditem{fd2ioport(+Pipe, -IOport)}{fd2ioport/2}{shell} Take a
socket descriptor and convert it to an XSB I/O port that can be used
for regular file I/O.
\end{description}
%%
Once {\tt IOport} is obtained, all normal I/O primitives can be used by
specifying the {\tt IOport} as their first argument. This is, perhaps, the
easiest and the most convenient way to use sockets in XSB. (This feature
has not been implemented for Windows.)

\noindent
Here is an example of the use of this feature:
%%
\begin{samepage}
\begin{verbatim}
:- compiler_options([xpp_on]).
#include "socket_defs_xsb.h"

?- (socket(Sockfd, SOCK_OK)
   ->   socket_connect(Sockfd1, 6020, localhost, Ecode),
        (Ecode == SOCK_OK
        -> fd2ioport(Sockfd, SockIOport),
           file_write(SockIOport, 'Hello Server!')
        ;  writeln('Can''t connect to server')
        ),
    ;   writeln('Can''t open socket'), fail
    ).
\end{verbatim}
\end{samepage}
%%


\section{Arrays}
%===============

The module {\tt array1} provides a simple backtrackable array
implementation that requires no copying.  In Version 3.2, this package
was changed to make use of the backtrackable destructive assignment
made possible by {\tt setarg/3}.  We note that as of Version 3.2 this
library provides simple syntactic sugar for {\tt functor/3}, {\tt
  arg/3} and {\tt setarg/3} and relies on error messages for these
predicates.

\begin{description}
\ourmoditem{array\_new(-Array,+Size)}{array\_new/2}{array}
%
Creates a one dimensional empty array of size {\tt Size}.  All the
elements of this array are variables.  

\ourmoditem{array\_elt(+Array, +Index, ?Element)}{array\_elt/3}{array}
%
Succeeds iff {\tt Element} unifies with the {\tt Index}-th element of
array {\tt Array}.  
%
\ourmoditem{array\_update(+Array, +Index, +Elem)}{array\_update/3}{array}
%
Updates the array {\tt Array} such that the {\tt Index}-th element of
the new array is {\tt Elem} using destructive assignment.  The
implementation is quite efficient in that it avoids the copying of the
entire array.
\end{description}

The following example shows the use of these predicates:
{\footnotesize
\begin{verbatim}
| ?- import array_new/2, array_elt/3, array_update/4 from array.

yes
| ?- array_new(A,3), array_update(A,1,1), array_update(A,2,2),
     ( array_update(A,3,3), writeln(first(A))
     ; array_update(A,3,6), writeln(second(A))
     ; array_update(A,3,7), writeln(third(A))),fail.

first(array(1,2,3))
second(array(1,2,6))
third(array(1,2,7))

no
\end{verbatim}
}

\section{The Profiling Library} \label{sec:profile}
%==============================

XSB can provide Prolog-level profiling for Prolog programs, which
allows the Prolog programmer to estimate what proportion of time is
spent executing code for each predicate, and also what modes have been
used to call a given predicate.  It also helps to find unindexed
accesses to dynamic predicates which may be the cause of poor
performance.  To enable profiling, XSB must be
started with the command line parameter of {\tt -p}.  The module {\tt
  xsb\_profiling} contains the predicate {\tt profile\_call/1} that
invokes profiling.  The profiling library should only be used with the
single-threaded engine in \version{}.

\begin{description}
\ourmoditem{profile\_call(+Goal)}{profile\_call/1}{xsb\_profiling}

Calls {\tt Goal}, and when it first succeeds, prints to {\tt userout}
a table of predicate names indicating for each, the percentage of time
spent executing that predicate's code.  Within the table, the sum of
the predicate times for each module is also given.  {\tt Goal} may
backtrack, but profiling is done only for the time to the first
success, so it is most appropriate to profile succeeding deterministic
goals~\footnote{This includes tabled subgoals under Local Evaluation,
  as such as goal will only succeed after deriving all of its
  answers.}.
\end{description}

Profiling works by starting another thread that interrupts every 100th
of a second and sets a flag so that the XSB emulator will determine
the predicate of the currently executing code.  The printout also
includes the total number of interrupts and for each predicate, the
raw number of times its code was determined to be executing.  A
predicate is printed only if its code was interrupted at least once.
The numbers will be meaningful only for relatively long-running
predicates, taking more than a couple of seconds.

\index{interrupt instruction} 
%
When an interrupt occurs, the {\bf next} {\em interrupt instruction}
to be executed -- a WAM {\sf call}, {\sf execute}, {\sf proceed} or
{\sf trust} instruction -- will charge its associated predicate by
logging that predicate to a table.  The system does not keep track of
code addresses for tries (used to represent the results of completed
tables, and trie-indexed asserted code), so for some interrupts the
associated executing predicate cannot be determined.  In these cases
the interrupt is charged against an ``unknown/?''  pseudo-predicate,
and this count is included in the output.

Profiling does not give the context from which the predicate is
called, so you may want to make renamed copies of basic predicates to
use in particular circumstances to determine their times.

Predicates compiled with the ``optimize'' option may provide
misleading results under profiling.  Note that all system predicates
(including those in {\tt basics}) are compiled with the ``optimize''
option, by default.  That option causes tail-recursive predicates to
use a ``jump'' instruction rather than an ``execute'' instruction to
make the recursive call, and so an interrupt in such a loop will not
be charged until the next interrupt instruction is executed.  If much
time is spent in the recursion, this might not be for a long time, and
the interrupt might ultimately be charged to another predicate.  (If
an interrupt has not been charged by the time of the next interrupt,
it is lost.)

Profiling is currently available under Windows, Mac OS X, and Linux.
However, for the profiling algorithm to provide a good estimation, the
thread that wakes and sets the interrupt flag must be of high priority
and given the CPU when it wants it.  Accordingly, the estimates may be
better or worse depending on the scheduling strategy of a given
platform~\footnote{Windows and Mac OS X 10.6 provide good estimates.
  Some Linuxes however, do not charge about 20\% of their interrupts
  due to thread scheduling issues.  This loss of interrupts makes the
  profile estimate inefficient, but does not bias the estimate.  We
  haven't figured out how to get priority scheduling for interrupts on
  all machines, so if you want profiling to work more efficiently,
  maybe you can help figure out how to get appropriate scheduling.}.

The profiling module also provides support for determining when a
dynamic predicate is invoked in a mode that isn't supported by any
index.  The XSB programmer can set a flag that will cause a message to
be printed when a dynamic predicate is invoked, no index is
applicable, and there are more than 20 potentially matching clauses.
See {\tt profile\_unindexed\_calls/1} below for details.

\begin{description}
\ourmoditem{profile\_mode\_call(+Goal)}{profile\_mode\_call/1}{xsb\_profiling}

Calls the goal {\tt Goal} and constructs a table of the modes in which
the predicate is called and the number of times it is called in that
mode.  Modes are simply ``b'' for ground and ``f'' for variable.
Counts are kept in a table with entries of the form {\tt
Pred(Md1,Md2,..,Mdn)} where Pred is the name of the called predicate
and the Mdi are either 'f' or 'b', indicating free or bound for the
corresponding argument.  The table can be printed using {\tt
profile\_mode\_dump/0} and can be cleared using {\tt profile\_mode\_init/0}.
\end{description}

\begin{description}
\ourmoditem{profile\_mode\_dump}{profile\_mode\_dump/0}{xsb\_profiling}

Prints out the counts of calls in particular modes as accumulated using \\
{\tt profile\_mode\_call(+Goal)}.
\end{description}

\begin{description}
\ourmoditem{profile\_mode\_init}{profile\_mode\_init/0}{xsb\_profiling}

Clears the table that accumulates counts of calls in particular modes
(done by\\ {\tt profile\_mode\_call(+Goal)}.
\end{description}

\begin{description}

\ourmoditem{profile\_unindexed\_calls(+Par)}{profile\_unindexed\_calls/1}{xsb\_profiling}

Sets the kind of unindexed profiling to perform.  If {\tt Par} is
\texttt{off}, no unindexed logging will be done.  This is the default.
If {\tt Par} is \texttt{once} each call to a dynamic predicate that
cannot use any index (and would backtrack through more than 20
clauses) will generate a log message to userout.  Note that the
predicate of the goal may have indexes, but the particular goal may
not be able to take advantage of them.  E.g., a totally open call to a
predicate with many clauses will generate an unindexed message.  By
setting the {\tt once} parameter, each unindexed call to a predicate
will be logged only once; after logging is done, the log instruction
is changed to a branch, so it will never produce another log message
for that dynamic code.  If {\tt Par} is \texttt{on}, logging is done
as for \texttt{once}, except every unindexed call to any dynamic
predicate will be logged; i.e. the logging instruction is not changed
after logging.  If {\tt Par} is a predicate specification (of the form
Pred/Arity, Module:Pred/Arity, Term, or Module:Term), only unindexed
calls to the indicated goal will be logged, and when each is logged a
back-trace will be printed.  This allows the programmer to find the
location of an unindexed call.
\end{description}

%----------------------------------------------------------------------
\comment{
% Replaced by tries section in volume 1.

\section{Low-level Trie Manipulation Utilities}

Chapter 8 of Volume 1 indicates how tries can be used as an efficient
mechanism to store thread-private and thread-shared terms.  However,
while they are efficient, the trie manipulation routines in Chapter 8
of Volume 1 may not be suitable

and thread-safe manner Interned tries can be used to assert an retract
facts in an extremely efficient manner.  Terms can be interned within
different tries, designated by a {\tt Root}, or left as default.  Like
trie-indexed dynamic code and unlike normal asserted code, interned
tries have a set semantics: the variant of a term can occur in a trie
only once, additional asserts have no effect.  In addition, no
ordering of terms within the trie is guaranteed.  In \version{} of
XSB, when multiple threads are used, all interned tries are private to
a thread, so that information interned by one trie will not be visible
to other threads.  In order to ensure efficiency for these routines,
not all possible error checking is done on their inputs.


Section~\ref{sec:intern-basic} below provides an overview for the
basic interface to interned tries, while
Section~\ref{sec:intern-advanced} describes more advanced features.
See also the module {\tt storage} in Volume 1 which uses interned
tries for backtrackable updates and associative arrays.

\subsection{A Basic API for Interned Tries} \label{sec:intern-basic}
%%
\begin{description}
  %%
\ourmoditem{new\_trie(-Root)}{newtrie/1}{intern} 
        {\tt Root} is instantiated to a handle for a new trie.

\ourmoditem{trie\_intern(+Term,+Root)}{trie\_intern/2}{intern}
% 
Effectively asserts \texttt{Term} by interning into the trie
designated by {\tt Root}.  If a variant of {\tt Term} is already in
{\tt Root} the predicate succeeds, but a new copy of {\tt Term} is not
added to the trie.

{\bf Error Cases}
\begin{itemize}
\item 	{\tt Root} is uninstantiated
\bi
\item 	 {\tt instantiation\_error}
\ei
\item 	{\tt Root} is instantiated, but not an integer (trie handle)
\bi
\item 	 {\tt type\_error(integer,Root)}
\ei
\end{itemize}
%%

\ourmoditem{trie\_interned(?Term,+Root)}{trie\_interned/2}{intern}
%%
This predicate backtracks through the terms that unify with {\tt Term}
and that are interned into the trie represented by the handle {\tt
  Root}.  {\tt Term} may be free, or partially bound.

{\bf Error Cases}
\begin{itemize}
\item 	{\tt Root} is uninstantiated
\bi
\item 	 {\tt instantiation\_error}
\ei
\item 	{\tt Root} is instantiated, but not an integer (trie handle)
\bi
\item 	 {\tt type\_error(integer,Root)}
\ei
\end{itemize}
%%

\ourmoditem{delete\_trie(+Root)}{delete\_trie/1}{intern}
%%
Deletes all the terms in the trie pointed to by {\tt Root}.

{\bf Error Cases}
\begin{itemize}
\item 	{\tt Root} is uninstantiated
\bi
\item 	 {\tt instantiation\_error}
\ei
\item 	{\tt Root} is instantiated, but not an integer (trie handle)
\bi
\item 	 {\tt type\_error(integer,Root)}
\ei
\item 	Failure continuations point to one or more nodes in the trie with root {\tt Root}
\bi
\item 	{\tt misc\_error}
\ei
\end{itemize}
%%
\end{description}

%=======================

\subsection{Low-level Trie Manipulation Utilities} \label{sec:intern-advanced}

The utilities in this section are very low-level and require good
understanding of the trie mechanism in XSB.  Improper calls to these
predicates can cause core dumps, particularly if an improper trie
value is passed to them.  Despite these drawbacks, they can be used to
implement modules such as the {\tt storage} module described in Volume
1.

%%
\begin{description}
  %%
\ourmoditem{trie\_intern(+Term,+Root,-Leaf,-Flag,-Skel)}{trie\_intern/5}{intern}
%%
Acts as {\tt trie\_intern/2} but returns additional information: {\tt
  Leaf} is the handle for the interned {\tt Term} in the trie.  {\tt
  Flag} is 1 if the term is ``old'' (already exists in the trie); it
is 0, if the term is newly inserted.  {\tt Skel} represents the
collection of all the variables in {\tt Term}. It has the form {\tt
  ret(V1,V2,...,VN)}, exactly as in {\tt get\_calls} (see Vol. 1 of
the XSB manual).

{\bf Error Cases}
\begin{itemize}
\item 	{\tt Root} is uninstantiated
\bi
\item 	 {\tt instantiation\_error}
\ei
\item 	{\tt Root} is instantiated, but not an integer (trie handle)
\bi
\item 	 {\tt type\_error(integer,Root)}
\ei
\end{itemize}
%%

\ourmoditem{trie\_intern(+Term,-Leaf,-Skel)}{trie\_intern/3}{intern}
%%
Acts as {\tt trie\_intern/5}, but interns {\tt Term} into the default
trie and does not return the new/old flag.

\ourmoditem{trie\_interned(?Term,+Root,?Leaf,-Skel)}{trie\_interned/4}{intern}
%%
This predicate backtracks through the terms that unify with {\tt Term}
interned into the trie represented by the handle {\tt Root} if {\tt
  Leaf} is a free variable.  Otherwise, if {\tt Leaf} is bound, it
will backtrack over the terms in the trie that unify with the term
pointed to by Leaf {\tt to}.  {\tt Term} is the term to be retrieved;
it can be either (partially) bound or free.  {\tt Skel} is the
collection of all the variables in {\tt Term}; it has the form {\tt
  ret(V1,...,Vn)}.

{\bf Error Cases}
\begin{itemize}
\item 	{\tt Root} is uninstantiated
\bi
\item 	 {\tt instantiation\_error}
\ei
\item 	{\tt Root} is instantiated, but not an integer (trie handle)
\bi
\item 	 {\tt type\_error(integer,Root)}
\ei
\end{itemize}
%%

\ourmoditem{trie\_interned(?Term,?Leaf,-Skel)}{trie\_interned/3}{intern} 
%%
Similar to {\tt trie\_interned/4}  but uses the default trie.

\ourmoditem{trie\_unintern(+Root,+Leaf)}{trie\_unintern/2}{intern}
%%
Uninterns a term indicated by {\tt Leaf} from the trie indicated by
root.  If there are failure continuations that point to trie nodes in
{\tt Root}, the term is marked as deleted but space is not reclaimed.
Otherwise, the term is really deleted.
%%

\ourmoditem{trie\_unintern\_nr(+Root,+Leaf)}{trie\_unintern\_nr/2}{intern}
%%
This is a safe version of {\tt trie\_unintern/2}. The term pointed to by
{\tt Leaf} is marked as deleted, but is not deleted from the trie. This
permits an efficient implementation of backtrackable updates.

{\bf Error Cases}
\begin{itemize}
\item 	{\tt Root} or {\tt Leaf} is uninstantiated
\bi
\item 	 {\tt instantiation\_error}
\ei
\item 	{\tt Root} or {\tt Leaf} is instantiated, but not an integer
  (trie handle or trie leaf) 
\bi
\item 	 {\tt type\_error(integer,Root)} or {\tt type\_error(integer,Leaf)}
\ei
\end{itemize}
%%

\ourmoditem{unmark\_uninterned\_nr(+Root,+Leaf)}{unmark\_uninterned\_nr/2}{intern}
The term pointed to by {\tt Leaf} should have been previously marked for
deletion using
{\tt trie\_unintern\_nr/2}. This term is then ``unmarked'' (or undeleted)
and becomes again a normal interned term.

{\bf Error Cases}
\begin{itemize}
\item 	{\tt Root} or {\tt Leaf} is uninstantiated
\bi
\item 	 {\tt instantiation\_error}
\ei
\item 	{\tt Root} or {\tt Leaf} is instantiated, but not an integer
  (trie handle or trie leaf) 
\bi
\item 	 {\tt type\_error(integer,Root)} or {\tt type\_error(integer,Leaf)}
\ei
\end{itemize}
%%

\ourmoditem{reclaim\_uninterned\_nr(+Root)}{reclaim\_uninterned\_rn/1}{intern}
%%
Runs through the chain of leaves of the trie {\tt Root} and deletes
the terms that have been marked for deletion by {\tt
  trie\_unintern\_nr/2}. This is a garbage collection step that should
be done just before returning to the top level or when it can
otherwise be ensure that no failure continuations point to a node in
{\tt Root}.

{\bf Error Cases}
\begin{itemize}
\item 	{\tt Root} is uninstantiated
\bi
\item 	 {\tt instantiation\_error}
\ei
\item 	{\tt Root} is instantiated, but not an integer (trie handle)
\bi
\item 	 {\tt type\_error(integer,Root)}
\ei
\end{itemize}


%%
\end{description}
}
\section{Gensym}
%================================

The Gensym library provides a convenient way to generate unique
integers or constants.  

\begin{description}
\ourmoditem{prepare(+Index)}{prepare/1}{gensym}
%
Sets the initial integer to be used for generation to {\tt Index}.
Thus, the command {\tt ?- prepare(0)} would cause the first call to
{\tt gennum/1} to return {\tt 1}.  {\tt Index} must be a non-negative
integer.

\ourmoditem{gennum(-Var)}{gennum/1}{gensym} 
%
Unifies {\tt Var} with a new integer.

\ourmoditem{gensym(+Atom,-Var)}{gensym/2}zzzz{gensym} 
%
Generates a new integer, and concatenates this integer with {\tt
Atom}, unifying the result with {\tt Var}.  For instance a call {\tt
?- gensym(foo,Var)} might unify {\tt Var} with {\tt foo32}.
\end{description}

\section{Random Number Generation and Sampling Predicates}
%================================

The following predicates are provided in module \texttt{random} to
generate random numbers (both integers and floating numbers), based on
the Wichmann-Hill Algorithm \cite{WicH82,McLe85}.  The Wichmann-Hill
Algorithm generates the uniform distribution produced, but
transformations are used to also generate Gaussian, Exponential and
Weibull distributions.  In addition predicates are provided to
probabilistically sample derivations by extending {\tt setof/3}, {\tt
  bagof/3} and {\tt findall/3}.

This library and the random number generator it is based on is
entirely portable, and does not require any calls to the operating
system random number routines.  As noted below, it does require 3
seeds, each of which must be an integer in a given range.  These seeds
are thread-specific: thus different threads may generate independent
sequences of random numbers.

\index{random variables}

\begin{description}

\ourmoditem{random(-Number)}{random/1}{random} 
%
Binds \texttt{Number} to a random float in the interval [0.0, 1.0).
Note that 1.0 will never be generated.

\ourmoditem{random(+Lower,+Upper,-Number)}{random/3}{random}
    Binds \texttt{Number} to a random integer in the interval
    [\texttt{Lower},\texttt{Upper}) if \texttt{Lower} and \texttt{Upper}
    are integers.  Otherwise \texttt{Number} is bound to a random float
    between \texttt{Lower} and \texttt{Upper}.  \texttt{Upper} will
    never be generated.

\ourmoditem{getrand(?State)}{getrand/1}{random}
    Tries to unify \texttt{State} with the term \texttt{rand(X,Y,Z)}
    where \texttt{X},\texttt{Y},and \texttt{Z} are integers describing
    the state of the random generator.

\ourmoditem{setrand(rand(+X,+Y,+Z))}{setrand/1}{random} 
    Sets the state of the random generator.  \texttt{X},\texttt{Y}, and
    \texttt{Z} must be integers in the ranges [1,30269), [1,30307),
    [1,30323), respectively.

\ourmoditem{datime\_setrand}{datime\_setrand/0}{random}
This simple initialization utility sets the random seed triple based on a
function of the current day, hour, minute and second. 

\ourmoditem{randseq(+K, +N, -RandomSeq)}{randseq/3}{random}
    Generates a sequence of \texttt{K} unique integers chosen randomly
    in the range from 1 to \texttt{N}.  \texttt{RandomSeq} is not
    returned in any particular order.

\ourmoditem{randset(+K, +N, -RandomSet)}{randset/3}{random} 
    Generates an ordered set of \texttt{K} unique integers chosen
    randomly in the range from 1 to \texttt{N}.  The set is returned in
    reversed order, with the largest element first and the smallest
    last.

\index{random variables!normal}
\ourmoditem{gauss(-G1,-G2)}{gauss/2}{random}
%
Generates two random numbers that are normally distributed with mean 0
and standard deviation 1.  It uses the polar form of the Box-Muller
transformation~\cite{BoxM58} of uniform random variables as generated by
{\tt random/1}.

\index{random variables!Weibull}
\ourmoditem{weibull(K,Lambda,X)}{weibull/3}{random}
%
Generates a random number for the Weibull distribution:
\[
  f(x;k,\lambda) = \frac{k}{\lambda}(\frac{x}{\lambda})^{k-1}e^{-(x/\lambda)^h}
\]
based on the transformation
\[
  x = \lambda(-ln(U))^{1/k}
\]
of a uniformly distributed random variable produced by {\tt random/1}

\index{random variables!exponential}
\ourmoditem{exponential(K,X)}{exponential/2}{random}
%
Generates a random number for the exponential distribution:
\[
  f(x;k,\lambda) = \frac{e^{-(x/\lambda)^h}}{\lambda}
\]
based on the transformation
\[
  x = \lambda(-ln(U))
\]
of a uniformly distributed ranom variable produce by {\tt random/1}.
This is the same as the Weibull distribution with $k = 1$.

\end{description}

\subsection{Sampling Predicates} \label{sec:sample-aggrs}

\begin{description}
  \ourmoditem{sample\_all(+Prob,?Template,+Goal,-List}{sample\_all/4}{random}
  Acts as {\tt findall/3} but with a probablity of {\tt Prob} that a
  given success of {\tt Goal} will be added to {\tt List}. As with
  {\tt findall/3} all variables in {\tt Goal} are taken to be
  existential.

  In {\tt sample\_all/4} the sampling probablity is applied before a
  solution is added to {\tt List} so that samples can be taken of
  solution lists that are too big to materialize in memory.
  
  \ourmoditem{sample\_bag(Prob,Template,Goal,-Bag)}{sample\_bag/4}{random}
  Acts as {\tt bagof/3} but with a probablity of {\tt Prob} that a
  given success of {\tt Goal} will be added to {\tt Bag}. As with {\tt
    bagof/3} each variable $V$ in {\tt Goal} is taken to be
  non-existential unless $V$ occurs in {\tt Goal} or ``quantified'' by
  the {\tt \^\ /2} term (cf. Section~\ref{sec:setofs}).

  In {\tt sample\_bag/4} the sampling probablity is applied before a
  solution is added to {\tt Bag} so that samples can be taken of
  solution lists that are too big to materialize in memory.
  
  \ourmoditem{sample\_set(Prob,Template,Goal,-Set)}{sample\_set/4}{random}
  Acts as {\tt setof/3} but with a probablity of {\tt Prob} that a
  given success of {\tt Goal} will be added to {\tt Bag}. As with {\tt
    bagof/3} each variable $V$ in {\tt Goal} is taken to be
  non-existential unless $V$ occurs in {\tt Goal} or ``quantified'' by
  the {\tt \^\ /2} term (cf. Section~\ref{sec:setofs}).

  The semantics of {\tt sample\_set/4} is to sample from the solution
  {\em set}.  For this reason, unlike {\tt sample\_all/4} and {\tt
    sample\_bag/4}, {\tt sample\_set/4} constructs {\tt Set} {\bf
    before} applying the sampling probability.
\end{description}

\section{Loading .CSV and Other Delimiter-Separated Files} \label{sec:procfiles}

\index{csv file format}
\index{dsv file format}
\index{comma-separated file format}
\index{delimiter-separated file format}

Delimiter-separated files, such as {\em csv} (comma-separated values)
or {\em tsv} (tab-separated values) are a common format to store
relational data and so are a useful way for XSB to communicate data
with spreadsheets, Python, Java, and other tools.  XSB's {\tt
  proc\_files} library supports I/O of Prolog predicates to
delimiter-separaed files. \footnote{The name of the library, {\tt
    proc\_files} is unfortunately vague.}

\begin{description}

\ourrepeatmoditem{load\_csv(+File,+PredSpec)}{load\_csv/2}{proc\_files}
\ourrepeatmoditem{load\_tsv(+File,+PredSpec)}{load\_tsv/2}{proc\_files}
\ourrepeatmoditem{load\_psv(+File,+PredSpec)}{load\_psv/2}{proc\_files}

{\tt load\_csv/2}, {\tt load\_tsv/2}, and {\tt load\_psv/2}, each
respecively takes a file in csv format, tsv (tab-delimited) format or
psv (pipe-delimited format) along with a predicate specification, and
loads its contents into memory to define the predicate {\tt PredSpec}.

The simplest form of {\tt PredSpec} is {\tt PredName/Arity}. e.g.,

{\tt load\_csv('weather\_stats.csv',weather\_stats/7)}.

In this form {\tt Arity} must equal the number of fields in {\tt
  File}, and {\tt PredSpec} is inferred to be dynamic.  Each line in
the file will define one fact of {\tt PredSpec}.  In accordance with
csv formatting, any filed in {\tt File} enclosed in double quotes will
be treated as a single field, and thus can contain commas and
new-lines.  Any clauses for {\tt PredSpec} will be retracted before
the facts from the file are added.  With this simple form of predicate
specification, each field will be loaded as an atom (including fields
that contain just integers.)

{\tt PredSpec} may also be of the form {\tt Module:PredName/Arity}.
In this case, the facts loaded will be in the module {\tt Module}.
Other allowed forms are $Pred(_1,_2,...,_k)$, which is
equivalent to {\tt Pred/k}, and $Mod:Pred(_1,_2,...,_k)$,
which is equivalent to {\tt Mod:Pred/k}.

If more control over typing is needed, {\tt PredSpec} may be of the
form:

{\tt predName(TypeSpec$_1$,...,TypeSpec$_n$)}

where each type specification, {\tt TypeSpec$_i$} indicates the type
of the corresponding field in {\tt File}.  The permitted type
specifications are:
\begin{itemize}
\item[{\tt atom}] The corresponding field value will become an atom in
the loaded fact.
\item[{\tt string}] The corresponding field value will become a string
  (a list of integers as produced from an atom by {\tt atom\_codes/2})
  in the loaded fact.
\item[{\tt integer}] The corresponding field value will be converted
to an integer in the loaded fact.
\item[{\tt float}] The corresponding field value will be converted to
a float in the loaded fact.
\item[{\tt term}] The corresponding field must contain a term in
  Prolog canonical syntax, and it will be converted to that term in
  the loaded fact.  This option can be useful for loading, e.g.,
  Python lists.
\item[\_] (A variable) This default syntax causes a field to be
  treated as an {\tt atom}.
\item[{\tt user-defined}]
%
  A \emph{user-defined type name} is any atom other than the above,
  for example, \texttt{date} or \texttt{time}.  In that case, the user
  must supply a type conversion hook for that type by defining a
  conversion routine for the type in advance via
  \texttt{add\_cvt\_type\_hook/2} (defined below).  For instance, one
  can use the data type \texttt{date} after executing
  
  \texttt{add\_cvt\_type\_hook(date,date\_converter(\_,\_))}

  where \texttt{date\_converter/2} could be a predicate that takes
  atoms of the form \texttt{'2017-11-26'} in the first argument and
  binds the second argument to terms of the form
  \texttt{date(2017,11,26)}.To remove a type conversion hook, use
  \texttt{remove\_cvt\_type\_hook/1}, e.g.,
  \texttt{remove\_cvt\_type\_hook(date)}.
\end{itemize}

{\tt load\_csv/2}, {\tt load\_tsv/2}, and {\tt load\_psv/2} are all
convenience predicates defined in terms of the general predicate {\tt
  load\_dsv/3} described below.  Their definitions are as follows:

\begin{verbatim}
load_csv(FileName,PSpec) :-
    load_dsv(FileName,PSpec,[]).
	
load_psv(FileName,PSpec) :-
    load_dsv(FileName,PSpec,[separator="|",order=reverse]).

load_tsv(FileName,PSpec) :-
    load_dsv(FileName,PSpec,[separator="\t",order=reverse]).
\end{verbatim}


\ourmoditem{load\_dsv(+FileName,+PredSpec,+Options)}{load\_dsv/3}{proc\_files}

This predicate supports the loading of more general forms of files
with value-separated fields.  The {\tt FileName} and {\tt PredSpec}
parameters are exactly as in {\tt load\_csv/2}, as described above.
{\tt Options} is a list of options.  (With an empty {\tt Options}
list, {\tt load\_dsv} acts as {\tt load\_csv/2}.)  The options are:
\begin{itemize}
\item[{\tt separator=''Sep''}] (or {\tt separator(''Sep'')}) which
  indicates that the character(s) {\tt Sep} will be used as the field
  separator.  There may be one or more characters.

  {\tt Example: } To load a tab-separated file (tsv) use 
    \verb|separator="\t"| in the options list.
\item[{\tt delimiter=''C''}] (or {\tt delimiter(''C'')}) which
  indicates that the single character {\tt C} will be used as the
  field delimiter, the default being the double quote, ``''''''.
%  (and  I've yet to find a situation in which I want to change it).
\item[{\tt titles}] which indicates that the first line of the file should
  be ignored and not contribute a fact to the dynamic predicate.
\item[{\tt titles=$N$}] (or {\tt titles($N$)}), where $N$ is an integer,
  indicates that the first $N$ lines of the file should be ignored and
  not contribute a fact to the dynamic predicate.
\item[{\tt types=TypeList}] ( or {\tt types(TypeList)}), where {\tt
  TypeList} is a list of types (atoms) as described above for {\tt
  load\_csv/2}.  If types are given both in {\tt PredSpec} and in the
  options list, the one in {\tt PredSpec} is used.
\item[{\tt returns=''S''}] (or {\tt returns(''S'')}), where {\tt S} is
  a string to be substituted for any newlines in non-delimited fields.
\item[{\tt order=reverse}] causes {\tt asserta/1} to be used to assert the
  input terms, so {\tt PredSpec} will contain the file records in
  reverse order.  This can significantly improve performance if {\tt
    File} has a large number of lines or if indexing is particularly
  weak.
\item[{\tt pad=$N$}] (or {\tt pad($N$)}, where $N$ is a (positive or
  negative) integer.  If $N$ is positive, then the input records must
  contain $N$ more fields than {\tt PredSpec}, and the extra fields in
  the records will be ignored.  If $N$ is negative, then {\tt
    Predspec} must have $|N|$ more fields than the input records
  contain, and they will be padded with terms of the form {\tt
    'NULL'(\_\_)}.
\item[{tt module=$ModName$}], where $ModName$ is a module name.  With
  this option, the loaded predicate will be defined in module
  $ModName$.  (If an explicit module name is given in the {\tt
    PredSpec} argument, then this $ModName$ will be ignored.)
\end{itemize}

{\bf Error Cases} 

In \version{}, {\tt load\_dsv/3} does not yet perform error checking of its
inputs; this will be fixed in a future version of XSB.

For each line in {\tt File} where the number of fieldsdoes not match
the number of arguments in {\tt PredSpec} a warning is issued.

\ourmoditem{save\_dsv(+FileName,PredSpec,Options)}{save\_dsv/3}{proc\_files}
Writes out the solutions of a Prolog predicate, {\tt PredSpec}, into a
a separated/delimited file.  {\tt FileName} is the name of the file to
create.  {\tt PredSpec} specifies the predicate to use, which may be
of the form {\tt Pred/Arity}, a skeleton, or simply an atom -- in
which case the unary predicate of that name is called to get a list of
fields to output.

{\tt Options} is as in {\tt load\_dsv/3}.

%\begin{itemize}
%\item
%\end{itemize}

%{\em Sep} is a separator ascii-code or list of ascii-codes.  Delim is
%currently ignored, but should get used when necessary.  The final
%argument is for future expansion if/when we get options. (It should
%support types, to write the arguments correctly.  In particular it
%should support the term type.  But this is not (yet) done.)


\ourmoditem{add\_cvt\_type\_hook(+Type,+Hook)}{add\_cvt\_type\_hook/2}{proc\_files}
%
This predicate adds a type hook for a user-defined field type {\tt
  Type}.  {\tt Hook} must be a predicate with two or more arguments
with arguments 1 and 2 unbound and distinct. For instance, a hook
could be a predicate of the form

{\tt convert\_date(+InField,-OutField)}.

When a field declared as being of type {\tt Type} is read {\tt Hook}
will be called to convert the field according to the supplied
hook-predicate.

{\bf Error Cases} 
\bi
\item 	A hook for type {\tt Type} is already defined
\bi
\item    {\tt misc\_error}
\ei
\item {\tt Hook} is not defined as a term of arity 2 or greater in which the first two arguments are distinct variables \bi
\item    {\tt misc\_error}
  \ei
  \ei
  
\ourmoditem{remove\_cvt\_type\_hook(?Type)}{remove\_cvt\_type\_hook/1}{proc\_files}
%
{\tt remove\_cvt\_type\_hook(Type)} removes type conversion hook
associated with {\tt Type}.  If no hook is associated with {\tt Type},
the predicate succeeds silently.

\ourmoditem{data\_records(+FileName,+Format,?RecTerm)}{data\_records/3}{proc\_files}
This predicate is used to read files of various formats.  It succeeds
once for each record in the file, returning the record contents in
{\tt RecTerm}.  The {\tt FileName} denotes a file, and is found by
looking through the directories in XSB's current search path in the
same manner as static or dynamic loading.

Unlike {\tt load\_dsv/3},{\tt data\_records/3} reads {\tt FileName}
lazily, returning records upon backtracking.  Further details of how
it works depend on the value of {\tt Format}.

The permissible values for {\tt Format} are:
\begin{itemize}

\item[{\tt read}] indicates that the file contains a sequence of
  Prolog terms that can be read with the {\tt read/1} predicate; the
  terms must be syntactically valid, but can use any operators defined
  in the current Prolog session.  {\tt RecTerm} will be unified with
  each term that is read.

  {\bf Example:} Suppose {\tt terms.txt} is a file that contains the
  two lines:
\begin{verbatim}
p(a,b,c). p(c,b,a). p(b,c,a).
g(d,e,f).
\end{verbatim}
In this case, the goal

{\tt data\_records('bar.txt',read,RecTern).}

\noindent
would succeed four times, unifying {\tt RecTerm} with each term in
turn.

\index{canonical format}
\item[{\tt canonical}] acts in a similar manner as when {\tt Format}
  is {\tt read}, but in this case the Prolog terms much be in
  canonical format, and are read by {\tt read\_canonical/1}.  By
  calling {\tt data\_records/3} using {\tt canonical} can be
  significantly faster for large files than using {\tt read}.

%\item[{\tt canonical}] indicates that the file contains Prolog terms
%    that can be read with Prolog's {\tt read\_canonical/1} predicate.
%    So the terms in the file must be canonical terms.  In this case
%    {\tt RecTerm} will be unified with each term that is read.  So
%    normally {\tt RecTerm} is a variable in this case.  This can be
%    significantly faster {\tt read} for large files.

\item[{\tt dsv(Options)}] where {\tt Options} is a list of options as
  described for {\tt load\_dsv/3} above.  In this case, {\tt
    data\_records/2} will read records (as specified by {\tt Options})
  from a delimited-separated file, in the same manner as {\tt
    load\_dsv/3}, but lazily succeeding once for each record.

%\item[{\tt sql(??)}] reserved for future functionality.
\end{itemize}
  
{\bf Example:} {\em (Loading a DSV file into an index-moded table.)}
A use of {\tt data\_records/2} that leverages tabling is as follows.
\begin{verbatim}
:- table employee/6 as index([1]).

employee(FileName,EmpNum,EmpName,EmpAddr,EmpDept,EmpSalary) :
    data_records(FileName,
                 dsv([separator="\t"]),
                 [EmpNum,EmpName,EmpAddr,EmpDept,EmpSalary]).
\end{verbatim}
Here the first field of {\tt employee/6} is the file from which to get
the employee records, and the 2nd through 6th fields contain the data
about a given employee.  The {\tt table/1} directive declares that on
all calls to {\tt employee/6}, the first argument ([1] indicating here
the {\tt FileName}) must be bound.  As explained in
Section~\ref{sec:table-index}, the first call will be abstracted to
having just the first argument bound, and that call will be
(subsumptively) tabled.  Every subsequent call will retrieve its
answer(s) directly from that table.  So this use loads the contents of
a file (joined with the file name) into an index-moded subsumptive
table.
%whose information can be incrementally updated (due to the {\tt
%  incremental} option).

{\bf Example:} {\em (Loading a file containing Prolog facts into an
  index-moded table)} If a file contains canpnical Prolog
facts instead of delimiter separated values, the previous example
needs only a slight change.
\begin{verbatim}
:- table pp_f/4 as index([1]).

pp_f(FileName,F1,F2,F3) :-
    data_records(FileName,canonical,pp(F1,F2,F3)).
\end{verbatim}

where {\tt pp\_f/4} is the ``join'' of {\tt FileName} with the {\tt
  pp/3} facts it contains.  (Prolog facts for predicates other than
{\tt pp/3} will be automatically filtered out.)  Note that since
index-moded subsumptive tables use trie indexing, the {\tt FileName}
is stored only once in the trie, and the records are appropriately
indexed.  Of course, as described in Section~\ref{sec:table-index},
additional index patterns could be used.

As compared to the use of, say {\tt load\_dync/1}, the use of {\tt
  data\_records/3} plus tabling is declarative in the sense that
thereis no need to have an ``initialization'' operation that must be
called to load the data and then shouldn't be called again.  Here, the
programmer simply calls {\tt pp\_f/4} whenever she needs those values,
and the system automatically loads the data on the first call,
building the necessary indexes.  

\end{description}

\section{Scanning in Prolog}

Scanners, (sometimes called tokenizers) take an input string, usually
in UTF-8 or similar format, and produce a scanned sequence of tokens.
The requirements that various applications have for scanning differ in
small but important ways -- a character that is special to one
application may be part of the token of another; or some applications
may want lower case text converted to upper-case test.  The {\tt
stdscan.P} library provides a simple scanner written in XSB that can
be configured in several ways.  While useful, this scanner is not
intended to be as powerful as general-purpose scanners such as {\em
lex} or {\em flex}.

\begin{description}

\ourmoditem{scan(+List,-Tokens)}{scan/2}{stdscan}

Given as input a {\tt List} of character codes, {\tt scan/2} scans
this list producing a list of atoms constituting the lexical tokens.
Its parameters are set via {\tt set\_scan\_pars/1}.

Tokens produced are either a sequence of {\em letters} and/or {\em
numbers} or consist of a single {\em special character} (e.g. {\tt (}
or {\tt )}).  Whitespaces may occur between tokens.

\ourmoditem{scan(+List,+FieldSeparator,-Tokens)}{scan/3}{stdscan} 

Given as input a {\tt List} of character codes, along with a character
code for a field separator, {\tt scan/3} scans this list producing a
list of list of atoms constituting the lexical tokens in each field.
{\em scan/3} thus can be used to scan tabular information.  Its
parameters are set via {\tt set\_scan\_pars/1}.

\ourmoditem{set\_scan\_pars(+List)}{set\_scan\_pars/1}{stdscan} 

{\tt set\_scan\_pars(+List)} is used to configure the tokenizer to a
particular need.  {\tt List} is a list of parameters including the
following:

\begin{itemize}

\item{{\tt whitespace}}.  The default action of the scanner is to return a
list of tokens, with any whitespace removed.  If {\tt whitespace} is a
parameter, then the scanner returns the token {\tt ''} when it finds
whitespace separating two tokens (unless the two tokens are letter
sequences; since two letter sequences can be two tokens ONLY if they
are separated by whitespace, such an indication of whitespace would be
redundant.)  Including the parameter {\tt no\_whitespace} undoes the
effect of previously including {\tt whitespace}.

\item{{\tt upper\_case}} The default action of the parser is to treat
lowercase letter differently from uppercase letters.  This parameter
should be set if conversion to uppercase should be done when producing
a token that does {\em not} consist entirely of letters (e.g. one with
mixed letters and digits).  Including the parameter {\tt no\_case}
undoes the effect of previously including {\tt upper\_case}.

\item{{\tt upper\_case\_in\_lit}} The default action of the parser is
to treat lowercase letter differently from uppercase letters.  This
parameter should be set if conversion to uppercase should be done when
producing a token that consists entirely of letters.  Including the
parameter {\tt no\_case\_in\_lit} undoes the effect of previously
including {\tt upper\_case}.

\item{{\tt whitespace(Code)}} adds {\tt Code} as a whitespace code.
By default, all ASCII codes less than or equal to {\tt 32} are
regarded as whitespace.

\item{{\tt letter(Code)}} adds {\tt Code} as a letter constituting a
token. By default, ASCII codes for characters {\tt a--z} and {\tt
A--Z} are regarded as letters.

\item{{\tt special\_char(Code)}} adds {\tt Code} as a special
  character.  By default, ASCII codes for the following characters are
  regarded as special characters:

\begin{verbatim}
| { } [ ] " % $ & ' ( ) * + , - . / : ; < = > ? @ \ ^ _ ~ `
\end{verbatim}
\end{itemize}

\ourmoditem{get\_scan\_pars(-List)}{get\_scan\_pars/1}{stdscan} 
%
{\tt get\_scan\_pars/1} returns a list of the currently active
parameters.

\end{description} 

\section{XSB Lint}
%================================

The {\tt xsb\_lint\_impexp.P} file contains a simple tool to analyze
import/exports along with definitions and uses of predicates.  It
tries to find possible inconsistencies, producing warnings when it
finds them and generating {\tt import}, {\tt document\_import}, and
{\tt document\_export} declarations that might (or might not) be
useful.  It can be used after a large multi-file, multi-module XSB
program has been written to find possible inconsistencies in (or
interesting aspects of) how predicates are defined and used.

We emphasize that the import and export statements generated by {\tt
  checkImpExps/1/2} are suggestions only.  The user is responsible for
determining if they are indeed correct and should be added to the
corresponding source file.  There are situations in which adding such
a generated import declaration may break existing code.

XSB source files that contain an {\tt export} compiler directive are
considered as modules.  Predicates defined in modules, but not
exported, are local to that module.  When compiling a module, the XSB
compiler generates useful warnings when predicates are used but not
defined or defined but not used.  All predicates that are defined in
source files that do not contain an {\tt export} directive are
compiled to be defined in a global module, called {\tt usermod}, and
no undefined/unused warning messages are generated.  The user may add
{\tt document\_export} and {\tt document\_import} compiler directives
(exactly analogous to the {\tt export} and {\tt import} directives) to
non-module source files.  These directives are ignored by the compiler
in terms of code generation, but cause the define-use analysis to be
performed, issuing warning messages as appropriate.  This allows a
user to get the benefit of the define-use analysis without using
modules.  (See Volume 1, Chapter 3 for more details.)

The {\tt xsb\_lint\_impexp} utility processes both modules and regular
XSB source files that may or may not contain {\tt document\_export}
statements.
%
{\tt xsb\_lint\_impexp} is itself a module.  To use it, one may
explicitly call {\tt xsb\_lint\_impexp:checkImpExps(...)}, or may
consult {\tt [xsb\_lint\_impexp]} and then call the {\tt
  checkImpExps/\{1,2\}} predicate.

\begin{description}
  \ouritem{checkImpExps(+Options,+FileNameList)}{checkImpExps/2}{xsb\_lint}
%
{\tt checkImpExps/2} reads all the XSB source files named in the list
{\tt FileNameList}, and all files they reference (recursively), and
produces a listing that describes properties of how they reference
predicates.  All referenced files are found using the XSB {\tt
  library\_directory/1} predicate of directories.  {\tt
  checkImpExps/2} uses {\tt add\_lib\_dir/1/2} directives in code
files to update the directory paths.  The user may also explicitly add
paths by calling {\tt add\_lib\_dir/1/2} before calling this
predicate.

{\tt Options} is a list of atoms (from the following list) indicating
details of how {\tt checkImpExps} should work and the messages it
should produce.

\begin{enumerate}
\item
   {\tt used\_elsewhere}: Print a warning message in the case of a
   predicate defined in a file, not used there, but used elsewhere (in
   a file in {\tt FileNameList} or a recursively referenced
   file). This can be useful to see whether it might be better to move
   the predicate definition to another file, but it produces many
   (irrelevant) warnings for predicates in multi-use libraries.
\item
   {\tt unused}: Print a warning message in the case of a predicate
   that is exported but never used.  This can be useful to see if a
   predicate might be deleted.  Again this option produces many
   (irrelevant) warnings for predicates in multi-use libraries.
\item
   {\tt all\_files}: By default, only predicates in files that contain
   a {\tt :- document\_export} or {\tt :- export} declaration are
   processed for warnings.  This option causes predicates of {\em all}
   files (and modules) to be processed.  This means that usermod code
   files without {\tt document\_export} declarations will have them
   generated.
\item
   {\tt all\_symbol\_uses}: Treat {\em all} uses of symbols (even
   constants) as predicate uses for the purpose of generating imports.
   This means that symbols used as functor symbols but not a predicate
   symbols, will be treated as referring to the predicate symbol.
   This can be useful when a program defines its own meta-predicates
   and passes predicate terms to another module to be called.
   However, it can generate spurious messages when a common symbol is
   used as both a predicate and as an unrelated functor symbol.  This
   differs from the default behavior of {\tt checkImpExps/1/2} only in
   that the default does {\em not} consider 0-ary functor symbols as
   predicate uses, whereas this option does.
\item
   {\tt no\_symbol\_uses}: Don't treat any purely functor uses of
   symbols as predicate uses for the purpose of generating imports.
   This means that a term that appears in an argument position (i.e.,
   not as a called predicate) will {\em not} be considered as a use of
   the predicate symbol at the root of the term.  Only symbols that
   are called (or appear in meat-argument positions of system-defined
   meta-predicates) will be considered as used.
\end{enumerate}

The final two options allow the user to control {\em predicate usage
  analysis} -- analysis of when a symbol $s$ might be used as a
predicate symbol.  We term occurrences of $s$ as a body literal of a
rule or in a callable argument in a meta-predicate, as {\em strict
  predicate contexts}.  The default behavior of this library is as
follows.
\begin{itemize}
\item if $s$ is a constant symbol, the predicate usage analysis of $s$
  is restricted to strict predicate contexts.
\item if $s$ is not a constant symbol, the predicate usage analysis of
  $s$ is based on non-strict predicate contexts.  I.e, {\em all}
  occurrences of $s$ count as predicate contexts.
\end{itemize}
Predicate usage analysis can be restricted to strict predicate
contexts for all symbols by the option {\tt
  no\_symbol\_uses}. Alternatively analysis can be expanded so that
non-struct predicate contexts are used for all symbols (including
constants) by the option {\tt all\_symbol\_uses}.


\ourmoditem{checkImpExps(+FileNameList)}{checkImpExps/1}{xsb\_lint}
%
{\tt checkImpExps/1}  is
currently equivalent to {\tt checkImpExps([],FileNameList)}.
\end{description}

\section{Set Processing and Meta-programming with prolog\_db}
\index{code authors!Warren, David S.}

The {\tt prolog\_db} library provides predicates that support a form
of ``pure'' meta-programming in XSB.  A programmer can create a term
data structure that represents a Prolog database (i.e., a set of
rules, and herein called a {\em Prolog DB} (PDB)), and then ask for a
goal to be proved in such a PDB.  Since a PDB is a Prolog term, it is
the value of a Prolog variable that is restored on backtracking,
unlike other XSB facilities for storing program clauses.  A Prolog DB
can also be used to manipulate sets of facts (or rules), as it
provides efficient set operations of union, intersection, subset, etc.

A PDB represents an unordered set of clauses using a trie, which can
be seen as a ground Prolog term.  Each branching point in the trie is
implemented by a variant of a radix tree.  Because a set of clauses is
canonically represented it does not matter what sequence of operations
(such as {\tt assert\_in\_db}'s, {\tt retractall\_in\_db}'s, or {\tt
  union\_db}'s) that one uses to construct a particular set of
clauses: the resulting PDBs (i.e., Prolog terms) are
identical.~\footnote{Since a Prolog DB is a term and must be passed as
  an argument through the predicates that use it, it differs from
  other implementations of tries in XSB, such as interned tries
  (cf. Chapter~\ref{chap:tries}), or trie-indexed facts
  (cf. Section~\ref{sec:assert}), both of which are persistent.}
Furthermore, since a PDB represents an unordered set of clauses, the
order in which clauses are returned from {\tt clause\_in\_db} (and
thus for {\tt call\_in\_db}) is indeterminate.


A PDB term is automatically interned (or hash-consed) by all
predicates operating on this data type, i.e., it is copied to a global
space where all sub-terms are stored only once.  PDBs are then
accessed through their ``names,'' which are Prolog atoms that encode
the terms' addresses in the global space.  (See {\tt term\_name/2}.)
Only these atomic names are passed through the Prolog program that
uses these PDBs, so they can be tabled efficiently.

PDB's, understood as representing sets of clauses, can be combined
using the binary set operations of union, intersection, and set
difference, and they can be compared for subset, size, and
disjointness.  These operations take advantage of the trie (and radix
tree) structure of the PDB representation, using common substructures
of the input PDB's where possible, to be quite efficient.  Thus it can
be effective to maintain distinct databases and then combine them with
set union, intersection, or difference.  Also this is a good
representation for sets of terms where an efficient subset operation is
required, for example for use with aggregation and answer subsumption.

The most important predicates provided by the {\tt prolog\_db}
interface are described next.  More extensive documentation, including
examples and other useful predicates, is available in the file
distributed as:

{\tt <XSB\_INSTALL\_DIR>/lib/prolog\_db\_doc/prolog\_db.pdf}

\begin{description}
\ourmoditem{empty\_db(-EmptyPrologDB)}{empty\_db/1}{prolog\_db} {\tt
  empty\_db/1} returns an empty PDB.  It is used to create an initial
PDB to pass to other in\_db predicates.

\ourmoditem{assert\_in\_db(+Clause,+DB0,-DB)}{assert\_in\_db/3}{prolog\_db}
{\tt assert\_in\_db/3} adds the clause, {\tt Clause}, to the PDB, {\tt
  DB0}, and returns a new PDB, {\tt DB}.  {\tt Clause} may be a list
of clauses, in which case they will all be added.  A PDB is a {\em
  set} of clauses, so asserting a clause that is already in {\tt DB0}
just returns that same database.  No ordering of clauses is preserved,
so cuts do not make sense and cannot be used in clauses.  (The if-then
else ('->'/3) should be used instead.)

\ourmoditem{retractall\_in\_db(+ClauseHead,+DB0,-DB)}{retractall\_in\_db/3}{prolog\_db}
{\tt retractall\_in\_db/3} removes all clauses whose heads unify with
{\tt ClauseHead} from {\tt DB0}, returning the result in {\tt DB}.  If
no clauses in {\tt DB0} unify, then {\tt DB} is simply {\tt DB0}
unchanged. {\tt ClauseHead} may be a list of clause heads, in which
case they each in turn will be used to remove clauses from {\tt DB0}.

\ourmoditem{clause\_in\_db(?ClauseHead,?ClauseBody,+DB)}{clause\_in\_db/3}{prolog\_db}
{\tt clause\_in\_db/3} returns all clauses in {\tt DB} whose heads and
bodies unify with {\tt ClauseHead} and {\tt ClauseBody}, respectively.
Note that, unlike {\tt clause/2} in Prolog, {\tt clause\_in\_db} can
be called with {\tt ClauseHead} as a variable.  Note also that the
order of clauses is not preserved and is indeterminate.

\ourmoditem{call\_in\_db(?Goal,+DB)}{call\_in\_db/2}{prolog\_db}
{\tt call\_in\_db/2} calls {\tt Goal} in {\tt DB} and returns all
instances of {\tt Goal} provable by rules and facts in {\tt DB}.
Clauses must not contain cuts (!).  They can contain most Prolog
constructs, including and, or, if-then-else, \textbackslash+, calls to
standard predicates, and calls explicitly modified by a module
name. Such calls will be satisfied by calling the goal in the
indicated module. So in this case one can think of a PDB as
being extended by the code in any available module.

\ourmoditem{load\_in\_db(+FileName,+DB0,-DB)}{load\_in\_db/3}{prolog\_db}
{\tt load\_in\_db/3} reads the clauses from the file named {\tt
FileName} and asserts them into database {\tt DB0} returning {\tt DB}.

\ourmoditem{load\_in\_db(+FileName,-DB)}{load\_in\_db/2}{prolog\_db}
{\tt load\_in\_db/2} reads the clauses from the file named {\tt
FileName} and asserts them into an empty database returning {\tt DB}.

\ourmoditem{union\_db(+DB1,+DB2,?DB3)}{union\_db/3}{prolog\_db} {\tt
  union\_db/3} returns in {\tt DB3} the union of the sets of clauses
in {\tt DB1} and {\tt DB2}.  This operation takes advantage of the
structure of the representations of the input {\tt DBs} to share
representations where possible.  For example, if no clauses for a
given predicate appear in one of the input {\tt DBs}, then the output
{\tt DB} will fully share the representation of that predicate from
the other {\tt DB} (and not even look at its contents.)

\ourmoditem{intersect\_db(+DB1,+DB2,?DB3)}{intersect\_db/3}{prolog\_db}
{\tt intersect\_db/3} returns in {\tt DB3} the intersection of the
sets of clauses in {\tt DB1} and {\tt DB2}.  Again, the algorithm
looks at only required portions of the input tries, and the result
shares representations where possible.

\ourmoditem{difference\_db(+DB1,+DB2,?DB3)}{difference\_db/3}{prolog\_db}
{\tt difference\_db/3} returns in {\tt DB3} the set difference of the
sets of clauses in {\tt DB1} and {\tt DB2}.  I.e., the result set
contains clauses in {\tt DB1} but not in {\tt DB2}.

\ourmoditem{sym\_diff\_db(+DB1,+DB2,?DB3)}{sym\_diff\_db/3}{prolog\_db}
           {\tt sym\_diff\_db/3} returns in {\tt DB3} the symmetric
           difference of the sets of clauses in {\tt DB1} and {\tt
             DB2}.  I.e., the result set contains clauses in {\tt DB1}
           but not in {\tt DB2} and clauses in {\tt DB2} but not in
           {\tt DB1}.

\ourmoditem{subset\_db(+DB1,+DB2)}{subset\_db/2}{prolog\_db}
{\tt subset\_db/2} succeeds if every clause in {\tt DB1} is in {\tt
  DB2}, i.e., considered as sets of clauses, {\tt DB1} is a subset of
{\tt DB2}.

\ourmoditem{disjoint\_db(+DB1,+DB2)}{disjoint\_db/2}{prolog\_db}
{\tt disjoint\_db/2} succeeds if no clause is in both {\tt DB1} and
{\tt DB2}, i.e., considered as sets of clauses, {\tt DB1} and {\tt
  DB2} are disjoint sets.

\end{description}

\section{Range Trees}
\index{code authors!Warren, David S.}

This library contains predicates that provide support for range
queries via {\em range trees}.  Range trees store a set of keys and
associated values.  Arbitrary ranges of keys (along with their
associated values) can be retrieved efficiently; such keys do not need
to be numeric..  Such a library is needed because all engine-level
indexes in XSB are hash-based and as such do not support efficient
range queries.

%Range trees can be used directly to implement key-value stores that
%provide efficient range access.  
Given a set of ground facts to be range-indexed on various arguments,
several predicates support the easy construction of range trees from
these facts.  This basic interface includes predicates to create, add
to, retrieve from, and delete from range trees.

Range trees use a balanced sort tree, similar to a B-Tree in that all
leaf nodes are equidistant from the root, and nodes are at least half
full. (Deletion of key-value pairs is supported, but trees are not
rebalanced on delete, so significant use of delete can seriously
degrade performance.)  Range trees are stored in dynamic predicates 
% in the Prolog database 
and are identified by user-provided handles.

%The user defines a predicate {\tt range\_tree\_compare/2} that defines
%the comparison operation ({\tt =<}) that is used for the range
%ordering.  If no such predicate is defined by the user in {\tt
%  usermod}, Prolog's {\tt @@=<} is used.  If the user defines
%{\tt range_tree_compare/2}, she must also define predicates
%{\tt range_tree_min_value/1} and {\tt range_tree_max_value/1},
%giving minimum and maximum values that can ever appear in a range.
%").

{\bf Example}
A user may have a predicate {\tt data(Key,Val1,Val2,Val3)}
and want to be able to do efficient range-valued queries for the
second field, {\tt Val1}, or for the fourth field, {\tt Val3}.  {\tt
  Key} is a key for the {\tt data/4} relation, i.e., no two different
tuples have the same value for the {\tt Key} fields.  The user would
make the following definitions: (See the specific predicate
documentation below for these range predicates to understand their
parameters in detail.)

\begin{verbatim}
:- import range_call/4, range_assert/3, range_retractall/4 from range_trees.

% to retrieve data by range from a range-indexed predicate
range_data(K,RV1,V2,RV3) :- range_call(data(K,_,V2,_),[1],[2,4],[RV1,RV3]).

% to add data to range-indexed predicate
assert_data(K,V1,V2,V3) :- range_assert(data(K,V1,V2,V3),[1],[2,4]).

% to delete data from range-indexed predicate
retractall_data(K,RV1,V2,RV3) :- 
      range_retractall(data(K,_,V2,_),[1],[2,4],[RV1,RV3]).
\end{verbatim}	   

With these definitions, the user will (init) and then add all data to
{\tt data/4} using {\tt assert\_data/4}, which will both assert a
tuple to {\tt data}/4 and add the tuples to the necessary range trees
to support efficient range queries on the second or fourth argument.

A range query to {\tt data/4} now is posed by calling the
user-defined predicate {\tt range\_data/4}.  For example to retrieve
all tuples whose second field is between 4000 and 5000 (inclusive),
the user would pose the query:

\begin{verbatim}
| ?- range_data(K,V1:[4000,5000],V2,V3).
\end{verbatim}

 This query will bind {\tt K,V1,V2,V3} to quadruples in the
{\tt data/4} relation where {\tt V1} is between 4000 and 5000
(inclusive.)  Ranges are indicated by terms of the form
{\tt Var:[Low,High]}, where {\tt Var} is the variable that will be
bound on return, and {\tt Low} and {\tt High} are ground values
indicating the lower and upper bounds of the desired range,
respectively.  The order of answers returned to a range query is
indeterminate (i.e., not necessarily in increasing order on the range
variable.)

One may also use a query such as:
\begin{verbatim}
| ?- range_data(K,4015,V2,V3).
\end{verbatim}
\noindent
which is treated as a range query with the same lower and upper
bounds.  For the declarations shown above, which indicate two
range-indexed fields for {\tt data/4}, one may also pose a query:

\begin{verbatim} 
| ?- range_data(K,V1,V2,V3:[6015,7000]).  
\end{verbatim} 

\noindent
 which would efficiently retrieve data/4 tuples whose fourth
field is between 6015 and 7000 (inclusive.)

When multiple range-indexed arguments are given ranges (or constants)
in a range query, only the first will be used for indexing.


\begin{description}

\ourmoditem{init\_range\_tree(+TreeId)}{init\_range\_trees/1}{range\_trees}
This predicate initializes a named tree which will provide access to
key-value pairs through range queries on the keys.  {\tt TreeId} is an
arbitrary user-supplied ground term that identifies the particular
tree.

\ourmoditem{get\_from\_range\_tree(+TreeId,+Lo,+Hi,?Key,-Val)}{get\_from\_range\_tree/6}{range\_trees}
Gets a range of Keys (and their associated values) between {\tt Lo}
and {\tt Hi} (inclusive), using the ordering defined by {\tt
  range\_tree\_compare/2}.  {\tt TreeId} is a user-provided tree
identifier.  
%{\tt Key} is the value of a key in the tree in the Lo-Hi
%range, and Val is its associated value..

\ourmoditem{add\_to\_range\_tree(+TreeId,+Key,+Val)}{add\_to\_range\_tree/3}{range\_trees}
  Adds a key-value pair to a range tree.

\ourmoditem{delete\_from\_range\_tree(+TreeId,+Key,?Val)}{delete\_from\_range\_tree/3}{range\_trees}
Deletes all key-value pairs with the given {\tt Key} and {\tt Val}
from the range tree.  It does not re-balance the range tree, so after
many deletes the tree may give bad performance.

\ourmoditem{delete\_from\_range\_tree(+TreeId,+Key)}{delete\_from\_range\_tree/2}{range\_trees}
%
Deletes all key-value pairs with the given {\tt Key} from the range
tree.  It does not re-balance the range tree, so after many deletes
the tree may give bad performance.

\ourmoditem{delete\_range\_tree(+TreeId)}{delete\_range\_tree/1}{range\_trees}
Deletes everything from the named tree (i.e., deletes the tree).

\ourmoditem{delete\_all\_range\_trees}{delete\_all\_range\_trees/0}{range\_trees}
Deletes all range trees, reinitializing everything.

%\ourmoditem{print\_rt/1, {range\_trees}
%print\_re(+TreeId) prints out the range tree
%with ID TreeId, in indented format.  Mainly for debugging..

\ourmoditem{range\_call(Goal,KeyPosList,RangePosList,RangeFormList)}{range\_call/4}{range\_trees}
  Calls {\tt Goal} using range indexing specifications in {\tt
    RangeFormList}, which binds variables in Goal in positions {\tt
    RangePosList}.  For example, to call a predicate {\tt data/4},
  whose key is field 1 and which has range indexes on fields 2 and 4,
  one could call:

\begin{verbatim}
| ?- range_call(data(K,X,Y,Z),[1],[2,4],[_,_:[4000,4500]]).
\end{verbatim}

This will efficiently return all triples of the stored predicate {\tt
  data/4} whose fourth field is in the range 4000-4500.  It is assumed
that range indexes have been built for the second and fourth fields of
{\tt data/4} (normally by using {\tt range\_assert/3}.)  This
predicate is intended to be used by the user to define a predicate
that can be used to get range-indexed access to another data
predicate, as in the example above.

The {\tt KeyPosList} is the list of positions in {\tt Goal} that
provide a key to the base predicate of {\tt Goal}.  That predicate
should be indexed on this key.  (If it is not a key or if it is not
indexed on this key, there may be serious degradation of performance.)
The {\tt RangeFormList} is a list of range specifications, i.e., terms
of form {\tt Var:[Low,High]}, or constants or variables.  The first in
this list that has a value or range-specification (i.e., not a
variable) will be used for range-indexing.  The {\tt RangePosList} is
the (corresponding) list of argument positions in {\tt Goal} that are
range-indexed.  These argument position lists must correspond to those
used in {\tt range\_assert/3}. 

\ourmoditem{range\_retractall(Goal,KeyPosList,RangePosList,RangeFormList)}{range\_retractall/4}{range\_trees}
%
Removes all tuples in the database that would be retrieved by a call
to {\tt range\_call/4}, with the same arguments.  This also updates
the range-indexes for this predicate.  Notice that this supports
efficient retraction through use of range-restricted arguments in
{\tt Goal}.  Note also, however, that since range trees are not
rebalanced after deletion, heavy use of this predicate may cause
performance degradation.  

\end{description}

\section{Set and Bag Aggregation} \label{sec:set-aggregation}
\index{random sampling}
\index{sampling!random}

The {\tt c\_aggregate} library provides aggregation operators over the
solutions of a predicate. The operations are a generalisation of the
{\tt bagof/3}, {\tt setof/3} and {\tt findall/3} built-in
predicates. The defined aggregation operations are counting, computing
the sum, minimum, maximum, a bag of solutions and a set of
solutions. \footnote{The {\tt c\_aggregation} library was originally
  written for Quintus Prolog, later extended and modified for SWI
  Prolog and finally included in the Prolog commons library. The
  documentation in this section follows the documentation in the
  Prolog Commons source file.}

We first give a simple example that computes the country with the
smallest area using a predicate that relates countries to their areas.

\begin{verbatim}        
smallest_country(Name, Area) :-
     aggregate(min(A, N), country(N, A), min(Area, Name)).
\end{verbatim}

\noindent
The four predicates in {\tt c\_aggregate} are as follows.
\begin{description}
\ourrepeatmoditem{aggregate(Template,Distinguish,Goal,Result)}{aggregate/4}{c\_aggregate}
\ourrepeatmoditem{aggregate(Template,Goal,Result)}{aggregate/3}{c\_aggregate}
\ourrepeatmoditem{aggregate\_all(Template,Distinguish,Goal,Result)}{aggregate\_all/4}{c\_aggregate}
\ourmoditem{aggregate\_all(Template,Goal,Result)}{aggregate\_all/3}{c\_aggregate}
%
Broadly speaking, these predicates compute an aggregate function,
specified in {\tt Template} based on solutions to {\tt Goal} and store
the result in {\tt Result}.

There are four aggregation predicates, distinguished by
two properties.

\begin{itemize}
\item Free variable quantification ({\tt aggregate/[3,4]} vs. {\tt
  aggregate\_all/[3,4]})
\bi
\item 
    The aggregate predicates use {\tt setof/3} ({\tt aggregate/4}) or
    {\tt bagof/3} ({\tt aggregate/3}), so they expect any existential
    variable {\tt Var} to be quantified: i.e., connected to the goal
    via a {\tt \verb|^|/2} operator as in {\tt Var\verb|^|Goal}.  As a result,
    multiple solutions may be provided by backtracking through
    bindings of the remaining free variables in {\tt Goal}.
%    
\item The {\tt aggregate\_all/3} predicate uses {\tt findall/3},
  implicitly treating all free variables as existential and providing
  exactly one solution. {\tt aggregate\_all/4} uses {\tt sort/2} over
  solutions and Distinguish (see below) generated using {\tt
    findall/3}.
  \ei
\item The {\tt Distinguish} argument.
\bi
  \item The {\tt aggregate/4} and {\tt
  aggregate\_all/4} each provides a {\tt Distinguish} argument that allows
  duplicate bindings of a variable in the result.
  For example, if we wish to compute the total population of all
  countries in our knowledge base, we do not want to lose results
  because two countries have the same population.  Therefore we use
  {\tt Name} as a distinguishing argument.
\begin{verbatim}
	aggregate(sum(P), Name, country(Name, P), Total)
\end{verbatim}
\ei
\end{itemize}  
\end{description}

All aggregation predicates support the following basic operators in
{\tt Template}.
\bi
\item  {\tt count}
	Count number of solutions.  Same as {\tt sum(1)}.
\item {\tt sum(Expr)}
	Sum of {\tt Expr} for all solutions.
\item  {\tt min(Expr)}
	Minimum of {\tt Expr} for all solutions.
\item {\tt min(Expr, Witness)}
%
       A term {\tt min(Min, Witness)}, where {\tt Min} is the minimal
       version Of {\tt Expr} over all solutions and {\tt Witness} is
       any other template applied to the solution that produced {\tt
         Min}.  If multiple solutions provide the same minimum,
       {\tt Witness} corresponds to the first solution.
\item  {\tt max(Expr)}
	Maximum of {\tt Expr} for all solutions.
\item  {\tt max(Expr, Witness)}
	As {\tt min(Expr, Witness)}, but producing the maximum result.
\item  {\tt set(X)}
	An ordered set with all solutions for {\tt X}.
\item  {\tt bag(X)}
	A list of all solutions for {\tt X}.
\ei

In addition, all aggregation predicates allow {\tt Template} to be an
arbitrarily named compound term each of whose arguments is a basic
operator.  For instance, the compound term {\tt minmax(min(X),
  max(X))} computes both the minimum and maximum binding for {\tt X}.
The choice of the functor {\tt minmax} is arbitrary, the term {\tt
  foo(min(X), max(X))} in {\tt Template} would produce results the
same min and max values.

%=======================================================================
\section{Ordered Sets}

In the {\tt ordset} library, sets are represented by ordered lists
with no duplicates.  Thus the set {\tt \{c,r,a,f,t\}} would be
represented as \verb|[a,c,f,r,t]|.  The ordering is defined by the
\verb|@<| family of term comparison predicates
(Section~\ref{Comparison}), which is the ordering used by {\tt sort/2}
and {\tt setof/3}.

The benefit of the ordered representation is that the elementary set
operations can be done in time proportional to the sum of the argument
sizes rather than their Product.  Some of the unordered set routines,
such as {\tt member/2}, {\tt length/2}, {\tt select/3} can be used
unchanged.  The main difficulty with the ordered representation is
remembering to use it.\footnote{The ordsets library was originally
  written for Edinburgh Prolog by Richard O'Keefe.  The simple {\tt
    jacquard\_similarity/2} and {\tt jacquard\_distance/2} predicates
  were added in the XSB version.}

In the documentation below, the term ordset simply means an ordered
list, and the length of an ordset is simply its length as an ordered
list.

\begin{description}
\ourmoditem{is\_ordset(+Term)}{is\_ordset/1}{ordsets}
%
  Succeeds if {\tt Term} is an ordset, and executes in time
  $\cO(length(Term))$.
  
\ourmoditem{list\_to\_ordset(+List,?Set)}{list\_to\_ordset/2}{ordsets}
%
Succeeds if {\tt Set} is the ordered representation of {\tt List}, and
executes in time $\cO(length(List)log(length(List)))$

\ourrepeatmoditem{ord\_add\_element(+Set1,+Elt,?Set2)}{ord\_add\_element/3}{ordsets}
\ourmoditem{ord\_del\_element(+Set1,+Elt,?Set2)}{ord\_del\_element/3}{ordsets}
%
Adds or deletes an element to an ordset in  $\cO(length(Set1))$
 time.

 \ourrepeatmoditem{ord\_intersect(+Set1,+Set2)}{ord\_intersect/2}{ordsets}
\ourmoditem{ord\_disjoint(+Set1,+Set2)}{ord\_disjoint/2}{ordsets}
%
Succeeds when {\tt Set1} and {\tt Set2} have an element in common/
don't have an element in common.  Both execute in $\cO(length(Set1) +
length(Set2))$.

\ourmoditem{ord\_intersection(+Set1,+Set2,?Intsct)}{ord\_intersection/3}{ordsets}
%
Succeeds when {\tt Intsct} is the interesection of {\tt Set1} and {\tt
  Set2}, and executes in $\cO(length(Set1) + length(Set2))$ tome.

\ourmoditem{ord\_intersection(+Set1,+Set2,?Intsct,?Diff)}{ord\_intersect/4}{ordsets}
%
Succeeds when {\tt Intsct} is the interesection of {\tt Set1} and {\tt
  Set2}, and {\tt Diff} are their disjoint elements.  Executes in
$\cO(length(Set1) + length(Set2))$ time.

\ourmoditem{ord\_intersection(+Sets,?Intsct)}{ord\_intersect/2}{ordsets}
%
Succeeds when {\tt Intsct} is the intersection of all sets in the list
{\tt Sets}.  Executes in time proportional to the sum of the lengths
of all the sets in {\tt Sets}.

\ourmoditem{ord\_seteq(+Set1,+Set2)}{ord\_seteq/2}{ordsets}
%
Succeeds if {\tt Set1} and {\tt Set2} are equal. Executes in
$\cO(length(Set1))$.

\ourmoditem{ord\_setproduct(+Set1,+Set2,?SetProduct)}{ord\_setproduct/3}{ordsets}
%
Succeeds if {\tt SetProduct} is the product of {\tt Set1} and {\tt
  Set2}.\\  Executes in $\cO(length(Set1)length(Set2))$.
  
\ourmoditem{ord\_subset(+Set1, +Set2)}{ord\_subset/2}{ordsets}
%
Succeeds if {\tt Set1} is a subset of {\tt Set2}; executes in
$\cO(length(Set1))$.

\ourmoditem{ord\_subtract(+Set1, +Set2, ?Diff)}{ord\_subtract/3}{ordsets}
%
Succeeds if {\tt Diff} is the {\tt Set1} - {\tt Set2}; executes in
$\cO(length(Set2))$.

\ourmoditem{ord\_symdiff(+Set1, +Set2, ?Diff)}{ord\_symdiff/3}{ordsets}
%
Succeeds if {\tt Diff} is the symmetric difference of {\tt Set1} and
{\tt Set2}; executes in $\cO(length(Set1) + length(Set2))$.

\ourmoditem{ord\_union(+Set1, +Set2, ?Union)}{ord\_union/3}{ordsets}
%
Succeeds if {\tt Union} is the union of {\tt Set1} and {\tt Set2}; executes in
$\cO(length(Set1)+length(Set2))$.

\ourmoditem{ord\_union(+Set1, +Set2, ?Union, ?New)}{ord\_union/4}{ordsets}
%
Succeeds when {\tt Union} is the union of {\tt Set1} and {\tt Set2},
and {\tt New} is {\tt Set2} - {\tt Set1}.  This is useful if you are
accumulating members of a set and you want to process new elements as
they are added to the set; executes in $\cO(length(Set1) + length(Set2))$.

\ourmoditem{ord\_union(+Sets, +Set2)}{ord\_union/2}{ordsets}
%
Succeeds when {\tt Union} is the union of all the sets in {\tt Sets}. 
Executes in time proportional to the sum of the lengths
of all the sets in {\tt Sets}.

\ourrepeatmoditem{jaccard\_similarity(+Set1,+Set2,-Coeff)}{jaccard\_similarity/3}{ordsets}
\ourmoditem{jaccard\_distance(+Set1,+Set2,-Coeff)}{jaccard\_distance/3}{ordsets}
%
Computes the jaccard similarity of two sets:

\[ \frac{|Set1 \cup Set2)|}{|Set1 \cap Set2|}\]

which is a real number between 0 and 1.  Jaccard distance is 1 -
jaccard similarity.
\end{description}

\section{Further Unicode Support: {\tt xp\_unicode}}
\index{Unicode!UTF-8}

As discussed in Section~\ref{standard}, XSB supports UTF-8 unicode in
its I/O and stream handling predicates, in standard predicates like
{\tt atom\_codes/2} and {\tt atom\_chars/2} and in libraries like {\tt
  prpc\_files} that provides I/O for files in {\tt csv} and similar
formats.  Beyond this basic support the {\tt xp\_unicode} library
provides further unicode support for capitalizing and lowering unicode
characters and for finding the caregory of a character.

{\tt xp\_unicode} is in fact a wrapper for Python functions and
methods available in its standard libraries such as {\tt unicodedata}.
It therefore will only work if XSB has been configured with Python
support for {\tt xsbpy}..

The extra functionality provided by this initial version of {\tt
  xp\_unicode} is limited, but it is expected to expand.

\begin{description}

  \ourpylibitem{unicode\_upper(+Lower,-Upper)}{unicode\_upper/2}{xp\_unicode}
  %
  Capitalizes any applicable Unicode characters in the atom {\tt Upper}.

\ourpylibitem{unicode\_lower(+Upper,-Lower)}{unicode\_lower/2}{xp\_unicode}
  Lower-cases any applicable Unicode characters in the atom {\tt Upper}.

\ourpylibitem{unicode\_capitalize(+Input,-Capitalized)}{unicode\_capitalize/2}{xp\_unicode}
Capitalizes the first Unicode character in {\tt Input}.

\ourpylibitem{unicode\_char\_category(+Char,-Cat)}{unicode\_char\_category/2}{xp\_unicode}
Obtains the category {\tt Cat} of any unicode character {\tt Char}:
e.g., {\tt Lu} for an uppercase letter, {\tt Sz} for a space, and so on.

\end{description}

\section{Miscellaneous Predicates}

\begin{description}

\ourmoditem{term\_hash(+Term,+HashSize,-HashVal)}{term\_hash/3}{machine}
%
Given an arbitrary Prolog term, {\tt Term}, that is to be hashed into
a table of {\tt HashSize} buckets, this predicate returns a hash value
for {\tt Term} that is between {\tt 0} and {\tt HashSize -1}.

\ourmoditem{crypto\_hash(+Type,+Input,-Output)}{crypto\_hash/3}{machine}
%
Given an atom {\tt Input}, produces an encrypted string {\tt Output}
according to the algorithm specified in {\tt Type}, which currently
can be {\tt sha1} or {\tt md5}.

\ourrepeatmoditem{pretty\_print(+ClausePairs)}{pretty\_print/1}{pretty\_print}
\ourmoditem{pretty\_print(+Stream,+ClausePairs)}{pretty\_print/2}{pretty\_print}



%
The input to {\tt pretty\_print/1}, {\tt ClausePairs}, can be either a
list of clause pairs or a single clause pair.  A clause pair is either
a Prolog clause (or declaration) or a pair:
%
\begin{center}
{\tt (Clause,Dict)} 
\end{center}
%
Where {\tt Dict} is a list of the form {\tt A = V} where {\tt V} is a
variable in {\tt Clause} and {\tt A} is the string to be used to
denote the variable \footnote{Thus the list of variable names returned
by {\tt read\_term/\{2,3\}} can be used directly in {\tt Dict}.}.

By default, {\tt pretty\_print/1} outputs atomic terms using {\tt
writeq/1}, but specialized output can be configured via asserting in
{\tt usermod} a term of the form
%
\begin{center}
{\tt user\_replacement\_hook(Term,Call)}
\end{center}
%
which will use Call to output an atomic literal $A$ whenever $A$
unifies with {\tt Term}.  For example, pretty printing weight
constraints in XSB's {\tt XASP} package is done via the hook
%
\begin{center}
{\tt user\_replacement\_hook(weight\_constr(Term),output\_weight\_constr(Term))}
\end{center}
%
which outputs a weight constraint in a (non-Prolog) syntax that is 
used by several ASP systems.

\ourmoditem{module\_of\_term(+Term,?Module)}{module\_of\_term/2}{machine}
%
Given a term {\tt Term}, {\tt module\_of\_term/2} returns the module
of its main functor symbol in {\tt Module}.  If the module cannot be
determined wither {\tt unknown1} or {\tt unknown2} is returrned,
depending on the reason the module name cannot be determined.

\end{description}


%%% Local Variables: 
%%% mode: latex
%%% TeX-master: "manual1"
%%% End: 
