\section{The Type-1 API} \label{sec:type1query}

The Type-1 interface is radically different than the Type-0 interface.
While the Type-0 interface uses tabling and logical preferences to
return correct answers according to the {\sc inh} proof system, it is
still resolution-based.  However, when the disjunction and negation of
virdual identifiers is added such an approach is no longer possible,
so that the Type-1 query interface is based on computing logican
consistency and entailment.  Since logical entailment of class
expressions can be reduced to consistency, the Type-1 interface is
based on consistency checking of CDF instances that have been
transformed into class expressions.

Consistency checking of class expressions such as those of
Definition~\ref{def:ce} is decidable, but {\em P-space}
complete~\footnote{Assuming a linear encoding of the integers in the
{\em atLeast} and {\em atMost} constructs.  Formally, the CDF prover
is complete for ${\cal ALCQ}$ description logics extended with
relational hierarchies and product classes.}, so that determening
whether a Type-1 instance has a model requires radically different
checking techniques than Type-0 instances.  Query answering for Type-1
instances is performed by using theorem proving techniques.

\subsection{The CDF Theorem Prover}

Specialized theorem-provers are generally implemented to check
consistency of class expressions.  These provers may use based on
structural subsumption techniques (e.g. as used in
CLASSIC~\cite{PMBRB91}, LOOM~\cite{MacB94} and GRAIL~\cite{RBGHNS97});
tableaux construction~\cite{HoPS99}; or stable model
generation~\cite{Swif04} --- in \version{} of CDF a tableaux-style
prover is used.

At a high level, the CDF prover first translates a class expression
$CE$ to a formula $\psi$ in an ontology language according to
Definition~\ref{def:fot}.  It then attempts to construct a model for
$\psi$: if it succeeds $CE$ is consistent, otherwise $CE$ is
inconsistent (since the prover can be shown to be complete).  The CDF
prover has access to the relational and class hierarchies of a CDF
instance during its execution.  As a result, only the principle
classes and relations of an identifier (Definition~\ref{def:redund})
need be entered in class expressions.  Finally, since objects in the
semantics of CDF are indistinguishable from singleton sets, an object
identifier $O$ can be used in any context that a class identifier can
be used.  The prover takes accont of this by enforcing a cardinality
constraint for the set $O$.

The theorem prover of \version{} uses exhaustive backtracking, rather
than the dependency-directed backtracking that is typical of recent
provers such as the DLP prover \cite{}, the FaCT prover \cite{} or the
Racer prover \cite{}.  As a result, the CDF prover may be slow for
certain types of queries relative to these other provers.
Dependency-directed backtracking has not been added to the CDF prover
largely to keep it simple enough to experiment with different
extensions to the types of class expressions it supports~\footnote{In
particular, work is underway on extending the CDF prover to handle
functional attribute chains and concrete domains (see \cite{}).}.  On
the other hand, the CDF prover is relatively efficient on how it
traverses a CDF instance to check consistency.

When a CDF Type-1 instance is checked, the instance is translated into
either a class expression before it can be sent to the CDF prover.
Due to the high worst-case complexity of consistency checking, input
strings to the prover should be kept as small as possible.  The CDF
system accomplishes this by translating information about a given CDF
identifier into a series of local class expressions
(\refsec{sec:lce}), sending a local class expresion to the CDF prover,
then producing and checking other local class expressions as needed.
Since CDF instances differ in philosophy from terminological systems,
they may be expected to be cyclic, so that a given class identifier
may occur in a level $n$ local class expression of itself.

\subsection{The Type-1 API}

\begin{description}

\ourpreditem{checkIdConsistency/1}{cdftp\_chkCon}
In {\tt checkIdConsistency(IdList)} {\tt IdList} is a (list of) class
or object identifier(s) which is taken as a conjunction.  The
predicate succeeds if {\tt IdList} is consistent in the current CDF
instance.

{\bf Exceptions} {\tt Domain Exception: IdList} is not a class
identifier, an object identifier, or a list of class or object
identifiers.

%-----------------------------------------------------------------------------
\mycomment{
/* The algorithm is described in detail in the CDF system paper.  What
we do here is to prove consistency of an identifier by an iterative
process.  Given an identifier {\tt Id}, a local class expression is
constructed for Id, and a consistency check made for that class
expression.  In other words, we prove the consitency of Id by trying
to construct a model in which Id is a non-empty set if it is a cid, or
a non-empty unique set if Id is an oid.

Local class expressions dont contain all information for an
identifier.  Accordingly, in the model constructed for Id we need to
check the *contexts* for each individual in the model, i.e. if an
individual i belongs to classes C1 and C2 in our model we must ensure
that a model can be constructed for both C1 and C2.  The checker thus
traverses through all the contexts in the model and checks them
recursively.

An important issue occurs if a check for an identifier recursively
leads to a context in which the identifier itself is present.  If this
is the case, we succeed, as it can be shown that the identifier is
consistent.  If an identifier Id depends on itself negatively, we
fail, as we cannot be sure of constructing a model in this case.  A
more elaborate algorithm would take into account even and odd loops,
but that seems a little arcane for our purposes.

This code doesn't use XSBs tabling for two reasons.  First, we want to
succeed on positive loops, and second, we only want a single solution
for each consistency check.  In this homespun tabling, information is
entered about whether a context we are traversing has been queried,
and whether it has succeeded or failed if it is complete.  Once a
consistency check succeeds, all of its choice points are cut away.
Success on positive loops is addressed by passing around an ancestor
list and performing an ancestor check at each call to the sat routine.
If the context is in the ancestor list we succeed, otherwise we call
the sat routine (which succeed or fail on table check).  Note that we
do not need to table the ancestor list -- it is not a set of
assumptions, its just used to succeed on loops.  Also, since all code
requires only a single solution for any consistency check do not have
to worry about incomplete tables that are not in the ancestor list.

Various cases.  
1) Not called before 
2) Called but incomplete 
3) Complete, succeed or fail 

checkIdConsistency_1 handles case 2).  
Cases 1) and 3) are handled by checkIdConsistency_1
*/
}
\ourpredmoditem{consistentWith/2}{cdftp\_chkCon}
In {\tt consistentWith(Id,CE)}, {\tt Id} can either be a class or an
object identifier and {\tt CE} is a class expression.  This predicate
checks whether {\tt CE} is logically consistent with all that is known
about {\tt Id} in the current CDF instance.  {\tt consistentWith/2}
determines whether there is a model of the current CDF instance that
satisfies the expression {\tt Id,CE}.

This predicate assumes that all class and object identifiers in a
given CDF instance are consistent.

{\bf Exceptions} 

{\tt Domain Exception: Id} is not a class or object identifier.

{\tt Domain Exception: CE} is not a well-formed class expression.

\ourpredmoditem{allModelsEntails/2}{cdftp\_chkCon}
In {\tt allModelsEntails(Id,CE)}, {\tt Id} is a class or object
identifier and {\tt CE} is a class expression. {\tt
allModelsEntails/2} succeeds if {\tt CE} is entailed by what is known
about {\tt Id} in the current CDF instance.  In other words, {\tt
allModelsEntails/2} determines whether in all models of the current
CDF instance, if an element is in {\tt Id} then it is also in {\tt
CE}.  It does this by checking the inconsistency of {\tt Id,CE}.

This predicate assumes that all class and object identifiers in a
given CDF instance are correct.

{\bf Exceptions} 

{\tt Domain Exception: Id} is not a class or object identifier.

{\tt Domain Exception: CE} is not a well-formed class expression.

\ourpredmoditem{localClassExpression/3}{cdftp\_chkCon}
In {\tt localClassExpression(+IdList,+N,-Expr)} {\tt IdList} is a list
of class identifiers, and {\tt N} is a positive integer.  In its
semantics, {\tt IdList} is interpreted as a conjunction of
identifiers, and upon success, {\tt Expr} is a class expression,
unfolded to depth {\tt N}, that describes {\tt IdList} according to
gthe current CDF instance.

{\bf Exceptions} 

{\tt Domain Exception: IdList} is not a class identifier or object
identifier, or a list of class or object identifiers.

{\tt Type Exception: N} is not a positive integer.

{\tt Instantiation Exception: Expr} is not a variable.

\ourpredmoditem{check\_lce/2}{cdftp\_chkCon}
In the goal {\tt check\_lce(+IdList,+N)} {\tt IdList} is a list of
class identifiers, and {\tt N} a positive integer.  In its semantics,
{\tt IdList} is interpreted as a conjunction of identifiers, and {\tt
check\_lce(+IdList,+N)} pretty-prints a class expression, unfolded to
depth {\tt N}, that describes {\tt IdList} according to the current
CDF instance.

{\bf Exceptions} 

{\tt Domain Exception: IdList} is not a class identifier or object
identifier, or a list of class or object identifiers.

{\tt Type Exception: N} is not a positive integer.

\end{description}
