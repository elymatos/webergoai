\section{CDF and Description Logics}
\label{sec:comp} 

\begin{example} \rm \label{ex:classexpr}
Consider the (rather abstract) CDF instance: 
\begin{verbatim}
     isa(cid(a),cid(b))                  isa(cid(a),cid(c))
     hasAttr(cid(a),rid(r1),cid(e))      allAttr(cid(a),rid(r2),cid(f)).
\end{verbatim}
%     isa(b,d).
%     isa(f,g).                  hasAttr(f,r3,h)
The axioms for this instance are logically equivalent to 
%----------
\begin{tabbing}
fooo\=foo\=foo\=foo\=fooooooooooooooooooooooooooooooo\=ooooooooooooo\=\kill
\> $isClass(a) \wedge isClass(b) \wedge 
	(\forall X)[elt(X,a) \Ra elt(X,b)] $ \\
\> $isClass(a) \wedge isClass(c) \wedge 
	(\forall X)[elt(X,a) \Ra elt(X,c)] $ \\	
\> $isClass(a) \wedge isRel(r1) \wedge isClass(e) \wedge 
     (\forall X)[elt(X,a) \Ra \exists Y.(rel(X,r1,Y) \wedge elt(Y,e))]$ \\
\> $isClass(a) \wedge isRel(r2) \wedge isClass(f) \wedge 
     (\forall X)[elt(X,a) \Ra \forall Y.(rel(X,r1,Y) \wedge elt(Y,f))]$ 
\end{tabbing}
%----------
which in turn are logically equivalent to
%----------
\begin{tabbing}
fooo\=foo\=foo\=foo\=fooooooooooooooooooooooooooooooo\=ooooooooooooo\=\kill
$isClass(a) \wedge isClass(b) \wedge isClass(c) \wedge isClass(e) 
	\wedge isClass(f) \wedge isRel(r1) \wedge isRel(r2)  \wedge $ \\
$(\forall X)[elt(X,a) \Ra$ \\
\> \> $(elt(X,b) \wedge  elt(X,c) \wedge$ \\
\> \> $(\exists Y)[rel(X,r1,Y) \wedge elt(Y,e)] \wedge
      (\forall Y)[rel(X,r2,Y) \Ra elt(Y,f)]]$ 
\end{tabbing}
%----------
Semantics for description logics are formulated in various ways.
In~\cite{Swif04}, the semantics for ${\cal ALC}$-style description
logics are given in terms of first order formulas over an ontology
language similar to that used in Definition~\ref{def:ontolang}.  Using
the translation of \cite{Swif04} the class expression
\[ b \sqcap c \sqcap exists(r2,f) \sqcap all(r2,f) \]
translates into the formula.
%----------
\begin{tabbing}
fooo\=foo\=foo\=foo\=fooooooooooooooooooooooooooooooo\=ooooooooooooo\=\kill
\>  $(\exists X)[ $ 
$ elt(X,b) \wedge elt(X,c)  \wedge $ \\
\> \> $(\exists Y)[rel(X,r1,Y) \wedge elt (Y,e)] \wedge 
       (\forall Y)[rel(X,r2,Y) \Ra elt(Y,f)]]$
\end{tabbing}
%----------
\end{example}

Clearly there is a strong similarity between the meaning of certain
class expressions and certain CDF instances.  To formalize this
similarity, we use the definition of an {\em atomic relational CDF
instance}.
\begin{definition}
Let $\cO$ be a CDF intance.  $\cO$ is {\em atomic} if it contains no
product identifiers.  $\cO$ is a {\em relational instance} if it
contains no object identifiers, and no {\tt classHasAttr/3}
predicates.
\end{definition}

%-------------------------
\mycomment{
Next, note that, ignoring sorting predicate, each instance axiom for
an atomic relatoinal class instance $\cO$ has the form: $\forall
X.(elt(X,C) \Ra Template)$.  Thus, $\cO^{\cI}$ is equivalent to the
set of formulas
\begin{tabbing}
fooo\=foo\=foo\=foo\=fooooooooooooooooooooooooooooooo\=ooooooooooooo\=\kill
\> $\forall X.(elt(X,C_1) \Ra Template_1) 
       \wedge \ldots \wedge\forall X.(elt(X,C_n) \Ra Template_n) $ 
\end{tabbing}
Where $\{C_1 \ldots C_n\}$ is the set of class identifiers in the
first argument of each fact in $\cO$.  
}
%-------------------------

We also need to formalize the language of class expressions that we
will use.

\begin{definition}
The syntax of an \omsdl{} class expression has the following form, in
which $A$ is an atomic class identifier, $R$ a relation identifer, $N$
a non-negative integer, and $C_1$ and $C_2$ \omsdl{} class expressions.
\[ C \leftarrow A | C_1 \sqcap C_2 | all(R,C_1) | exists(R,C_1) 
	| atLeast(N,R,C) | atMost(N,R,C) \]
\end{definition}

If $C$ is an \omsdl{} class expression, it can be translated according
to Definition 1.3 of \cite{Swif04}, into a first order sentence
(denoted as $C^{\cT}$) over what are called first-order description
logic languages.  These languages contain slightly different predicate
symbols from $\cL$, so we make use of a function $f$ from structures
over ontology languages to structures over first-order description
logic languages~\footnote{Given a structure $\cM$ over an ontology
language, $\cL_{CDF}$, we construct a language $f(\cL_{CDF})$ by
setting the set of atomic class names in $f(\cL_{CDF})$ equal to the
class identifiers in $\cL_{CDF}$, and the atomic relation names in
$\cL'_D$ equal to the relation identifiers in $\cL_{CDF}$.  A new
structure $f(\cM)$ is constructed by restricting $\cM$ to $elt/2$
and $rel/3$ predicates.}.

In the following theorem, $TH(\cO)/4$ denotes $TH(\cO)$ minus Core
Axiom~\ref{ax:contained} (Domain Containment).  A model $\cM$ is
$C$-reified for a class expression $C$ if for each sub-expression $C'$
of $C$, there is a constant $c'$ such that $isClass(c')$ and $\cM
\models (\exists X)[elt(X,c')]$ and $C'^{\cT}$ holds for all $d$ such
that $\cM \models elt(d,c)$.   $c'$ is called a witnessing constant for
$C$.

\begin{theorem}
Let $C$ be an \omsdl{} class expression.  Then 
\begin{enumerate}
\item There exists a $C$-reified model $f(\cM)$ for $C$.  
\item There is an atomic CDF instance $\cO$ such that 
\begin{enumerate}
\item For any $C$-reified model $f(\cM)$ of $C$, there is a CDF instance
$\cO$ such that $\cM \models TH(\cO)/4$
\item For any model $\cM$ of $TH(\cO)/4$, $f(\cM) \models C^{\cT}$.
\end{enumerate}
\end{enumerate}
\end{theorem}
\begin{proof}
The proof is contained in the appendix.
\end{proof}

Atomic CDF class instances thus have an expressive power that is
equivalent to a weak description logic.  However general CDF class
instances can also specify axioms about objects and their attributes,
as well as making use of non-inheritable relations and product
identifiers.  From Theorme~\ref{thm:poly}, however, consistency of an
CDF instance can be checked in polynomial time, while consistency
checking for ${\cal ALC}$ is {\em P-space} complete~\cite{ShmS91}.
CDF instances can be considered as ``special-cases'' of class
descriptions that can be extended into more elaborate descriptions if
needed, but otherwise are efficient for consistency checking,
subsumption checking and other operations.

%------------------------------------------------
\mycomment{
However, CDF can be compared to FLORA facts, for which it is somewhat
simpler.  CDF does not provide for a constraint that a given attribute
is functional as is allowed in FLORA, although this extension can be
provided if CDF instances are extended to have unqualified number
restrictions

Also, CDF does not provide for non-monotonic inheritance, but
substitutes the monotonic inheritances described in
Section~\ref{sec:inheritance}.  }
