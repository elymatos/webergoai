
\chapter{{\tt curl}: The XSB Internet Access Package}

\begin{center}
  {\Large {\bf By Aneesh Ali}}
\end{center}



\section{Introduction}

The {\tt curl} package is an interface to the {\tt libcurl}  library,
which provides access to most of the standard Web protocols.
The supported protocols include
FTP, FTPS, HTTP, HTTPS, SCP, SFTP, TFTP, TELNET, DICT, LDAP, LDAPS, FILE,
IMAP, SMTP, POP3 and RTSP. Libcurl supports SSL certificates, HTTP
GET/POST/PUT/DELETE, FTP uploading, HTTP form based upload, proxies,
cookies, user+password authentication (Basic, Digest, NTLM, Negotiate,
Kerberos4), file transfer resume, HTTP proxy tunneling etc. 
The \texttt{curl} package of XSB supports a subset of that functionality,
as described below. 


The {\tt curl} package accepts input in the form of URLs and Prolog
atoms. To load the {\tt curl} package, execute the following query in the
XSB shell or loaded file:
%%
\begin{verbatim}
 ?- [curl].  
\end{verbatim}
%%
The {\tt curl} package is integrated with file I/O of XSB in a transparent
fashion and for many purposes Web pages can be treated just as yet another
kind of a file. We first explain how Web pages can be accessed
using the standard file I/O feature and then describe other predicates,
which provide a lower-level interface.

\section{Integration with File I/O}

The {\tt curl} package is integrated with XSB File I/O so that a web page
can be opened as any other file.Once a Web page is opened, it can be read
or written just like the a normal file.


\subsection{Opening a Web Document}\label{sec-web-open}

Web documents are opened by the usual predicates {\bf see/1}, {\bf open/3},
{\bf open/4}.

\begin{description}

%  \index{\texttt{see/1}}
%\item[see({\it url}({\it +Url}))]\mbox{}
%  \index{\texttt{see/1}}
%\item[see({\it url}({\it +Url,Options}))]\mbox{}
%  \index{\texttt{open/3}}
%\item[open({\it url}({\it +Url}), {\it +Mode}, {\it -Stream})]\mbox{}
%  \index{\texttt{open/4}}
%\item[open({\it url}({\it +Url}), {\it +Mode}, {\it -Stream}, {\it +Options})]\mbox{}

  %  \index{\texttt{see/1}}
%\item[see({\it url}({\it +Url}))]\mbox{}
%  \index{\texttt{see/1}}
%\item[see({\it url}({\it +Url,Options}))]\mbox{}
%  \index{\texttt{open/3}}
%\item[open({\it url}({\it +Url}), {\it +Mode}, {\it -Stream})]\mbox{}
%  \index{\texttt{open/4}}
%\item[open({\it url}({\it +Url}), {\it +Mode}, {\it -Stream}, {\it +Options})]\mbox{}

\ourrepeatmoditem{see(url(+Url))}{see/1}{curl}
\ourrepeatmoditem{see(url(+Url,Options))}{see/1}{curl}
\ourrepeatmoditem{open(url(+Url),+Mode,-Stream)}{open/3}{curl}
\indourmoditem{open(url)(+Url),+Mode,-Stream,+Options)}{open/4}{curl}
\\

{\it Url} is an atom that specifies a URL.
\emph{Stream} is the file stream of the open file.
{\it Mode} can be
\textbf{read}, to create an input stream, or
{\bf write},  to create an output stream.
For reading, the contents of the Web page are cached in a temporary file.
For writing, a temporary empty file is created. This file is posted to the
corresponding URL at closing.

The {\it Options} parameter is a list that controls loading. Members of that list can be of the following form:

  \begin{description}
  \item[{\bf redirect}{\bf (}{\it Bool}{\bf )}]\mbox{}
    \\
    Specifies the redirection option. The supported values are true and
    false. If true, any number of redirects is allowed. If false,
    redirections are ignored.
    The default is true.

  \item[{\bf secure}{\bf (}{\it CrtName}{\bf
    )}]\mbox{}
    \\
    Specifies the secure connections (https) option. \emph{CrtName} is the name of the file holding one or more certificates to verify the peer with. 

  \item[{\bf auth}{\bf (}{\it UserName, \it Password}{\bf )}]\mbox{}\\Sets the username and password basic authentication.

  \item[{\bf timeout}{\bf (}{\it Seconds}{\bf )}]\mbox{}\\Sets the maximum time in seconds that is allowed for the transfer operation.

  \item[{\bf user\_agent}{\bf (}{\it Agent}{\bf )}]\mbox{}\\Sets the User-Agent: header in the http request sent to the remote server.

  \item[{\bf header}{\bf (}{\it String}{\bf )}]\mbox{}\\This allows one
    to specify an HTTP header. Several \texttt{header(...)} options can be
    specified in the same list. Specifying the headers is useful mostly
    when closing
    Web pages that are open for writing, which corresponds to POST HTTP
    requests.

  \end{description}


\end{description}

\subsection{Closing a Web Document}

Web documents opened by the predicates {\bf see/1}, {\bf open/3}, and {\bf
  open/4} above must be closed by the predicates {\bf close/2} or {\bf
  close/4}. The stream corresponding to the URL is closed. If the stream
was open for writing, the data written to the stream is POSTed
to the URL, which corresponds to HTTP POST.
If writing is unsuccessful for some reason,
a list of warnings is returned.

\begin{description}

%\index{\texttt{close/2}}
%\item[close({\it +Source, +Options})]\mbox{}
\ourrepeatmoditem{close(+Source, +Options)}{close/2}{curl}
%\index{\texttt{close/4}}
%\item[close({\it +Source, +Options, -Response, -Warnings})]\mbox{}
\indourmoditem{close(+Source, +Options, -Response, -Warnings)}{close/4}{curl}
  \\
These versions of \texttt{close} are typically used for sources that
are open for writing. URL-streams open for reading can be closed using the
usual \texttt{close/1} predicate. 
{\it Source} is of the form {\tt url({\it {url-string}})},
where \emph{url-string} must be an atom. {\it Options} is a list
of options like those for the open predicate of Section~\ref{sec-web-open}.
If the HTTP server returns a response, the \emph{Response} variable is
bound to that string. \emph{Warnings} is a list of possible warnings. If
everything is fine, this list is empty.  
Closing often requires the \texttt{header(...)} option because it
is often necessary to specify \texttt{Content-Type} and other header
attributes when posting to a Web site. 

\end{description}


\section{Low Level Predicates}

This section describes additional predicates provided by the {\tt
  curl} packages, which extend the functionality provided by the file I/O
integration.

\subsection{Loading Web Documents}

Web documents are loaded by the predicate {\bf load\_page/5}, which has
many options. The parameters of this predicate are described below.


\begin{description}
%  \index{\texttt{load\_page/5}}
%\item[load\_page({\it +Source, +Options, -Properties, -Content, -Warn})]\mbox{}
\indourmoditem{load\_page(+Source,+Options,-Properties,-Content,-Warn)}{load\_page/5}{curl}
  \\
  {\it Source} is of the form {\tt url({\it {url}})}.
  The document is returned in {\it Content}.
  {\it Warn} is bound to a (possibly empty) list of warnings generated during the process.

  {\it Properties} is bound to a \emph{list} of properties of the document. They
  are {\it Page size}, {\it Page last modification time}, and \emph{Redirection
  URL}.  
  The {\tt load\_page/5}  predicate caches a copy of the Web page that it
  fetched from the Web in a local file, which is identified by the URL's
  directory, file name portion, and  its file
    extension. The first two parameters indicate the size and the last
  modification time of the fetched Web page.
  The last parameter, \emph{Redirection URL}, is the source URL, if no
  redirection happened or, if the original URL was redirected then this
  parameter shows the final URL.
  The directory and the file name 
  The \emph{Options} parameter is the same as in
  Section~\ref{sec-web-open}.

  \texttt{load\_page} has additional options that can appear in the
  \emph{Options} list:
  \begin{description}
  \item[{\tt post\_data}{\bf (}{\it String}{\bf )}]\mbox{}\\
    This allows one to post data (HTTP POST)
    to a web page and is an alternative to
    posting by opening-writing-closing URLs, which was described above.

    If several \texttt{post\_data(...)} options are given, only the last one is
    used. This option often goes with the \texttt{header(...)} option because it
    is often necessary to specify \texttt{Content-Type} and other header
    attributes.  

    If this option is specified with an \texttt{open/4} predicate, it is
    ignored.  If it is specified in the \texttt{close/2} or 
    \texttt{close/4}   predicate, \emph{String} is posted instead of what was
    written to the closed stream. This goes to say that the \texttt{post\_data}
    option in \texttt{open} and \texttt{close} predicates makes little sense. 
    \item[{\tt put\_data}{\bf (}{\it String}{\bf )}]\mbox{}\\
    This allows one to put data (HTTP PUT) to a web page.

    If several \texttt{put\_data(...)} options are given, only the last one is
    used. This option often goes with the \texttt{header(...)} option because it
    is often necessary to specify \texttt{Content-Type} and other header
    attributes.  

    If this option is specified with an \texttt{open/4},
    \texttt{close/2}, or \texttt{close/4} predicate, it is
    ignored.
  \item[\tt delete] \mbox{} \\
    This sends a DELETE HTTP request to the server.
  \end{description}

\end{description}

\subsection{Retrieving Properties of a Web Document}

The properties of a web document are loaded by the predicates {\bf
  url\_properties/3} and {\bf url\_properties/2}. 

\begin{description}
%  \index{\texttt{url\_properties/3}}
%\item[url\_properties({\it +Url, +Options, -Properties})]\mbox{}

\indourmoditem{url\_properties(+Url,+Options,-Properties)}{url\_properties/3}{curl}
  \\
  The {\it Options} and {\it Properties} are same as in {\bf load\_page/5}:
  a \emph{list} of properties of the document, which
  are {\it Page size}, {\it Page last modification time}, \emph{RedirectionURL} 
  in that order. If the original page has no redirection then
  \emph{RedirectionURL} is the same as \emph{Url}. 
  Some Web servers will not report page sizes or modification times (or
  both) in which case they appear as -1.
%  \index{\texttt{url\_properties/2}}
%\item[url\_properties({\it +Url, -Properties})]\mbox{}
\indourmoditem{url\_properties(+Url,-Properties)}{url\_properties/2}{curl}
  \\
  This uses the default options (\texttt{secure(false)}, \texttt{redirect(true)}). 

\end{description}

\subsection{Encoding URLs}

Sometimes it is necessary to convert a URL string into something that can
be used, for example, as a file name. This is done by the following
predicate.

\begin{description}
%  \index{\texttt{encode\_url/2}}
%\item[encode\_url({\it +Source, -Result})]\mbox{}
\indourmoditem{encode\_url(+Source,-Result)}{encode\_url/2}{curl}
  \\
{\it Source} has the form {\it url(url-string)}, where
\emph{url-string} is an atom.
{\it Result} is bound to a list of components of the URL:
the URL-encoded {\it Directory Name}, the URL-encoded {\it File Name}, and the {\it
  Extension} of the URL.

\end{description}


%%\subsection{Obtaining the Redirection URL}
%%
%%If the originally specified URL was redirected, the URL of the page that
%%was actually fetched by {\tt load\_page/5} can be found with the help of
%%the following predicate: 
%%
%%\begin{description}
%%  \index{\texttt{get\_redir\_url/2}}
%%\item[get\_redir\_url({\it +Source, -UrlNew})]\mbox{}
%%  \\
%%  {\it Source} can be of the form {\tt url({\it {url}})}, {\tt file({\it {file%%name}})} or a string.
%%
%%\end{description}

\section{Installation and configuration}

The {\tt curl} package of XSB requires that the {\tt libcurl} package is
installed.  For Windows, the {\tt libcurl} library files are included with
the installation. For Linux and Mac, the {\tt libcurl} and {\tt
  libcurl-dev} packages need to be installed using a suitable
package manager (e.g., deb or rpm in Linux, Homebrew in Mac). In some
systems, {\tt libcurl-dev} might be
called {\tt libcurl-gnutls-dev} or {\tt libcurl-openssl-dev}.  In addition,
the release number might be attached, as in {\tt libcurl4} and {\tt
  libcurl4-openssl-dev}.

The {\tt libcurl} package can also be downloaded and
built manually from
%% 
\begin{quote}
  \url{http://curl.haxx.se/download.html} 
\end{quote}
%% 
To configure {\tt curl} on Linux, Mac, or on some other Unix variant,
switch to the {\tt XSB/build} directory and type

%%
\begin{verbatim}
    cd XSB/packages/curl
    ./configure
    ./makexsb
\end{verbatim}
%%

%%% Local Variables: 
%%% mode: latex
%%% TeX-master: "manual2"
%%% End: 
