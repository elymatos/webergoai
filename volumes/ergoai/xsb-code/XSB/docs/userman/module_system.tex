% attempt to re-document the XSB module system

\index{modules}
\section{The Module System of XSB} \label{Modules}

XSB has been designed as a module-oriented Prolog system.  Modules
provide a step towards {\em logic programming ``in the large''} that
facilitates the construction of large programs or projects from
components that are developed, compiled and tested separately.  Also,
module systems support the principle of information hiding and can
provide a basis for data abstraction.  And modules form the basis for
XSB's implementation of its standard predicates, and for the on-demand
loading of standard and user predicates.

The module system of XSB is {\em file based} -- one module per file --
and {\em flat} -- modules cannot be nested.  In addition, XSB's module
system is essentially {\em structure-symbol-based}, where any compound
symbol in a module can be imported, exported or be a local symbol, as
opposed to a {\em predicate-based} one where this can be done only for
predicate symbols.~\footnote{Operator symbols can be exported as any
  other symbols, but their precedence must be redeclared in the
  importing module.}  In XSB, every structure symbol (and thus
structured term) is associated with a module, and structure symbols
with the same name but in different modules are different symbols and
thus do not unify.  On the other hand, XSB does not require a
meta-predicate to be explicitly declared as such in order to be used
in a module.  As we will discuss, these differences lead to certain
incompatibilities with the predicate-based module systems supported by
most other Prologs.  Although the differences are important, XSB's
module system is quite often compatible with those of other Prologs:
for instance, XSB's module system is able to support the Prolog
Commons libraries.

In Prolog systems a predicate definition (aka a set of rules) may be
associated with a structure symbol.  In other words, a predicate
symbol is just a structure symbol with an associated definition.
Accordingly, predicates are either static or dynamic depending on how
they are defined.  Static predicates get their definitions from source
code files that are compiled and loaded into memory.
The definition of a static predicate cannot be changed by adding ar
deleting clauses using {\tt assert/1} or {\tt retract}, or similar
predicates.
A dynamic
predicate $P$ may also occur in a compiled source file if there is a
dynamic declaration for $P$, of if $P$ is in a file that is loaded as
dynamic code via {\tt load\_dyn/[1,3]} or similar predicates.  $P$ is
also dynamic if rules for $P$ have been asserted to the Prolog data
store.  Dynamic predicates can have their definitions changed by {\tt
  assert/1} and {\tt retract/1}.

In XSB every structure symbol is associated with a module -- not just
predicates.  A term is said to be in the module of its main structure
symbol.  Terms in different modules are different terms and do {\bf
  not} unify.  So two terms whose main structure symbols (or any
structure symbols) have the same name but different modules, are
different terms and do not unify.  So, for example, terms printed as
{\tt p(a,b)} and {\tt p(a,b)} would not unify if the first structure
symbol named {\tt p/2} is in a different module from the second
structure symbol named {\tt p/2}. \footnote{This lack of unification
  also occurs in most other Prologs if the structure symbols are
  predicate symbols that occur in other modules.}

\index{modules!declaration}
So how are terms in a module created?  The most common way
is for a term to occur in a Prolog source file $F_{mod}.P$ that contains
a {\em module declaration} such as

\demo{:- export $Pred/Arity$.}

or

\demo{:- module($F_{mod}$,[$Pred/Arity$]).}

\index{modules!qualification}
\noindent 
where $F_{mod}$ is compiled as static code and loaded into a session,
or loaded as dynamic code into a session.  In either case, the name of
the module is the name of $F_{mod}$, the file name without its extension.
In addition, a file $F_{nomod}.P$ can be loaded into a module as dynamic
code via the appropriate option in {\tt load\_dyn\_gen/2} and similar
predicates.  Finally, a dynamic clause for a predicate $P$ can be
explicitly asserted into a module $M$ using the {\em module
  qualification} syntax for predicates, $M:P$.  For instance, via a
goal like

\demo{assert(M:P)}

\noindent
a rule can be asserted for $P$ in module $M$.  More generally, the
module qualification $M:P$ also causes P to be treated as belonging to
$M$ if $M:P$ occurs as a goal in any source file or input stream.

\index{modules!usermod}
\index{modules!declaration}

A predicate or goal $Pred$ is considered to be in the ``default''
module {\tt usermod} if none of the above situations occur,  i.e., if
$Pred$ does not have a module qualification, does not occur in a file
with a module declaration, and is not loaded into a module as dynamic
code.  Thus a term that has no module qualification when read from an
input stream is put into {\tt usermod}.  In addition, if a term is
created via {\tt functor/3} or {\tt =../2} it is also considered to be
in usermod.

Given a predicate $P$ in a module $M$, $P$ is usually accessed by
exporting it from $M$ and importing it to {\tt usermod} or some other
module via an import directive, such as:

\demo{:- import $Pred/Arity$ from $Module$.}

or

\demo{:- use\_module(Module,$Pred/Arity$).}

\noindent
For example, the file:
\begin{verbatim}
%% file: mod1.P
:- export p/2.

p(a,b).
p(X,Y) :- q(X,Y).

q(b,c).
\end{verbatim}
defines predicates {\tt p/2} and {\tt q/2} in {\tt mod1}.
%(i.e., {\tt p/2} terms
%that define the facts have their main structure symbols put in module
%{\tt mod1}, and the code implementing those clauses are associated
%with that structure symbol in that module.)  It also defines a
%predicate {\tt q/2} in the same module.  (And, of course, the call to
%{\tt q(X,Y)} in the body of the rule for {\tt p/2} is also put in that
%same module.)  The predicate {\tt p/2} is exported and is thus
%available for use by other code.
%
Continuing, we can create another file (not a module in this case),
which uses the definition of {\tt p/2} above:
\begin{verbatim}
%% file: my_code.P
:- import p/2 from mod1.

q(X,Y) :- p(X,Y).
\end{verbatim}
\noindent Here there is no {\tt export} directive, so all definitions
in this file will go into module {\tt usermod}.  The clause here
defines {\tt q/2} in {\tt usermod}, which is a different predicate
from the {\tt q/2} defined above in the module {\tt mod1}.  The {\tt
  import} of {\tt p/2} in this file causes the {\tt p(X,Y)} term in
the body of the rule for {\tt q/2} to be interpreted as being in
module {\tt mod1}.  Thus, {\tt my\_code} is compiled and loaded, {\tt
  q/2} is defined in {\tt usermod} and its code calls {\tt p/2} in
module {\tt mod1}.  Alternately, one could accomplish the same
functionality via:
\begin{verbatim}
%% file: my_other_code.P

q(X,Y) :- mod1:p(X,Y).
\end{verbatim}

A module source file may want to access a predicate defined in {\tt
  usermod}, which can be done by explicitly importing the predicate
from {\tt usermod}, or using {\tt usermod} as part of a module
qualification.

%There are situations in which a programmer wants to explicitly provide
%a module name to ``override'' the module associated with a term.  For
%example, one might want to call the goal {\tt p(X,Y)} to invoke the
%code associated with {\tt p/2} in module {\tt mod1} at the top level,
%%regardless of what module the {\tt p/2} structure symol is associated
%with.  In this case, one can write:
%
%\demo{| ?- call(mod1:p(X,Y)).}
%
%\noindent Here {\tt call} will construct the term {\tt p(X,Y)} with
%structure {\tt p/2} in module {\tt mod1} (ignoring the module associated
%with the {\tt p/2} structure symbol) and then call that term, which
%accesses the code of {\tt p/2} in module {\tt mod1}.  In this
%particular case the original term {\tt mod1:p(X,Y)} had the {\tt p/2}
%structure in {\tt usermod}, since that's where the top-level read puts
%it.  But {\tt call/1} interprets this term (with main structure symbol
%{\tt :/2}) as a coercion of the term {\tt p(X,Y)} into the module {\tt
%  mod1}.  In XSB, in most contexts in which a term is interpreted as a
%goal, the syntax of {\tt Mod:Goal} is interpreted as a coercion of
%term {\tt Goal} into the module {\tt Mod}.  And in fact, the top-level
%goal:
%
%\demo{| ?- mod1:p(X,Y).}
%
%\noindent is equivalent to the goal above.
%
%And instead of:
%\begin{verbatim}
%:- import pr/2 from mod3.
%...
%q(X,Y) :- ... pr(X,Z), ....
%\end{verbatim}
%one can directly write:
%\begin{verbatim}
%q(X,Y) :- ... mod3:pr(X,Z), ....
%\end{verbatim}

%In general, the use of {\tt import} is recommended, even though it may
%sometimes be more verbose.  The use of {\tt imort} allows for better
%visibility and easier analysis of module dependencies.

Returning to the topic of module declarations, in XSB the declaration:

\demo{:- module($filename$,[$sym_1$, \ldots, $sym_l$.]).}

\noindent
is syntactic sugar for:

\demo{:- export $sym_1$, \ldots, $sym_l$. }

\noindent
as long as the $filename$ is the same as the name of the file in which
it was contained.  Similarly,

\demo{:- use\_module($module$,[$sym_1$, \ldots, $sym_l$.]).}

\noindent
is treated as semantically equivalent to 

\demo{:- import $sym_1$, \ldots, $sym_n$ from $module$. }

\noindent
Accordingly, {\tt use\_module/2} and {\tt module/1} can be used
interchangeably with {\tt import/2} and {\tt export/1}.  However the
declaration

\demo{:- use\_module($module$).}

\noindent
which is often used in other Prolog systems, is {\em not} equivalent
to an XSB import statement, as each XSB import statement must
explicitly declare a list of predicates that are used from each
module.  Such a declaration will raise a compilation error. The main
reason for this is that the compiler needs to associate a module with
each structured term and this is facilitated by listing the imported
predicates explicitly.  In addition, including specific predicates in
an import statement facilitates the loading of modules on demand.

Finally, the declaration 

\demo{:- import $sym/n$ from $module$ as $sym'/n$. }

\noindent
allows a predicate to be imported from a module, but renamed as
$sym'/n$ within the importing module.  In this case the structure
symbol $sym'/n$ is placed in the current module and its code pointer
is identified with that of the structure symbol $sym/n$ in module
$module$.  Such a feature is useful when porting a library written for
another Prolog (e.g. a constraint library) to XSB.  As shown in the
next section, is also useful when one wants to import two predicates
with the same name from different modules.  In that case (at least)
one of the names needs to be changed on import.

For modules, the base file name is stored in its byte-code ({\tt
  .xwam}) file, so that renaming a byte-code file for a module may cause
problems, as the renaming will not affect the information within the
byte-code file.  (But see below for ways to support modules with
names different from the files that contain their code.)
However, byte-code files generated for non-modules
can be safely renamed.

\subsection{How the Compiler Determines the Module of a Term}

When XSB compiles a source code file, it must determine the module for
every term it encounters.  For non-module source files (i.e., those
lacking a module declaration), all terms are associated with {\tt
  usermod} except for those whose structure symbols are imported.  Any
occurrence of an imported structure symbol is associated with the
module from which it is imported.

For a module source code file $M$, i.e., where $M$ contains at least
one module declaration, the process of determining the module of a
structure symbol $t/n$ is more complicated.  The idea is that all
terms in $M$ that refer to predicates are placed in the module of the
file and all terms that are non-predicate structures are by default
placed in {\tt usermod}.  All occurrences of $T/N$ in $M$ are in a
file are normally associated with the same module.~\footnote{but see
  {\tt import .. as ..}  for an exception.}  So if $T/N$ appears both
as a predicate symbol and as a non-predicate structure (e.g., as an
argument in a body literal, both occurrences will be associated with
the current module.  Of course, imported structure symbols are
associated with the module from which they are imported.

The compiler recognizes as predicate symbols any symbol that:
\begin{enumerate}
\item appears as the main structure symbol in the head of a rule,

\item appears as a subgoal in the body of a rule,

\item appears as the main structure symbol of terms passed to known
  meta-predicates.  In \version{} these are listed in
  Table~\ref{tab:meta-predicates}.

\item is declared as {\tt dynamic}.
\end{enumerate}

\begin{table}[tbp]\centering{\tt
\begin{tabular}{lll}
abolish/2       & abolish\_table\_subgoals/[1,2] & abolish\_table\_subgoal/[1,2] \\
assert/1        & asserta/1                    & assertz/1\\
bagof/3         & call/[1-10]                  & call\_tv/1 \\ 
call\_cleanup/2 & catch/3                      & clause/2 \\ 
fail\_if/1      & findall/[3,4]                & forall/2 \\ 
not/1           & once/1                       & retract/1 \\ 
retractall/1    & setof/3                      & table\_once/1 \\ 
time/1          & tnot/1                       & \verb|\+|/1 \\ 
\verb|;|/2      & \verb|/|/2                   & \verb|->| /2   \\ 
do\_all/[1,2]   &                              &
\end{tabular}}
\caption{Meta-predicates known to XSB's Compiler}\label{tab:meta-predicates}
\end{table}

Otherwise a structure symbol is associated with {\tt usermod}.

Note that these rules imply that all structure symbols used solely as
non-predicate structures are placed by default in {\tt usermod}.  This
is usually what a user wants.  But there are times a user might want a
non-predicate structure to be associated with the current module --
for instance if a meta predicate is not among those listed in
Table~\ref{tab:meta-predicates}.  The programmer can tell the
compiler to place a particular structure symbol in the current module
by using the {\tt local} directive:

\demo{:- local $Sym/Arity$.}

\noindent which will force all uses of the indicated structure symbol
to be associated with the current module. \footnote{The low-level
  builtin {\tt term\_new\_mod/3} can be used to explicitly coerce a
  term into an arbitrary module, without using module qualification.}

%Associating a non-predicate
%structure with the module in which it appears also can be used to
%provide a measure of information hiding: Since no other module (or
%usermod) will construct a term associated with this module, another
%module can't use unification to look at the subfields inside such a
%term.  So in this way, a module can return to a caller a complex term,
%and the caller can pass it around and back to the module in a later
%call, and only the module code can manipulate thatq data
%structure.\footnote{The hiding is only partial, since other code can 
% use {\tt functor/3} or {\tt univ/2} to look inside such terms.  Also
%  the very low-level builtin {\tt term\_new\_mod/3} can be used to
%  explicitly coerce a term into an arbitrary module.}

An XSB programmer can also export a structure symbol (that is not used
as a predicate), and others can import and use it as a structure
symbol.

Standard predicates (those defined in the XSB environment) are
actually defined in system modules and the compiler implicitly
provides the necessary imports to allow the programmer to use them.
Standard predicates are described in Section \ref{sec:standard}.

For clarity, we state a few consequences of the above discussion.
\begin{itemize}
\item The module for a particular symbol appearing in a module must be
  uniquely determined.  As a consequence, a symbol of a specific
  $functor/arity$ generally cannot be declared as both exported and
  local, or both exported and imported from another module or declared
  to be imported from more than one module, etc.  Such environment
  conflicts are detected at compile-time and abort the compilation.
  One exception is the {\tt import ... as ...} directive.  Using this
  directive, one can import $symbol_1$ from a module {\tt as}
  $symbol_2$ and then export $symbol_2$.  (In fact, $symbol_1$ and
  $symbol_2$ are allowed to be the same symbol.)
%
\item If the {\tt import ... as ...} declaration is not used, a module
  {\em cannot} export a predicate symbol that is directly imported
  from another module, since this would require that symbol to be in
  two modules.
%
% Perhaps discuss symbols, if that isn't too confusing.
\index{Prolog flags!{\tt unknown}}
\item If a module {\tt m1} imports a predicate {\tt p/n} from a module
  {\tt m2}, but {\tt m2} does not export {\tt p/n}, nothing is
  detected at the time of compilation.  If {\tt p/n} is defined in
  {\tt m2}, a runtime warning about an environment conflict will be
  issued.  However, if {\tt p/n} is not defined in {\tt m2}, a runtime
  existence error will be thrown~\footnote{This behavior can be
    altered through the Prolog flag {\tt unknown}.}.
\end{itemize}

\subsection{Atoms and 0-Ary Structure Symbols}

XSB uses different internal representations for {\bf atoms} (i.e.,
constant functions) and for {\bf 0-ary structure
  symbols}. \footnote{Prolog terminology denotes non-numeric constants
  as atoms, which conflicts with the definition of an atomic term in
  mathematical logic.}  Atoms cannot have definitions associated with
them (i.e., cannot be predicates) and are not associated with modules.
But 0-ary predicates can and are.

In order to reduce ambiguity, a 0-ary predicate $p/0$ can be
designated as $p()$, in addition to $p$.

XSB automatically coerces atoms to 0-ary structure symbols and vice
versa in a manner analogous to the coercion for non-predicate
structures.
%But when coercing an atom to a 0-ary structure symbol, it
%{\bf always} associates the generated structure symbol with {\tt
%  usermod}.
This can sometimes lead to unexpected results.  All works as expected
for a 0-ary predicate $p/0$ as long as the programmer explicitly
exports and imports $p/0$, uses the $p/0$ predicate in a
meta-predicate known to XSB's compiler, or declares $p/0$ as dynamic.
%However, passing an atom as an argument, and then calling it will always
%call it in {\tt usermod}.

\subsection{Importing and Loading On Demand} 

XSB automatically loads modules in two situations.  Most commonly if
predicates are imported from a module $M$, $M$ is loaded at the first
call to any of these predicates, compiling the file for $M$ if
necessary.  In addition, if a predicate $P$ is not explicitly
imported, but a call to $P$ is explicitly qualified with a module,
e.g., $M:P$, $M$ is located, compiled if necessary, and loaded just as
if it were imported.  See Section~\ref{LibPath} for the details of how
the system finds and processes the appropriate XSB source files.

%\subsection{Consulting a Module}
Usually, access to a predicate $P$ exported from a module $M$ occurs
by means of an import statement or module qualification, as stated
above.  However, to debug code in $M$ it is often convenient simply to
consult it at the top-level and then call $P$ in the same way as if it
were not in a module.
For this reason, such an exported predicate $P$ is defined in
{\tt usermod} in addition to being defined in $M$.  
However, this leads to pollution of names in {\tt usermod}, and so the
consulting (and ensure\_load-ing) of modules is {\em strongly}
discouraged outside of testing contexts.

%(In effect, for exported {\tt p/2}, XSB implements {\tt :- import p/2
%from $module$ as p/2.} in {\tt usermod} to provide direct access to
%{\tt p/2}'s code in $module$ from the {\tt p/2} predicate in {\tt
%usermod}.)  It is bad form to use this property and consult a module
%in an executing program to get access to its exported predicates
%through {\tt usermod}.  One should always explicitly import the
%predicates one wants to use and let the dynamic loader automatically
%load the module code on demand.

\index{usage inference}
\index{static analysis!usage}

% Took out -- redundant.
%\subsection{Dynamically loading predicates in the interpreter}
%%=============================================================
%Modules are usually loaded into an environment when they are consulted
%(see Section~\ref{Consulting}).  Specific predicates from a module can
%also be imported into the run-time environment through the standard
%predicate {\tt import PredList from
%  Module}\index{declarations!\texttt{import/1}}.  Here, {\tt PredList}
%can either be a Prolog list or a comma list.  ({\tt import/1} can also
%be used as a directive in a source module (see
%Section~\ref{Modules}). \index{standard predicates}

We provide a sample session for compiling, automatically loading, and
querying a user-defined module named {\tt quick\_sort}.  For this
example we assume that {\tt quick\_sort.P} is a file in the current
working directory, and contains the definitions of the predicates {\tt
  concat/3} and {\tt qsort/2}, both of which are exported.

{\footnotesize
\begin{verbatim}
             | ?- compile(quick_sort).
             [Compiling ./quick_sort]
             [quick_sort compiled, cpu time used: 1.439 seconds]

             yes
             | ?- import concat/3, qsort/2 from quick_sort. 

             yes
             | ?- concat([1,3], [2], L), qsort(L, S).

             L = [1,3,2]
             S = [1,2,3]

             yes.
\end{verbatim}
}

The standard predicate {\tt import/1} does not load the module 
containing the imported predicates, but simply informs the system 
where it can find the definition of the predicate when (and if) the
predicate is called.
If the module byte-code file is out of date (or non-existent), it will
automatically be compiled before loading.
Thus, in the example above, the {\tt compile(quick\_sort)} command is not
strictly necessary; calling {\tt concat} would cause the compilation
of its module if necessary.

\subsection{Usage Inference and the Module System} \label{sec:useinfer}
In addition to supporting loading on demand, the import and export
statements of a module $M$ are used by the compiler for inferring {\em
  usage} of predicates.  The following properties are determined at
compile time.
\begin{itemize}
\item {\em Undefined Predicates} If a predicate $P/N$ occurs as
  callable in the body of a clause defined in $M$, but $P$ is neither
  defined in $M$ (including via a {\tt dynamic} declaration), nor
  imported into $M$ from some other module, a warning is issued that
  $P/N$ is undefined.  Here {\em occurs as callable,} means that
  $P/N$ is found as a literal in the body of a clause, or within a
  system meta-predicate listed in Table~\ref{tab:meta-predicates}.
  Currently, occurrences of a term inside user-defined meta-predicates
  are not detected as callable by XSB's usage inference algorithm.

    On the other hand, since all modules are compiled independently,
    the usage inference algorithm has no way of checking whether a
    predicate imported from a given module is actually exported by
    that module.
%
\item Alternatively, if $P/N$ is defined in $M$, it is {\em used} if
  $P/N$ is exported by $M$, or if $P/N$ occurs as callable in a clause
  for a predicate that is used in $M$.  The compiler issues warnings
  about all unused predicates in a module.  
\end{itemize}

\index{xsbdoc} \index{declarations!\texttt{document\_export/1}}
\index{declarations!\texttt{document\_import/1}} Usage inference can
be highly useful during code development for ensuring that all
predicates are defined within a set of files, for eliminating dead
code, etc.  In addition, import and export declarations are used by
the {\tt xsbdoc} documentation system to generate manuals for
code.\footnote{Further information on {\tt xsbdoc} can be found in
  {\tt \$XSB\_DIR/packages/xsbdoc}.}  For these reasons, it is
sometimes the case that usage inference is desired even in situations
where a given file is not ready to be made into a module, or it is not
appropriate for the file to be a module for some other reason.  In
such a case the directives {\tt document\_export/1} and {\tt
  document\_import/1} can be used, and have the same syntax as {\tt
  export/1} and {\tt import/1}, respectively.  These directives affect
only usage inference and {\tt xsbdoc}.  A file is treated as a module
if and only if it includes an {\tt export/1} (or {\tt module/2})
  statement, and only {\tt import/1} statements affect automatic
  loading and name resolution for predicates.

\index{pseudo-modules}
\index{modules!pseudo}
\subsection{Importing with Pseudo-Modules}

As discussed, when the module system is used to import predicates,
code files for modules are automatically found and automatically
(compiled and) loaded on first access.  But normally non-module source
files must be explicitly consulted or loaded via{\tt []/2}, {\tt
  ensure\_loaded/[1,2,3]} and so on.  To provide the convenience (and
declarativity) of automatic loading to non-modularized source files,
XSB supports a directive of the form:

\demo{:- import $Pred/Arity$ from usermod($File$).}

\noindent 
In the above directive, {\tt usermod($File$)} is called a {\em
  pseudo-module reference}, and can be used in place of module
references in import statements.  $File$ must be the name of an XSB
source code file that defines $Pred/Arity$ and is {\bf not} a module,
i.e., its predicates will be loaded into {\tt usermod}.  In this case,
when a goal to $Pred/Arity$ is called and does not yet have a
definition, the file $File$ is (compiled and) loaded, and the goal is
then called.  If $File$ is a base file name (without a slash), then the
{\tt library\_directory/1} paths are used to find the correct file as
for normal modules.
%%% ??? If the predicate already has a definition, that
%%one is used.

Pseudo-module imports allow code in non-module files to be treated
somewhat like module files.  In fact this facility, when carefully
used, can eliminate the need for runtime {\tt consult/1} and {\tt
  ensure\_loaded/1} commands.  But, as usual, it is the user's
responsibility to ensure that different imported non-module files do
not define the same predicate.  If $Pred/Arity$ has been defined in
{\tt usermod} prior to the load of $File$, and the prior definition
does not come from $File$, a warning is issued.

Pseudo-module imports can also load dynamic code using a form such as:

%~\footnote{This
%  terminology may be somewhat confusing.  Dynamic loading, as seen,
%  loads code on its first call.  Dynamic code, as opposed to static
%  code, can be modified by {\tt assert/1}, {\tt retract/1} and similar
%  predicates.  Thus, either static code or dynamic code can be loaded
%  dynamically.} 

\demo{:- import $Pred/Arity$ from usermod(load\_dyn($File$)).}

\noindent This will cause the file $File$ to be loaded automatically
on the first call to $Pred/Arity$.  ($File$ must of course define
$Pred/Arity$.)  The {\tt load\_dyn} in this example may be replaced by
any file-loading predicate whose first argument is the name of the
file to load.  For example, one might also use:

\demo{:- import $Pred/Arity$ from usermod(consult:load\_dync($File$,a)).}

\noindent to load a file as dynamic code such that its contents are
canonical terms to be asserted in reverse order.  In fact, one can
also use:

\demo{:- import $Pred/Arity$ from usermod(proc\_files:load\_csv($File$,$Pred/Arity$)).}

\noindent to load a comma-separated file with each line containing two
fields to define the predicate $Pred/Arity$.  (See \ref{sec:procfiles}
for details.)

As an aside, for the purposes of usage inference
(Section~\ref{sec:useinfer}) 

\demo{:- document\_import $Pred/Arity$ from $File$.}

\noindent is equivalent to 

\demo{:- import $Pred/Arity$ from usermod($File$).}
  
\noindent However, the two forms are not equivalent in general: a
pseudo-module import is not considered by {\tt xsbdoc}, and {\tt
  document\_import} does not automatically load a file.

\index{parameterized modules}
\index{modules!parameterized}
\subsection{Parameterized Modules in XSB}

The XSB module system now supports {\em parameterized modules} which
allow a form of higher-order programming that makes it possible to
program some tasks more declaratively -- and modularly.  More
specifically, parameterized modules are higher-order constructs that
use module parameters to represent sets of predicates.  At the same
time, the semantics of parameterized modules is a conservative
extension of the unparameterized module system described in previous
sections. From a software engineering viewpoint, parameterized modules
offer functionality similar to that of code injection in Java and
other languages.

Simply put, any module can be parameterized by other modules.  A
parameterized module is declared by including a directive of the form:

\demo{:- module\_parameters($atom_1$, \ldots, $atom_n$).}

\noindent in the module code file, $M$.  The ground atoms, $atom_1$,
\ldots, $atom_n$, are formal module parameters; when the module is
loaded, those module parameters will be replaced by the actual module
names passed to the load operation.  A parameterized module can also
be declared in the {\tt module} directive; for example:

\demo{:- module($baseModName$($atom_1$, \ldots, $atom_n$), [..export..list]).}

Therefore, the general form of a module name (i.e., a module with its
parameters replaced by names of actual modules) is specified by a
ground term: In other words, a module name has the general form:

\demo{$M(module\_name_1,\ldots, module\_name_n)$}

\noindent
where the main structure symbol specifies the base name module file;
arguments of the module term, if any, indicate the names of the other
modules that parameterize $M$.  Syntactically, each $module\_name_i$
may be an atom if $module\_name_i$ is not parameterized, or a compound
term if $module\_name_i$ is parameterized.  There is no limit on the
nesting of parameterized modules.

As a simple example, consider a module that takes a graph, an initial
node in the graph, and a set of final nodes in the graph, and returns
all final nodes reachable through the graph from the initial node.  A
parameterized module for this task, named {\tt reachability}, is:
\begin{verbatim}
%% file: reachability.P

:- module_parameters(some_graph).
:- export reachable_final/1.
:- import initial/1, move/2, final/1 from some_graph.

reachable_final(F) :- reachable(F), final(F).

:- table reachable/1.
reachable(N) :- initial(N).
reachable(N) :- reachable(P), move(P,N).
\end{verbatim}
This module, whose base name is {\tt reachability}, is parameterized
by an argument module that exports (at least) 3 predicates: {\tt
  initial/1}, {\tt move/2}, and {\tt final/1}.  For {\tt reachability}
to be loaded, the argument parameter {\tt some\_graph} must be
replaced by the name of an actual module that exports the three
required predicates.  I.e, let us assume that a (non-parameterized)
module named {\tt my\_graph} exports those 3 predicates: the actual
module name would be {\tt reachability(my\_graph)}.  When {\tt
  reachable\_final} is called, the {\tt reachability} module will be
loaded and all references to predicates in {\tt my\_graph} will be
resolved.  Thus the predicates imported from {\tt my\_graph} will be
used here in {\tt reachable\_final/1} and {\tt reachable/1}.  The
module {\tt my\_graph} will actually be loaded the first time one of
those predicates is called.

In this case, the module could be loaded by an import statement and
executed:
\begin{verbatim}
| ?- import reachable_final/1 from reachability(my_graph).
| ?- reachable_final(ReachableFinalState).
\end{verbatim}
Or the module could be loaded by an explicit qualification:
\begin{verbatim}
| ?- reachability(my_graph):reachable_final(ReachableFinalState).
\end{verbatim}
Both of these forms return the reachable final states for the problem
parameterized by {\tt my\_graph}.

Note that this example is declarative in the sense that there is no
explicit procedural code necessary to load the code for a particular
example.  All loading is handled by XSB's existing lazy loader.  And
this same search module can be run with many different examples.

Parameterized modules are implemented in XSB as follows.  When a
parameterized module is to be loaded into memory, the formal
parameters in the byte-code file are replaced by the actual parameters
and that code is loaded.  (This is actually done by renaming symbols
as they are loaded, so there is minimal effect on loading time.)  This
implementation has two consequences: 1) the performance of code in
parameterized modules is {\em exactly} the same as if the user had
explicitly written the module with the actual parameter modules, and
2) every instance of a parameterized module has its own version of
code.  So loading a thousand different instances of a parameterized
module will take a thousand times the space of a single instance.  In
practical uses
%(that I know of so far) 
this should not be a significant problem, but it should be kept in
mind.

As an example of this, one could load another instance of {\tt
  reachability} to execute the search algorithm on a different graph
by:
\begin{verbatim}
| ?- reachability(my_other_graph):reachable_final(ReachableFinalState).
\end{verbatim}
This would load two instances of the {\tt reachability} module, the
first loaded:

\demo{{\tt reachability(my\_graph)}.}

\noindent
and the newly loaded:

\demo{{\tt reachability(my\_other\_graph)}.}

\noindent
Both modules would behave exactly as other modules, with their code
accessible to the user as well as to other programs and modules.

So modules in XSB's runtime system can now be identified by module names
that are terms, not simply constants. Accordingly, anywhere a module
name is required, a parameterized module name, i.e., a module term,
can be used.  The module name must be ground at the time it is
required for use in order to resolve specific code; and all structure
symbols and atoms in the structured module identifier must identify
actual files that contain the appropriate module's code; and finally
those files must be on a library path for the session in order to be
found by the XSB loader.

To write well-structured and maintainable code, it is strongly
recommended that all uses of parameterized modules be done through
{\tt use\_module/2} or {\tt import} directives explicitly appearing in
XSB code.  The use of module qualification via {\tt :/2} at runtime
should usually be avoided.  (The sole exception is when the user types
in such a goal on the top-level command line.)  Using only explicit
import directives allows better compile-time analysis of modules and
their dependencies, a factor that can be critical in maintaining large
XSB code bases. This also requires that the run time search paths be
available at compile time, which implies programmer discipline in
changing that predicate as well
(cf. Section~\ref{sec:search-path}).\footnote{As may be obvious, this
  has been learned through much painful experience. -dsw}.

While parameterized modules can be used in many ways, one of the most
important is in the construction of so-called {\em view systems}.  In
the traditional relational database sense, a view is a relational
operator that takes a set of input relations and views, and produces
an output relation.  By composing views one can build large and
complex systems of data transformations in a completely declarative
way.  With such systems, one often receives base (i.e., input) data
from a source, and then wants to apply a view system to that data,
generating the derived views for use in other applications.  One can
do this declaratively by using parameterized modules.  Each module is
a view definition, exporting the view it defines, and importing the
base and view relations it depends on.  These input relations can be
defined in base modules, and a view module is parameterized by the
base modules it depends on.  Then the same view module can be applied
to the particular input tables obtained from a particular source.
Finally, such a view system can be efficiently implemented and
maintained by XSB's incremental tabling
(Section~\ref{sec:incremental_tabling}).

%[-dsw: Provide a simple example of a view system...]

\index{modules!explicit filenames}
\subsection{Modules with Explicit File Names}

Normally, the Prolog code that defines predicates in a module is in a
file whose name is the same as the module name (without parameters, if
any), with a {\tt .P} extension.  However, XSB provides a way for the
programmer to explicitly provide a different name for the file that
contains the code of a module.

Module specifications (in import statements and in goals with an
explicit module qualifier) may explicitly include the name of the file
from which to load the module.  The syntax uses the {\tt '>>'} symbol as
in the following example:
\begin{verbatim}
:- import p/2 from moda>>moda_file.
\end{verbatim}
In this case when {\tt p/2} is loaded on demand, {\tt moda\_file}
(with the Prolog extension added) is used as the name of the
file to load.  (The system uses the same search strategy to find the
file as for module names, as described in Section~\ref{LibPath}.)

N.B. {\em Every} reference to a module whose code is stored in an
explicitly named file {\em must} include the explicit file name.  The
one exception is that when asserting (or retracting) clauses, only the
base module name should be used to qualify the head of a clause.
(This is because no on-demand loading can be involved in such uses).

A subgoal that is qualified by an explicit module name may also use
this form.  For example, the following goal sequence is supported:
\begin{verbatim}
| ?- Mod = moda>>moda_file, Mod:p(X,Y).
\end{verbatim}

Pseudo-module references, such as {\tt usermod(}{\em FileName}{\tt
  )}, can alternatively be written as {\tt usermod>>}{\em
  FileName}{\tt )}.

Parameterized modules and the actual parameters supplied to them may
also have the explicit filename qualifier.  For example, in the {\tt
  reachability} example above, we might store the code for {\tt
  my\_graph} parameter in a file named {\tt my\_graph\_file.P}.  In
this case we could invoke the goal:
\begin{verbatim}
| ?- reachability(my_graph>>my_graph_file):reachable_final(ReachableFinalState).
\end{verbatim}
And if the code for the paramaterized module {\tt reachability} (with
formal parameter {\tt some\_graph}) were itself stored in the file {\tt
  reachability\_file.P}, we would write:
\begin{verbatim}
| ?- reachability(my_graph>>my_graph_file)>>reachability_file:
             reachable_final(ReachableFinalState).
\end{verbatim}

Because of the extra complexity of the syntax of modules with explicit
file names (as is easily seen in this example), it is recommended that
they be used only in situations for which the default file names
cannot be made to work.

The compile command supports a {\tt module(}{\em ModName}{\tt )}
option.  With this parameter, the compiled file will contain byte-code
for the module name {\em ModName}, which will be defined when it is
loaded.  This option is not normally used by a programmer, but is used
internally by the system when compilation is required during on-demand
loading.

%%% Local Variables: 
%%% mode: latex
%%% TeX-master: "manual1"
%%% End: 
