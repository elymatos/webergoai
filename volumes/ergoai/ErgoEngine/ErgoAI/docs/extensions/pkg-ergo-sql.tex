\chapter[Querying SQL Databases]
{Querying SQL Databases\\
  {\Large by Michael Kifer}}


This chapter describes the API for SQL queries against relational
databases.

\section{Connecting to a Database}

The \ERGO-to-SQL API is available in the system module
\texttt{\bs{}sql}
and calling anything \texttt{@\bs{}sql}  will load that module. If, for
some reason, it is necessary to load this module without executing any
operations, one can accomplish this by calling 
%% 
\begin{itemize}
\item  \texttt{ensure\_loaded@\bs{}sql}. 
\end{itemize}
%% 

Prior to performing any operation on an SQL database the user must
\emph{open} a \emph{connection} to that database.
\ERGO supports two database drivers:
%% 
\begin{itemize}
\item  \texttt{odbc}: the general driver to all relational databases that
  support the ODBC protocol. All major database products and open-source
  databases support this protocol.\footnote{
    There have been serious problems with ODBC support on Linux and Mac for
    MySQL server 5.7.
    }
  The user must be familiar with the
  basics of setting up ODBC data sources (called DSNs),
  which specify database drivers and the target databases.
\item \texttt{mysql}: the native driver for MySQL databases (for Linux,
  Mac, Windows (64 bit)). 
\end{itemize}
%% 

The commands to connect to a database for these two drivers are slightly
different.
%% 
\begin{itemize}
\item The ODBC driver:\footnote{
    The ODBC driver for MySQL 5.7 has a number of problems on Linux and
    Mac, so we recommend to use MySQL 5.6, if ODBC is required.
    }
  \\
  \texttt{odbc[open(?ConnectId,?DSN,?User,?Password)]@\bs{}sql}.
  \\
  Here \texttt{?ConnectId} must be bound to a Prolog atom (note: an atom,
  not a variable) that will henceforth identify the connection.
  \texttt{?DSN} must be bound to an ODBC DSN (data source name), and
  \texttt{?User} and \texttt{?Password} must be the user name and the
  password to be used to log into the database---both must be Prolog atoms.   
  \\
  Example:
  \texttt{odbc[open(id1,mydbn,me,mypwd)]@\bs{}sql.}
\item The MySQL driver (Linux, Mac, Windows (64 bit)):\\
  \texttt{mysql[open(?ConnectId,?Server,?Database,?User,?Password)]@\bs{}sql}.
  \\
  \texttt{?Server} must be bound to the address of the desired database server.
  Usually this is an IP address such as 123.45.67.89 (with optional
  port number, e.g., 123.45.67.89:6666) or a domain name, like
  \texttt{abc.example.com} --- again with optional port number. On a local
  machine, the server would usually be just \texttt{localhost}.
  
  The meaning of the other parameters is the same as for the ODBC driver.

  Example: \texttt{mysql[open(id2,localhost,test,me,mypwd)]@\bs{}sql.}  

  Note that one can use the two drivers simultaneously for different
  connections. However, the connection Ids must be distinct whether the
  same or different drivers are used. A connection Id can be \emph{reused}
  if it was previously \emph{closed} (see below).  
\end{itemize}
%% 
When done with the database, it is recommended to close the connection to
that database for two
reasons:
%% 
\begin{itemize}
\item To avoid hitting the limit of 200 on the number of databases that one
  can work with at the same time.
\item  To release the resources allocated by the OS to work with that open
  connection.
\end{itemize}
%% 
The syntax for closing connections is
%% 
\begin{verbatim}
    ?ConnectId[close]@\sql.
\end{verbatim}
%% 
For example,  \texttt{id2[close]@\bs{}sql.}  

\section{Queries}

The \ERGO-to-SQL API provides a simple query interface to send SQL queries
(\texttt{SELECT}), updates (\texttt{INSERT}, \texttt{DELETE}, etc.),
schema definition
(\texttt{CREATE}),  and other commands.
%% 
\begin{itemize}
\item  \texttt{?ConnectId[query(?QueryId,?QueryList,?ReturnList)]@\bs{}sql.}
  \\
  \texttt{?ConnectId} is the Id of a previously open (and not closed)
  connection. \texttt{?QueryId} must be bound to an atom that will
  represent the query statement that will be created as a result of this
  command. \texttt{?QueryList} is a list that must concatenate into a
  Prolog atom that forms a valid
  SQL statement. Components of the list can be variables and terms, and in
  this way the query can be constructed at run time. \texttt{?ReturnList}
  is a list of variables that must correspond to the list of items in the
  \texttt{SELECT} query. For other types of SQL statements,
  \texttt{?ReturnList} should be an empty list.

  Examples: Assume that our database has a table
  \texttt{Person(name char(40),addr char(100),age integer)}. Then
  the following is a legal query:
  %% 
\begin{verbatim}
    ?- ?Tbl=Person, ?Age = 33,
       id1[query(qid,['SELECT name, addr FROM ',?Tbl, ' WHERE age=', ?Age],
                     [?Name,?Address]
                )
          ]@\sql.  
\end{verbatim}
  %% 
  Observe how the SQL query here is constructed at runtime: the table and the
  value of \texttt{age} are bound only when the above \ERGO query is
  executed. 

  Here is an example of an update statement:
  %% 
\begin{verbatim}
    id2[query(qa,
              ['insert into Person(name,addr,age)
                            values("mike","unknown",NULL)'],
              []
             )
       ]@\sql.  
\end{verbatim}
  %% 
  

\item Preparing queries.\\
  Frequent databases queries can be precompiled and optimized once and then
  executed multiple times, which is the recommended modus of operandi.
  (The previously described query interface is more flexible, but less
  efficient; it is typically used for infrequent queries or queries that
  must be constructed at run time, as in the above example.)

  For frequent queries that are known in advance, a two-step process is
  used. First, the query is \emph{prepared} (i.e., compiled and optimized)
  and then \emph{executed}. The preparation and execution of such queries
  allows certain level of flexibility by letting the user to place question
  marks \texttt{?} in lieu of some of the constants (these cannot be column
  names, table names, variable names, etc. --- only regular constants).
  These question marks can be replaced by actual constants at the query
  execution time.
  %% 
  \begin{itemize}
  \item    \texttt{?ConnectId[prepare(?QueryId,?QueryList)]@\bs{}sql.} \\
    The meaning of the parameters is the same as before.

    Example:
    %% 
\begin{verbatim}
      id1[prepare(qid,['SELECT T.addr FROM ', Person,
                       ' T where T.name = ? and T.age = ?']
                 )
         ]@\sql.    
\end{verbatim}
    %%
    The query Id \texttt{qid} can then be used to execute the above query,
    as shown below. 
    \item  \texttt{?QueryId[execute(?BindList,?ReturnList)]@\bs{}sql.} \\
      \texttt{?QueryId} must be bound to the query Id of a previously
      prepared query. \texttt{?BindList} must be a list of values that
      is supposed to be substituted for the \texttt{?}'s in the
      \texttt{prepare} command; the \texttt{?}'s are substituted in the
      order in which they appear in the \texttt{prepare} statement.
    
      Example:
      %% 
\begin{verbatim}
       qid[execute([mike,44],[?Address])]@\sql.      
\end{verbatim}
      %% 
  \end{itemize}
  %% 
  \item Closing query Ids.\\
    Like database connections, query Ids must be closed in order to release
    the resources that the OS allocates to the query. There is also a limit of
    2000 on the number of active queries, which can be easily reached in
    applications that query the database heavily. The command for closing the
    query Ids is:
    %% 
\begin{verbatim}
      ?QueryId[qclose]@\sql.    
\end{verbatim}
    %% 
    For instance,
    %% 
\begin{verbatim}
      qid[qclose]@\sql.    
\end{verbatim}
    %% 
\end{itemize}
%% 
Finally, we need to mention that when a NULL value is returned as a result
of a query, it is returned as a Prolog term \texttt{NULL(?)@\bs{}plg},
which is the internal representation of the \FLSYSTEM null quasi-constant
\texttt{\bs{}@?}. 
This implies that if such a term is used as an argument to a
literal that is to be inserted into the database, it
will be converted to the NULL value.


%%% Local Variables: 
%%% mode: latex
%%% TeX-master: "../ergo-packages"
%%% End: 
