\chapter[Querying SPARQL Endpoints]
{Querying SPARQL Endpoints\\
  {\Large by Paul Fodor and Michael Kifer}}


This chapter describes the \ERGO interface to 
SPARQL endpoints (i.e., remote processors that
support the SPARQL protocol---both querying and update statements),
which is based on Apache Jena. It should be noted from the outset
that several triple stores
implement SPARQL extensions that go well beyond the SPARQL 1.1 protocol and
Jena might not support some of them. The user will see syntax errors
whenever such extensions are used in SPARQL queries or update statements.

\section{General}

The \ERGO-to-SPARQL API is available through the \ERGO system module
\texttt{\bs{}sparql}
and calling anything \texttt{@\bs{}sparql}  will load that module. If, however, for
some reason it is necessary to load this module without executing any
operations, one can accomplish this by calling 
%% 
\begin{itemize}
\item  \texttt{ensure\_loaded@\bs{}sparql}. 
\end{itemize}
%% 

Prior to performing any queries against a SPARQL endpoint the user must
\emph{open} a \emph{connection} to that endpoint.
A connection is identified via \ERGO symbols, like \texttt{MyConnection123},
which are chosen by the user.
An endpoint is usually capable of supporting either queries (\emph{query
endpoint})   or updates (\emph{update endpoint}), but not both.  
%% 
\begin{itemize}
\item
  \texttt{System[open(?ConnectionId,?EndpointURL,?Username,?Password)]@\bs{}sparql}.
  Binds \texttt{?ConnectionId} to a \emph{query} endpoint specified by the
  \texttt{?EndpointURL} URL.  (See about \emph{update} endpoints
  below.) 
  \texttt{?ConnectionId}  must be bound to an \ERGO symbol (Prolog atom);
  it is a connection identifier, and it is chosen by the user.
  After opening, the connection Id
  can be used to query the endpoint without re-authentication.
  \texttt{?EndpointURL} must be the URL of a valid \emph{query}
  endpoint to which the user wishes
  to connect. It must be an atom.
  \texttt{Username}, and 
  \texttt{?Password}  must be bound to Prolog atoms (\ERGO
  symbols).  
  \\
  \emph{Example}:\\
  \texttt{System[open(DBPEDIAConnectionID,'http://dbpedia.org/sparql','','')]@\bs{}sparql}. 
  \\
  Binds the symbol \texttt{DBPEDIAConnectionID}  to the given
  \emph{query} endpoint
  with empty credentials (no user id or password). 
  If the connection fails due to an error at the endpoint URL or the user 
  credentials, an error will be issued.
  If the connection is successful, the query will succeed and one can use
  \texttt{DBPEDIAConnectionID} to query the specified endpoint.
\item
  \texttt{System[open(update(MyConnection),'http://localhost:7200/repositories/test/\\statements','','')]@\bs{}sparql}.
  \\
  Due to the peculiarities of the SPARQL 1.1 protocol, triple stores
  usually maintain \emph{different} endpoints (with different URLs!) for
  query and update operations. So, to both query and update the same triple
  store one must open two connections. The above form of the \texttt{open} 
  statement is used if one wants to connect to an \emph{update}
  endpoint.
\item \texttt{System[connectionType(?ConnectionId) -> ?Type]@\bs{}sparql}.\\
  Sometimes one might need to test programmatically if a particular
  connection is already open and get its connection type. This can be
  accomplished with the above call.
  \\
  If the connection is open, \texttt{?Type} gets bound to \texttt{query} or
  \texttt{update}---whichever applies. If the connection is not open, the
  call fails.   
\item \texttt{System[connectionURL(?ConnectionId) -> ?URL]@\bs{}sparql}.\\
  Like \texttt{connectionType} but returns the URL of the connection's 
  target endpoint instead of the connection's type.
\item  \texttt{System[close(?ConnectionId)]@\bs{}sparql}.
  \texttt{ConnectionId} must be an id of a previously open (and not yet
  closed) connection to a SPARQL end point.
  The method closes the connection and releases the space it holds.
  \\
  \emph{Example}:\\
  \texttt{System[close(DBPEDIAConnectionID)]@\bs{}sparql}. 
\end{itemize}
%% 

It should be noted that closing a connection is usually \emph{not} necessary
because each connection involves a relatively small memory overhead and the
memory is released when \ERGO exits.
This only becomes a problem if
the user opens (and keeps open) hundreds of thousands
connections. The only real inconvenience with keeping many connections open
is that one must keep all the names distinct.

Finally, it should be kept in mind
that all the definitions and examples in this chapter
show \ERGO statements in the context of a query or of a rule body. It should be
clear that these statements cannot be put in rule heads. If one wants to
execute them from within a file, they have to be prefixed with a
\texttt{?-}, as usual. For instance,
%% 
\begin{verbatim}
    ?- System[close(DBPEDIAConnectionID)]@\sparql.
\end{verbatim}
%% 



\section{Queries and Updates}

The \ERGO-to-SPARQL API supports several kinds of queries:
 \texttt{select},
 \texttt{selectAll},
 \texttt{construct}, 
 \texttt{ask},
 \texttt{describe}, 
 \texttt{describeAll}, 
 and 
 \texttt{update}.   
 Recall that SPARQL normally uses different endpoints for queries and
 updates. Accordingly,
 the first six statements utilize connections that were previously open and
 bound to SPARQL \emph{query} endpoints. The last (\texttt{update})
 statement utilizes connections that are bound to \emph{update} endpoints.   
%% 
\begin{itemize}
\item  \texttt{Query[select(?ConnectionId,?Query)->?Result]@\bs{}sparql} 
  \\
  runs a SPARQL SELECT 
  \texttt{?Query} and successively binds \texttt{?Result} to each answer via
  backtracking. 
  The \texttt{?Query} must be an \ERGO atom \emph{or} a  list. In the former case,
  the atom must form a valid SPARQL query. In the
  latter case, the list elements (which typically are \ERGO 
  atoms and variables)
  are converted into atoms and concatenated
  to form a valid SPARQL query.
  If the query is not valid, a syntax error is issued.
  Forming a query using lists is usually necessary only if one wants to pass
  values through variables from \ERGO to the query. The first example below
  does not pass any
  variables to the query, so we represent the query simply as an atom.
  The second example is more interesting, as it passes the \ERGO variable
  \texttt{?Subj}   into
  the query and so we use a list.
  \\
  \emph{Example}: \\
  \texttt{Query[select(DBPEDIAConnectionID,'SELECT * WHERE \{?x ?r ?y\}
    LIMIT 2')\\\hspace*{2cm}
    -> ?Result]@\bs{}sparql}.
  \\
\emph{Output}:
\begin{verbatim}
?Result=
   ["http://www.openlinksw.com/virtrdf-data-formats#default-iid"^^\iri,
    rdf#type,
    "http://www.openlinksw.com/schemas/virtrdf#QuadMapFormat"^^\iri]
?Result=
   ["http://www.openlinksw.com/virtrdf-data-formats#default-iid-nullable"^^\iri,
    rdf#type,
    "http://www.openlinksw.com/schemas/virtrdf#QuadMapFormat"^^\iri]
\end{verbatim}
  %% 
  \emph{Example}: \\
  \verb|?Subj="http://dbpedia.org/ontology/person"^^\iri,|
  \texttt{Query[select(DBPEDIAConnectionID,
    \\\hspace*{25mm}
    ['SELECT * WHERE \{', ?Subj, '?r ?y\} LIMIT 2'])
    \\ \hspace*{12mm}
    -> ?Result]@\bs{}sparql.}
  \\
  Note that this query passes the binding from the variable
  \texttt{?Subj} into the query. 
  It is important to not confuse \ERGO variables, like \texttt{?Subj}, with
  SPARQL variables, like \texttt{?r} and \texttt{?y}, in the above query.
  From the \ERGO perspective, \texttt{?Subj} is a real logical variable 
  and its binding is substituted into the
  list that forms the query. Without knowing anything about the actual SPARQL
  variables, \ERGO nevertheless ``magically'' successively binds the variable
  \texttt{?Result}  to the lists of pairs
  $[r_1,y_1],[r_2,y_2],...,[r_k,y_k]$, where each $r_i,y_i$ are
  the answers returned by SPARQL.
  In contrast, \texttt{?r} and \texttt{?y} are
  seen by \ERGO simply as sequences of characters that form the string
  \texttt{'?r ?y\} LIMIT 2'} that becomes part of the query after the list
  is concatenated. In fact, \ERGO does not even look inside that string.
  From SPARQL perspective, on the other hand, \texttt{?r}
  and \texttt{?y} are real variables through which it passes the answers to
  the query. In contrast, SPARQL does not see the \ERGO
  variable \texttt{?Subj} at all, as the binding for
  that variable becomes part of the query list before the actual query is
  formed and sent to SPARQL processor.
  
\item \texttt{Query[selectAll(?ConnectionId,?Query)->?ResultList]@\bs{}sparql}
  \\
  runs a SPARQL query, similarly to \texttt{select}, 
  except that \emph{all} results are returned at once
  in the list \texttt{?ResultList}. In contrast, the \emph{select} query
  returns the results from the query one-by-one. 
  Since we do not pass any values from \ERGO to the query, we represent the
  query simply as an atom.
  \\
   \emph{Example}: \\
  \texttt{Query[selectAll(DBPEDIAConnectionID,'SELECT * WHERE \{?x ?r ?y\} LIMIT 2')
    \\\hspace*{12mm}
    -> ?ResultList]@\bs{}sparql}.
   \\
\emph{Output}:
\begin{verbatim}
?Result=
   [["http://www.openlinksw.com/virtrdf-data-formats#default-iid"^^\iri,
     rdf#type,
     "http://www.openlinksw.com/schemas/virtrdf#QuadMapFormat"^^\iri],
    ["http://www.openlinksw.com/virtrdf-data-formats#default-iid-nullable"^^\iri,
     rdf#type,
     "http://www.openlinksw.com/schemas/virtrdf#QuadMapFormat"^^\iri]]
\end{verbatim}
  %% 
  

\item \texttt{Query[construct(?ConnectionId,?Query)->?Result]@\bs{}sparql}
  \\
  runs a 
  SPARQL \texttt{CONSTRUCT}  query.
  As before, \texttt{?Query} must be bound either to an atom (which must be
  a valid CONSTRUCT query) or to a list, which must concatenate into such a
  valid query. The latter, again, is used to pass values to the query via
  variables. 
  The CONSTRUCT query is an alternative
  query to SELECT, that instead of returning a table of results
  returns an RDF graph. The resulting RDF graph is created by taking
  the results of the equivalent SELECT query and filling in the values of
  variables that occur in the CONSTRUCT clause.
  The resulting graph (a list of triples) is then bound to \texttt{?Result}. 
  \\
  \emph{Example}: \\
  \texttt{Query[construct(DBPEDIAConnectionID,'CONSTRUCT
    {<http://example3.org/person> ?r ?y} WHERE {?x ?r ?y}  LIMIT
    2')->?Res]@\bs{}sparql}.

  Note that the query refers to a URL constant
  \texttt{<http://example3.org/person>}  using the SPARQL syntax for
  URLs (angle brackets). This syntax differs from the syntax for URLs in
  \ERGO, which is
  \texttt{"http://example3.org/person"$\hat{~}\hat{~}$\bs{}iri}. Note that
  in the second example for SELECT we passed an IRI to the query using
  the \ERGO syntax. \ERGO IRIs are converted to SPARQL URLs automatically.
  However, in that example, we could as well use an atom that represents
  the desired URL. For instance,
  \verb|?Subj = '<http://dbpedia.org/ontology/person>'|.

\item \texttt{Query[ask(?ConnectionId,?Query)]@\bs{}sparql}\\
  runs a SPARQL ASK query.
 An ASK query tests whether or not a query pattern has a solution.
 It does not return any results and simply succeeds or fails.
  \\
  \emph{Example}: \\
  \texttt{Query[ask(DBPEDIAConnectionID,'ASK \{?x ?prop  "Alice"\}')]@\bs{}sparql}.\\
  \emph{Output}: 'Yes' because DBpedia has a matching triple.
  
\item \texttt{Query[describe(?ConnectionId,?Query)->?Result]@\bs{}sparql}
  \\
  runs a SPARQL DESCRIBE query, which returns descriptions of
  RDF resources. These descriptions are bound to \texttt{?Result}. 
  \\
  \emph{Example}: \\
  \texttt{Query[describe(DBPEDIAConnectionID,'DESCRIBE ?y WHERE \{?x ?r ?y\} LIMIT 1')->?Result]@\bs{}sparql}.

\item \texttt{Query[update(?ConnectionId,?Query)]@\bs{}sparql}\\
  runs update operations on connection \texttt{?ConnectionId}, which must
  be bound to an \emph{update} endpoint.  The operations are
  \emph{insert}, \emph{delete},
  \emph{modify}, \emph{load}, and \emph{clear} (described in the standard: 
  \url{https://www.w3.org/TR/sparql11-update/}). 
  The update requires an update-enabled RDF triple server (e.g., GraphDB,
  Jena TDB, Virtuoso Universal Server).      
  \\
  \emph{Examples}: 
  \\
  \texttt{Query[update(ServerConnectionID,\\
    \hspace*{22mm}
    'PREFIX dc: <http://purl.org/dc/elements/1.1/> 
    \\
    \hspace*{24mm}
          INSERT DATA \{ <http://example/john> dc:title "A new book" ;
    \\
    \hspace*{54mm}
    dc:creator	"A.N.Other" . \}')]@\bs{}sparql}.
  \\
  \texttt{Query[update(ServerConnectionID,\\
    \hspace*{22mm}
    'PREFIX dc: <http://purl.org/dc/elements/1.1/>
    \\
    \hspace*{24mm}
    DELETE DATA \{ <http://example/john> dc:title "A new book" ;
    \\
    \hspace*{54mm}
    dc:creator	"A.N.Other" . \}')]@\bs{}sparql}.
\end{itemize}
%%
Here \texttt{ServerConnectionID} must be an endpoint that was previously
open on an update endpoint.

In addition, there are \texttt{constructAll} and \texttt{describeAll}
queries, which are related to \texttt{construct} and \texttt{describe}
queries the same way \texttt{selectAll} is related to \texttt{select}: the
variable \texttt{?Result} gets bound to a list that contains all answers
rather than one answer at a time.       

Additional examples of queries to standard endpoints (e.g., DBpedia and Wikidata
SPARQL endpoints) are provided in Coherent's ErgoAI Tutorial,
in the section on \ERGO connectors,
at
\url{https://sites.google.com/a/coherentknowledge.com/ergo-suite-tutorial/home/ergo-connectors}.

\section{Creating Your Own Triple Store}

A number of public SPARQL endpoints, such as DBpedia, exist in order to
play with SPARQL queries. However, if one wants to modify triples in the
store and create endpoints, a local (or a cloud) installation is needed.
In this section, we provide the instructions for two triple stores: GraphDB
fro Ontotext and Apache's Jena TDB with Fuseki server.

\subsection{GraphDB}

We found that GraphDB from Ontotext (\url{http://graphdb.ontotext.com/}) is
one of the easiest to install, maintain, and experiment with. This is a
commercial triple store, but by registering
(\url{http://info.ontotext.com/graphdb-free-ontotext} one can obtain a free
license, which supports all major features of the product for small
projects.  To
install GraphDB, use the installation package appropriate for your system.
Below are the instructions for Ubuntu Linux (Mint Linux with Cinnamon, to
be precise).

After installing the \texttt{graphdb-free-7.1.0.deb} package (provided to
you by Ontotext after registering), you will find
GraphDB in the Programming category in the Start menu.
Choosing GraphDB from the menu will open a console and a Firefox browser 
with a tab open on the GraphDB workbench. If you don't have Firefox installed,
just head to \texttt{localhost:7200}  in your favorite browser.
The Workbench lets you create new triple stores (in the \texttt{Admin}
menu), put information into the store, and query it.
Since we want to query our triple store using \ERGO, skip the query/update
form: just use the \texttt{Admin} menu to create/administer your
store.

Let's suppose we created a triple store called \texttt{Test}. 
In response, GraphDB creates two endpoints:
\url{http://localhost:7200/repositories/Test} --- a query endpoint and
\url{http://localhost:7200/repositories/Test/statements} --- an update endpoint.
By opening an \ERGO query connection to the former endpoint
and an update connection to the latter you will be able to
use \ERGO to manage your own triple store!

\subsection{Jena TDB}

Jena TDB from Apache is an open source triple store with full support for
the SPARQL 1.1 protocol.
To install it, visit \url{http://jena.apache.org/download/#jena-fuseki}
and download the latest Apache Jena Fuzeki. As of this writing, the latest
release is
\texttt{apache-jena-fuseki-2.4.0.zip} (or you can choose a tar.gz file).

Unzip the above file in a desired directory (say, \texttt{TDB}), change to the directory
\texttt{TDB/apache-jena-fuseki-2.4.0/} and type
%% 
\begin{verbatim}
   fuseki-server --update --mem /test
\end{verbatim}
%% 
(\texttt{fuseki-server.bat} on Windows).
This will create an \emph{in-memory} triple store called \texttt{test}.
Since it is an in-memory store, any data inserted into it will be deleted
when the Fuseki server terminates (kill it by typing Ctrl-C).
In addition, Fuseki will create two SPARQL endpoints: a query endpoint at
\url{http://localhost:3030/test/query} and an update endpoint at
\url{http://localhost:3030/test/update}. Use these endpoints to perform
operations on this triple store via \ERGO.

To create a persistent triple store, you need to create a subdirectory in
\texttt{TDB/apache-jena-fuseki-2.4.0/}, say \texttt{MyTestDB} and then
start the Fuseki server like this:
%% 
\begin{verbatim}
   fuseki-server --update --loc=MyTestDB /test
\end{verbatim}
%% 
Note that \texttt{MyTestDB} is the name of the directory in which to store
the data while \texttt{test} is the name of the \emph{service}. So, the
SPARQL endpoints for this persistent store would be the same as in the
previous example:    \url{http://localhost:3030/test/query} and 
\url{http://localhost:3030/test/update}.

You can manage this and other triple stores on this server by heading to
the Fuseki workbench site at \url{localhost:3030} in your favorite browser.

To protect the triple stores with a password, edit the file 
\texttt{TDB/apache-jena-fuseki-2.4.0/run/shiro.ini} and add
users under the \texttt{[users]} section. For instance,
%% 
\begin{verbatim}
[users]
its_me=its_my_pw
\end{verbatim}
%% 

%%% Local Variables: 
%%% mode: latex
%%% TeX-master: "../ergo-packages"
%%% End: 
