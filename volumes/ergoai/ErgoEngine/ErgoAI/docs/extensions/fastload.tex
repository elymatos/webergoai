
\subsection{Fast Loading Vast Amounts of Data}
\label{sec-fastload}

When one needs to load vast amounts of data (hundreds of thousands to
millions of facts), running them through the regular \FLSYSTEM compiler can
take too long because that compiler is built to recognize the rich syntax
of \FLSYSTEM. In contrast, large data sets invariably have very simple
structure and it would be a waste to run a complex compiler to import all
that data. To this end, \FLSYSTEM provides special primitives to load,
query, and destroy such vast amounts of data provided the input has very
regular structure: Prolog (not HiLog)
terms with no infix, prefix, or postfix operators.
It can contain \FLSYSTEM preprocessor commands (Section~\ref{sec-gpp})
and comments, but the comments must be either enclosed between \texttt{/*
  ... */} or be one-line comments beginning with a \texttt{\%} ({\it
  i.e.}, Prolog style, not \FLSYSTEM style).   

For instance, the following input would work
%% 
\begin{verbatim}
   foo(1,bar(abc,cde),q2).
   foo(2,bar(ab,ce),q3).
   bar(23,foo(123,[a,b,c]),"q33").
\end{verbatim}
%% 
but the following will cause errors and loading will be aborted:
%% 
\begin{verbatim}
   foo(2,bar(ab,ce),q3).  %% ok
   foo(1+2,abc).          %% nope: + is an infix operator
   bar(2,foo(123,"q33").  %% nope: parentheses missing
   pqr(1,?f(2,g(3))).     %% nope: ?f(2,...) is a HiLog term
   pqr(1,ff(2,${g(3)})).  %% nope: reification (${g(3)}) isn't allowed
\end{verbatim}
%% 
The fast-loaded facts are placed in a special storage container that is not
accessible to regular queries: these facts are to be queried via a special
primitive. The primitives for fast-loading and working with these storage
containers are as follows:
%% 
\begin{itemize}
\item  \texttt{fastload\{{\it FileName\/},{\it Storage\/}\}} --- fast load
  \emph{Filename} into storage container \emph{Storage}. \emph{Filename}
  must have the extension \texttt{.P} and
  \emph{Storage} must be an alphanumeric string.
  Executing multiple \texttt{fastload} commands for the same storage name
  will add all facts up from the files involved.
  Duplicates are \emph{not} eliminated.
  \\
  Example: \texttt{?- fastload\{myfile,mystorage\}}. 
\item \texttt{fastquery\{{\it Storage\/},{\it Query\/}\}} --- query    
  the information stored in the storage container \emph{Storage}.
  \emph{Query} must be a HiLog term. Other forms are also allowed, but
  they will return nothing, as they will not match the input of the form
  described above. Variables in the query term are allowed.
  Note also that even though the input consists of Prolog terms, they are
  converted to HiLog so HiLog terms can be used to query the information
  stored in these containers.
  \\
  Examples:\\
  \hspace*{5mm}\texttt{?- fastquery\{mystorage,?X\}}. \\
  \hspace*{5mm}\texttt{?- fastquery\{mystorage,?Y(p,q)\}}. \\
  \hspace*{5mm}\texttt{?- fastquery\{mystorage,r(p,q)\}}.
\item \texttt{fasterase\{{\it Storage\/}\}} --- erase all the information
  in the given storage container.   
  \\
  Example: \texttt{?- fasterase\{mystorage\}}. 
\end{itemize}
%% 

\paragraph{Notes:}
%% 
\begin{itemize}
\item  The \texttt{fastload} command should not occur in the body of any
  rule that has a tabled head-literal---directly or indirectly (via other
  rules). The most straightforward way to call \texttt{fastload} is via a
  top-level query, as in the above examples. 
\item The intended use of \texttt{fastload} is that the data will be loaded
  early in the process, before any kind of significant queries are
  computed. It is not an error to call \texttt{fastload} \emph{after} such
  queries, but then the tables associated with those
  queries will be abolished. The result will be that, if the same or related
  queries are asked again, they will be recomputed from scratch,
  resulting in some delay. 
\end{itemize}
%% 


\index{storage name!isfastloaded\{StorageName\}}
\index{isfastloaded\{StorageName\} primitive}
\index{primitive!isfastloaded\{...\}}
%%
Sometimes it is useful to check which storage names are fast-loaded or if a
particular storage is loaded (say, because you might want to load it,
if not).  To find out which storages are loaded at present time, use the
primitive {\tt isfastloaded\{...\}}. For instance, the first query,
below, succeeds if the storage {\tt foo} is fast-loaded. The second query
succeeds and binds {\tt L} to the list of all fast-loaded storages.
%%
{\tt
\begin{quote}
 ?- isfastloaded\{foo\}.\\
 ?- ?L= setof\{?X|isfastloaded\{?X\}\}.
\end{quote}
}
%%
\noindent
The primitive
\texttt{setof} and other aggregate operators are discussed in
Section~\ref{sec-aggregates}.

\index{isfastloaded\{FileAbsName,StorageName\} primitive}
\index{isfastloaded\{FileAbsName,StorageName,\\FileLocalName\} primitive}
One can also check which files are fast-loaded into what storages using the
primitive\\
\texttt{isfastloaded\{FileAbsName,StorageName\}}.
\\
There is also a three-argument version of this primitive:
\\
\texttt{isfastloaded\{FileAbsName,StorageName,FileLocalName\}}
\\
This one
also includes the local name of the file (sans the directory)
from which \emph{StorageName} was loaded. 


%%% Local Variables: 
%%% mode: latex
%%% TeX-master: "../ergo-manual"
%%% End: 
