
\subsection{Querying Rules via the Rule Structure Database}

\index{rule structure database}
%%
The \emph{rule structure database} provides a simple yet powerful
interface for querying the rule structure of \ERGO knowledge bases. In many
respects, this interface goes beyond what \texttt{clause\{...\}} can do,
but it does not replace \texttt{clause\{...\}}.
The major advance over \texttt{clause\{...\}} is that it allows one to find
all rules \emph{containing} a particular atomic formula---either in the
head or in the body. This is possible even if the formula occurs under
\texttt{\RULELOGNEG}  and/or \texttt{\RULELOGNAF},  under a quantifier,
aggregate, and other
constructs. In addition, the query can have complex goals such as 
those allowed under the Lloyd-Topor and omniformity extensions described in
Sections~\ref{sec-lt} and \ref{sec-omniform}.


\paragraph{Requesting the structural database feature.}
\ERGO does not build the structure database by default because of a slight
overhead in compile time, load time, and additional demands on RAM.
The user can request that \ERGO builds a structure database for a file by
placing the directive
%%
\index{use\_rule\_structure\_db directive}
\index{directive!use\_rule\_structure\_db}
%% 
\begin{verbatim}
    :- use_rule_structure_db.
\end{verbatim}
%% 
in the file (or in a file \texttt{\#include}'ed in that file).
If the above directive is not present, the calls to the structural query
API will simply fail (will return false).

If, for some reason, it is preferable to not include
the above directive in a file, structure database may be requested at
\emph{runtime} by executing the primitive
\texttt{use\_rule\_structure\_db\{Module,Flag\}} as a query. For instance,
%% 
\begin{verbatim}
    ?- use_rule_structure_db{foo,on}.
\end{verbatim}
%% 
From then on, any file compiled for loading into the module \texttt{foo} will
have the rule structure database available. To turn this mode off, use
%% 
\begin{verbatim}
    ?- use_rule_structure_db{foo,off}.
\end{verbatim}
%% 
Note, these commands \underline{\emph{affect only the files that need
  recompilation}} (because they were never compiled or because they were
changed since their last compilation). If a file was compiled \emph{before}
the change in the default mode for rule structure databases (from
\texttt{on} to \texttt{off} or vice versa) then reloading such a file will
not cause a recompilation and thus the availability of that database with
respect to the aforesaid file will not change.

\paragraph{Structural queries.}
The structural database provides special primitives for querying the rule
structure:
%% 
\begin{itemize}
\item 
  \texttt{structdb\{\textnormal{\textit{?Goal},\textit{?SearchSpec},\textit{?Id},\textit{?Module},\textit{?File}}\}
      }
      \\
      Here \emph{?Goal} and \emph{?SearchSpec} must be bound. The \emph{?Goal}
      argument is a formula that is allowed in the head or the body of a
      rule; it does not need to be reified, as \ERGO knows what to expect
      here.
      The \emph{?Goal}  formula is decomposed into a list of atomic frames
      or predicates
      and then the rule base is searched according to \emph{?SearchSpec}.
      The latter can be as follows:
      \begin{itemize}
      \item \texttt{any} --- means that a rule satisfies the search
        condition if it contains \emph{any} of the aforesaid atomic formulas
        (in the head or body)
      \item \texttt{all} --- means a rule satisfies the search
        condition if it matches  \emph{all}  of the aforesaid atomic formulas
        (whether they appear in the head or body)
      \item $X+Y$, where $X$ can be \texttt{any} or \texttt{all} and
        $Y$ can be \texttt{head}, \texttt{body} or a variable. For instance,
        \texttt{all+head} means that for a rule to satisfy the search
        condition it must contain all of the aforesaid atomic formulas in
        the head; \texttt{all+body} is similar except that all the formulas
        must occur in the body of that rule. If \texttt{any+head} is given,
        it means that it is enough to have just one of the aforesaid atomic
        formulas in the rule head, and so on.        
      \end{itemize}
      %%
      The output variables are \emph{?Id}, \emph{?Module}, and \emph{?File}.
      Why these three? Recall from Section~\ref{sec-rule-id} that a rule
      Id is composed out of these three components and knowing them opens
      the door to the host of rule manipulation meta-facilities described
      in this manual.
  \item
    \texttt{structdb\{\textnormal{\textit{?Id},\textit{?Module},\textit{?File},\textit{?Goal},\textit{?LocationInRule},\textit{?Context}}\}
    }
    \\
    Here \emph{?Id} and \emph{?Module} must
    be bound before issuing a call to this primitive. That is, given a
    partial (or complete) rule Id, this primitive returns the atomic
    formulas that occur in the corresponding rules (via the \emph{?Goal}
    variable).
    In addition, for each such formula, it tells us (via the
    \texttt{?LocationInRule} variable) whether that formula occurs in the
    rule head or in its body. Finally, the semantic context of the formula
    is also returned as a list of semantic attributes.
    The relevant semantic attributes are $\RULELOGNAF$, $\RULELOGNEG$,
    $\PLGNAF$, and all the aggregate functions \texttt{avg}, \texttt{sub},
    \texttt{setof}, and the others.   
    For instance, if, say, $max\{?X\mid p(?X,?Y),\, \RULELOGNAF\, \RULELOGNEG \;q(?Y)\}$
    appears in the rule then the semantic context of $p(?X,?Y)$ would be
    the list \texttt{[max]} while for $q(?Y)$  it will be
    \texttt{[\RULELOGNEG,\RULELOGNAF,max]} (the semantic attributes are
    listed in the reverse order of their appearance). 
\end{itemize}
%% 
Here are some examples that illustrate the use of the above primitives.
\begin{verbatim}
  :- use_rule_structure_db.
  @!{r1} p(?X) :- q(?X), r(?X).
  @!{r2} p(?X)==>r(?X) :- q(?X), ?X=avg{?V|\neg t(?V)}.
  @!{r3} \neg q(?X) :- \neg p(?X).
  ?- structdb{(p(?)==>r(?)),all+head,?Id,?Mod,?File}.
  ?- structdb{(p(?)==>r(?)),all,?Id,?Mod,?File}.
  ?- structdb{(p(?)==>r(?)),any,?Id,?Mod,?File}.
  ?- structdb{r2,main,?,?P,?H,?C}.
\end{verbatim}
%%
Then the first query returns
%% 
\begin{verbatim}
  ?Id = r2
  ?Mod = main
  ?File = 'foo.ergo'
\end{verbatim}
%% 
because \texttt{p(?)} and \texttt{r(?)} both occur in the rule \emph{head}
only in rule \texttt{r2}. The second query returns 
%% 
\begin{verbatim}
  ?Id = r1
  ?Mod = main
  ?File = 'foo.ergo'

  ?Id = r2
  ?Mod = main
  ?File = 'foo.ergo'
\end{verbatim}
%% 
because \texttt{p(?)} and \texttt{r(?)} both occur in \emph{some} part of a
rule (head or body) in rules \texttt{r1} and \texttt{r2}. The third query   
returns information about all three rules because it is looking for rules
where either \texttt{p(?)} or \texttt{r(?)} occur (in the head \emph{or}
body). 
Finally, the last query (which uses the 6-argument version of the
\texttt{structdb} primitive) returns the information about all the literals
found in rule \texttt{r2}. For each literal it says where the literal
occurs and in what context:
%% 
\begin{verbatim}
  ?P = ${p(?_h2847)@main}
  ?H = body
  ?C = []

  ?P = ${p(?_h2875)@main}
  ?H = head
  ?C = ['\\neg']

  ?P = ${q(?_h2819)@main}
  ?H = body
  ?C = []

  ?P = ${r(?_h2729)@main}
  ?H = head
  ?C = []

  ?P = ${r(?_h2757)@main}
  ?H = body
  ?C = ['\\neg']

  ?P = ${t(?_h2787)@main}
  ?H = body
  ?C = ['\\neg', avg]

  ?P = ${\neg p(?_h2963)@main}
  ?H = head
  ?C = []

  ?P = ${\neg r(?_h2905)@main}
  ?H = body
  ?C = []

  ?P = ${\neg t(?_h2933)@main}
  ?H = body
  ?C = [avg]
\end{verbatim}
%% 
Of particular interest here are the answers such as
%% 
\begin{verbatim}
  ?P = ${r(?_h2757)@main}
  ?H = body
  ?C = ['\\neg']

  ?P = ${t(?_h2787)@main}
  ?H = body
  ?C = ['\\neg', avg]

  ?P = ${\neg t(?_h2933)@main}
  ?H = body
  ?C = [avg]
\end{verbatim}
%% 
The first says that \texttt{r(...)} occurs in the body under negation
\texttt{\RULELOGNEG}, the second says that
\texttt{t(...)} occurs in the body under first 
\texttt{\RULELOGNEG} and then the aggregation \texttt{avg}. The last answer
rephrases the second one, saying that the literal \texttt{\RULELOGNEG}
\texttt{t(...)} occurs under aggregation \texttt{avg}. 

Finally, we give an example of a 5-argument \texttt{structdb} query where
the mode has form \texttt{all+?Y}  or \texttt{any+?Y}:
%% 
\begin{verbatim}
  ?- structdb{(p(?)==>r(?)),all+?H,?Id,?Mod,?File}.

  ?H = body
  ?Id = r2
  ?Mod = main
  ?File = 'foo.ergo'

  ?H = head
  ?Id = r2
  ?Mod = main
  ?File = 'foo.ergo'
\end{verbatim}
%% 
We do know that \texttt{r2} is the right rule here, but 
one might wonder why in the first answer \texttt{?H} is bound to
\texttt{body} while in the second to \texttt{head}. On the surface,
the first answer seems incorrect since in rule \texttt{r2} both
\texttt{p(?)} and \texttt{r(?)} seem to appear in the head only.   
The explanation is that that rule
  \texttt{p(?X)==>r(?X) :- q(?X), ?X=avg\{?V|\RULELOGNEG t(?V)\}}
has a composite head, which gets
transformed by the omniformity transformation (Section~\ref{sec-omniform})
into two rules:
%% 
\begin{verbatim}
  r(?X) :- q(?X), ?X=avg{?V|\neg t(?V)}, p(?X).
  \neg p(?X) :- q(?X), ?X=avg{?V|\neg t(?V)}, \neg r(?X).
\end{verbatim}
%% 
Thus, in reality, both \texttt{r(?)} and \texttt{p(?)} appear in the
head as well as in the body of the rules created out of the original rule.


%%% Local Variables: 
%%% mode: latex
%%% TeX-master: "../ergo-manual"
%%% End: 
